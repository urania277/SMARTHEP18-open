%%Explain the appropriateness of the infrastructure of each participating organisation, as outlined in Section 5 (Participating Organisations), in light of the tasks allocated to them in the action.
%
All nodes in the consortium have infrastructure appropriate to the tasks
they are responsible for. \lundentity's management is supported by its Research Services, 
which are appropriate for a large university. 
All ESRs will have access to the complete range of \acronym
expertise (Sec.~\ref{ss:competence_44}) and infrastructure (part B2),
and be provided with office space and a laptop for use during \acronym. 
Beneficiaries and the secondment officer will also help with accommodation,
visas and practical issues, making use of the EURAXESS Services for Relocation Assistance. 
Academic support granted to ESRs includes close supervision, mentoring by senior scientists and 
institutional academic literature subscriptions, as well all the facilities supporting research at the nodes. 
All institutes and research laboratories have software divisions with excellent computing
capabilities and infrastructure, including rack-mounted
servers and large data storage systems, locally and on the Worldwide Computing Grid. 

As detailed for each node in part B2, \acronym participants
contribute with the specific facilities and equipment required for the successful completion of the WPs.
In particular, partners hosting secondments involving GPUs and hybrid architectures host those on their premises
and ESRs will be able to use them during the secondment and remotely on demand. 
Moreover, a number of industrial entities have agreed with Prof. Lacassagne that ESR10
will be able to make use of beyond-state-of-the-art machines to test algorithms developed
throughout the duration of \acronym, and that ESR10 will
be able to disseminate results in open access journals as soon as
the machines become public, within the course of their PhD project. 

%While \dq is a SME, it possesses independent and fully equipped office space and
%extensive private computing facilities, on par with what students
%would have access to at the academic nodes.
%
\subsection{Competences, experience and complementarity of the participating organisations and their commitment to the program}
\label{ss:competence_44}

\subsubsection{Consortium composition and exploitation of participating organisations' complementarities}
\label{sub:composition}

The academic beneficiaries and partners of \acronym are 
the pioneers and experts of RTA in each of the LHC experiments. They have a strong
background in the physics topics that are the focus of the network, as the network includes 5 ERC grantees
and scientists who have covered and are covering positions of responsibilities within large
international collaborations as detailed in Secs.~\ref{subsub:qual_supervisors} and part B2.
Even though the network is mostly composed and led by early-career, emergent researchers, 
they all have extensive experience in student supervision and research. 
Researchers in \acronym have already been successfully collaborating within their experiments on complementary topics. 
Scientists in \nikhefentity, \cnrsentity, \dortmundentity (LHCb), \oregonentity, \ohioentity, \lundentity, \heidelbergentity (ATLAS),
\helsinkientity (ATLAS)  and \lundentity (ALICE) lead the trigger and RTA techniques
developments for each of the collaborations. Expertise in ML is given by \pointeightentity, \unigeentity, 
by LHCb researcher Ustyuzhanin from the Yandex School of Computing and \yandexentity.  

%Gligorov and Albrecht
%hold ERC grants on real-time physics analysis on the LHCb collaboration and have served in multiple senior coordinating roles on the experiment.
%%and on the first triggerless detector readout. 
%Their expertise is complemented by Raven from \nikhefentity, former LHCb HLT project leader and one of the main authors of LHCb trigger software over the last decade, and by the expertise in ML training
% 

%Doglioni from (\lundentity, ERC StG 2015), Boveia (\ohioentity),  Starovoitov and
%Dunford in (\heidelbergentity) are the pioneers of the first trigger-level search for new physics in the
%ATLAS collaboration. Together with the trigger and physics experts Strom and Majewski (\oregonentity), 
%Petersen (\cern), Sfyrla (\unigeentity), Igonkina (\nikhefentity), with the computing grid and data analysis expert 
%Smirnova (\lundentity) and with the machine learning expert Schramm (\unigeentity), 
%they represent a team that can design analyses targeting discoveries that
%advance high energy physics with novel techniques. 
%The same can be said of the CMS researchers, M. Voutilainen, P. Eerola and H. Kirschenmann (\helsinkientity)
%and M. Pierini (\cernentity, ERC CoG 2017 on Machine Learning at the LHC), who have designed and 
%published the first search for new particles in CMS using trigger objects. P. Christiansen at \lundentity
%and R. Shahoyan from \cernentity complement the data analysis in real-time from the ALICE collaboration,
%attempting the first physics measurement with readout at the LHC collision rate for their Time Projection Chamber. 

The synergy in the RTA physics program of \acronym is cross-collaboration,
with researchers from \nikhefentity, \cnrsentity and
\dortmundentity researching LFV/LFU in ATLAS and LHCb,
researchers from \unigeentity, \cernentity, \helsinkientity, \heidelbergentity, \ seek new physics through 
Higgs and dark sector particles in ATLAS and CMS,
while \lundentity and \helsinkientity precisely measure the Standard Model of particle physics in CMS and ALICE. 
The network also benefits from theory expertise with the affiliated theorists of
Dark Matter and LFV/LFU from \ohioentity and \cincinnatientity. 
The combination of WP4, led by one researcher per collaboration, and WP7 led by \cernentity
will ensure that the results of physics research are disseminated and communicated effectively. 
Hybrid architectures (WP5, led by L. Lacassagne of \cnrsentity) are another necessary ingredient of RTA,
as software-only solutions on standard CPUs are not sufficiently fast for the purposes of HEP and industry alike. 

Software and machine-learning experts from \pointeightentity,
\ximantisentity, \heidelberginstrumentsentity, \yandexentity and \ibmentity
as well as  experts in hybrid infrastructures in \pisaentity, \cnrsentity, \fleetmaticsentity,
\lightboxentity, and \cathientity provide the complementary
tools to develop efficient solutions to both industry and research problems, and in the case of the
industrial partners, practical implementations with a commercial value. 
The combination of Machine Learning, novel data analysis techniques and hybrid architectures 
are applied to commercial examples of real-time data analysis. This cross-pollination can only happen with the 
involvement of the non-academic participants, within WP6 coordinated by \heidelbergentity and \dqentity. 
All commercial partners have a concrete interest in these techniques. 

The status of \cernentity as an international collaboration that
hosts all HEP experiments provides the ideal framework in which
to exploit the research complementarity within the network and support its 
advancements beyond the state of the art. An example is the recently created Internet-of-Things
Openlab effort at CERN\footnote{First workshop of Internet-of-Things at CERN, \url{https://indico.cern.ch/event/669690/overview}, Nov 2017}
whose organizers informally agreed to support this network through a half-day 
training and presentation day. 
The participation of all LHC collaborations to \acronym and the explicit support of the HEP Software Foundation offers a unique chance to form lasting collaborations and shape the future data taking strategies of LHC experiments and beyond, building from the  results obtained in \acronym.  

%%%%Non-academic



%As explained in Sec.~\ref{sec:supervision}, all the beneficiaries 
%%are either able to enrol the ESRs in their doctoral school programs
%%or have established co-supervision arrangements 
%%for the ESRs to obtain their  in another institution within the Network. 
%training in industry and knowledge transfer at \dq [mention heidelberg school]. 
%In particular the dedicated and hands-on \cern HEP schools and the [school of data analysis and ML]
%will complement the more traditional university-based lectures of other academic nodes, while
%\dq 
%%, \technopolis 
%and the partner companies will provide training which will greatly increase the employability of
%the ESRs. %Summer schools, academic lecture series as well as the full spectrum
%%of training courses offered by the \acronym participants will be open to all ESRs.



%%explain the compatibility and coherence between the tasks attributed to each
%beneficiary/partner organisation in the action, including in light of their experience;


%
%The research synergies and complementarities
%between the \acronym participants have been described in detail in Sec.~\ref{sec:synergy}.
%
%
%The non-academic beneficiaries and partners have been chosen for their active involvement in research
%and development related to optimising decisions with a commercial purpose, 
%leading to a common interest in RTA shared with the \acronym participants, 
%as detailed in Sec.~\ref{sec:EXCELLENCE}. 
%
%%Of particular importance are the training courses and secondments opportunities
%%offered by our industrial partners, which will give first-hand insight
%%and hands-on experience in the commercial world.
%
%TO BE FILLED AFTER TRAINING PART IS DONE
%

\subsubsection{Commitment of beneficiaries and partner organisations to the program}
%(for partner organisations, please see also sections 5 and 7).

The first in-person meeting for the members of the \acronym network took place in \lundentity 
in May 2017\footnote{\href{https://indico.lucas.lu.se/event/656/}{SMARTHEP kickoff meeting}, supported by the Grace och Philip Sandbloms Fund}, and a
number of subsequent meetings have happened at CERN culminating in a follow-up 
workshop dedicated to the preparation and collaborative editing of this ITN
grant\footnote{\href{https://indico.lucas.lu.se/event/775/}{SMARTHEP ETN writing meeting}, 
supported by the Grace och Philip Sandbloms Fund and \lundentity Research Services}. 
It is clear from the enthusiasm and participation of all members, as well
as from the support offered to this network by \lundentity, that all \acronym
participants are fully committed to this project, as 
it is the first network of its kind including all major LHC experiment
that will consistently advance RTA through
Machine Learning and hybrid architectures for modern particle physics experiments
and industry alike. The majority of the \acronym academic participants are already successfully
collaborating in research projects within the LHC experiments.

High quality education of students in large-scale
international projects is already an essential part of the mission
of the universities and research labs in \acronym. \acronym will
greatly strengthen and extend their capacities to carry out these activities. 
The doctoral advisors will devote a significant fraction of their time
(see node descriptions in part B2) both to ESR supervision and to \acronym activities. 
All industrial nodes are committed to \acronym, to benefit from the experience in RTA of LHC experts, to 
foster connections to the academic world and to expand their activities. 
The industry parties have a strong motivation to participate in cutting edge developments
and the subsequent transfer of the technologies to new commercial applications.
\acronym partners will not only contribute training to \acronym,
but will also benefit significantly in their own research and commercial objectives
through the secondment projects of the ESRs. 

The relations between the nodes,
established starting from the preparation of the project will last even beyond \acronym:
for example there have already been discussions to open the series of training events to the HEP community
as there is a great need for this kind of training, organize post-graduate recruitment 
events for companies in the network that specifically target \acronym ESRs after the completion of their PhD. 

%\acronym participants therefore have already established long-lasting
%and fruitful professional relations, marked by mutual respect and trust.
%In short, there is no doubt that all participants in \acronym
%will show the highest level of commitment to the research program. 


%i) Funding of non-associated third countries (if applicable): Only entities from EU Member States, from Horizon 2020 Associated Countries or from countries listed in General Annex A to the Work program are automatically eligible for EU funding. If one or more of the beneficiaries requesting EU funding is based in a country that is not automatically eligible for such funding, the application shall explain in terms of the objectives of the action why such funding would be essential. Only in exceptional cases will these organisations receive EU funding.16 The same applies for international organisations other than IEIO.

%ii) Partner organisations: The role of partner organisations and their active contribution to the research and training activities should be described. A letter of commitment shall also be provided in section 7 (included within the PDF file, but outside the page limit).
