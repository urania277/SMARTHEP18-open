%%Explain the appropriateness of the infrastructure of each participating organisation, as outlined in Section 5 (Participating Organisations), in light of the tasks allocated to them in the action.
%
All nodes in the consortium have infrastructure appropriate to the tasks
they are responsible for. \lundentity's management is supported by its Research Services, 
which are appropriate for a large university. 
A number of the nodes already hold the \href{https://euraxess.ec.europa.eu/jobs/charter/declaration-endorsement}{European HR Excellence in Research Award}, or are taking steps to obtain it.
All ESRs will have access to the complete range of \acronym
expertise (Sec.~\ref{ss:competence_44}) and infrastructure (part B2),
and be provided with office space and a laptop for use during \acronym. 
Beneficiaries and the secondment officer will also help with accommodation,
visas and practical issues, making use of the EURAXESS Services for Relocation Assistance. 
Academic support granted to ESRs includes close supervision, mentoring by senior scientists and 
institutional academic literature subscriptions, as well all the facilities supporting research at the nodes. 
All institutes and research laboratories have software divisions with excellent computing
capabilities and infrastructure, including rack-mounted
servers and large data storage systems, locally and on the Worldwide LHC Computing Grid (WLCG),
of which Smirnova from \lundentity is an architect.

As detailed for each node in part B2, \acronym participants
contribute with the specific facilities and equipment required for the successful completion of the WPs.
In particular, partners hosting secondments involving GPUs and hybrid architectures host those on their premises
and ESRs will be able to use them during the secondment and remotely on demand. 
Moreover, a number of industrial entities have agreed with Prof. Lacassagne that \ESRg
will be able to make use of beyond-state-of-the-art machines to test algorithms developed
throughout the duration of \acronym, and that \ESRg will
be able to disseminate results in open access journals as soon as
the machines become public, within the course of their PhD project. 

\subsection{Competences, experience and complementarity of the participating organisations and their commitment to the program}
\label{ss:competence_44}

\subsubsection{Consortium composition and exploitation of participating organisations' complementarities}
\label{sub:composition}

%The academic participants in \acronym are the pioneers and experts of RTA in each of the LHC experiments, and include ERC grantees and scientists who have covered and are covering positions of responsibilities within large international collaborations as detailed in Secs.~\ref{subsub:qual_supervisors} and part B2. 

A strong expertise in software and computing tools is crucial for the success of the research projects. This is provided by \nikhefentity and \cernentity with the main authors of the LHCb trigger software over the last decade, as well as computer scientists in \sorbonneentity and \uniboentity. From the industrial side, having \ibmentity, one of the largest producers of computing hardware and software in the world, is an unique asset for \acronym with the participation of experienced members. The software competences are matched by the detector hardware expertise of \lundentity and \cernentity. 

ML experts from \liegesentity, \uniboentity, \pointeightentity, \ximantisentity, \cernentity and \ibmentity, as well as experts in hybrid infrastructures in \sorbonneentity, \santiagoentity, \pisaentity, \cnrsentity, \oregonentity, \ohioentity, \fleetmaticsentity and \lightboxentity provide the complementary tools to develop efficient solutions to both industry and research problems, and in the case of the industrial partners, practical implementations with a commercial value. 

Scientists in \nikhefentity, \cnrsentity, \dortmundentity (LHCb), \oregonentity, \ohioentity, \lundentity, \heidelbergentity (ATLAS), \helsinkientity (CMS)  and \lundentity (ALICE) lead the trigger and RTA techniques developments for each of the collaborations as detailed in Secs.~\ref{subsub:qual_supervisors} and part B2. The synergy in the RTA physics program of \acronym is also cross-collaboration, with a synergy in physics topics that allow the ESRs to acquire expertise from different node members. Similarly, the industrial partners have similar interests in common tools (see Fig. 2) so that they can learn from each other. 

The participation of all LHC collaborations to \acronym and the explicit support of the HEP Software Foundation (see letter of support) offers a unique chance to form lasting collaborations and shape the future data taking strategies of LHC experiments and beyond, building from the results obtained in \acronym. 
The status of \cernentity as an international collaboration that hosts all HEP experiments provides the ideal framework in which to exploit the research-industry complementarity and support its advancements beyond the state of the art.
%An example is the recently Internet-of-Things Openlab effort at CERN\footnote{First workshop of Internet-of-Things at CERN, \url{https://indico.cern.ch/event/669690/overview}, Nov 2017} whose organizers informally agreed to support this network through a half-day training and presentation day. 
\acronym will allow its participants and the ESRs involved in it to work on projects with cross-pollination not only between different LHC experiments, but also between research and industry. 

%%%%Non-academic

%As explained in Sec.~\ref{sec:supervision}, all the beneficiaries 
%%are either able to enrol the ESRs in their doctoral school programs
%%or have established co-supervision arrangements 
%%for the ESRs to obtain their  in another institution within the Network. 
%training in industry and knowledge transfer at \dq [mention heidelberg school]. 
%In particular the dedicated and hands-on \cern HEP schools and the [school of data analysis and ML]
%will complement the more traditional university-based lectures of other academic nodes, while
%\dq 
%%, \technopolis 
%and the partner companies will provide training which will greatly increase the employability of
%the ESRs. %Summer schools, academic lecture series as well as the full spectrum
%%of training courses offered by the \acronym participants will be open to all ESRs.



%%explain the compatibility and coherence between the tasks attributed to each
%beneficiary/partner organisation in the action, including in light of their experience;


%
%The research synergies and complementarities
%between the \acronym participants have been described in detail in Sec.~\ref{sec:synergy}.
%
%
%The non-academic beneficiaries and partners have been chosen for their active involvement in research
%and development related to optimising decisions with a commercial purpose, 
%leading to a common interest in RTA shared with the \acronym participants, 
%as detailed in Sec.~\ref{sec:EXCELLENCE}. 
%
%%Of particular importance are the training courses and secondments opportunities
%%offered by our industrial partners, which will give first-hand insight
%%and hands-on experience in the commercial world.
%
%TO BE FILLED AFTER TRAINING PART IS DONE
%

\subsubsection{Commitment of beneficiaries and partner organisations to the program}
%(for partner organisations, please see also sections 5 and 7).

While the majority of the \acronym academic participants are already successfully collaborating in research projects within the LHC experiments, this is the first network that including all major LHC experiment that will consistently advance RTA through Machine Learning and hybrid architectures for modern particle physics experiments and industry alike. 
In-person meetings for the members of the \acronym network have been taking place since May 2017\footnote{\href{https://indico.lucas.lu.se/event/656/}{SMARTHEP kickoff meeting}, supported by the Grace och Philip Sandbloms Fund}. A number of subsequent meetings have taken place, including the RAPID2018 workshop (Sec.~\ref{sec:synergy}), and culminated in a three-day workshop dedicated to the preparation and collaborative editing of this ITN \footnote{\href{https://indico.cern.ch/event/777802/}{SMARTHEP writing meeting at CERN}, supported by Lund Research Services}. SMARTHEP members have also secured funding for a 12 day RTA workshop at the \href{https://www.universite-paris-saclay.fr/en/institut-pascal}{Institut Pascal} (Sec.~\ref{sec:synergy}).
It is clear from the enthusiasm and participation of all members, as well as from the support offered to this network by \lundentity, that all \acronym participants are fully committed to this project.

High quality education of students in large-scale international projects is already an essential part of the mission of the universities and research labs in \acronym. 
\acronym will greatly strengthen and extend their capacities to carry out these activities, by dedicating a significant fraction of research and supervision time to \acronym activities (see node descriptions in part B2).
All industrial nodes are committed to \acronym, to benefit from the experience in RTA of LHC experts, to foster connections to the academic world and to expand their activities. 
The industry parties have a strong motivation to participate in cutting edge developments and the subsequent transfer of the technologies to new commercial applications.
\acronym partners will not only contribute training to the network, but will also benefit in their own research and commercial objectives through the secondment projects of the ESRs. 

The relations between the nodes, established starting from the preparation of the project will last even beyond \acronym: 
to organize post-graduate recruitment events for companies in the network that specifically target \acronym ESRs after the completion of their PhD and to continue the series of training events and offer them to the European graduate student community.  
%\acronym participants therefore have already established long-lasting
%and fruitful professional relations, marked by mutual respect and trust.
%In short, there is no doubt that all participants in \acronym
%will show the highest level of commitment to the research program. 


%i) Funding of non-associated third countries (if applicable): Only entities from EU Member States, from Horizon 2020 Associated Countries or from countries listed in General Annex A to the Work program are automatically eligible for EU funding. If one or more of the beneficiaries requesting EU funding is based in a country that is not automatically eligible for such funding, the application shall explain in terms of the objectives of the action why such funding would be essential. Only in exceptional cases will these organisations receive EU funding.16 The same applies for international organisations other than IEIO.

%ii) Partner organisations: The role of partner organisations and their active contribution to the research and training activities should be described. A letter of commitment shall also be provided in section 7 (included within the PDF file, but outside the page limit).
