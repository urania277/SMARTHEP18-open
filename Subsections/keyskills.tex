%In this section, please explain the impact of the research and training on the fellows' careers. 

ESRs will gain a skillset that spans from research to industry,
in an European context, as the proposed network builds on the foundation of
\textbf{interdisciplinary, public-private} exchanges
and \textbf{international scientific collaborations}
To describe the impact of the research and training on the ESR careers, we follow the template of the European Doctoral Training Principles\footnote{ERA Steering Group Human Resources and Mobility, \href{https://euraxess.ec.europa.eu/belgium/jobs-funding/doctoral-training-principles}{European Doctoral Training Principles}, 2011}. 

\noindent\color{blue}{Research excellence, attractive institutional environment. }\color{black}
The research and peer-reviewed publications on topics that are crucial
for the advancement of HEP, stemming from the RTA centered research program of \acronym,
as well as a \phd degree from renowned European universities, 
will provide the ESRs with the best possible ground towards a career in academia. 
All institutes in \acronym have long
experience in supervising and bringing students to graduation of master and \phd degrees, 
and employ world-renowned scientists that are leaders in RTA and 
can serve as supervisors as well as role models. 

\noindent\color{blue}{Interdisciplinary Research Options, Exposure to Industry and other relevant employment sectors. }\color{black}
Training in large-scale Data analysis from HEP experiments, especially those trained in ML and data science, already provides a good ground for employability\footnote{
\href{https://www.wired.com/2017/01/move-coders-physicists-will-soon-rule-silicon-valley/}{Wired Jan 2017: Move Over, Coders? Physicists Will Soon Rule Silicon Valley}}.
Bridging the gap between academia and industry therefore remains essential for most 
physics graduates\footnote{P. Heron, Preparing physics students for
21st-century careers, Physics Today 70, 11, 38 (2017);}. 
ESRs will not only be exposed to their
own topic of research at their local institute, but they will 
also be seconded at other institutes and industrial partners with practical
applications that are sufficiently diverse but still match
their research topic, leading to a consistent thesis project as well
as a broader expertise for the ESRs at the end of their studies.

%That is why the doctoral programs proposed at \acronym institutes
%will be complemented by the training 
%offered by the Network activities described in Sec.~\ref{sec:training}. 

\noindent\color{blue}{International networking. }\color{black}
The complexity and cost of HEP research has pushed scientists to form large, 
international collaborations. Scientists have an ideal ground to do so at CERN: 
the idea of trans-national scientific collaboration promoting a unified Europe
is at its heart.
All \acronym ESRs research focused on developing RTA within the 
CERN experiments, and CERN will host PhD students and secondments. 
By virtue of this, the ESRs will be immersed in an
international environment that trains them to work in teams and present their work regularly.
It also encourages the ESRs to consider Europe as a single, open labour market,
so that they can either continue their research quest towards a deeper understanding or nature
or convert their ideas into products and services for a thriving knowledge-based European
economy and society. 

\noindent\color{blue}{Transferable skills training. }\color{black}
The academic and industry training foreseen for the 
ESRs in one of the most data intense fields
prepares them for careers in academia, industry, or both, 
following the example of their supervisors. The training program of \acronym
is crucial for this purpose, and includes a significant portion of transferrable skills training
described in Sec.~\ref{sec:training}.

\noindent\color{blue}{Quality Assurance. }\color{black}
Both ESRs, and the \acronym research program, will be continuously reviewed 
through feedback of their local supervisors and peers. Further internal quality assurance is
provided by the other members of the network, before coming to the peer-review
and selection of papers and conference talks, as well as the exploitation of the research
of products developed together with the industrial partners. 
This combination will allow \acronym ESRs to complete an excellent PhD degree 
that will be recognized by future employers. The secondments to the industrial partners will
allow the ERSs to gain valuable, documented work experience that will boost
their employability and recruitment possibilities, 
both within and outside the the companies of the network
and prepare them for joining the European work force towards successful careers. 

%Thanks to the balanced composition of the \acronym network, the
%trained ESRs can expect to be employable in research, development,
%consultancy and management in sectors as diverse as information
%technology, medical engineering, the Internet Of Things or traffic control.
%
%%CERN Courier Viewpoint: Birth of the high-energy network~\url{http://cerncourier.com/cws/article/cern/68436}}. 
%

%The ESRs will be exposed to the whole range of 
%RTA research topics and applications throughout the network,
%thanks to the participation in \acronym events and to the close communication foreseen
%between all PIs and students. 

