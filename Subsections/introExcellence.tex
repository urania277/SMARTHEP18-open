%context
\vspace{-2mm}
Due to the ever increasing availability of vast amounts 
of information in research and industry, 
new \textbf{fast and efficient automated decision making techniques} are needed. 
\textit{Fast} means that the decision 
needs to be taken in real-time, on timescales that allow changing subsequent actions. 
\textit{Efficient} means that all information available for decision-making is
retrieved and processed in the best possible way. 
\acronym ({\color{blue}\textbf{S}}ynergies between {\color{blue}\textbf{M}}ultivariate {\color{blue}\textbf{A}}nalysis, {\color{blue}\textbf{R}}eal {\color{blue}\textbf{T}}ime analysis and {\color{blue}\textbf{H}}ybrid architectures for {\color{blue}\textbf{E}}vent {\color{blue}\textbf{P}}rocessing) 
is an interdisciplinary ETN aiming to train
Early Stage Researchers (ESRs) to tackle these challenges using novel
software and hardware tools. 
\vskip2pt
%enrolling them
%in interdisciplinary and intersector doctoral programs. 

%Blurb from doc NCP
% ?The overarching objective of this ITN is to provide high-level training in X to a new generation of high achieving early stage researchers to provide them with the transferable skills necessary for thriving careers in a burgeoning area that underpins innovative technological development across a range of diverse disciplines. This goal will be achieved by a unique combination of ?hands-on? research training, non-academic placements and courses and workshops on scientific and complementary so-called ?soft? skills facilitated by the academic-non-academic composition of the consortium?
 
%problem 
Traditional decision-making processes require that data is collected first, 
then analyzed. This model is not sustainable when the decision needs to be made faster than traditional data analysis allows, or when the quantity of data is too large to be recorded for subsequent analysis. 
%cost, examples
Examples abound, in research environments in High Energy Physics (HEP) and industry. 
The Large Hadron Collider (LHC) at CERN collides protons up to 30 million times per second: 
the majority of these events need to be rejected to comply
with data processing and storage constraints. 
The advent of mobile apps for e.g. traffic prediction and fleet control, as well as
Internet-of-Things (IoT) sensors\footnote{G. Santucci, \href{http://cordis.europa.eu/fp7/ict/enet/documents/publications/iot-between-the-internet-revolution.pdf}{The Internet of Things: Between the Revolution of the Internet and the Metamorphosis of Objects}, FP7 document},
mean that the amount of information 
largely grows while the resources and time to take a decision using 
that information do not scale accordingly, so industry faces a similar challenge as HEP
to make the most of the available data. 
\vskip2pt
%%solution
With \acronym, we intend to bring forward a paradigm shift in the way these
decisions are made. With \textbf{real-time analysis (RTA)}, where the data is analyzed 
over the timescale of microseconds to seconds, 
we move from an asynchronous
data collection and analysis to a nearly-simultaneous data collection and analysis
perspective. An example of a real-life, real-time decision can be found in
traffic predictions: the information on the position of the vehicle and the planned route are 
transmitted to a central analysis system, the traffic conditions around it are analyzed and a forecast is made. 
All of this needs to be done in the course of less than a second so that the driver can change their route if needed. 
Real-time analyses are made possible by machine learning (ML) and multivariate techniques as well as artificial intelligence (AI)
to speed up the reconstruction and the analysis of the data, and by 
novel hybrid software/hardware solutions,
where data are collected and reconstructed on the same computing platform. 
\vskip2pt
\acronym pools the expertise of the leaders of the RTA programs in 
all four major HEP experiments at the LHC. 
Their knowledge and experience is complemented by a set of industrial partners that apply RTA
to automotive traffic, and to applications optimizing IoT sensors for industrial processes, 
medical applications, computing and instrumentation. 
The \acronym ESRs will be \textbf{trained} by a
consortium formed by academic and industrial
members on scientific, technological, and entrepreneurship
aspects of both HEP and the interdisciplinary field of Data science (DS), 
i.e. the analysis of massive amounts of complex data to extract
useful information\footnote{C. Hayashi\href{https://link.springer.com/chapter/10.1007/978-4-431-65950-1_3}{What is Data Science ? Fundamental Concepts and a Heuristic Example}, Springer}. 
\acronym will implement a training program for
the ESRs that not only will allow them to achieve and be recognized by
high academic degrees but will also prepare them as professionals
in the fast-evolving areas of DS and RTA.
\acronym will target concrete
commercial deliverables such as 
%KKT - ITA
mobile apps using ML for efficient vehicle fleet control, 
%Ximantis - SWE
optimized algorithms for traffic predictions,
%CATHi - DE
and a variety of software for sensors that will improve medical simulation,
%WildTree - SWI
optimize computing resources,
%%Lightbox - I
monitor industrial production
%HeidelbergInstruments - DE
and assemble particles detectors. 
It will benefit the HEP community by
providing cutting edge computing technologies and algorithms
necessary for the detection and measurement of fundamental particles,
and for the discovery of new phenomena that explain e.g. 
the mystery of the missing matter in the universe. 
Researchers and entrepreneurs members of \acronym 
will also work together beyond this ETN proposal. 
They are committed to follow the future careers of the 
scientists they train, and seek further collaborations
%that develop more efficient ways to take decisions on short timescales
for the benefit of research, industry and society. 
