%%\begin{wraptable}{r}{0.8\textwidth}
%%\begin{table}[!h]%{0.8\textwidth}
%%	\caption{
%%\label{tab:Events}}
%	\begin{center}\scriptsize
%			%\resizebox {\textwidth }{!}{%
%			\begin{tabular}{@{}lp{56mm}p{7mm}p{40mm}p{20mm}@{}}
%				\toprule
%				\multicolumn{2}{p{4cm}}{\pbox{8cm}{Training Events \& Conferences}} &
%				\pbox{8cm}{Credits} &%http://ec.europa.eu/education/ects/users-guide/docs/ects-users-guide_en.pdf 
%				\pbox{8cm}{Lead Institution} & 
%				\pbox{8cm}{Action Month}
%				\tabularnewline 
%				\toprule
%				1. & Kick-off meeting & - & \lundentity & 2  \tabularnewline\midrule
%				2. & Yearly meeting & 2 & \lundentity & 10,22,34  \tabularnewline\midrule
%				3. & Introductory school & 3 & \nikhefentity & 10  \tabularnewline\midrule %when all recruitment is complete
%				4. & Physics and machine learning school & 3 & \unigeentity & 18  \tabularnewline\midrule
%				5. & Basic FPGA course, FPGA boot-camp & 1.5 & \cernentity, \ohioentity, \pisaentity & 27, 28 \tabularnewline
%				6. & CPU and hybrid architectures school & 1.5 & \santiagoentity  & 29 \tabularnewline \midrule
%				7. & Intermediate conference & - & \dortmundentity  & 30 \tabularnewline \midrule
%				8. & Industry, career and transferrable skills school & 1.5 & \heidelbergentity  & 36 \tabularnewline \midrule
%				9. & Final conference & 2 & \cnrsentity & 42  \tabularnewline \midrule
%				10. & Closing meeting & - & \lundentity & 48  \tabularnewline\midrule
%				\bottomrule
%			\end{tabular}
%		%}%
%	\end{center}
%%	\vspace{-5mm}
%%\end{wraptable}
%%\end{table}


%\begin{wraptable}{l}{0.3\textwidth}
%\vspace{-1mm}
%%\vskip-10pt
%	\caption{Recruitment deliverables per beneficiary\label{tab:recruitmentDeliverables}}
%	\begin{center}\scriptsize
%			\resizebox {0.3\textwidth }{!}{%
%			\begin{tabular}{@{}lp{10mm}p{10mm}p{10mm}p{10mm}@{}}
%\toprule
%ESR &
%\pbox{8cm}{\Tstrut Recruiting \\ Participant\Bstrut} &
%\pbox{8cm}{\Tstrut Planned \\ Start\Bstrut} & %month 0-45
%\pbox{8cm}{\Tstrut Duration \Bstrut}%duration 3-36
%\tabularnewline 
%\toprule
%\tabularnewline 
%\toprule
%1 & \helsinkientity  & 8 & 36\tabularnewline\midrule
%2 & \helsinkientity  & 8 & 36\tabularnewline\midrule
%3 & \unigeentity  & 8 & 36\tabularnewline\midrule
%4 & \cernentity  & 8 & 36\tabularnewline\midrule
%5 & \cernentity  & 8 & 36\tabularnewline\midrule
%6 & \dortmundentity  & 8 & 36\tabularnewline\midrule
%7 & \dortmundentity  & 8 & 36\tabularnewline\midrule
%8 & \cnrsentity  & 8 & 36\tabularnewline\midrule
%9 &  \dqentity  & 8 & 36\tabularnewline\midrule
%10 & \sorbonneentity  & 8 & 36\tabularnewline\midrule
%11 & \nikhefentity  & 8 & 36\tabularnewline\midrule
%12 & \nikhefentity & 8 & 36\tabularnewline\midrule
%13 & \ibmentity  & 8 & 36\tabularnewline\midrule
%14 & \lundentity  & 8 & 36\tabularnewline\midrule
%15 & \heidelbergentity & 8 & 36\tabularnewline\midrule
%Total & & & 540 \tabularnewline
%\bottomrule
%\end{tabular}
%}%
%\end{center}
%\vspace{-6mm}
%%\vskip-30pt
%\end{wraptable}


%Training is at the heart of all activities within \acronym,
%with a dedicated Work Package, WP2, with Anna Sfyrla from \unigeentity as responsible. 
%In this section, we list the recruitment deliverables for the 15 ESRs, 
%describe the Personal Career Development Plan (PCDP),
%then describe the complementary network-wide training and local training, 
%and finally outline the \acronym credit system designed for this ETN. 

%1.2.a
%\begin{table}[t]
%\caption{\label{tab:DelivPart}}

%\textbf{Table 1.2a: Recruitment deliverables per beneficiary}
%\begin{center}\footnotesize
%\resizebox {\textwidth }{!}{%
%\begin{tabular}{@{}p{5mm}p{105mm}p{20mm}p{15mm}p{15mm}@{}}
%\toprule
%ESR &
%Topic &
%\pbox{8cm}{\Tstrut Recruiting \\ Participant\Bstrut} &
%\pbox{8cm}{\Tstrut Planned \\ Start\Bstrut} & %month 0-45
%\pbox{8cm}{\Tstrut Duration \Bstrut}%duration 3-36
%\tabularnewline 
%\toprule
%\tabularnewline 
%\toprule
%1 & Discovery of new physics with jets in CMS with RTA & \helsinkientity  & 8 & 36\tabularnewline\midrule
%2 & Use of machine learning and RTA for discovering new physics and measuring the SM in CMS & \helsinkientity  & 8 & 36\tabularnewline\midrule
%3 & Machine learning for pattern recognition in searches for exotic physics at the ATLAS experiment & \unigeentity  & 8 & 36\tabularnewline\midrule
%4 & Efficient RTA in ATLAS using multi-threaded processing & \cernentity  & 8 & 36\tabularnewline\midrule
%5 & Search for LFV in tau, strange and charmed mesons decays using RTA in LHCb & \cernentity  & 8 & 36\tabularnewline\midrule
%6 & Real-time implementation of multivariate analysis for LFV in unflavoured meson decays in LHCb & \dortmundentity  & 8 & 36\tabularnewline\midrule
%7 & Event Triggering in LHCb & \dortmundentity  & 8 & 36\tabularnewline\midrule
%8 & Real-time particle trajectory reconstruction for online event selection and analysis in ATLAS & \cnrsentity  & 8 & 36\tabularnewline\midrule
%9 & Real-time analysis and machine learning in industry and new physics searches with LHCb data & \dqentity  & 8 & 36\tabularnewline\midrule
%10 & Enabling RTA on heterogeneous computing architectures & \cnrsentity  & 8 & 36\tabularnewline\midrule
%11 & Smart optimization of resources for efficient trigger and analysis and use for ATLAS measurements of LFV & \nikhefentity  & 8 & 36\tabularnewline\midrule
%12 & Smart optimization of resources for efficient trigger and analysis and use for LHCb measurements of LFV & \nikhefentity & 8 & 36\tabularnewline\midrule
%13 & Anomaly detection in industry and ATLAS trigger for new physics searches & \ibmentity  & 8 & 36\tabularnewline\midrule
%14 & Real-time calibrations and analysis of the ALICE Time Projection Chamber & \lundentity  & 8 & 36\tabularnewline\midrule
%15 & Real-time noise reduction in searches for new physics phenomena beyond the Standard Model in ATLAS & \heidelbergentity & 8 & 36\tabularnewline\midrule
%Total & & & 540 \tabularnewline
%\bottomrule
%\end{tabular}
%}%
%\end{center}
%%\end{table}



%\FloatBarrier
%\begin{table}[!htb]
%\centering
\begin{center}
\scriptsize
%\resizebox {\textwidth }{!}{%
%\begin{tabular}{@{}p{5mm}p{40mm}p{25mm}p{22mm}p{22mm}p{12mm}p{12mm}p{12mm}}

			\begin{tabular}{@{}|c|p{56mm}|p{7mm}|p{40mm}|p{10mm}@{}}
				\toprule
				\multicolumn{2}{|p{4cm}|}{\pbox{8cm}{Training Events \& Conferences}} & 
				\pbox{8cm}{Credits} &%http://ec.europa.eu/education/ects/users-guide/docs/ects-users-guide_en.pdf 
				\pbox{8cm}{Lead Institution} & 
				\pbox{8cm}{Action Month} &
%				\pbox{8cm}{New network event} &		
				\tabularnewline 
				\hline
				\hline
				%\toprule
				
				%%%
				
%				\cellcolor{red!70!black}1. & Kick-off meeting & - & \lundentity & 2 & - \tabularnewline\hline
%				
%				\multicolumn{6}{p{\textwidth}}{
%The \textbf kick-off meeting  will be dedicated to organizing the project management, signing the consortium agreement, and monitoring the recruitment of the ESRs.
%			    } \tabularnewline \hline %\midrule
			    
			    %%%
			    
%				\cellcolor{red!70!black}2. & Yearly meeting & 2 & \lundentity & 10,22,34 & - \tabularnewline \hline
%				
%			 	\multicolumn{6}{p{\textwidth}}{
%				
%\acronym will hold \textbf{yearly in-person network-internal conferences}, with a duration of 4 or 5 days. 
%On the first two days, the ESRs will gain experience in presenting their research by giving presentations to the other \acronym members and participating in a poster session.
%Days 3--5 will be dedicated to lectures on research, technical and transferable skills, as well as communication and dissemination activities tailored to local circumstances (see Sec.~\ref{sec:CommPub}). 
%All yearly meetings will include a dedicated half-day of lectures on gender, diversity and inclusion and sustainable research, to complement local training on transferrable skills. 
%\acronym members believe these are necessary soft skills to work in a positive environment. 
%Within the Yearly Meetings, there will be time devoted to management activities such as the Executive Board, ESR Board and Supervisory Board meetings (see Sec.~\ref{sub:networkOrganization}).
%%Example topics that have been agreed upon are "Optimizing workspaces for productivity" (\dqentity); 
%%"Writing software in collaborative environments" (\wildtreeentity), "Gender and inclusion" (\lundentity),
%%"Innovation and entrepreneurship, including IPR" (\lundentity, in a one-day workshop similar to the successful
%%\href{http://indico.hep.lu.se/conferenceDisplay.py?confId=697}{Innovation and Entrepreneurship for PhD Students} event).
%			    } \tabularnewline \hline %\midrule
%			    
%			    %%%
%			    
%				\cellcolor{lime} 3. & \textbf{Introductory school} & 3 & \nikhefentity & 10 & Yes \tabularnewline\hline%\midrule %when all recruitment is complete
%				
%			    \multicolumn{6}{p{\textwidth}}{
%			    
%A first introduction to both HEP and RTA will be provided, by experts from both HEP and industry. 
%This school will be an opportunity for all ESR members to get to know each other, as they are all recruited together. 
%They will also receive basic training in gender issues and research integrity. 
%This school will be hosted shortly after the ESRs have been recruited. 
%%We will also invite visiting scientists outside the network, and names will be agreed during the organization of each of the schools.   
%
%			    } \tabularnewline \hline %\midrule
%			    
%			    %%%
%			    
%			    \cellcolor{orange} 4. & \textbf{Physics and machine learning school} & 3 & \unigeentity & N & Yes\tabularnewline\hline
%				
%				\multicolumn{6}{p{\textwidth}}{
%
%This school will provide the ESRs with introductory courses on the physics topics tackled in the network, a few months after the ESRs have started their projects.  
%It will also provide a first introduction to how to design a physics analysis, as well as to machine learning concepts and their connections to HEP and industrial applications. 
%
%			    } \tabularnewline \hline %\midrule
%
%			    %%%
%
%				\cellcolor{orange} 4. & Machine Learning for HEP (MLHEP) school & 3 & \unigeentity & M & Yes \tabularnewline\hline
%
%			    \multicolumn{6}{p{\textwidth}}{
%			    
%Describe MLHEP. 
%
%			    } \tabularnewline \hline %\midrule
%			    			    
%			    %%%
%			    
%			    \cellcolor{green} 5. & \textbf{Non-academic skills workshop} (joint with INSIGHTS ITN, TBC) & 1.5 & \lundentity  & N & Yes \tabularnewline \hline
%				
%				\multicolumn{6}{p{\textwidth}}{		
%							
%Describe transferrable skills school. 
%
%			    } \tabularnewline \hline %\midrule	
%			    
%			    %%%
%			    
%				\cellcolor{yellow} 6. & Basic FPGA course, FPGA boot-camp & 1.5 & \cernentity, \ohioentity, \pisaentity & 27, 28 & Yes \tabularnewline\hline
%				
%				\multicolumn{6}{p{\textwidth}}{				
%				
%This school will include lectures on technologies and architecture and hands-on exercises on triggering applications. 
%This school will consist of introductory courses given at \cernentity, and a follow-up bootcamp in Pisa where practical problems are solved for ESRs specializing in this topic. 
%Lectures will be given from researchers in \pisaentity and \ohioentity, both leading institutes in research and development on hardware track triggers.
%				
%			    } \tabularnewline \hline %\midrule	
%			    
%			    %%%%
%			    			
%				\cellcolor{yellow} 7. & GPU and hybrid architectures school & 1.5 & \santiagoentity  & 29 & Yes \tabularnewline \hline
%				
%				\multicolumn{6}{p{\textwidth}}{				
%				
%In this school hosted by \santiagoentity, the ESRs will learn how to compare architectures and practical solutions (e.g. GPU programming). 
%Lectures will be given by \sorbonneentity
%The ESRs will learn not only about what is available on the market and what is being planned, but also ways to best evaluate the chosen hardware solution for the software they are developing.
%
%				} \tabularnewline \hline %\midrule				
%				%7. & Intermediate conference & - & \dortmundentity  & 30 \tabularnewline \midrule
%				
%				%%%%
%				
%				\cellcolor{green} 8. & \textbf{Industry and career development school} & 1.5 & \dortmundentity  & 36 & Yes \tabularnewline \hline
%				
%				\multicolumn{6}{p{\textwidth}}{		
%							
%This school will include lectures and workshops in collaboration with industry.
%This school is dedicated to an in depth study of strategies for intellectual property rights (IPR), commercializing research output, presenting research results to policy-makers, and knowledge transfer.
%One day of this school will be dedicated to group work on case studies prepared by the industrial beneficiaries and partners. 
%This school will include experiences and Q\&A sessions with the CEOs and founders of the companies within \acronym. 
%This school will be organized at \dortmundentity, in collaboration with \ximantisentity. 
%				
%%An example of the lectures:
%%\begin{itemize}
%%\item {Description of the company;}
%%\item {Experience from transitioning from physics to industry;}
%%\item {The daily job within the industry (e.g. of a quant in finance or of a quant in a Big Data company) and why physicists are particularly good at that when it comes to real-life (or real-economy) applications.}
%%\item {An example of the life-cycle of a basic commercial strategy: from the initial idea to real-time deployment}
%%\item {Group work: analysis of a case study of big data applications to real economy}
%%\end{itemize}
%				} \tabularnewline \hline %\midrule				
%				
%				%%%%
%				
%				\cellcolor{cyan} 9. & \textbf{Final conference} & 2 & \cnrsentity & 42 & Yes \tabularnewline \hline
%				
%				\multicolumn{6}{p{\textwidth}}{					
%				
%				We will hold a five-day \textbf{public conference}
%				%, as detailed in Table~\ref{tab:Events}. These conferences 
%				which will showcase the work of the network to the wider scientific community. 
%				As opposed to the yearly meetings, the conference will not feature management meetings,
%				and will be dedicated to presentations on \acronym research
%				and invited topical presentations on state-of-the art developments within the HEP and DS fields. 
%				Each of the conferences will dedicate between
%				half and one day to dissemination activities, both by \acronym members and by invited external experts. 
%				The conference will have two additional days reserved for ESRs' presentations and 
%				public lectures showcasing the work done during their PhD program with \acronym. ESRs will also
%				take active part in the organization of this conference, to add to their transferrable skillset. 
%				} \tabularnewline \hline %\midrule
%						
%				\cellcolor{red!70!black} 10. & Closing meeting & - & \lundentity & 48  \tabularnewline\midrule
%				
%				\multicolumn{6}{p{\textwidth}}{					
%				
%				In the closing meeting, the network will take stock of the experience of the ETN and plan the next steps. The PIs will give 
%				public lectures. The closing meeting is beyond the doctoral period for most of the ESRs,
%				but they will be invited to participate as network alumni, also to give precious feedback on the ETN experience. 
%
%				} \tabularnewline \hline %\midrule

				%\bottomrule
			\end{tabular}

%}%end of resizebox
\end{center}
%\vspace{-5mm}
%\end{table}
%\FloatBarrier
%\vspace{-5mm}

