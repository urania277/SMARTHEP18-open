% HEADER PLEASE READ!!!!!!!!
% HEADER PLEASE READ!!!!!!!!
% HEADER PLEASE READ!!!!!!!!
% HEADER PLEASE READ!!!!!!!!
% HEADER PLEASE READ!!!!!!!!
% ALL NODES TO FILL IN THE FIRST TABLE IN THIS TEX FILE DEFINING THE SUPERVISION
% HEADER PLEASE READ!!!!!!!!
% HEADER PLEASE READ!!!!!!!!
% HEADER PLEASE READ!!!!!!!!
% HEADER PLEASE READ!!!!!!!!
% HEADER PLEASE READ!!!!!!!!

%CD: Table will be added? Bullet points seem more efficient. 

%The following section of the European Charter for Researchers refers specifically to supervision:
%
%Employers and/or funders should ensure that a person is clearly identified to
%whom Early-Stage Researchers can refer for the performance of their
%professional duties, and should inform the researchers accordingly.
%Such arrangements should clearly define that the proposed supervisors are
%sufficiently expert in supervising research, have the time, knowledge,
%experience, expertise and commitment to be able to offer the research trainee
%appropriate support and provide for the necessary progress and review
%procedures, as well as the necessary feedback mechanisms.

%The network structure including the strong DS nodes at CNRS/LRI as
%well as the forefront of HEP research institutes (CERN, Lund, RHUL, 
%NIKHEF, Dortmund, CNRS/LAL) ensures that \acronym will be a truly 
%strategic partnership of experts in DS and HEP. 

%This partnership will be
%further strengthened by the strength of the DS experience at
%the chosen partners \nyu and \yandex. 
%The physics students will all be seconded by data
%scientists, allowing them a more stringent theoretical training on
%methods like deep learning. On the other hand, all DS 
%students will be seconded by particle physicists giving them access to
%the the data of the LHC and to actual problems physicists are working
%to solve with DS methods.


%Required sub-headings:
\subsubsection{Qualifications and supervision experience of supervisors}
\label{subsub:qual_supervisors}

%To avoid duplication, the role and profile of the supervisors should only be listed
%in the "Capacity of the Participating Organisations" tables (see section 5 below)

The ESRs at \acronym\ will be supervised by the main experts in in RTA, machine learning and hybrid architectures from all LHC collaborations, with an excellent track record demonstrated by the European and national grants received by the PIs (5 ERC H2020 grants and numerous national grants). 

Many of the supervisors are young yet successful scientists with long- or medium-term academic positions, who dedicate a large fraction of their time to research and direct supervision of students. 
This guarantees that, as specified in the European Charter for Researchers, they will have the time, knowledge, experience and commitment to offer support and feedback so to guarantee the supervision of excellent research. 
To complement the supervision, ESRs supervised by young professors also have a senior professor/researcher from their local node included in their SC.
The addition of non-permanent researchers (postdocs) with long-term contracts to the ESR SC guarantees that the ESRs receive day-to-day hands-on local or remote supervision, and allows \acronym postdocs to gain supervision experience necessary towards their future career to positions in academia and industry. 

All members of the ESR SC in Table 1.2 have proven experience in supervising PhD and Master's theses, as shown by the numbers of supervised early stage researchers from Master's level onwards. 
Most of the researchers in the network, including the postdocs, have also held positions of responsibility within their large international collaborations. 
Within these roles, they have trained numerous students who performed research tasks under their supervision, and they have managed large groups of researches to successfully operate complex detector systems. 
Supervisors in industrial nodes have supervision expertise from introducing trainees to the work in their company, and are therefore well placed to deliver innovative cross-disciplinary training within \acronym. 

The supervisors chosen for each ESR for both PhD-awarding entity and secondments are those with the most relevant experience and expertise to enhance both the research and training potential of the ESR project. 
While we leave further details on the qualification of the supervisors and member of consortium to the node description, the main expertise and qualifications of the academic and industrial supervisors needed for the success of the ESR projects are descried below, according to research topic. 

Most supervisors in \acronym are \textbf{experts in HEP and industrial real-time data analysis}. 
Gligorov (\parisUentity), Raven (\nikhefentity), Strom (\oregonentity) and Sfyrla (\unigeentity) have held positions of responsibility in Trigger and Data Acquisition  in LHCb and ATLAS respectively. 
Pierini (\cernentity),  Voutilainen (\helsinkientity), Gligorov, Albrecht (\dortmundentity), Doglioni (\lundentity), Boveia (\ohioentity), Starovoitov and Dunford in (\heidelbergentity) and Schramm (\unigeentity) are pioneers of the first fully RTA-based searches and measurements in their experiments, leading groups of PhD students to peer-reviewed publication on high impact journals. Gligorov, Albrecht and Doglioni hold ERC grants on this kind of analyses and application to LFV/LFU in LHCb and dark matter in ATLAS, respectively. 
Further trigger expertise is brought to the consortium by Petersen (\cern) and Igonkina (\nikhefentity), who have successfully designed dedicated trigger selections for Supersymmetry and LFV in ATLAS together with their students. 
RTA in industry is a specialty of all industrial partners, from industrial digital services (Catastini and Borri, \lightboxentity), transport optimization (Dungs and Brambach, \pointeightentity), fleet safety (Sambo and Taccari, \fleetmaticsentity) and fraud detection (Juille and Shaw, \ibmentity). 

A strong expertise in \textbf{software and computing tools} is crucial for the success of the research projects in this ITN. This is provided by Raven, one of the main authors of the LHCb software over the last decade, Matev (\cernentity) who is a lead author of the LHCb trigger software, and Smirnova (\lundentity), who is one of the architects of the Worldwide LHC Computing Grid. From the industrial side, having \ibmentity, one of the largest producers of computing hardware and software in the world, is an unique asset for \acronym with the participation of experienced members such as De Sainte Marie.  
Christiansen (\lundentity), Shahoyan (\cernentity) complement the software competences with detector hardware expertise, as leading developers of the ALICE Time Projection Chamber.  

\textbf{Machine learning techniques} and their software implementations are an emerging research tool that all ESRs will be exposed to. Here \acronym ESRs can rely on the expertise of Louppe (\liegesentity), a ML expert with many ongoing collaborations in particle physics; Pierini, holder of an ERC Consolidator Grant for ML at the LHC; Ustyuzhanin from the Yandex School of Computing and \cernentity, and for specific ML computing vision expertise within the CVLab founded by Di Stefano (\uniboentity).  Sopasakis (\ximantis), Sambo and Taccari (\fleetmaticsentity) use advanced ML techniques for transport and fleet control on a day to day basis.  

Expertise in \textbf{hybrid computing architectures} is provided by Lacassagne (\sorbonneentity), leader of the Hardware and Software for Embedded Systems team at LIP6. 
The know-how of Annovi and Roda from \pisaentity, Boveia of \ohioentity, and Crescioli of \cnrsentity in FPGA design and commissioning from their experience on the ATLAS FTK, provide training and research using one of the first hybrid supercomputers in HEP as testing ground. 
GPU expertise is provided by Santos of \santiagoentity, also holding an ERC StG on applications of GPUs for HEP. 

The thesis topics of all ESRs will involve a component of \textbf{data analysis in high energy physics}. 
Those working on Dark Sectors and physics beyond the SM will benefit from the knowledge of the LHC Dark Matter Working Group organizers, 
Boveia and Doglioni, as well as from the expertise of in supersymmetric dark matter and new physics searches of Petersen, Sfyrla, 
Voutilainen, Dunford and . 
The network expertise in LFV/LFU is given by Albrecht, Gligorov, Raven, Igonkina and . 
Malaescu (\cnrsentity) is currently the ATLAS Standard Model convener, while Christiansen holds several national grants for heavy ion measurements in ALICE. 

For \textbf{career mentoring purposes}, Dungs and Catastini transitioned to industry after a successful research experience in HEP, Salti (\uniboentity) spent a period in industry before returning to academia, and Sopasakis is the CEO of a start-up alongside his assistant professorship at \lundentity, so they are all extremely well placed to supervise and mentor students with intersector projects and career perspectives. 

The ESRs also will receive \textbf{further academic mentoring} at their nodes and during the yearly meetings from the Internal Advisory Board, a \textbf{group of senior physicists} that includes Dr. Monica Pepe-Altarelli (also head of \cern node), Vice-President Elected at Large of the Executive Council of the International Union of Pure and Applied Science  (\href{http://iupap.org/}{IUPAP}), Prof. Torsten Akesson from \lundentity, a member of the Nobel Committee and ex-CERN Council president, Prof. Paula Eerola from \helsinkientity, currently Vice Rector of the University of Helsinki, Prof. Stephanie Hansmann- Menzemer who is a deputy managing director at the \heidelbergentity, Prof. Bernard Spaan who has been Dean of the Physics Faculty of \dortmund and Prof. Giuseppe Iacobucci, the Director of the DPNC at \unigeentity. 
Other \textbf{permanent and senior post-doctoral researchers} in the various nodes (see part B2) will also be present for additional \textbf{day-to-day supervision} of the ESRs. 


%The expertise and the most relevant qualifications of the supervisors are in the table, with full 
%details in Sec.~\ref{sec:capacities}. 

% and will be the main
%mentor for the ESR on career and intellectual development. The main supervisor will also be responsible
%for helping the ESR compile their PCDP, discussed in the previous section.




\vskip-10pt

\subsubsection{Quality of the joint supervision arrangements}
\label{sec:jointsuperqual}

%Make sure to signal that ESR will be a vehicle to get supervisors to talk to each other regularly. 

%To do if there is more room: 
%- explain better why a given supervisor is the best co-supervisor for a given project. 

Both academic and industrial partners play a key role in supervision of \acronym, with the secondment responsibles acting as secondary supervisors. 
In the cases where the ESR is not seconded/hosted by an industry node, the ESR will have a supervisor from an industrial partner for mentoring and career advice purposes. 

More than half of the ESRs will have a female scientist within their Supervisory Committee, which is more than the norm in the field and meets the Commission's target of 40\% of the under-represented gender, and will set an example for gender balance to follow throughout the ESR's career.

All supervisors will direct the research and training of the ESRs and work together on the topic of RTA. 
They will also assist the ESRs during the course of \acronym: this includes providing help with travel, installation, accommodation, integration, etc., with the help of the dedicated secondment officer, Dr. O. Smirnova. 
The supervisors will also provide mentoring and support towards the ESR's future career. 
%The ESRs will also receive informal mentoring by the high profile experienced scientists
%located at the various nodes that have agreed to participate in the Supervisory Board
%(see Sec.~\ref{sub:jointGoverningStructure}). 

[mention the supervisory committee and what it does]

%In  case the node granting the PhD title is not the same as the host,
%one or two of the supervisors will be from the host and another one or two from the second institution.
During the secondments, all the ESRs will have a responsible \textbf{tutor} in the institution in which the secondment is taking place. These tutors will ensure that the internship is fruitful for the research project of the student and for their training, and will provide all the necessary
resources. 
%All members of the node to which the ESR is seconded will participate in training. and be available to the ESRs throughout their secondments.
We have also taken care to place the secondment in the same location or country wherever possible, to ensure an effective joint supervision 
with maximal cross-talk, and that the overhead of ESRs moving countries (especially those with family) is reduced. 
%In this way the benefit of the secondment to the ESRs will be maximized.

As described in the previous section (Sec.~\ref{sec:training}), the ESRs and the supervisors will create a PCDP. 
Supervisors and ESRs will meet at least weekly, to review the work done and agree on next steps. 
Whenever the students are on a secondment, this meeting will be with the tutors, and the supervisors will have the chance to join via Skype. 
%Supervisors 
%who are not in the same physical location of the ESR (either because they are from a different 
%institution or because the student is on a secondment) will join these meetings via 
%teleconference. 
A "virtual corridor" for efficient communication between students and remote supervisors will be created in the form of instant messaging channels on the Mattermost platform for those who wish to use it, as described in Sec.~\ref{sub:networkOrganization}. 
The supervisors will report on the progress to the WP responsibles at the Executive Board meetings (see Sec.~\ref{sub:jointGoverningStructure}). 

As research organizations, neither \cernentity, nor \nikhefentity award Doctoral degrees, and neither do the private sector beneficiaries, \dqentity and \ibmentity. ESRs are enrolled in partner universities of the network, mainly where the beneficiary members hold positions, are assigned co-supervisors from the partner university, and obtain their PhD degree there. 

%In the case of \cnrsentity, node members also
%hold positions at \sorboneeentity, and
%the ESRs will be enrolled in this university to obtain
%their Doctoral degrees. Similar arrangements exist between 
%\nikhefentity 
%and its partner universities, \amsterdamentity and \radboudentity.
%In the case of \cernentity, the ESRs
%are assigned co-supervisors from \unigeentity and \dortmundentity, 
%and will be awarded PhD degrees in the same manner as ESRs based at these institutions. 



%Because \dqentity is based in Paris, no specific secondments to the academic
%supervisors between \dqentity and \parisUentity are foreseen, and the 
%$students will be able to alternate working between the nodes as best fits the research and training goals.

%Each ESR will, together with their supervisor, create a PCDP. 
%This will include the scientific contents 
%of the research project that the ESR will carry out and the
%list of their secondments and foreseen work. 
%The PCDP will contain a list of objectives for the 
%student. The PCDP will be presented by the ESR six months after
%the starting of their project to 
%be approved first by the local node coordinator, then by the
%relevant WP coordinator, and finally by 
%the Consortium Supervision Board. Updates to the PCDP,
%depending upon the outcome of some of the WP 
%milestones, will be allowed provided they are agreed
%with the specific WP coordinator. By the end of 
%their project, the ESR will be asked to present a report in
%which they compare the final 
%result of their work to the initial objectives. 
%This report will also have to be approved by the 
%relevant WP coordinators.

