% HEADER PLEASE READ!!!!!!!!
% HEADER PLEASE READ!!!!!!!!
% HEADER PLEASE READ!!!!!!!!
% HEADER PLEASE READ!!!!!!!!
% HEADER PLEASE READ!!!!!!!!
% ALL NODES TO FILL IN THE FIRST TABLE IN THIS TEX FILE DEFINING THE SUPERVISION
% HEADER PLEASE READ!!!!!!!!
% HEADER PLEASE READ!!!!!!!!
% HEADER PLEASE READ!!!!!!!!
% HEADER PLEASE READ!!!!!!!!
% HEADER PLEASE READ!!!!!!!!

%CD: Table will be added? Bullet points seem more efficient. 

%The following section of the European Charter for Researchers refers specifically to supervision:
%
%Employers and/or funders should ensure that a person is clearly identified to
%whom Early-Stage Researchers can refer for the performance of their
%professional duties, and should inform the researchers accordingly.
%Such arrangements should clearly define that the proposed supervisors are
%sufficiently expert in supervising research, have the time, knowledge,
%experience, expertise and commitment to be able to offer the research trainee
%appropriate support and provide for the necessary progress and review
%procedures, as well as the necessary feedback mechanisms.

%The network structure including the strong DS nodes at CNRS/LRI as
%well as the forefront of HEP research institutes (CERN, Lund, RHUL, 
%NIKHEF, Dortmund, CNRS/LAL) ensures that \acronym will be a truly 
%strategic partnership of experts in DS and HEP. 

%This partnership will be
%further strengthened by the strength of the DS experience at
%the chosen partners \nyu and \yandex. 
%The physics students will all be seconded by data
%scientists, allowing them a more stringent theoretical training on
%methods like deep learning. On the other hand, all DS 
%students will be seconded by particle physicists giving them access to
%the the data of the LHC and to actual problems physicists are working
%to solve with DS methods.


%Required sub-headings:
\subsubsection{Qualifications and supervision experience of supervisors}
\label{subsub:qual_supervisors}

%To avoid duplication, the role and profile of the supervisors should only be listed
%in the "Capacity of the Participating Organisations" tables (see section 5 below)

The ESRs at \acronym\ will be supervised by a group of competent researchers with proven experience in supervising PhD and Master's theses, as shown in Tab. 1.2. 
The consortium gathers the main experts in in RTA, machine learning and hybrid architectures from all LHC collaborations, with a demonstrated track record. 
Many of the supervisors are young yet successful scientists with long- or medium-term academic positions, who dedicate a large fraction of their time to research and direct supervision of students. This guarantees that, as specified in the European Charter for Researchers, they will have the time, knowledge, experience and commitment to offer support and feedback so to guarantee the supervision of excellent research. 

Both academic and industrial partners play a key role in supervision of \acronym, with the secondment responsibles acting as secondary supervisors. 
In the cases where the ESR is not seconded/hosted by an industry node, the ESR will have a supervisor from an industrial partner for mentoring and career advice purposes. 

In the table below, we indicate the main expertise and qualifications of the academic and industrial supervisors, leaving the details for the node description in part B2. 

%In all cases, the supervisors chosen for each ESR are those with the most relevant experience and expertise to
%enhance both the research and training potential of the ESR project. The same logic has guided the selection
%of secondments for each ESR. 

%- specify expertise of non-academic supervisors

\begin{center}\scriptsize
\begin{tabular}{|p{0.250\textwidth}|p{0.700\textwidth}|}
\hline
\textbf{Node / ESR} & \textbf{Supervisors and qualifications}
\tabularnewline \hline
\lundentity (\ESRj, \ESRk, \ESRm) & 
\textbf{C. Doglioni}: Associate senior lecturer. Expert on RTA and dark matter/new physics in ATLAS. ERC Starting Grant 2015. Author of first trigger-level search in ATLAS. Convener of HEP Software Foundation Trigger and Reconstruction group. Steering group of COMPUTE research school. Responsible for IPPOG Masterclasses at \lundentity. 
\tabularnewline
& \textbf{O. Smirnova}: Senior lecturer. Computing grid and distributed data analysis expert. National Grant on RTA and grid computing. Steering group of COMPUTE research school. 
\tabularnewline
& \textbf{P. Christiansen}: Professor. Expert in real-time detector reconstruction and analysis in ALICE. National Grants on Heavy Ion physics and detector building. Extensive teaching and outreach experience. 
\\
\hline
\cnrsentity (\ESRf, \ESRx) & 
\textbf{V. Gligorov}: Senior researcher. Expert in RTA, ML in trigger systems and flavor physics. ERC Consolidator Grant 2016. LHCb Real-time analysis project coordinator, former LHCb High Level Trigger project leader and deputy Physics Coordinator. Initiator and coordinator of LHCb's Masterclass programme. 
\tabularnewline
& \textbf{F. Crescioli}: Research engineer. Expert in ASIC design and design and commissioning of highly parallel FPGA based reconstruction systems, such as FTK for ATLAS. Technical coordinator of LPNHE ATLAS group and national projects. 
\tabularnewline
& \textbf{B. Malaescu}: Senior researcher. Expert in RTA, statistics, SM and BSM physics. ATLAS Standard Model Working Group convener. \\
\hline
\dortmundentity & \\
\heidelbergentity & \\
\helsinkientity & \\
\cernentity & \\
\nikhefentity &  \\
\unigeentity & \\
\sorbonneentity & \\
\ibmentity & \\
\fleetmaticsentity & \\
\ximantisentity & \\
\pointeightentity & \\
\lightboxentity & \\
\pisaentity & \\
\santiagoentity & \\
\oregonentity & \\
\liegesentity & \\
\uniboentity & 
%\ibmentity & \href{http://hr-training.web.cern.ch/hr-training/}{Academic training program} including transferrable skills.\\
\tabularnewline\hline
\end{tabular}
\end{center}

%%WP3
All ESRs will have a component of HEP real-time data analysis. 
Gligorov and Albrecht hold ERC grants on real-time physics analysis on the LHCb collaboration and have served in
multiple senior coordinating roles on the experiment. 
%and on the first triggerless detector readout. 
Their expertise is complemented by Raven from \nikhefentity, former LHCb HLT project leader and one of
the main authors of LHCb trigger software over the last decade, by further software experience from Matev from \cernentity
and by the expertise in ML training by LHCb researcher Ustyuzhanin from the Yandex School of Computing and \cernentity.   
Doglioni from (\lundentity, ERC StG), Boveia (\ohioentity),  Starovoitov and
Dunford in (\heidelbergentity) are the pioneers of the first trigger-level search for new physics in the
ATLAS collaboration. Together with the trigger and physics experts Strom and Majewski (\oregonentity), 
Petersen (\cern), Sfyrla (\unigeentity), Igonkina (\nikhefentity), with the computing grid and data analysis expert 
Smirnova (\lundentity) and with the machine learning expert Schramm (\unigeentity), 
they represent a team that can guide the ESRs to design analyses targeting discoveries and
advance high energy physics with novel techniques. 
The same can be said of the CMS researchers, M. Voutilainen, P. Eerola and H. Kirschenmann (\helsinkientity)
and M. Pierini (\cernentity, ERC CoG on Machine Learning at the LHC), who have designed and 
published the first search for new particles in CMS using trigger objects. P. Christiansen at \lundentity
and R. Shahoyan from \cernentity complement the data analysis in real-time from the ALICE collaboration,
attempting the first physics measurement with readout at the LHC collision rate for their Time Projection Chamber. 
%
%%WP4
For ESRs on hybrid architectures (3, 4, 8-11, 15), the help of
Lacassagne from \sorbonneentity, leader of the Hardware and Software for Embedded Systems team at LIP6 and
coordinator of WP6, will be crucial for a coherent program of studies. 
The know-how of Annovi and Roda from \pisaentity, Boveia of \ohioentity, and Crescioli of \cnrsentity in FPGA design
and commissioning from their experience on the ATLAS FTK, provide training and research 
using one of the first hybrid supercomputers in HEP as testing ground. 
GPU expertise is provided by Santos of \santiagoentity,
also holding an ERC StG on applications of GPUs for HEP. 
%
%%WP5
All ESRs except the most architecture-oriented ESR10 will perform a search or a measurement. Those
working on Dark Sectors and physics beyond the SM will benefit from the knowledge of the LHC Dark Matter Working Group organizers, 
Boveia and Doglioni, as well as from the expertise of in supersymmetric dark matter and new physics searches of Petersen, Sfyrla, 
Majewski, Voutilainen and Dunford. The network expertise in LFV/LFU is given by Albrecht, Gligorov, Teubert (\cernentity), Raven and Igonkina. 
Eerola of \helsinkientity is an expert of b-physics measurements,
and Malaescu of \cnrsentity is currently the ATLAS Standard Model convener, while Christiansen holds
several national grants for measurements in ALICE. 
%
%%WP6
Supervision towards commercial problems and solutions that complement the HEP component of the ESRs research projects 
are provided by all industrial nodes, with Sambo (\fleetmaticsentity), Catastini (\lightboxentity), Sopasakis (\ximantisentity), Dungs (\pointeightentity), 
Feillet and Julli\'{e} (\ibmentity). Dungs and Catastini transitioned to
industry after a successful research experience in HEP, Salti (\uniboentity) spent a period in industry before returning to academia, and Sopasakis is the CEO of a start-up alongside his
assistant professorship at \lundentity, so they are all extremely well placed to supervise and mentor students with intersector projects and career perspectives. 
%

%The expertise and the most relevant qualifications of the supervisors are in the table, with full 
%details in Sec.~\ref{sec:capacities}. 

% and will be the main
%mentor for the ESR on career and intellectual development. The main supervisor will also be responsible
%for helping the ESR compile their PCDP, discussed in the previous section.


%\begin{table}[bt]
%\caption{Specific supervisors and tutors per ESR. The number of
%  Diploma/MSc/PhD students supervised to date is given in square
%  brackets. \checkme{PS: not sure if these numbers are indicative} \label{tab:supervisors}}
%\vspace{-5mm}
%\end{table}
%\vspace{-2mm} 

%\begin{center}\scriptsize
%\resizebox {\textwidth }{!}{%
%\begin{tabular}{lm{80mm}m{100mm}}
%\toprule
%ESR & Host and main \phd supervisor & Second supervisors and tutors \tabularnewline
%\toprule
%%Here we have: physics, responsibility, award
%1 & \textbf{\underline{Voitulainen} \helsinkientity [7]}, jet reconstruction and top expert, convener of CMS Standard Model (SM) and Jets group, Wu-Ki Tung postdoc award in QCD. & 
%\underline{Doglioni} \lundentity [11], jet, trigger and Dark Sectors expert in ATLAS, LHC Dark Matter (DM) Working Group organizer, ERC StG 2015; 
%\textit{\underline{Sopasakis} [14]. Mathematics and AI expert, \ximantisentity CEO and \lundentity associate professor, start-up experience}.
%\tabularnewline\midrule	
%
%2 & \textbf{\underline{Eerola} \helsinkientity [19]}, B-physics and data taking expert in CMS, Director of Helsinki Institute of Physics, Responsibility positions in multiple scientific societies. &
%\underline{Pierini} \cernentity [0], ML and RTA expert in CMS, CERN Staff and Fermilab Fellow, ERC CoG 2017; 
%\textit{\underline{Sambo}, ML expert and Senior Data Scientist at \fleetmaticsentity, transitioned from HEP to industry}. 
%\tabularnewline\midrule	
%
%3 & \textbf{\underline{Sfyrla} \unigeentity [4]}, SUSY DM and trigger expert, various convenership positions at ATLAS in trigger, searches for new physics, HL-LHC upgrade & 
%\underline{Schramm} \unigeentity [1], ML, jet and new physics expert, convener of the Interexperiment Machine Learning group, Banting Fellow 2017; 
%\textit{\underline{Catastini}, data analysis expert, Quantitative Analyst at \lightboxentity coordinating a team of 10, transitioned from HEP to industry}. 
%\tabularnewline\midrule	
%
%4 & \textbf{\underline{Petersen} \cernentity [6]}, SUSY DM and trigger expert, past ATLAS trigger convener, current upgrade physics convener; \textbf{\underline{Sfyrla} \unigeentity [4]}. &
%\underline{Crescioli} \cnrs [2], FTK and FPGA expert, technical coordinator of various national and international R\&D projects; 
%\underline{Lacassagne} \sorbonneentity [13], Hybrid architectures expert, leader of the Hardware and Software for Embedded Systems team at LIP6; 
%\textit{\underline{Catastini} \lightboxentity}. 
%\tabularnewline\midrule	
%
%5 & \textbf{\underline{Teubert} \cernentity [35]}, LFV/LFU and trigger expert, CERN LHCb deputy leader, has convened numerous working groups across experiments; 
%    \textbf{\underline{Albrecht} \dortmundentity [13]}, LFV/LFU, tracking, ML and trigger expert, ERC StG 2016. &
%\textit{\underline{Head}, ML and data analysis expert, mentor, lecturer and consultant at \wildtreeentity, transitioned from HEP to industry}. 
%\tabularnewline\midrule
%
%6 & \textbf{\underline{Albrecht} \dortmundentity [13]}. &
%{\underline{Teubert} \cernentity [35]}, 
%\textit{\underline{Sopasakis \ximantisentity [14]}}. 
%\tabularnewline\midrule
%
%7 & \textbf{\underline{Albrecht} \dortmundentity [13]}. &
%{\underline{Smirnova} \lundentity [13]}, expert in distributed computing and data analysis in ATLAS, LHC Computing Grid Architect;
%\textit{\underline{Head} \wildtreeentity}. 
%\tabularnewline\midrule
%
%8 & \textbf{\underline{Crescioli} \cnrsentity [2]}, \textbf{\underline{Malaescu} \cnrsentity, \sorbonneentity [4]}, jet measurements and searches expert, ATLAS SM group convener, past Statistics Forum convener. &
%\underline{Annovi} \pisaentity [2], ATLAS FTK, tracking and trigger expert, FTK upgrade coordinator;
%\underline{Roda} \pisaentity [14], expert in searches beyond the SM, ATLAS Pisa group leader, coordination roles in ATLAS detector and analysis;
%\textit{Sambo \fleetmaticsentity}. %,  
%\tabularnewline\midrule
% 
%9 & \textit{\textbf{\underline{Meric}, CEO of \dqentity}, expert in ML and big data techniques, transitioned from HEP to industry};  
%\textbf{\underline{Gligorov} \cnrsentity [4]}, expert in ML, LFV/LFU, trigger, past LHCb Deputy Physics Coordinator, ERC CoG 2016. & 
%\underline{Santos} \santiagoentity [0], expert in rare decays, trigger and detector performance in LHCb, LHCb Particle ID Convenor; 
%\underline{Teubert} \cernentity [35].
%\tabularnewline\midrule
%
%10 & \textbf{\underline{Lacassagne} \sorbonneentity [13]}. &
%{\underline{Petersen} \cernentity [6]}, 
%\textit{\underline{Head} \wildtreeentity}.
%\tabularnewline\midrule
%
%11 & \textbf{\underline{Igonkina} \nikhefentity, \radboudentity [14]}, expert in LFV/LFU and triggering in LHCb, LHCb time-dependent CP-violation physics group convener, receiver of several Dutch grants &
%\underline{Strom} \oregonentity [6], expert in trigger and data acquisition in ATLAS, convenor of all aspect of trigger and data acquisition including upgrade in 2017-2018, 
%\textit{\underline{Hlindzich}, expert in medical simulations and computing, software engineer at \cathientity} 
%\tabularnewline\midrule
%
%12 & \textbf{\underline{Raven} \nikhefentity, \amsterdamentity [14]}, expert in LFV/LFU and triggering in LHCb, convened trigger and new physics working groups, receiver of several Dutch grants.&
%\underline{Albrecht} \dortmundentity [13], 
%\textit{\underline{Head} \wildtreeentity}.
%\tabularnewline\midrule
%
%13 & \textit{\textbf{\underline{Julli\'{e}}, \ibmentity}, software engineer, expertise in modelling optimization problems, ML, anomaly detection.} 
%\textbf{\underline{Doglioni} \lundentity [11]}& 
%\underline{Boveia} \ohioentity [1], expert in Dark Sectors, trigger and FTK in ATLAS, responsible for first trigger-level analysis in ATLAS. 
%\tabularnewline\midrule
%
%14 & \textbf{\underline{Christiansen} \lundentity [17]}, expert in real-time detector reconstruction and analysis in ALICE, receiver of several Swedish grants.&
%\underline{Shahoyan} \cernentity [4], expert in calibration and data analysis in ALICE, main developer of real-time O2 reconstruction project, 
%\textit{\underline{Sopasakis \ximantisentity [14]}}. 
%\tabularnewline\midrule
%
%15 & \textbf{\underline{Starovoitov} \heidelbergentity [7]}, expert in ATLAS detector, trigger and data analysis, receiver of several international fellowships. &
%\underline{Dunford} \heidelbergentity [8], Dark Matter, SM measurement, detector and trigger expert in ATLAS, editor of first trigger-level analysis paper in ATLAS, Young Researcher group leader; \textit{\underline{Kaplan}, expert in software for litography at \heidelberginstrumentsentity}. 
% \tabularnewline\midrule
%
%\end{tabular}
%}%
%\end{center}

All supervisors will direct the research and training of the ESRs and work together on the topic
of RTA. They will also assist the ESRs during the course of \acronym:
this includes providing help with travel, installation, accommodation, integration, etc., 
with the help of the dedicated secondment officer, Dr. O. Smirnova. The supervisors 
will also provide mentoring and support towards the ESR's future career. 
The ESRs will also receive informal mentoring by the high profile experienced scientists
located at the various nodes that have agreed to participate in the Supervisory Board
(see Sec.~\ref{sub:jointGoverningStructure}). 

\vskip-10pt

\subsubsection{Quality of the joint supervision arrangements}
\label{sec:jointsuperqual}

%To do if there is more room: 
%- explain better why a given supervisor is the best co-supervisor for a given project. 

More than half of the ESRs will have a female scientist within their Supervisory Committee, which is more than the norm in the field and meets the Commission's target of 40\% of the under-represented gender, and will set an example for gender balance to follow throughout the ESR's career.

%Justification for these choices can be found in the individual ESR descriptions. 
As shown in Table in Sec.~\ref{tab:recruitmentDeliverables},
each ESR will be assigned one or two \textbf{main supervisors}\footnote{Two main supervisors are foreseen when the student is receiving their PhD by a university different from the beneficiary.}, alongside one or more \textbf{second supervisors}. 
%In  case the node granting the PhD title is not the same as the host,
%one or two of the supervisors will be from the host and another one or two from the second institution.
During the secondments, all the ESRs will have a responsible \textbf{tutor} in the institution in which the secondment is taking place. These tutors will ensure that the internship is fruitful for the research project of the student and for their training, and will provide all the necessary
resources. 
%All members of the node to which the ESR is seconded will participate in training. and be available to the ESRs throughout their secondments.
We have also taken care to place the secondment in the same location or country wherever possible, to ensure an effective joint supervision 
with maximal cross-talk, and that the overhead of ESRs moving countries (especially those with family) is reduced. 
%In this way the benefit of the secondment to the ESRs will be maximized.

As described in the previous section (Sec.~\ref{sec:training}), the ESRs and the supervisors will create a PCDP. 
Supervisors and ESRs will meet at least weekly, to review the work done and agree on next steps. 
Whenever the students are on a secondment, this meeting will be with the tutors, and the supervisors will have the chance to join via Skype. 
%Supervisors 
%who are not in the same physical location of the ESR (either because they are from a different 
%institution or because the student is on a secondment) will join these meetings via 
%teleconference. 
A "virtual corridor" for efficient communication between students and remote supervisors will be created in the form of instant messaging channels on the Mattermost platform for those who wish to use it, as described in Sec.~\ref{sub:networkOrganization}. 
The supervisors will report on the progress to the WP responsibles at the Executive Board meetings (see Sec.~\ref{sub:jointGoverningStructure}). 

As research organizations, neither \cernentity, nor \nikhefentity award Doctoral degrees, and neither do the private sector beneficiaries, \dqentity and \ibmentity. ESRs are enrolled in partner universities of the network, mainly where the beneficiary members hold positions, are assigned co-supervisors from the partner university, and obtain their PhD degree there. 

%In the case of \cnrsentity, node members also
%hold positions at \sorboneeentity, and
%the ESRs will be enrolled in this university to obtain
%their Doctoral degrees. Similar arrangements exist between 
%\nikhefentity 
%and its partner universities, \amsterdamentity and \radboudentity.
%In the case of \cernentity, the ESRs
%are assigned co-supervisors from \unigeentity and \dortmundentity, 
%and will be awarded PhD degrees in the same manner as ESRs based at these institutions. 



%Because \dqentity is based in Paris, no specific secondments to the academic
%supervisors between \dqentity and \parisUentity are foreseen, and the 
%$students will be able to alternate working between the nodes as best fits the research and training goals.

%Each ESR will, together with their supervisor, create a PCDP. 
%This will include the scientific contents 
%of the research project that the ESR will carry out and the
%list of their secondments and foreseen work. 
%The PCDP will contain a list of objectives for the 
%student. The PCDP will be presented by the ESR six months after
%the starting of their project to 
%be approved first by the local node coordinator, then by the
%relevant WP coordinator, and finally by 
%the Consortium Supervision Board. Updates to the PCDP,
%depending upon the outcome of some of the WP 
%milestones, will be allowed provided they are agreed
%with the specific WP coordinator. By the end of 
%their project, the ESR will be asked to present a report in
%which they compare the final 
%result of their work to the initial objectives. 
%This report will also have to be approved by the 
%relevant WP coordinators.

