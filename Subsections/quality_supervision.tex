% HEADER PLEASE READ!!!!!!!!
% HEADER PLEASE READ!!!!!!!!
% HEADER PLEASE READ!!!!!!!!
% HEADER PLEASE READ!!!!!!!!
% HEADER PLEASE READ!!!!!!!!
% ALL NODES TO FILL IN THE FIRST TABLE IN THIS TEX FILE DEFINING THE SUPERVISION
% HEADER PLEASE READ!!!!!!!!
% HEADER PLEASE READ!!!!!!!!
% HEADER PLEASE READ!!!!!!!!
% HEADER PLEASE READ!!!!!!!!
% HEADER PLEASE READ!!!!!!!!

%CD: Table will be added? Bullet points seem more efficient. 

%The following section of the European Charter for Researchers refers specifically to supervision:
%
%Employers and/or funders should ensure that a person is clearly identified to
%whom Early-Stage Researchers can refer for the performance of their
%professional duties, and should inform the researchers accordingly.
%Such arrangements should clearly define that the proposed supervisors are
%sufficiently expert in supervising research, have the time, knowledge,
%experience, expertise and commitment to be able to offer the research trainee
%appropriate support and provide for the necessary progress and review
%procedures, as well as the necessary feedback mechanisms.

%The network structure including the strong DS nodes at CNRS/LRI as
%well as the forefront of HEP research institutes (CERN, Lund, RHUL, 
%NIKHEF, Dortmund, CNRS/LAL) ensures that \acronym will be a truly 
%strategic partnership of experts in DS and HEP. 

%This partnership will be
%further strengthened by the strength of the DS experience at
%the chosen partners \nyu and \yandex. 
%The physics students will all be seconded by data
%scientists, allowing them a more stringent theoretical training on
%methods like deep learning. On the other hand, all DS 
%students will be seconded by particle physicists giving them access to
%the the data of the LHC and to actual problems physicists are working
%to solve with DS methods.


%Required sub-headings:
\subsubsection{Qualifications and supervision experience of supervisors}
\label{subsub:qual_supervisors}

%To avoid duplication, the role and profile of the supervisors should only be listed
%in the "Capacity of the Participating Organisations" tables (see section 5 below)

Table 1.2 summarizes the qualifications and supervision experience of the supervisors in \acronym. 
Each ESR will have a SC that includes the main experts in in RTA, machine learning and hybrid architectures from all LHC collaborations. 
They have an excellent track record demonstrated by the European and national grants received by the PIs (5 ongoing ERC H2020 grants and numerous national grants). 
Each SC includes an industrial supervisor related to the topic developed in the PhD project that will also provide career-related mentoring. 

Many of the \acronym\ academic participants are young yet successful scientists with long-term academic positions, who dedicate a large fraction of their time to research and direct supervision of students. 
This guarantees that, as specified in the European Charter for Researchers, they will have the time, knowledge, experience and commitment to offer support and feedback so to guarantee the supervision of excellent research. 
To complement the supervision, ESRs supervised by young professors also have a senior professor/researcher from their local node included in their SC.
The addition of research staff the ESR SC guarantees that the ESRs receive day-to-day hands-on local or remote supervision, and allows \acronym postdocs to gain supervision experience necessary towards their future career to positions in academia and industry. 

All members of the ESR SC in Table 1.2 have \textbf{proven experience in supervising PhD theses}, as shown by the numbers of supervised early stage researchers from the PhD level onwards. 
Most of the researchers in the network, including the postdocs, have also held positions of responsibility within their large international collaborations. 
Within these roles, they have trained numerous students who performed research tasks under their supervision, and they have managed large groups of researches to successfully operate complex detector systems. 
Supervisors in industrial nodes have supervision expertise from \textbf{trainees to working in their company}, and are therefore well placed to deliver innovative cross-disciplinary training within \acronym. 

The ESRs also will receive \textbf{further academic mentoring} at their nodes and during the yearly meetings from the Internal Advisory Board, a \textbf{group of senior physicists} that includes Dr. Monica Pepe-Altarelli (also head of \cern node), Prof. Torsten Akesson from \lundentity, Prof. Paula Eerola from \helsinkientity, Prof. Stephanie Hansmann- Menzemer who was managing director at the PI of \heidelbergentity, Prof. Bernard Spaan who has been Dean of the Physics Faculty of \dortmund and Prof. Giuseppe Iacobucci, the Director of the DPNC at \unigeentity. 

%Other \textbf{permanent and senior post-doctoral researchers} in the various nodes (see part B2) will also be present for additional \textbf{day-to-day supervision} of the ESRs. 

\vskip-10pt

\subsubsection{Quality of the joint supervision arrangements}
\label{sec:jointsuperqual}

%Make sure to signal that ESR will be a vehicle to get supervisors to talk to each other regularly. 

%To do if there is more room: 
%- explain better why a given supervisor is the best co-supervisor for a given project. 

All academic and industrial partners play a key role in supervision of \acronym, with the secondment responsibles acting as co-supervisors~\footnote{In the cases where the ESR is not seconded/hosted by an industry node, the ESR will have a supervisor from an industrial partner for mentoring and career advice purposes.}.
As described in the previous section (Sec.~\ref{sec:training}), the ESRs and the supervisors will create a PCDP.  
The supervisors will direct the research and training of the ESRs and work with them to advance RTA, and provide mentoring and support towards the ESR's future career. 
The supervisors will also assist the ESRs in terms of integration in their research environment, both at the local node with the help of existing students, and during secondments with the help of the dedicated secondment officer, Dr. O. Smirnova, and are the first point of contact for issues that may arise with mobility. 
The training officer Starovoitov~\footnote{For students of the same node as the officers, the PC Doglioni will assume this role.} will act as ombudsman so that ESRs can raise issues in matter of training. 
During the secondments, all the ESRs will have a responsible \textbf{tutor} in the institution in which the secondment is taking place. These tutors will ensure that the internship is fruitful for the research project of the student and for their training, and will provide all the necessary resources. 
We have also taken care to place the secondment in the same location or country wherever possible, to ensure an effective joint supervision with maximal cross-talk, and that the overhead of ESRs moving countries (especially those with family) is reduced. 

The ESR will have regular meetings with the main supervisor on a weekly basis. 
There will be an "open-door" policy with the members of the SC most relevant for the day-to-day work at least weekly, to review the work done and discuss next steps. 
This arrangement guarantees that all ESRs will receive timely supervision even when the supervisors hold positions of responsibility, and that the ESR projects are vehicles for supervisors to discuss joint research and strengthen synergies between their organizations.  
Whenever the students are on a secondment, the main supervisor and the SC can meet the student via videoconference. 
A "virtual corridor" for efficient communication between students and remote supervisors will be created in the form of instant messaging channels for each ESR project on the Mattermost platform, as described in Sec.~\ref{sub:networkOrganization}. 
The main supervisors will report on the progress to the WP responsibles at the Executive Board meetings (see Sec.~\ref{sub:jointGoverningStructure}). 

%As research organizations, neither \cernentity, nor \nikhefentity award Doctoral degrees, and neither do the private sector beneficiaries, \dqentity and \ibmentity. ESRs are enrolled in partner universities of the network, mainly where the beneficiary members hold positions, are assigned co-supervisors from the partner university, and obtain their PhD degree there. 

More than half of the ESRs will have a female scientist within their Supervisory Committee, which is more than the norm in the field and meets the Commission's target of 40\% of the under-represented gender, and will set an example for gender balance to follow throughout the ESR's career. \acronym also has dedicated diversity and inclusion officers, as discussed in Sec.~\ref{sub:networkOrganization}. 
