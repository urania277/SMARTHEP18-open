%Concrete plans for these sections must be included in the corresponding implementation tables!
%Note that the following sections of the European Charter for Researchers refer specifically to  public engagement and dissemination:

%Dissemination, Exploitation of Results
%All researchers should ensure, in compliance with their contractual arrangements, that the results of their research are disseminated and exploited, 
%e.g. communicated, transferred into other research settings or, if appropriate, commercialised. Senior researchers, in particular, are expected to take 
%a lead in ensuring that research is fruitful and that results are either exploited commercially or made accessible to the public (or both) whenever the opportunity arises.

%Public Engagement
%Researchers should ensure that their research activities are made known to society at large in such a way that they can be understood by non-specialists, 
%thereby improving the public's understanding of science. Direct engagement with the public will help researchers to better understand public interest in 
%priorities for science and technology and also the public's concerns.
%You can also refer to the Communicating EU research and innovation guidance for project participants as well as to the "communication" section of the H2020 Online Manual.

%http://ec.europa.eu/research/participants/data/ref/h2020/other/gm/h2020-guide-comm_en.pdf
%http://ec.europa.eu/research/participants/docs/h2020-funding-guide/grants/grant-management/communication_en.htm



%Communicating to the tax-payers the activities of a publically funded activity Moreover, the information
%management is not a minor point in the success of any endavour. Conveying that a relevant activity is in progress, that partial and
%final results are achieved, and that the involved people are strongly committed to the effort is of the highest relevance.

Reaching audiences with a direct interest in \acronym research is covered in Section~\ref{sec:dissemination}. 
It is equally important to inspire the general public, young generations of students and their educators, and the entrepreneur and industry community with \acronym's basic research and novel techniques whose commercial applications impact everyday life. 
%Since HEP is mainly publicly funded, \acronym will profit from the familiarity of many of the proponents with communication and outreach activities. 

%From MSCA guidelines
%Outreach activities are meant to engage a large audience and to bring knowledge and expertise on a particular topic to the general public. Outreach activities can take several forms, such as school presentations, workshops, public talks and lab visits, etc. The objective of outreach is to explain the benefits of research to a larger public (the tax payers who fund your research). Outreach implies an interaction between the sender and the receiver of the message, there is an engagement and a two-way communication between the researcher and the public.
%Communication, on the other hand, only goes in one direction from the sender to the receiver. Communication refers to articles in mainstream newspapers and magazines, or on TV and radio channels. Successful communication requires a clear language and attractive scientific subject with outstanding results that can catch the media's attention.

%Already written above
%The audiences that will be reached through the \acronym communication and outreach program are 
%the general public reached mostly by the communication tools and local events, primary and secondary school students and teachers reached via the Masterclasses and the visits at CERN, entrepreneurs/industrial parties outside the network reached through local innovation events and career fairs.as listed below. 

\color{blue}Communication: \color{black} 
The ALICE, ATLAS, CMS and LHCb experiments, CERN, and academic institutions involved in \acronym have \textbf{public web pages} where the experiments main results are presented to large audiences. 
\acronym already has a webpage (\url{www.smarthep.org}), describing its members and their activities. 
If the network is successful, the page will be expanded with the ongoing activities, goals, achieved results and latest news with dedicated general public target. 
It will also continue being updated as the members engage in further collaborations beyond this ITN. 
The webpage will contain a link to a blog  curated by ESRs who would like to discuss their research activity in the style of \href{http://www.quantumdiaries.org}{http://www.quantumdiaries.org}, to allow the general public to have an insight into the life and work of young researchers. 
Here the ESRs can benefit from the experience of the INSIGHTS ESR seconded at \lundentity who is already curating such a blog. 
\textbf{Social media} are an important part of \acronym communication strategy: many of the network researchers are already active on Twitter (see e.g. @CatDogLund for the PC's account) and the network will have its own Twitter handle. 
We plan to run a scheme where an ESR curates the account for a period of time in the same fashion as @CMSVoices or @Sweden, with tweets approved by the PC and the WP7 responsible. 
Highlights and milestones of \acronym will be posted from this account. 
\acronym researchers will provide regular information to their institutions's press offices, and be proactive in seeking contact with \textbf{local newspapers and TV channels}, through press releases and articles in accessible language (e.g. we foresee an upcoming article on RTA in {The Conversation website\footnote{The Conversation is an independent research-oriented news source, \url{https://theconversation.com/}.}). 
The \textbf{logo} designed for \acronym and the H2020 logo will provide visibility to the network by being included on presentations in conferences,
%\footnote{We consciously do not say ``required'' as we recognize that the ESRs must consent to participating in public engagement, that some people
%have medical and particularly psychological conditions which make such activity hazardous to them,
%and that our equal opportunities policy mandates we do not discriminate against such applicants
%in our recruitment process.} 

\color{blue}Outreach: \color{black} 
Outreach activities are an integral part of \acronym training.% as they allow researchers to rediscover their own work from the fresh perspective of the non-specialist. 
The main innovative outreach element in \acronym is a \textbf{RTA data challenge} organized by Ustyuzhanin (\cernentity). 
This is a citizen science project\footnote{Citizen science engages the public in the creation of scientific results, reducing the gap between the scientific community and the general public.}, where open problems about RTA designed by the ESRs and their supervisors are published on an online platform, and prizes are offered to those solving them in the most efficient way. \cernentity and other \acronym participants have experience in citizen science and data challenges with e.g. the HiggsHunters project (Oxford U. and \lundentity) and the RTA-tailored RAPID2018 workshop (\dortmundentity and \cnrsentity). 
Since HEP is mainly publicly funded, \acronym members are already involved in outreach and educational activities such as \textbf{talks in high schools, conferences on popular science, summer schools, mentoring} and similar programs. 
They are also active in the EU-organized and funded events such as the \textbf{European Researchers Night}. 
These activities will be used as a platform for giving visibility to \acronym's multidisciplinary activities. 
\textbf{Open lectures} to, as well as Q\&A sessions with, the general public and local schools will be included as a part of \acronym-wide events. 
At the yearly meetings, we will organize \textbf{"Science on tap"} events following the successful example of the University of Hamburg and DESY (with 
which \lundentity has a strategic collaboration)\footnote{Science on tap 2017 edition, \url{http://www.cui.uni-hamburg.de/en/events/science-on-tap-2017/}}, where the ESRs present will organize short talks in local venues in an informal setting, and get to know each other and the attendants.  
In addition, \acronym members have extensive experience with the \textbf{International Masterclass} program and \textbf{World Wide Data Day}, which exposes tens of thousands of high-school students worldwide to HEP concepts and techniques. 
As part of WP7, we will create two new RTA-based Masterclass and WWDD exercises that can be chosen by any institute, moderated by \acronym members.   
In the same vein, it is vital to convey to undergraduate students in multiple disciplines (physics, mathematics/statistics and economy) the relevance of the network activities and the opportunities that the job market has for people that have completed a \phd in the multidisciplinary context created by \acronym. 
For this reason, ESRs will be encouraged to help with CERN visits from local schools when seconded or on field trips there. 
All ESRs will be expected to include at least one regular local outreach activity (e.g. chool and university visits; mentoring of younger students with a special eye on minority communities) within their PCDPs. 
All public engagement activities undertaken by ESRs will count as credits towards their training.  
To reach an audience with an interest beyond academia and inspire further collaboration, we foresee that the ESRs and the industrial supervisors will participate to local innovation events \footnote{e.g. Ideon Science Park Innovation Challenge, ~\url{https://ideon.se/event/lund-innovation-challenge/}} and career fairs, presenting the results of \acronym on RTA applications and the model of academic and non-academic training for PhD students. This will also be beneficial for the exposure and recruiting strategies of the industrial partners.

% such as
%\href{http://press.web.cern.ch}{press.web.cern.ch},
%\href{http://www.tu-dortmund.de/uni/Medien/index.html}{www.tu-dortmund.de/uni/Medien/index.html},
%\href{http://tuebingen.mpg.de/en/news-press.html}{tuebingen.mpg.de/en/news-press.html},
%\href{https://www.nikhef.nl/nieuws-events/}{www.nikhef.nl/nieuws-events/}.
%Twitter, Facebook, and Google$^{+}$ accounts under the control of the
%WP7 coordinator \textcolor{red}{Dr. xxx}
%will also be set up and used to disseminate information about network activities, events,
%and research progress. In addition, separate Twitter accounts will be created for each research WP,
%used by the WP coordinator to disseminate information about that WP.


%This could be achieved in talks for
%final year undergraduate students within the academic institutions involved and by proposing \phd thesis topics for final year undergraduates
%related to \acronym activities. These students would be encouraged to apply for the job offers with the other participants to \acronym.
%These job openings within \acronym will also be widely publicised on \acronym web page and on dedicated portals such as
%inspirehep.net etc. The job hiring procedures would follow a merit selection and ensure a transparent and fair recruitment processes.
