The training program of \acronym has three main {\bf objectives}, taking the Salzburg Principles\footnote{Salzburg II recommendations, \url{http://www.eua.be/Libraries/publications-homepage-list/Salzburg_II_Recommendations}} as a guidance: 
\begin{itemize}
\item {\bf Objective A}: provide ESRs with knowledge and training to conduct original research during \& beyond their PhD studies;
\item {\bf Objective B}: provide solid bases in a broad spectrum of topics related to their field of research, extending beyond their dedicated research and including soft skills; 
\item {\bf Objective C}: provide up-to-date and career-related training through interactions with multiple collaborators within academia as well as the industry, to meet the needs of a broad employment market.
\end{itemize}

Objectives A and B will be fulfilled by schools, workshops and events organized within the network; ESRs will profit from high quality lectures delivered by experts within the institutes associated to \acronym, complemented by doctoral training at local nodes. Objective C will be fulfilled by special network events where the industrial collaborators will provide seminars and training on commercial applications of \acronym's objectives, as well as career-focused aspects. All these objectives will be further addressed via the natural collaboration between the ESRs and the various institutes of the network, as well as the planned secondments in academia and industry. This is a more flexible and diverse structure with respect to standard PhD studies. It will allow the ESRs to develop an individual and unique research mindset within an inclusive environment, so that they can act as proactive participants in furthering HEP and industry goals through RTA techniques.
%Required sub-headings:
%Required sub-headings:
\vspace{-2mm}
\subsubsection{Overview and content structure of the training}
\label{sub:overviewTraining}
%(including network-wide training events and complementarity with those programs offered locally at the participating organisations (please include table 1.2a and table 1.2b)

%\acronym will facilitate exchanges between HEP and DS by training 15  of young scientists who can cross
%the disciplinary aisles more easily than their predecessors. For HEP
%students, the goal is to get acquainted with the latest research in DS so they know where to look for a solution when a new problem
%comes along. For DS students, the goal is to understand
%the basic notions and research goals of HEP so they can interact with physicists, understand and formalize novel
%DS problems coming from HEP. 
%We achieve this by ``embed'' students in laboratories of their secondary discipline, as
%prescribed by the ITN scheme. 
Training is at the heart of all activities within \acronym,
with a dedicated Work Package, WP2, with Anna Sfyrla from \unigeentity as responsible. 
In this section, we list the recruitment deliverables for the 15 ESRs, 
describe the Personal Career Development Plan (PCDP),
then describe the complementary network-wide training and local training, 
and finally outline the \acronym credit system designed for this ETN. 

\begin{wraptable}{l}{0.7\textwidth}
\vspace{-1mm}
%\vskip-10pt
	\caption{Recruitment deliverables per beneficiary, and supervisors/tutors with number of supervised students in brackets. Non-academic supervisors and tutors are in italics. \label{tab:recruitmentDeliverables}}
	\begin{center}\scriptsize
			%\resizebox {0.6\textwidth }{!}{%
			\begin{tabular}{p{5mm}p{13mm}p{7mm}p{9mm}p{30mm}p{35mm}}%{@{}lp{0.05\textwidth}p{0.05\textwidth}p{0.1\textwidth}p{0.15\textwidth}p{0.2\textwidth}@{}}
\toprule
ESR & \textbf{\Tstrut Recruiting Participant} & \textbf{\Tstrut Planned Start} & \textbf{\Tstrut Duration} & \textbf{\Tstrut Main supervisors} & \textbf{\Tstrut Secondary supervisors}
\tabularnewline 
\toprule
1 & \helsinkientity  & 8 & 36 & Voutilainen [9] & Kirschenmann [0], Doglioni [11], \textit{Sopasakis} [14] \tabularnewline\midrule
2 & \helsinkientity  & 8 & 36 & Eerola [19] & Pierini [0], \textit{Sambo} \tabularnewline\midrule
3 & \unigeentity  & 8 & 36 & Sfyrla [4] & Schramm [1], \textit{Catastini} \tabularnewline\midrule
4 & \cernentity  & 8 & 36 & Petersen [6], Sfyrla [4] & Crescioli [2], Lacassagne [13], \textit{Catastini}  \tabularnewline\midrule
5 & \cernentity  & 8 & 36 & Teubert [35] & Albrecht [13], \textit{Head}  \tabularnewline\midrule
6 & \dortmundentity  & 8 & 36 & Albrecht [13] & Teubert [35], \textit{Sopasakis}  \tabularnewline\midrule
7 & \dortmundentity  & 8 & 36 & Albrecht [13] & Smirnova [13], \textit{Head}  \tabularnewline\midrule
8 & \cnrsentity  & 8 & 36 & Crescioli [4], Malaescu [4] & Annovi [2], Roda [14], \textit{Sambo}  \tabularnewline\midrule
9 &  \dqentity  & 8 & 36 & \textit{Meric}, Gligorov [4] & Santos [1], Teubert [35]\tabularnewline\midrule
10 & \sorbonneentity  & 8 & 36 & Lacassagne[13] & Petersen [6], \textit{Head}  \tabularnewline\midrule
11 & \nikhefentity  & 8 & 36 & Igonkina [14] & Strom [6], \textit{Hlindzich}  \tabularnewline\midrule
12 & \nikhefentity & 8 & 36 & Raven [14] & Albrecht [13], \textit{Head}  \tabularnewline\midrule
13 & \ibmentity  & 8 & 36 & \textit{Julli\'{e}}, Doglioni [11] & Boveia [1]  \tabularnewline\midrule
14 & \lundentity  & 8 & 36 & Christiansen [17] & Shahoyan [4], \textit{Sopasakis}  \tabularnewline\midrule
15 & \heidelbergentity & 8 & 36 & Starovoitov [7] & Dunford [8], Majewski [3], \textit{Kaplan}  \tabularnewline\midrule
Total & & & 540 & &  \tabularnewline\bottomrule
\end{tabular}
%}%
\end{center}
\vspace{-6mm}
%\vskip-30pt
\end{wraptable}

%\begin{wraptable}{l}{0.3\textwidth}
%\vspace{-1mm}
%%\vskip-10pt
%	\caption{Recruitment deliverables per beneficiary\label{tab:recruitmentDeliverables}}
%	\begin{center}\scriptsize
%			\resizebox {0.3\textwidth }{!}{%
%			\begin{tabular}{@{}lp{10mm}p{10mm}p{10mm}p{10mm}@{}}
%\toprule
%ESR &
%\pbox{8cm}{\Tstrut Recruiting \\ Participant\Bstrut} &
%\pbox{8cm}{\Tstrut Planned \\ Start\Bstrut} & %month 0-45
%\pbox{8cm}{\Tstrut Duration \Bstrut}%duration 3-36
%\tabularnewline 
%\toprule
%\tabularnewline 
%\toprule
%1 & \helsinkientity  & 8 & 36\tabularnewline\midrule
%2 & \helsinkientity  & 8 & 36\tabularnewline\midrule
%3 & \unigeentity  & 8 & 36\tabularnewline\midrule
%4 & \cernentity  & 8 & 36\tabularnewline\midrule
%5 & \cernentity  & 8 & 36\tabularnewline\midrule
%6 & \dortmundentity  & 8 & 36\tabularnewline\midrule
%7 & \dortmundentity  & 8 & 36\tabularnewline\midrule
%8 & \cnrsentity  & 8 & 36\tabularnewline\midrule
%9 &  \dqentity  & 8 & 36\tabularnewline\midrule
%10 & \sorbonneentity  & 8 & 36\tabularnewline\midrule
%11 & \nikhefentity  & 8 & 36\tabularnewline\midrule
%12 & \nikhefentity & 8 & 36\tabularnewline\midrule
%13 & \ibmentity  & 8 & 36\tabularnewline\midrule
%14 & \lundentity  & 8 & 36\tabularnewline\midrule
%15 & \heidelbergentity & 8 & 36\tabularnewline\midrule
%Total & & & 540 \tabularnewline
%\bottomrule
%\end{tabular}
%}%
%\end{center}
%\vspace{-6mm}
%%\vskip-30pt
%\end{wraptable}


%Training is at the heart of all activities within \acronym,
%with a dedicated Work Package, WP2, with Anna Sfyrla from \unigeentity as responsible. 
%In this section, we list the recruitment deliverables for the 15 ESRs, 
%describe the Personal Career Development Plan (PCDP),
%then describe the complementary network-wide training and local training, 
%and finally outline the \acronym credit system designed for this ETN. 


\noindent \color{blue}Recruitment deliverables per participant and awarding of PhD degrees. \color{black}
Table~\ref{tab:recruitmentDeliverables} presents the recruitment deliverables per participant. 
ESRs will be recruited by month 8, complete
a total of 36 months of research and training, and be awarded a PhD degree. 
ESRs at non-academic beneficiaries and ESRs where
the beneficiary is an international or national laboratory will 
be awarded PhDs by universities within the network. 
%ESR9 and ESR13,
%will receive a PhD from \parisULong and \lund respectively. 
%ESR4, ESR5, ESR11 and ESR12 will be awarded laboratories 
For Finland, Netherlands and Sweden which mandate a
four-year PhD, all beneficiaries will provide support
for the ESRs to complete their PhD thesis.
%we include a letter of commitment from \nikhef and \lund
%guaranteeing funding for the final year of the ESR's studies.

%1.2.a
%\begin{table}[t]
%\caption{\label{tab:DelivPart}}

%\textbf{Table 1.2a: Recruitment deliverables per beneficiary}
%\begin{center}\footnotesize
%\resizebox {\textwidth }{!}{%
%\begin{tabular}{@{}p{5mm}p{105mm}p{20mm}p{15mm}p{15mm}@{}}
%\toprule
%ESR &
%Topic &
%\pbox{8cm}{\Tstrut Recruiting \\ Participant\Bstrut} &
%\pbox{8cm}{\Tstrut Planned \\ Start\Bstrut} & %month 0-45
%\pbox{8cm}{\Tstrut Duration \Bstrut}%duration 3-36
%\tabularnewline 
%\toprule
%\tabularnewline 
%\toprule
%1 & Discovery of new physics with jets in CMS with RTA & \helsinkientity  & 8 & 36\tabularnewline\midrule
%2 & Use of machine learning and RTA for discovering new physics and measuring the SM in CMS & \helsinkientity  & 8 & 36\tabularnewline\midrule
%3 & Machine learning for pattern recognition in searches for exotic physics at the ATLAS experiment & \unigeentity  & 8 & 36\tabularnewline\midrule
%4 & Efficient RTA in ATLAS using multi-threaded processing & \cernentity  & 8 & 36\tabularnewline\midrule
%5 & Search for LFV in tau, strange and charmed mesons decays using RTA in LHCb & \cernentity  & 8 & 36\tabularnewline\midrule
%6 & Real-time implementation of multivariate analysis for LFV in unflavoured meson decays in LHCb & \dortmundentity  & 8 & 36\tabularnewline\midrule
%7 & Event Triggering in LHCb & \dortmundentity  & 8 & 36\tabularnewline\midrule
%8 & Real-time particle trajectory reconstruction for online event selection and analysis in ATLAS & \cnrsentity  & 8 & 36\tabularnewline\midrule
%9 & Real-time analysis and machine learning in industry and new physics searches with LHCb data & \dqentity  & 8 & 36\tabularnewline\midrule
%10 & Enabling RTA on heterogeneous computing architectures & \cnrsentity  & 8 & 36\tabularnewline\midrule
%11 & Smart optimization of resources for efficient trigger and analysis and use for ATLAS measurements of LFV & \nikhefentity  & 8 & 36\tabularnewline\midrule
%12 & Smart optimization of resources for efficient trigger and analysis and use for LHCb measurements of LFV & \nikhefentity & 8 & 36\tabularnewline\midrule
%13 & Anomaly detection in industry and ATLAS trigger for new physics searches & \ibmentity  & 8 & 36\tabularnewline\midrule
%14 & Real-time calibrations and analysis of the ALICE Time Projection Chamber & \lundentity  & 8 & 36\tabularnewline\midrule
%15 & Real-time noise reduction in searches for new physics phenomena beyond the Standard Model in ATLAS & \heidelbergentity & 8 & 36\tabularnewline\midrule
%Total & & & 540 \tabularnewline
%\bottomrule
%\end{tabular}
%}%
%\end{center}
%%\end{table}


%%%PCDP 

\noindent \color{blue}Personal Career Development Plan. \color{black}
Two months after the start of their projects, ESRs will present a \textbf{Personal Career Development Plan} 
(PCDP) so that the \textbf{core, advanced and transferrable skills} to be acquired,
as well as the milestones for each of the ESR projects,
can be agreed between student and supervisor and consortium, 
planned and monitored throughout the course of the program. 
This will be done with the guidance of their supervisors, and will take into account the 
existing resources both at the ESR node and at foreseen secondments. The PCDP will be reviewed by the local node coordinator, who will bring it to the
Supervisory Board (SB) for approval. The main supervisor will be available throughout the 
course of the PhD, and meet with the ESR on a weekly basis. 
The local node coordinator will also review the progress
at minimum every six months (e.g. during staff appraisal meetings) and bring a short report to the consortium. 
%ESRs are expected to participate in one or two network-organized schools throughout
%their PhD, corresponding to up to 3 ECTS each, and one external summer or winter school. 
The PCDP\footnote{The \acronym PCDP will follow a common template based on that
from the MSCA website to provide coherence between the ESRs}
will include
the requirements, milestones and goals within the schedule of the doctoral program (including secondments),
the local and network-wide courses and schools to be attended, and
a list of dissemination, communication and outreach activities. 

%	requirements and goals of the planned training for the ESR
%	A list of courses (local and network-wide) to be taken by the ESR during their program, including any ECTS credit requirements
%	A list of communication and dissemination activities to be undertaken by the ESR
%	A schedule for their program, including secondments

%Core Research Skills (acquired via their ESR project)
%Advanced/Additional Research Skills (delivered by the consortium)
%Transferable Skills (delivered by the consortium - particularly those useful in non-academic careers)   

%The doctoral program will have a set of compulsory modules but also some degree of freedom for the 
%ESRs' preferences. ESRs are expected to complete 1 or 2 
%secondments, considered as a part of their training. 

\noindent \color{blue}Network-wide events: schools, yearly meetings and schools. \color{black}
%CD expand here on how the PhDs will benefit from the conferences and from interactions with other members of the network (students and secondment supervisors experts in what they do)
Network-wide schools, conferences and events %listed in the following table (Table 1.2b) 
shown in the table below will be organized by \acronym beneficiaries
and partners as part of the training and dissemination program. 
We expect ESRs to attend network events in person, but 
wherever possible will make network school, conferences and events available as Webinars using
the \href{http://information-technology.web.cern.ch/services/fe/vidyo}{Vidyo technology provided by \cernentity}, to allow all \acronym ESRs and PIs to attend if family/personal commitments would 
otherwise prevent it. A permanent record of
the lectures will be available as proceedings and in some cases video recordings. 
\acronym has the ambition to continue the training program beyond this Action, 
\begin{wraptable}{l}{0.45\textwidth}
    \vspace{-2mm}
	\caption{Example \acronym\ doctoral program\label{tab:docProg}
	}\vspace{4mm}
	%\begin{center}
	\small
	%\resizebox {\textwidth }{!}{%
	\begin{tabular}{p{50mm}r}
		\midrule
		Type of training & Number of credits \tabularnewline\midrule
		\textbf{Training through research}  & \textbf{135} \tabularnewline
		\hspace{5mm}At host & 75 \tabularnewline
		\hspace{5mm}Through secondment  & 60 \tabularnewline\midrule
		\textbf{Training through lectures, courses and dissemination} &  \textbf{45} \tabularnewline
%		\hspace{5mm}PhD courses  & \tabularnewline
		\hspace{5mm}Technical and Research Training & 30 \tabularnewline
		\hspace{5mm}Transferable Skills Training & 15 \tabularnewline
		\hspace{10mm}of which towards teaching & 5\tabularnewline
        \hspace{10mm}and dissemination & \tabularnewline
		\textbf{Total} &  180 \tabularnewline
		\bottomrule
	\end{tabular}
	%}%
	%\end{center}
    \vspace{-2mm}
\end{wraptable}
organizing these topical schools in a three-year rotation period. All network-organized
schools will be also open to the local students of the beneficiary and partners organizing the school
as part of enhancing the overall training program of the institutions involved in \acronym. 
In addition to network-specific events, ESRs will be encouraged
to attend the International School of Trigger And Data AcQuisition, (ISOTDAQ) as part of their training plan. 
ISOTDAQ is a yearly school dedicated to triggering and acquiring data for physics experiments
with lectures and hands-on exercises in equal proportions. The school organizers
have agreed that editions of this school from 2020 will feature a lecture on RTA taught by \acronym researchers if the network is funded. 

\noindent \color{blue}Local training. \color{black}
Training events organized specifically for \acronym are complemented
by local training provided by beneficiary nodes, as detailed in the table below. Students from the
network will be able to participate in this training when located at the node. 
All nodes include a range of graduate level courses in languages, career management, presentational skills, diversity and inclusion, as 
well as pedagogical courses in teaching and learning that the ESRs will be encouraged to follow
to obtain the necessary amount of credits. As we want to prepare ESRs to teach others, specific 
credits are assigned to teaching skills (taking pedagogical courses, but also supervising Master's students). 
Partners will also contribute with 
individual supervision and local training for employees while students
are seconded at their premises. 

%%Local training
\vspace{-4mm}
%\begin{table}[h]
%\caption{Local, existing training across the network.\label{tab:LocalTraining}}
\begin{center}\footnotesize
\begin{tabular}{p{\textwidth}}
\midrule
\textbf{\acronym :} Types of local training provided by the beneficiaries \tabularnewline\midrule 
\textbf{\helsinkientity:} HEP graduate school including courses equivalent to one year of full time study (40 ECTS points) of which 10 ECTS are to be earned in transferable skills such as e.g. \href{https://weboodi.helsinki.fi/hy/vl_kehys.jsp?Kieli=6&MD5avain=&vl_tila=4&Opas=5703&Org=98574586&KohtTyypHierAuk=33}{scientific/grant writing and efficient communication}. Regular seminars at the Helsinki Institute of Physics and training courses at the Finnish \href{https://www.csc.fi/web/training}{"IT Center For Science"} complement the classes. \\ %
\textbf{\unigeentity:} HEP doctoral school that provides courses ranging from theory and phenomenology to experimental aspects such as detectors and stats. These courses are given by UNIGE employees or other experts in the field that are especially invited to give lectures. \href{http://ple.unige.ch/fr/}{Dedicated center} organizes soft skills workshops and seminars.\\ %, awarding ECTS credits
\textbf{\cernentity}: 
%\href{https://indico.cern.ch/category/345/}{Summer students program}; 
\href{http://hr-training.web.cern.ch/hr-training/}{wide catalogue} of transferrable skills courses available, \eg, ``Making presentations'', ``Writing of professional documents'', ``Risk Management''\\%
\textbf{\dortmundentity:} Advanced lectures (Msc. / graduate student level)
in the departments of physics, computer science and
mathematics. Graduate school of the collaborative research center
(SFB876) on Resource-aware Machine Learning. Weekly seminars in
physics and computer science. Personal and professional development transferrable skills courses in the center for higher education. \\
\textbf{\cnrsentity:} Sorbonne university provides a full range of \href{http://ed560.ipgp.fr/index.php/Formations_scientifiques}{academic} and \href{http://ed560.ipgp.fr/index.php/Formations_g\%C3\%A9n\%C3\%A9ralistes}{non-academic} (transferrable skills) training courses.\\ %
\textbf{\nikhefentity}:  \href{https://www.nikhef.nl/en/education/onderzoekschool/}{Research school} in sub-atomic physics, with \href{https://www.nikhef.nl/en/education/onderzoekschool/topical-lectures/}{topical lectures} on subjects ranging from theoretical and experimental physics, to advanced statistical data analysis
techniques;
\href{https://www.nikhef.nl/en/education/academic-education/master/}{lectures}
from Masters program; personal development courses on time
management, grant writing, C++ and object oriented programming.
\textbf{\lundentity:} Advanced lectures (Msc. / graduate student
level) in the departments of physics, computer science and
mathematics. Yearly course for PhD students organized by the HEP division on a variety of topics (e.g. 2016: Dark Matter). 
Weekly seminars in physics, mathematics, astronomy and computer science. \href{http://cbbp.thep.lu.se/compute/Courses.php}{COMPUTE graduate school} with advanced courses on computing in research, awarding ECTS credits. \href{https://www.lunduniversity.lu.se/international-admissions/professional-education/professional-education-paid-by-your-employer}{Transferrable skills courses} such as "entrepreneurship and soft skills".\\
%Lectures in statistics and reproducibility in data science, statistics and astrophysics\\
\textbf{\heidelbergentity:} \href{https://www.physik.uni-heidelberg.de/highrr/}{HighRR graduate school}, lectures and tutorials on HEP detector development. School on physics beyond the SM, where lectures are prepared and delivered by advanced students, are also part of the training.%
\tabularnewline
\textbf{\ibmentity :} One of the \href{http://www.rudebaguette.com/2014/03/26/ibm-france-lab-hotbed-innovation-made-france/}{largest, top research labs in France}, IBM Research Lab France provides expert supervision and employee in algorithms, mobile app integration from design to market, interaction between large companies and startups, cloud computing and cognitive systems. \\
\textbf{\dqentity :} Expert supervision in deep learning, real-time control, project management, software development, insurance provision and risk assessment.\\
\midrule
\end{tabular}
\vspace{-4mm}
\end{center}

\noindent \color{blue}\acronym credits system: \color{black}
To ensure that such a diverse training program is 
coherent and recognized across the network, we have designed 
a \acronym credit system according to the ECTS standard. 
%The training side of the projects will be developed as a doctoral program.
Each ESR will complete 180 \acronym\ credits, as shown for an example ESR 
with two secondments in Table~\ref{tab:docProg}.

As PhDs in all participating institutions are awarded based
on local regulations, 
%solely on the basis of a produced PhD thesis;
\acronym credits also ensure that ESRs receive the appropriate training. %and are not required to be legally recognized by the awarding institutions.
%CD: does this make sense? Maybe reword
Each network-wide event below includes the
amount of assigned \acronym\ credits.

%The numbers in brackets indicate the division between {\it Technical and Research}' and
%{\it Transferable Skills}' credits. 
We have assigned 1 \acronym\ credit
per each $1/2$ day of lectures. 
%The open conferences of the network
%have only {\it Technical and Research} credits associated. For the
%{\it Transferable Skills} lectures in particular, 
%only examples are presented---the nature of
%private sector employment evolves very quickly, and our strong
%industrial partners
%will allow the network to stay on top of the emerging paradigms. 
All students will have to explicitly include 15 credits of 
transferrable skills training 
within their PCDP - they will be able to choose among the programs
of their local institute, or the institutes / industries they are
seconded in. By attending the yearly meetings of the network and the final conference,
%The institutions awarding the doctorates are listed in Sec.~\ref{sec:jointsuperqual}. 
\begin{wraptable}{l}{0.45\textwidth}
	%\vspace{-2mm}
	\caption{\acronym yearly meetings.\label{tab:YearlyMeeting}}
    \vspace{4mm}
	%\begin{center}%
	\small
	\begin{tabular}{m{75mm}}
		\midrule 
		\textbf{Days 1--2}\tabularnewline 
		{\begin{itemize}%[topsep=2pt,noitemsep,listparindent=3pt,leftmargin=*]
				\item Presentations and poster session by the ESRs.
				\item Meeting of the Executive Board and \linebreak
				preparation of Supervision Board meeting.
                \vspace{-5mm}
			\end{itemize}}
            \tabularnewline
            \midrule 
			\textbf{Days 3--4/5}\tabularnewline 
			\begin{itemize}%[topsep=2pt,noitemsep,listparindent=3pt,leftmargin=*]
				\item Transferable/research/technical skills lectures.
				\item Outreach activities. 
				\item Supervisory Board meeting. \vspace{-4mm}
			\end{itemize} \tabularnewline \midrule
		\end{tabular}
		%\end{center}
		\vspace{-3mm}
	\end{wraptable}
around $1/2$ of the \acronym\ credits
that they will need to complete in both ``Technical and Research'' and
``Transferable Skills'' categories, as presented at the beginning of
the section, will be provided in network-wide events. 
%Examples of the
%lectures foreseen in both categories can be found in
%Table~\ref{tab:Lectures}. 
ESRs will have freedom how to complete the rest of required credits through the local
resources. 
%Note that language training resources are available at
%every node and therefore are not
%made explicit. 
Finally, ESRs will be required to
present their work to at least one conference outside the network in their area of expertise. A list of conferences of interest for
\acronym topics is given in Sec.~\ref{sec:dissemination}.
The amount of credits awarded will depend on the targeted conference.  
The attribution of ECTS credits will require institutes to explicitly
include the network events in their course plan, but the conversion
from \acronym to ECTS credits will be justified and straight forward. 
% and this will be the responsibility of 
%each beneficiary main contact.

\vspace{-5mm}
%%\begin{wraptable}{r}{0.8\textwidth}
%%\begin{table}[!h]%{0.8\textwidth}
%%	\caption{
%%\label{tab:Events}}
%	\begin{center}\scriptsize
%			%\resizebox {\textwidth }{!}{%
%			\begin{tabular}{@{}lp{56mm}p{7mm}p{40mm}p{20mm}@{}}
%				\toprule
%				\multicolumn{2}{p{4cm}}{\pbox{8cm}{Training Events \& Conferences}} &
%				\pbox{8cm}{Credits} &%http://ec.europa.eu/education/ects/users-guide/docs/ects-users-guide_en.pdf 
%				\pbox{8cm}{Lead Institution} & 
%				\pbox{8cm}{Action Month}
%				\tabularnewline 
%				\toprule
%				1. & Kick-off meeting & - & \lundentity & 2  \tabularnewline\midrule
%				2. & Yearly meeting & 2 & \lundentity & 10,22,34  \tabularnewline\midrule
%				3. & Introductory school & 3 & \nikhefentity & 10  \tabularnewline\midrule %when all recruitment is complete
%				4. & Physics and machine learning school & 3 & \unigeentity & 18  \tabularnewline\midrule
%				5. & Basic FPGA course, FPGA boot-camp & 1.5 & \cernentity, \ohioentity, \pisaentity & 27, 28 \tabularnewline
%				6. & CPU and hybrid architectures school & 1.5 & \santiagoentity  & 29 \tabularnewline \midrule
%				7. & Intermediate conference & - & \dortmundentity  & 30 \tabularnewline \midrule
%				8. & Industry, career and transferrable skills school & 1.5 & \heidelbergentity  & 36 \tabularnewline \midrule
%				9. & Final conference & 2 & \cnrsentity & 42  \tabularnewline \midrule
%				10. & Closing meeting & - & \lundentity & 48  \tabularnewline\midrule
%				\bottomrule
%			\end{tabular}
%		%}%
%	\end{center}
%%	\vspace{-5mm}
%%\end{wraptable}
%%\end{table}


%\FloatBarrier
%\begin{table}[!htb]
%\centering
\begin{center}
\scriptsize
\resizebox {\textwidth }{!}{%
%\begin{tabular}{@{}p{5mm}p{40mm}p{25mm}p{22mm}p{22mm}p{12mm}p{12mm}p{12mm}}

			\begin{tabular}{@{}lp{56mm}p{7mm}p{40mm}p{20mm}@{}}
				\toprule
				\multicolumn{2}{p{4cm}}{\pbox{8cm}{Training Events \& Conferences}} &
				\pbox{8cm}{Credits} &%http://ec.europa.eu/education/ects/users-guide/docs/ects-users-guide_en.pdf 
				\pbox{8cm}{Lead Institution} & 
				\pbox{8cm}{Action Month}
				\tabularnewline 
				\toprule
				
				%%%
				
				\cellcolor{red!70!black}1. & Kick-off meeting & - & \lundentity & 2  \tabularnewline\hline
				
				\multicolumn{5}{p{\textwidth}}{
				
				The \textbf kick-off meeting  will be dedicated to organizing the project management, signing the consortium agreement, and monitoring the recruitment of the ESRs.
			    } \tabularnewline \hline\midrule
			    
			    %%%
			    
				\cellcolor{red!70!black}2. & Yearly meeting & 2 & \lundentity & 10,22,34  \tabularnewline \hline
				
			 	\multicolumn{5}{p{\textwidth}}{
				
				\acronym will hold \textbf{yearly in-person network-internal conferences}, with a duration of 4 or 5 days, whose structure is shown in
                %The structure of these yearly meetings is shown in 
                Table~\ref{tab:YearlyMeeting}. 
                The splitting in the length of these meetings refers to the days devoted to ESRs presentations and to lectures.
                Each yearly meeting will contain appropriate dissemination activities
                tailored to local circumstances like public poster sessions,
                lectures, Q\&A sessions or Masterclasses. These will generally take up
                between half a day and one day. 
                All yearly meetings will also include a dedicated half-day of lectures on gender, diversity and inclusion and transferrable skills
                including skills needed to work in their environment positively and effectively (soft skills)
                so that all ESRs have network training that complements their local training. 
                Example topics that have been agreed upon are "Optimizing workspaces for productivity" (\dqentity); 
                "Writing software in collaborative environments" (\wildtreeentity), "Gender and inclusion" (\lundentity),
                "Innovation and entrepreneurship, including IPR" (\lundentity, in a one-day workshop similar to the successful
                \href{http://indico.hep.lu.se/conferenceDisplay.py?confId=697}{Innovation and Entrepreneurship for PhD Students} event).
			    } \tabularnewline \hline\midrule
			    
			    %%%
			    
				\cellcolor{lime} 3. & Introductory school & 3 & \nikhefentity & 10  \tabularnewline\hline%\midrule %when all recruitment is complete
				
			    \multicolumn{5}{p{\textwidth}}{
			    
			    A first introduction to both HEP and RTA will be provided, by experts from both HEP and industry. 
			    This school will be an opportunity for all ESR members to get to know each other. 
			    They will also receive basic training in gender issues and research integrity. 
			    This school will be hosted shortly after the ESRs have been recruited. The lecturers for this and other
			    network schools have been identified from within the network based on
			    the expertise of the individuals or the groups they work with in, see Sec. B1.4.1%Sec.~\cite{sec:trainingcontrib}. 
			    We will also invite visiting scientists outside the network, and the invitations will be discussed during the organization of each of the schools.   
			    } \tabularnewline \hline\midrule
			    
			    %%%
			    
				\cellcolor{orange} 4. & Physics and machine learning school & 3 & \unigeentity & 18 \tabularnewline\hline
				
				\multicolumn{5}{p{\textwidth}}{
				
				 This school will provide an in-depth introduction to machine learning concepts and the connections to HEP,
				 as well as an overview of other academic or industrial applications, and it will take place at \unige a
				 few months after the ESRs have started with their projects. 
			    } \tabularnewline \hline\midrule
			    
			    %%%
			    
				\cellcolor{yellow} 5. & Basic FPGA course, FPGA boot-camp & 1.5 & \cernentity, \ohioentity, \pisaentity & 27, 28 \tabularnewline\hline
				
				\multicolumn{5}{p{\textwidth}}{				
				
				This school will include lectures on technologies and architecture and hands-on exercises on
				triggering applications. Lecturers will be given from researchers in \pisaentity, a leading
				institute in research and development on hardware track triggers, and from researchers from
				\ohioentity. This school will consist of introductory courses given at \cernentity, and a
				follow-up bootcamp where practical problems are solved for ESRs specializing in this topic. 
			    } \tabularnewline \hline\midrule	
			    
			    %%%%
			    			
				\cellcolor{yellow} 6. & GPU and hybrid architectures school & 1.5 & \santiagoentity  & 29 \tabularnewline \hline
				
				\multicolumn{5}{p{\textwidth}}{				
				
				In this school, the ESRs will learn how to compare architectures and practical
				solutions (e.g. GPU programming). The ESRs will learn not only about what is available on
				the market and what is being planned, but also ways to best evaluate the chosen hardware solution
				for the software they are developing.
				} \tabularnewline \hline\midrule				
				%7. & Intermediate conference & - & \dortmundentity  & 30 \tabularnewline \midrule
				
				%%%%
				
				\cellcolor{green} 7. & Industry, career and transferrable skills school & 1.5 & \heidelbergentity  & 36 \tabularnewline \hline
				
				\multicolumn{5}{p{\textwidth}}{		
							
				This school will include lectures and workshops in collaboration with industry.
				This school is dedicated to an in depth study of strategies for intellectual property rights (IPR),
				commercializing research output, presenting research results to policymakers, and knowledge transfer.
				One day of this school will be dedicated to group work on case studies prepared by the industrial
				beneficiaries and partners. This school will include experiences and Q\&A sessions with the CEOs
				and founders of the companies within \acronym. This school will be organized at \heidelbergentity together with \dqentity. 
%An example of the lectures:
%\begin{itemize}
%\item {Description of the company;}
%\item {Experience from transitioning from physics to industry;}
%\item {The daily job within the industry (e.g. of a quant in finance or of a quant in a Big Data company) and why physicists are particularly good at that when it comes to real-life (or real-economy) applications.}
%\item {An example of the life-cycle of a basic commercial strategy: from the initial idea to real-time deployment}
%\item {Group work: analysis of a case study of big data applications to real economy}
%\end{itemize}
				} \tabularnewline \hline\midrule				
				
				%%%%
				
				\cellcolor{cyan} 8. & Final conference & 2 & \cnrsentity & 42  \tabularnewline \hline
				
				\multicolumn{5}{p{\textwidth}}{					
				
				We will hold a five-day \textbf{public conference}
				%, as detailed in Table~\ref{tab:Events}. These conferences 
				which will showcase the work of the network to the wider scientific community. 
				As opposed to the yearly meetings, the conference will not feature management meetings,
				and will be dedicated to presentations on \acronym research
				and invited topical presentations on state-of-the art developments within the HEP and DS fields. 
				Each of the conferences will dedicate between
				half and one day to dissemination activities, both by \acronym members and by invited external experts. 
				The conference will have two additional days reserved for ESRs' presentations and 
				public lectures showcasing the work done during their PhD program with \acronym. ESRs will also
				take active part in the organization of this conference, to add to their transferrable skillset. 
				} \tabularnewline \hline\midrule
						
				\cellcolor{red!70!black} 9. & Closing meeting & - & \lundentity & 48  \tabularnewline\midrule
				
				\multicolumn{5}{p{\textwidth}}{					
				
				In the closing meeting, the network will take stock of the experience of the ETN and plan the next steps. The PIs will give 
				public lectures. The closing meeting is beyond the doctoral period for most of the ESRs,
				but they will be invited to participate as network alumni, also to give precious feedback on the ETN experience. 

				} \tabularnewline \hline\midrule

				\bottomrule
			\end{tabular}

}%end of resizebox
\end{center}
%\vspace{-5mm}
%\end{table}
%\FloatBarrier
%\vspace{-5mm}

%%Events of the network and yearly meetings
	

%\vspace{-5mm}
%\end{table}

%\textbf{\Tstrut Transferable skills\Bstrut} \\
%{\parbox{\textwidth}{%
%\Tstrut 
%Gender and unconscious bias (\cern); \\ \hline
%\end{tabular}
%}%
%\end{center}
%%\vspace{-5mm}
%\end{table} 
%\vspace{-2mm}
\subsubsection{Role of non-academic sector in the training program}

\acronym recognizes that basic research is at the heart of its PhD projects in the ESRs, but also
that modern academia is not a career for life: researchers can move between academic, industrial,
and entrepreneurial sectors dynamically. This enriches all sectors through the transfer of best practice, knowledge, and expertise.
For this reason we have included a comprehensive program of non-academic training in \acronym 
including hands-on experience in solving practical problems during secondments
at industrial partners, discussed further in Sec.~\ref{sec:qualityInteraction}.
The non--academic partners of \acronym will have the following crucial roles in the training of the ESRs:
\begin{itemize}
\item \textbf{Training through research: secondments}. One of \acronym's most important objectives is to increase the exposure of the 
students to the private sector, solving practical problems with RTA and creating commercial value. 
Mot ESRs will have secondments at private companies relevant for their tools and research topics.
These will place a particular emphasis on common methods between the commercial
applications which the non--academic partners specialize in, and the academic goals of the ESR 
projects. The connections are emphasized in the ESR project descriptions (Sec.~\ref{sec:FellowProj}). 
\item \textbf{Training through mentoring and supervision:} The secondment supervisor will be a co-supervisor for the
overall PhD project, while ESRs that don't have an industrial secondment will be assigned a non--academic tutor throughout
their PhD.
\item \textbf{Training through courses: transferrable skills} All industrial beneficiaries and partners have also agreed to
provide lectures and courses on transferrable skills and on their experience in occasion of the yearly meetings, and in 
the network-organized industry school. Since an additional goal of \acronym is to develop sustainable software that is
commonplace in industry, dedicated lectures will be given by T. Head of \wildtreeentity in the introductory school.
\end{itemize}

%allowing the seconded students to participate in their local training events. 
%
%We profit from the experience of the companies that take part in the project, and include secondments with them
%for most of the ESRs
%based at academic nodes. The secondments will be an essential contribution of the non--academic sector to the 
%training, also because 
%The secondments will ensure that ESRs are trained in a variety of skills throughout their participation in \acronym. 
%The industrial beneficiaries and partners in \acronym all share the challenge of high energy physics research
%of taking decisions fast and efficiently, and the secondments are naturally embedded in the research programs
%for each ESR as discussed in Sec.~\ref{sec:coherence} providing \textbf{training through research}. 
%  
%
%
%Finally, the non--academic sector is also expected to contribute intensely to the network-wide 
%events, both in the Technical/Research and Transferable Skills sides. 
%There will be several lectures on industry-related 
%Transferable Skills covered by experts from the non-academic sector.


%%%%%%%%

%\noindent \textbf{\color{blue}Importance of the secondments to the partner companies for the training purposes of the network\color{black}}
%\textbf{Text from last year's application: why the secondments to these companies are good} 
% Add drawing and explanation of the intersectoral secondments,but also take the broader look of all  intersectoral exposure and link to the solutions as for example mentorship by industry. 
% Consider the consequences for the Individual Research Projects. 
% Consider the consequences for the Gantt chart


%ESRs will be trained during the network-wide events
%and in addition receive one-to-one mentoring and supervision specific to their projects,
%as described in Tab.~\ref{tab:LocalTraining}. 
%All ESRs will work on an industrial project relevant
%to their research and deliverables during these secondments. 
%The precise topics will be defined in the PCDPs,
%to give the ESRs some freedom to follow their interests. %Some example topics are
%\vspace{-3mm}
%\begin{multicols}{2}[]
%\begin{enumerate}{\leftmargin=1em}
%    \item{Time analysis of a sequence of speech signals %(\dq);}%\vspace{-2mm}
%    \item{Combined speech/image signals analysis %(\dq);}%\vspace{-2mm}
%\end{enumerate} 
%\end{multicols}
%\vspace{-3mm}


%Below: TBC, but it looks like it's in
%\technopolis will play a key role in this area because
%of their unparalleled experience with training scientists to address policy makers and commercialize
%the results of their research.
%We are particularly aware of the very different ways that job applications work inside
%and outside academia, and this will be addressed directly, both in the lectures and as part of the
%final industry school which all ESRs will attend.

%Examples are: preparation for recruitment and interviews, management, and transitioning
%from academia to industry. 


%% HEADER PLEASE READ!!!!!!!!
%% HEADER PLEASE READ!!!!!!!!
%% HEADER PLEASE READ!!!!!!!!
%% HEADER PLEASE READ!!!!!!!!
%% HEADER PLEASE READ!!!!!!!!
%% ALL NODES TO FILL IN THE TWO TABLES AT THE BOTTOM OF THIS TEX FILE DEFINING THE TRAINING
%% HEADER PLEASE READ!!!!!!!!
%% HEADER PLEASE READ!!!!!!!!
%% HEADER PLEASE READ!!!!!!!!
%% HEADER PLEASE READ!!!!!!!!
%% HEADER PLEASE READ!!!!!!!!
%
%%%From instructions
%%4.1 Research and Training Activities
%%Applicants will primarily propose a dedicated and high-level joint research training program that focuses on promoting scientific excellence and exploiting the specific research expertise and infrastructure of the beneficiaries and of the collective expertise of the network as a whole. These training programs will address in particular the development and broadening of the research competences of the ESRs. Such training activities might include:
%% Training through research by means of individual, personalised projects, including meaningful exposure to different sectors;
%% Development of network-wide training activities (e.g. workshops, summer schools) that exploit the inter/multi-disciplinary and intersectoral aspects of the project and expose the researchers to different schools of thought. Such events could also be open to external researchers. For doctoral programs (i.e. EID and EJD), the broad structure of the curriculum should be outlined and preferably quantifiable by ECTS points;
%% Provision of structured training courses (e.g. tutorials, lectures) that are available either locally or at another participant. Training programs between the participants are expected to be coordinated to maximise added value (e.g. joint syllabus development, opening up of local training to other network teams, joint PhD programs, etc.);
%% Exchanging knowledge with the members of the network through undertaking intersectoral visits and secondments. A strong networking component is expected in each proposal;
%% Invitation of visiting researchers originating from the academic or non- academic sector. This would be aimed at improving the skills and know-how of the researchers and should be duly justified in the context of the training program. The network can cover costs of visiting researchers under the Research, Training and Networking cost category.
%%Further training activities with a particular view to widening the career prospects of the researchers would include transferable skills training both within and outside the network. Topics of interest could include:
%%Marie Skłodowska-Curie Actions, Guide for Applicants Innovative Training Networks 2016
%%Page 15 of 48
%% Training related to research and innovation: management of IPR, take up and exploitation of research results, communication, standardisation, ethics, scientific writing, personal development, team skills, multicultural awareness, gender issues, research integrity, etc.
%% Training related to management or grant searching: involvement in the organisation of network activities, entrepreneurship, management, proposal writing, enterprise start-up, task co-ordination, etc.
%%Each researcher recruited for a period of more than 6 months will establish, together with her/his personal supervisor(s) in the host organisation/s, a personal Career Development Plan. This plan shall aid in the provision of the research training program that best suits the researcher's needs. Attention should be paid to the quality of the joint research training program, with provision for supervision and mentoring arrangements and career guidance. Furthermore, the meaningful exposure of each researcher to other disciplines and sectors represented in the network through visits, secondments and other training events shall also be ensured.
%%Although mutual recognition is mandatory only for EJD, it is expected that both beneficiaries and partner organisations will mutually recognise the quality of the research and training and, if possible, of diplomas and other certificates awarded. The size of the joint research training program and of the network will depend on the nature and scope of the training activities to be undertaken by the network, as well as on considerations regarding management and effective interaction among the partners.
%

