The training program of \acronym has three main {\bf objectives}, taking the Salzburg Principles\footnote{Salzburg II recommendations, \url{http://www.eua.be/Libraries/publications-homepage-list/Salzburg_II_Recommendations}} as a guidance: 
\begin{itemize}
\item {\bf Objective A}: provide ESRs with knowledge and training to conduct original research during and beyond their PhD studies;
\item {\bf Objective B}: provide solid bases in a broad spectrum of topics related to their field of research, extending beyond their dedicated research and including soft skills; 
\item {\bf Objective C}: provide up-to-date and career-related training through interactions with multiple collaborators within academia as well as the industry, to meet the needs of a broad employment market.
\end{itemize}

%MLD what does 'career-focused aspects' supposed to mean?
Objectives A and B will be fulfilled by schools, workshops and events organized within the network; ESRs will profit from high quality lectures delivered by experts within the institutes associated to \acronym, complemented by doctoral training at local nodes. Objective C will be fulfilled by mentoring by senior scientists and special network events where the industrial collaborators will provide seminars and training on commercial applications of \acronym's objectives. All these objectives will be further addressed via the natural collaboration between the ESRs and the various institutes of the network, as well as the planned secondments in academia and industry. This is a more flexible and diverse structure compared to standard PhD studies and will allow the ESRs to develop an individual and unique research mindset within an inclusive environment, so that they can act as proactive participants in furthering HEP and industry goals through RTA techniques.

\vspace{-2mm}
\subsubsection{Overview and content structure of the training}
\label{sub:overviewTraining}
%(including network-wide training events and complementarity with those programs offered locally at the participating organisations (please include table 1.2a and table 1.2b)

%MLD be consistent with the use of names, some times the responsible is named, other times not
Training is at the heart of all activities within \acronym, with a dedicated Work Package, WP2. ESRs will have clear recruitment deliverables, a Personal Career Development Plan (PCDP), and they will be able to attend complementary network-wide and local training accounted through a credit system designed for this ETN. The recruiting deliverables, the composition of the Supervisory Committee (SC) for each ESR and the expertise and qualification of supervisors is shown in Tab. 1.2.

\begin{center}\scriptsize
\begin{tabular}{|p{0.05\textwidth}|p{0.07\textwidth}|p{0.1\textwidth}|p{0.06\textwidth}|p{0.05\textwidth}|p{0.45\textwidth}|}
\hline
\pbox{8cm}{\textbf{ESR}} & 
\pbox{8cm}{\Tstrut \textbf{Recruiting} \\ \textbf{node} \Bstrut} &  
\pbox{8cm}{\Tstrut \textbf{PhD-awarding} \\ \textbf{node} \Bstrut} &  
\pbox{8cm}{\Tstrut \textbf{Planned} \\ \textbf{start} \Bstrut} &  
\pbox{8cm}{\Tstrut \textbf{Duration}} & 
\pbox{8cm}{\Tstrut \textbf{Title}} 
\tabularnewline 
\hline
%%%
%Only PhDs now
\textbf{\ESRa} & \helsinkientity & \helsinkientity & 8 & 36 & ML and RTA for Higgs boson measurements and industry \tabularnewline \hline  
\textbf{SC} & \multicolumn{5}{p{0.9\textwidth}|}{
\textbf{Main supervisor: Voutilainen} (\helsinkientity), [2]. Assistant prof. Expertise: RTA, SM and Higgs measurements, with positions of responsibility in CMS. Grant from the Finnish Academy of Science, prizes for research during PhD and postdoc. } \tabularnewline %[8/3/2] in particle physics (HEP)
 & \multicolumn{5}{p{0.9\textwidth}|}{\textbf{Second supervisor: Pierini} (\cernentity) [6]. Staff scientist. Expertise: RTA, ML for HEP. ERC CoG for ML at the LHC.}\tabularnewline %[Al momento ho 3 undegraduate, 1 graduate e 3 fellow In piu?, ogni anno prendo 5-10 intern (summer student e similar) e co-supervisiono studenti PhD in CMS (al momento due di Caltech)]
 & \multicolumn{5}{p{0.9\textwidth}|}{\textbf{Industrial supervisor: Taccari} (\fleetmaticsentity). Data scientist. Expertise: mathematical optimization and ML. }\tabularnewline 
 & \multicolumn{5}{p{0.9\textwidth}|}{\textbf{Supervising postdoc: Kirschenmann} (\helsinkientity). Expertise: LHC jet physics and searches, with positions of responsibility in CMS. } \tabularnewline
 & \multicolumn{5}{p{0.9\textwidth}|}{\textbf{Senior mentor: Eerola} (\cernentity). Professor (HEP), Vice Rector of the University of Helsinki.}\tabularnewline %[19]
  \hline \hline
%%%
%%%
\textbf{\ESRb} & \unigeentity & \unigeentity & 8 & 36 & ML pattern recognition for exotic physics and industry \tabularnewline \hline 
\textbf{SC} & \multicolumn{5}{p{0.9\textwidth}|}{
\textbf{Main supervisor: Sfyrla} (\unigeentity), [2]. Assistant prof. (HEP). Expertise: SUSY DM and trigger expert, responsibility positions in ATLAS in trigger and HL-LHC upgrade. Grant from the Swiss National Foundation.} \tabularnewline %[3/2]
 & \multicolumn{5}{p{0.9\textwidth}|}{\textbf{Second supervisor: Martinez-Santos} (\santiagoentity) [6]. Assistant prof. (HEP). Expertise: rare decays, trigger and detectors. ERC StG for BSM searches. }\tabularnewline 
 & \multicolumn{5}{p{0.9\textwidth}|}{\textbf{Industrial supervisor: Catastini} (\lightboxentity). Quantitative Analyst. Expertise: analysis of financial markets, automated digital advertising trading.}\tabularnewline 
 & \multicolumn{5}{p{0.9\textwidth}|}{\textbf{Supervising postdoc: Schramm} (\unigeentity). Expertise: ML and new physics, convener of LHC interexperiment ML group until 2018, Banting Fellow.} \tabularnewline
 & \multicolumn{5}{p{0.9\textwidth}|}{\textbf{Senior mentor: Iacobucci} (\unigeentity). Director of the DPNC at \unigeentity. }\tabularnewline \hline \hline
%%%
%%%
\textbf{\ESRc} & \cernentity & \unigeentity & 8 & 36 &Efficient RTA in ATLAS using multi-threaded processing \tabularnewline \hline 
\textbf{SC} & \multicolumn{5}{p{0.9\textwidth}|}{
\textbf{Main supervisors: Petersen}(\cernentity), [2]. Staff scientist. SUSY DM and trigger expert, past ATLAS trigger convener, current upgrade physics convener; \textbf{Sfyrla} (\unigeentity), see \ESRb }\tabularnewline %[8/3/2]
 & \multicolumn{5}{p{0.9\textwidth}|}{\textbf{Second supervisor: Crescioli} (\cnrsentity) [2]. Research engineer. FTK and FPGA expert, technical coordinator of many R\&D projects;}\tabularnewline 
 & \multicolumn{5}{p{0.9\textwidth}|}{\textbf{Industrial supervisor: Catastini} (\lightboxentity), see \ESRb. }\tabularnewline   
 & \multicolumn{5}{p{0.9\textwidth}|}{\textbf{Senior mentor: Monica Pepe-Altarelli} (\cernentity). Staff scientist. Vice-President Elected at Large of the Executive Council of \href{http://iupap.org/}{IUPAP}}\tabularnewline \hline \hline
 %& \multicolumn{5}{p{0.9\textwidth}|}{\textbf{Supervising postdoc: E (if available)} (place). Expertise: [Text]} \tabularnewline \hline \hline
%%%
%%%
\textbf{\ESRd} & \dortmundentity & \dortmundentity & 8 & 36 &Real-time ML for LFV in unflavoured meson decays \tabularnewline \hline %Eerola [19], 
\textbf{SC} & \multicolumn{5}{p{0.9\textwidth}|}{
\textbf{Main supervisor: Albrecht}(\dortmundentity), [8]. Assistant prof. and Emmy Noether group leader. Expertise: LFV/LFU, tracking, ML and trigger expert, deputy Physics Coordinator of LHCb. ERC StG 2016 on LFV/LFU.  } \tabularnewline %8/18
 & \multicolumn{5}{p{0.9\textwidth}|}{\textbf{Second supervisor: Martinez-Santos} (\santiagoentity [6], see \ESRb.}\tabularnewline 
 & \multicolumn{5}{p{0.9\textwidth}|}{\textbf{Industrial supervisor: Sopasakis} (\ximantisentity). Mathematics and ML/AI expert, CEO and \lundentity associate professor, start-up experience.}\tabularnewline 
 & \multicolumn{5}{p{0.9\textwidth}|}{\textbf{Supervising postdoc: Matev} (\cernentity). Expertise: software and trigger design and maintenance.} \tabularnewline 
 & \multicolumn{5}{p{0.9\textwidth}|}{\textbf{Senior mentor: Spaan} (\dortmundentity). Head of experimental physics 5, team leader for the LHCb experiment, project leader of CS research area SFB876.}\tabularnewline \hline \hline
%%%
%%%
\textbf{\ESRe} & \dortmundentity & \dortmundentity & 8 & 36 & Global event triggering in LHCb \tabularnewline \hline %Eerola [19], 
\textbf{SC} & \multicolumn{5}{p{0.9\textwidth}|}{
\textbf{Main supervisor: Albrecht}(\dortmundentity), [13]. Assistant prof. See \ESRd.} \tabularnewline
 & \multicolumn{5}{p{0.9\textwidth}|}{\textbf{Second supervisor: Raven} (\nikhefentity) [14]. Professor. expert in LFV/LFU and triggering in LHCb with positions of responsibility, receiver of several national grants. }\tabularnewline 
 & \multicolumn{5}{p{0.9\textwidth}|}{\textbf{Industrial supervisor: Dungs} (\pointeightentity). Staff. Expertise: transitioned from LHCb trigger group to data science (via Google). }\tabularnewline 
% & \multicolumn{5}{p{0.9\textwidth}|}{\textbf{Additional supervision: Pearce} (\cernentity). Expertise: [Text]} \tabularnewline \hline \hline
 & \multicolumn{5}{p{0.9\textwidth}|}{\textbf{Senior mentor: Spaan} (\dortmundentity), see \ESRd. }\tabularnewline \hline \hline
%%%
%%%
\textbf{\ESRf} & \cnrsentity & \sorbonneentity & 8 & 36 & Real-time trajectory reconstruction in ATLAS \tabularnewline \hline %Eerola [19], 
\textbf{SC} & \multicolumn{5}{p{0.9\textwidth}|}{
\textbf{Main supervisors: Crescioli} (\cnrsentity and \sorbonneentity), see \ESRc \newline
\textbf{Malaescu}(\cnrsentity and \sorbonneentity), [4], Assistant prof. Expertise: jet measurements and searches, ATLAS SM group convener, past Statistics Forum convener. 
}\tabularnewline 
 & \multicolumn{5}{p{0.9\textwidth}|}{\textbf{Second supervisor: Roda} (\pisaentity) [14]. Expert in searches beyond the SM, ATLAS Pisa group leader, coordination roles in ATLAS.}\tabularnewline 
 & \multicolumn{5}{p{0.9\textwidth}|}{\textbf{Industrial supervisor: Sambo} (\fleetmaticsentity). Senior Data Scientist. Expertise: AI, ML, connected vehicles. }\tabularnewline 
 & \multicolumn{5}{p{0.9\textwidth}|}{\textbf{Additional supervision: Annovi} (\pisaentity). Expertise: tracking and track triggers for ATLAS} \tabularnewline 
 & \multicolumn{5}{p{0.9\textwidth}|}{\textbf{Senior mentor: Calderini} (\cnrsentity), Team leader, AIDA2020 management. Expertise: detectors, SM in ATLAS}\tabularnewline \hline \hline
%%%
%%%
\textbf{\ESRg} & \sorbonneentity & \sorbonneentity & 8 & 36 & Real-time trajectory reconstruction in ATLAS \tabularnewline \hline %Eerola [19], 
\textbf{SC} & \multicolumn{5}{p{0.9\textwidth}|}{
\textbf{Main supervisor: Lacassagne} (\sorbonneentity), [15], Professor. Hybrid architectures expert, leader of LIP6 Hardware and Software for Embedded Systems team. }\tabularnewline 
 & \multicolumn{5}{p{0.9\textwidth}|}{\textbf{Second supervisor: Petersen}, see \ESRc }\tabularnewline 
 & \multicolumn{5}{p{0.9\textwidth}|}{\textbf{Industrial supervisor: Borri} (\lightboxentity). Staff. Expertise in quantitative analysis and automated financial market trading expert.}\tabularnewline 
 & \multicolumn{5}{p{0.9\textwidth}|}{\textbf{Additional supervision: Couturier} (\cernentity). Expertise: software architect for various companies, now core team of LHCb software framework} \tabularnewline \hline \hline
%%%
%%%
\textbf{\ESRj} & \ibmentity & \lundentity & 8 & 36 &Novelty detection for industry and ATLAS searches \tabularnewline \hline %Eerola [19], 
\textbf{SC} & \multicolumn{5}{p{0.9\textwidth}|}{
\textbf{Main supervisors:  Julli\'{e}} (\ibmentity), Software Engineer. Expertise: modelling optimization problems, ML, anomaly detection. \newline
\textbf{Doglioni} (\lundentity), [6], Assistant prof. Expertise: jet, trigger and Dark Sectors expert in ATLAS, LHC Dark Matter WG organizer and HSF trigger and reconstruction WG convenor, ERC StG 2015 on RTA for ATLAS, national grants. }\tabularnewline 
 & \multicolumn{5}{p{0.9\textwidth}|}{\textbf{Second supervisor: Boveia} (\ohioentity). Expertise: Dark Sectors, trigger and track trigger ATLAS, responsible for first fully RTA-based search in ATLAS. } \tabularnewline 
 & \multicolumn{5}{p{0.9\textwidth}|}{\textbf{Additional supervision: Shaw} (\ibmentity). Expertise: constraint programming, optimization modelling and local search.} \tabularnewline 
 & \multicolumn{5}{p{0.9\textwidth}|}{\textbf{Senior mentor: Akesson} (\lundentity). Professor. Expert in searches at ATLAS. Member of the Nobel Committee and ex-CERN Council president. }\tabularnewline \hline \hline
 
%%%
%%%
 \multicolumn{6}{p{0.95\textwidth}}{
\footnotesize 
\vskip2pt
Table 1.2: Recruitment deliverables per beneficiary, Supervisory Committee (SC) with number of supervised PhD students and postdocs, and expertise/qualification of supervisors. 
\vskip2pt
\normalsize
}
\end{tabular}
\end{center}

\newpage


\begin{center}\scriptsize
\begin{tabular}{|p{0.05\textwidth}|p{0.07\textwidth}|p{0.1\textwidth}|p{0.06\textwidth}|p{0.05\textwidth}|p{0.45\textwidth}|}
\hline
\pbox{8cm}{\textbf{ESR}} & 
\pbox{8cm}{\Tstrut \textbf{Recruiting} \\ \textbf{node} \Bstrut} &  
\pbox{8cm}{\Tstrut \textbf{PhD-awarding} \\ \textbf{node} \Bstrut} &  
\pbox{8cm}{\Tstrut \textbf{Planned} \\ \textbf{start} \Bstrut} &  
\pbox{8cm}{\Tstrut \textbf{Duration}} & 
\pbox{8cm}{\Tstrut \textbf{Title}} 
\tabularnewline 
\hline
%\ESRx & \ibmentity  & \ & 8 & 36 & \textit{Feillet}, Lacassagne [15/21] & Gligorov [3/2]  \tabularnewline \hline
\textbf{\ESRx} & \ibmentity & \sorbonneentity & 8 & 36 & Real-time rule induction in fraud detection and HEP \tabularnewline \hline %Eerola [19], 
\textbf{SC} & \multicolumn{5}{p{0.9\textwidth}|}{
\textbf{Main supervisor:  Feillet} (\ibmentity), [N],[Position]. Expertise: [Text]. [Qualifications and grants] \newline
\textbf{Lacassagne} (\sorbonneentity), [15], See \ESRg.}\tabularnewline 
 & \multicolumn{5}{p{0.9\textwidth}|}{\textbf{Second supervisor: Gligorov} (\cnrsentity) [3]. Staff scientist. Expertise: ML, LFV/LFU, trigger, past LHCb Deputy Physics Coordinator, Real-time analysis project coordinator. ERC CoG 2016 on RTA and LFV/LFU in LHCb. }\tabularnewline 
 & \multicolumn{5}{p{0.9\textwidth}|}{\textbf{Additional supervision: Louppe} (\liegesentity) [3]. Assistant prof. Expertise: artificial intelligent and deep learning, Analysis Consultant Expert with ATLAS. }\tabularnewline \hline \hline
% & \multicolumn{5}{p{0.9\textwidth}|}{\textbf{Additional supervision: Louppe} (place) [N]. [Position]. Expertise: [Text]. [Qualifications and grants] }\tabularnewline 
%%%
%%%
\textbf{\ESRk} & \lundentity & \lundentity & 8 & 36 &Real-time calibration and analysis of the ALICE Time Projection Chambers (TPC) \tabularnewline \hline %Eerola [19], 
\textbf{SC} & \multicolumn{5}{p{0.9\textwidth}|}{
\textbf{Main supervisor: Christiansen} (\lundentity), [17], Professor. Expertise: ALICE TPC, real-time detector reconstruction and analysis in ALICE, receiver of several national grants. }\tabularnewline 
 & \multicolumn{5}{p{0.9\textwidth}|}{\textbf{Second supervisor: Shahoyan} (\cernentity) [2]. Staff scientist. Expertise: expert in calibration and data analysis in ALICE, main developer of real-time reconstruction project. }\tabularnewline 
 & \multicolumn{5}{p{0.9\textwidth}|}{\textbf{Industrial supervisor: Sopasakis} (\ximantisentity). See \ESRd. }\tabularnewline \hline \hline
 %& \multicolumn{5}{p{0.9\textwidth}|}{\textbf{Additional supervision: Shahoyan} (\cernentity). Expertise: [Text]} \tabularnewline 
\hline \hline
%%%
%%%
\textbf{\ESRh} & \nikhefentity & \radboudentity & 8 & 36 &Optimization of RTA resources and ATLAS LFV search \tabularnewline \hline %Eerola [19], 
\textbf{SC} & \multicolumn{5}{p{0.9\textwidth}|}{
\textbf{Main supervisor: Igonkina} (\nikhefentity and \radboudentity), [14], Professor. Expert in LFV/LFU and triggering in ATLAS with positions of responsibility, receiver of several national grants.}\tabularnewline 
 & \multicolumn{5}{p{0.9\textwidth}|}{\textbf{Second supervisor: Strom} (\oregonentity) [6]. Professor. Expert in trigger and data acquisition in ATLAS, responsible of all aspect of trigger and data acquisition (2017-2018) and FTK (2018-).}\tabularnewline 
 & \multicolumn{5}{p{0.9\textwidth}|}{\textbf{Industrial supervisor: Sopasakis} (\ximantisentity). See \ESRd. }\tabularnewline \hline \hline
% & \multicolumn{5}{p{0.9\textwidth}|}{\textbf{Additional supervision: } (place). Expertise: [Text]} \tabularnewline 
\hline \hline
%%%
%%%
\textbf{\ESRi} & \nikhefentity & \amsterdamentity & 8 & 36 &Optimization of RTA resources and LHCb LFV search \tabularnewline \hline %Eerola [19], 
\textbf{SC} & \multicolumn{5}{p{0.9\textwidth}|}{
\textbf{Main supervisor: Raven} (\nikhefentity and \amsterdamentity), [14]. Professor. See \ESRe. }\tabularnewline 
 & \multicolumn{5}{p{0.9\textwidth}|}{\textbf{Second supervisor: Albrecht} (\dortmundentity) [13]. Assistant prof. See \ESRd. }\tabularnewline 
 & \multicolumn{5}{p{0.9\textwidth}|}{\textbf{Industrial supervisor: Brambach} (\pointeightentity). Staff. Expertise: data science, sales and project management, PhD in LHCb. }\tabularnewline 
 & \multicolumn{5}{p{0.9\textwidth}|}{\textbf{Additional supervision: Petersen, Couturier} {\cernentity}. See \ESRc and \ESRg.} \tabularnewline \hline \hline
\hline \hline
%%%
%%%
\textbf{\ESRn} & \heidelbergentity & \heidelbergentity & 8 & 36 & RTA to search for Dark Photons in LHCb \tabularnewline \hline %Eerola [19], 
\textbf{SC} & \multicolumn{5}{p{0.9\textwidth}|}{
\textbf{Main supervisor: Hansmann-Menzemer } (\heidelbergentity), [15]. Professor. Expertise: tracking algorithms and software in LHCb. Co-spokesperson of research training "Particle Physics beyond the SM" (DFG). EPS Young Researcher award (2007), ERC StG 2010}\tabularnewline 
 & \multicolumn{5}{p{0.9\textwidth}|}{\textbf{Second supervisor: Albrecht} (\dortmundentity) [13]. Assistant prof. See \ESRd. }\tabularnewline 
 & \multicolumn{5}{p{0.9\textwidth}|}{\textbf{Industrial supervisor: Dungs} (\pointeightentity). See \ESRe. }\tabularnewline 
 & \multicolumn{5}{p{0.9\textwidth}|}{\textbf{Additional supervision: Borsato} (\heidelbergentity). Expertise: Dark sectors, trigger and software for LHCb. \textbf{Martinez-Santos} (\santiagoentity). See \ESRd. } \tabularnewline \hline \hline
%%%
%%%
\textbf{\ESRl} & \heidelbergentity & \heidelbergentity & 8 & 36 & Real-time noise reduction new physics searches \tabularnewline \hline %Eerola [19], 
\textbf{SC} & \multicolumn{5}{p{0.9\textwidth}|}{
\textbf{Main supervisors: Dunford} (\heidelbergentity), [8], Young Researcher group leader. Expertise: Dark Matter, SM measurements, detector and trigger expert in ATLAS, editor of first trigger-level analysis paper in ATLAS.\newline
\textbf{Starovoitov} (\heidelbergentity), [7], Postdoc. Expert in ATLAS detector, trigger and data analysis, SM measurements and searches, receiver of several international fellowships.}\tabularnewline 
 & \multicolumn{5}{p{0.9\textwidth}|}{\textbf{Second supervisor: Strom} (\oregonentity) [6]. Professor. See \ESRh. }\tabularnewline 
 & \multicolumn{5}{p{0.9\textwidth}|}{\textbf{Industrial supervisor: Sopasakis} (\ximantisentity). See \ESRd. }\tabularnewline 
 & \multicolumn{5}{p{0.9\textwidth}|}{\textbf{Additional supervision: Boveia} (\ohioentity). See \ESRj.} \tabularnewline 
 & \multicolumn{5}{p{0.9\textwidth}|}{\textbf{Senior mentor: Hansmann-Menzemer} (\heidelbergentity). See \ESRn. }\tabularnewline \hline \hline
\hline \hline
%%%
%%%

%& \textit{Sambo}, Di Stefano [24/NN], & Salti [0/13], Lacassagne [15/21], Doglioni [6/6]  \tabularnewline \hline
\textbf{\ESRm} & \fleetmaticsentity & \uniboentity & 8 & 36 & RTA through computer vision on dashcams \tabularnewline \hline %Eerola [19], 
\textbf{SC} & \multicolumn{5}{p{0.9\textwidth}|}{
\textbf{Main supervisors: Sambo} (\fleetmaticsentity), See \ESRf.}\tabularnewline 
 & \multicolumn{5}{p{0.9\textwidth}|}{\textbf{Di Stefano} (\uniboentity) [11]. Professor. Expertise: Computer Architecture, Computer Vision and Image Processing. }\tabularnewline 
 & \multicolumn{5}{p{0.9\textwidth}|}{\textbf{Second supervisor: Lacassagne} (\sorbonneentity) [15]. See \ESRg.}\tabularnewline
 & \multicolumn{5}{p{0.9\textwidth}|}{\textbf{Additional supervision: Salti} (\uniboentity). Expertise: computing engineering, computer vision. \textbf{Doglioni} (\lundentity). See \ESRj.\newline
} \tabularnewline 
\hline \hline
%%%
%%%
 %%%
 \multicolumn{6}{p{0.95\textwidth}}{
\footnotesize 
\vskip2pt
Table 1.2: Recruitment deliverables per beneficiary, Supervisory Committee (SC) with number of supervised PhD students and postdocs and expertise/qualification of supervisors. 
\vskip2pt
\normalsize
}
\end{tabular}
\end{center}



\noindent \color{blue}Recruitment deliverables per participant and awarding of PhD degrees. \color{black}
ESRs will be recruited by month 8, complete a total of 36 months of research and training, and be awarded a PhD degree. 
ESRs at non-academic beneficiaries and ESRs at international or national laboratories will be awarded PhDs by universities within the network and have supervisors in that university as well. 
Each ESR will have a Supervisory Committee (SC) composed of participants from the local and secondment academic and industrial nodes. 
For Finland, Netherlands and Sweden which mandate a four-year PhD, all beneficiaries will provide support for the ESRs to complete their PhD thesis.
%we include a letter of commitment from \nikhef and \lund guaranteeing funding for the final year of the ESR's studies.

%%%PCDP 
\noindent \color{blue}Personal Career Development Plan. \color{black}
Two months after the start of their projects, ESRs and their SC will present a \textbf{Personal Career Development Plan} (PCDP) so that the \textbf{core, advanced and transferrable skills} to be acquired, as well as the milestones for each of the ESR projects, can be agreed between student and supervisor and consortium, planned and monitored throughout the course of the program, taking into account the existing resources both at the ESR node and at foreseen secondments. The PCDP will be reviewed by the local node coordinator, who will bring it to the Supervisory Board (SB) for approval. The main supervisor will be available throughout the course of the PhD, and meet with the ESR on a weekly basis and with an open-door policy. 
%this is written
The local node coordinator will also review the progress described in the PCDP at least every six months (e.g. during staff appraisal meetings) and bring a short report to the consortium. 
%ESRs are expected to participate in one or two network-organized schools throughout
%their PhD, corresponding to up to 3 ECTS each, and one external summer or winter school. 
The PCDP\footnote{The \acronym PCDP will follow a common template based on that from the MSCA website to be coherent between the ESRs.} will include the requirements, milestones and goals within the schedule of the doctoral program (including secondments), the local and network-wide courses and schools to be attended, and a list of dissemination, communication and outreach activities. 
%	requirements and goals of the planned training for the ESR
%	A list of courses (local and network-wide) to be taken by the ESR during their program, including any ECTS credit requirements
%	A list of communication and dissemination activities to be undertaken by the ESR
%	A schedule for their program, including secondments

%Core Research Skills (acquired via their ESR project)
%Advanced/Additional Research Skills (delivered by the consortium)
%Transferable Skills (delivered by the consortium - particularly those useful in non-academic careers)   

%The doctoral program will have a set of compulsory modules but also some degree of freedom for the 
%ESRs' preferences. ESRs are expected to complete 1 or 2 
%secondments, considered as a part of their training. 

\noindent \color{blue}Network-wide events: schools, yearly meetings and schools. \color{black}
%\begin{wraptable}{l}{0.45\textwidth}
%	%\vspace{-2mm}
%	\caption{\acronym yearly meetings.\label{tab:YearlyMeeting}}
%    \vspace{4mm}
%	%\begin{center}%
%	\small
%	\begin{tabular}{m{75mm}}
%		\midrule 
%		\textbf{Days 1--2}\tabularnewline 
%		{\begin{itemize}%[topsep=2pt,noitemsep,listparindent=3pt,leftmargin=*]
%				\item Presentations and poster session by the ESRs.
%				\item Meeting of the Executive Board and \linebreak
%				preparation of Supervision Board meeting.
%                \vspace{-5mm}
%			\end{itemize}}
%            \tabularnewline
%            \midrule 
%			\textbf{Days 3--4/5}\tabularnewline 
%			\begin{itemize}%[topsep=2pt,noitemsep,listparindent=3pt,leftmargin=*]
%				\item Transferable/research/technical skills lectures.
%				\item Outreach activities. 
%				\item Supervisory Board meeting. \vspace{-4mm}
%			\end{itemize} \tabularnewline \midrule
%		\end{tabular}
%		%\end{center}
%		\vspace{-3mm}
%	\end{wraptable}
Network-wide schools, conferences and events shown in the table below will be organized by \acronym beneficiaries and partners as part of the training and dissemination program and its preparation. 
We expect ESRs to attend network events in person, but wherever possible will make network school, conferences and events available as Webinars using the \href{http://information-technology.web.cern.ch/services/fe/vidyo}{Vidyo technology provided by \cernentity}, to allow all \acronym ESRs and PIs to attend if family/personal commitments would otherwise prevent it. 
A permanent record of the lectures will be available as proceedings, and in some cases video recordings, as \acronym has the ambition to make the training program available beyond this Action and continue organizing successful schools. 
The table below summarizes all events included within the network, with compulsory schools marked in bold so that students attend a yearly meeting and a school each year, dedicating sufficient time to local training and research. 
The hosts and lecturers for these events schools have been identified within the network based on their expertise, see Sec. B1.4.1. 

%\FloatBarrier
%\begin{table}[!htb]
%\centering
\begin{center}
\scriptsize
%\resizebox {\textwidth }{!}{%
%\begin{tabular}{@{}p{5mm}p{40mm}p{25mm}p{22mm}p{22mm}p{12mm}p{12mm}p{12mm}}

			\begin{tabular}{@{}|c|p{45mm}|p{7mm}|p{30mm}|p{15mm}|p{45mm}|@{}}
				\hline
				\multicolumn{2}{|p{4cm}|}{\pbox{8cm}{\color{blue}{Main training events and conferences}}} & 
				\pbox{8cm}{\color{blue}{Credits}} &%http://ec.europa.eu/education/ects/users-guide/docs/ects-users-guide_en.pdf 
				\pbox{8cm}{\color{blue}{Lead (support) institution}} & 
				\pbox{8cm}{\color{blue}{Action month}} &
				\pbox{8cm}{\color{blue}{Notes}} 
				\tabularnewline 
				\hline
				\hline
				%\toprule
				%2 = october 2019
				\cellcolor{red!70!black}1. & \textbf{Kick-off meeting} & - & \lundentity & 2 & - \tabularnewline\hline
				
				\multicolumn{6}{|p{0.975\textwidth}|}{
The kick-off meeting  will be dedicated to organizing the project management, signing the consortium agreement, and monitoring the ESR recruitment.
			    } \tabularnewline \hline %\midrule

				%%%
				%10 = june 2020
				\cellcolor{red}2. & \textbf{Yearly conferences} & 2 & \lundentity (\nikhefentity, \ibmentity) & 9,23,36 & - \tabularnewline \hline
				
			 	\multicolumn{6}{|p{0.975\textwidth}|}{
				
\acronym will hold \textbf{yearly in-person network-internal conferences}, with a duration of 4 or 5 days. 
On the first two days, the ESRs will gain experience in presenting their research by giving presentations to the other \acronym members and participating in a poster session.
Days 3--5 will be dedicated to lectures on research, technical and transferable skills, open discussions of synergies between industrial participants, as well as communication and dissemination activities tailored to local circumstances (see Sec.~\ref{sec:CommPub}). 
The first yearly meeting on month 8 will serve as introductory school on HEP and RTA, as it will be hosted shortly after all ESRs have been recruited. 
All yearly meetings will include a dedicated half-day of lectures on non-academic training. 
Within the Yearly Meetings, there will be time devoted to management activities such as the Executive Board, ESR Board and Supervisory Board meetings (see Sec.~\ref{sub:networkOrganization}), 
that also train ESRs in scientific collaboration and governance. 
%Example topics that have been agreed upon are "Optimizing workspaces for productivity" (\dqentity); 
%"Writing software in collaborative environments" (\wildtreeentity), "Gender and inclusion" (\lundentity),
%"Innovation and entrepreneurship, including IPR" (\lundentity, in a one-day workshop similar to the successful
%\href{http://indico.hep.lu.se/conferenceDisplay.py?confId=697}{Innovation and Entrepreneurship for PhD Students} event).
			    } \tabularnewline \hline %\midrule
			    
			    %july 2020			   / Oslo would do this in July 2019
			    \cellcolor{green} 5. & \textbf{Non-academic training workshop} & 1.5 & \lundentity  & 10 & Network event joint with INSIGHTS ETN, compulsory for ESRs \tabularnewline \hline
				\multicolumn{6}{|p{0.975\textwidth}|}{
This workshop will follow the first yearly meeting and be held in either the University of Oslo (UiO) or \lundentity. 
From preliminary discussions, the coordinator of the INSIGHTS ETN (where \lundentity is a beneficiary and C. Doglioni is a diversity and inclusion officer) and node responsible for this training have agreed to make it a joint event if \acronym is funded. 
In this workshop, the ESRs will receive non-academic training by experts at \lundentity and UiO, as well as INSIGHTS' partner KPMG. 
The topics of the lectures provided by \acronym will be diversity and inclusion, team-work, research integrity and sustainable research, as an early complement to local training on transferrable and soft skills so that the ESRs can start their work in a positive environment. 
The joint organization of \lundentity and UiO, both within the INSIGHTS ETN, will allow ESRs from two different ITNs to meet, exchange ideas and experiences, and broaden their network. 
			    } \tabularnewline \hline %\midrule	

%			    %9 =  2020
%			    \cellcolor{green} 3. & \textbf{Introductory school} & 3 & \nikhefentity & 9 & New network event, compulsory for ESRs \tabularnewline\hline%\midrule %when all recruitment is complete
%				
%			    \multicolumn{6}{|p{0.975\textwidth}|}{
%			    
%A first introduction to HEP and RTA will be provided, by experts from both HEP and industry. 
%This school will be an opportunity for all ESR members to get to know each other, as they are all recruited together. 
%They will also receive basic training in gender issues and research integrity. 
%This school will be hosted shortly after all ESRs have been recruited. 
%%We will also invite visiting scientists outside the network, and names will be agreed during the organization of each of the schools.   
%			    } \tabularnewline \hline %\midrule

				%september 2019
			    \cellcolor{orange} 4. & \textbf{Physics and machine learning school} & 3 & \unigeentity & 15 & New network event, compulsory for ESRs\tabularnewline\hline
				\multicolumn{6}{|p{0.975\textwidth}|}{
This school will provide all ESRs with more advanced courses on the physics topics tackled in the network, a few months after the ESRs have started their projects.  
It will also provide an introduction to how to design a physics analysis, as well as to machine learning concepts and their connections to HEP and industrial applications. 
			    } \tabularnewline \hline %\midrule

				%february 2021 or 2022				
				\cellcolor{orange} 5. & International School of Trigger And Data AcQuisition (ISOTDAQ) & 3 & \oregonentity & 18 or 30 & - \tabularnewline\hline
			    \multicolumn{6}{|p{0.975\textwidth}|}{
ISOTDAQ is a yearly school dedicated to triggering and acquiring data for physics experiments with lectures and hands-on exercises in equal proportions.
A lecture about RTA by \oregonentity members of the \acronym network will be added to the program if funded. 
ESRs will be encouraged to follow one of the two editions of the school during their PhD. 
			    } \tabularnewline \hline %\midrule			    
			    			    
				%december 2019				
				\cellcolor{orange} 5. & Machine Learning for HEP school (MLHEP) & 3 & \cernentity & 11 or 22& - \tabularnewline\hline
			    \multicolumn{6}{|p{0.975\textwidth}|}{
MLHEP is a school on cutting-edge machine learning techniques featuring dedicated trigger lectures, whose main organizer (Ustyuzhanin) is attached to CERN as a LHCb member. ESRs will be suggested to follow one of the two editions of the school during their PhD, and the nodes in \acronym will be encouraged to place a bid to host the school as \lundentity did in 2016. 
			    } \tabularnewline \hline %\midrule			    
			    
			    %june 2021
				\cellcolor{yellow} 6. & Basic FPGA course, FPGA boot-camp & 3 & \ohioentity, \cnrsentity & 26, 27 & New network event \tabularnewline\hline
				\multicolumn{6}{|p{0.975\textwidth}|}{				
This school will include lectures on technologies and architecture and hands-on exercises on triggering applications. 
This school will consist of introductory courses given at \cernentity, and a follow-up bootcamp in Paris, where practical problems are solved for ESRs specializing in this topic. 
Lectures will be given from researchers in \ohioentity, \cnrsentity with the help of \pisaentity, all leading institutes in research and development on hardware track triggers.				
			    } \tabularnewline \hline %\midrule	
			    
			    %%%%
				\cellcolor{yellow} 7. & GPU and hybrid architectures school & 3 & \santiagoentity (\sorbonneentity)  & 29 & New network event \tabularnewline \hline
				\multicolumn{6}{|p{0.975\textwidth}|}{								
In this school, the ESRs will learn how to compare architectures and practical solutions, and programming on different platforms (e.g. GPU programming). 
The ESRs will learn not only about what is available on the market and what is being planned, but also ways to best evaluate the chosen hardware solution for the software they are developing.
%Lectures will be given by \sorbonneentity. 
				} \tabularnewline \hline %\midrule				
				%7. & Intermediate conference & - & \dortmundentity  & 30 \tabularnewline \midrule
				
				%%%%
				
				\cellcolor{green} 8. & \textbf{Industry and career development school} & 3 & \dortmundentity (\ximantisentity)  & 38 & New network event \tabularnewline \hline
				\multicolumn{6}{|p{0.975\textwidth}|}{		
This school will include lectures and workshops in collaboration with industry, and it is optimally placed towards the end of the ESR's PhD training in month 44.
This school is dedicated to an in depth study of strategies for intellectual property rights (IPR), commercializing research output, presenting research results to policy-makers, and knowledge transfer.
One day of this school will be dedicated to group work on case studies prepared by the industrial beneficiaries and partners. 
This school will include experiences and Q\&A sessions with the CEOs and founders of the companies within \acronym. 
%This school will be organized at \dortmundentity, in collaboration with \ximantisentity. 
%An example of the lectures:
%\begin{itemize}
%\item {Description of the company;}
%\item {Experience from transitioning from physics to industry;}
%\item {The daily job within the industry (e.g. of a quant in finance or of a quant in a Big Data company) and why physicists are particularly good at that when it comes to real-life (or real-economy) applications.}
%\item {An example of the life-cycle of a basic commercial strategy: from the initial idea to real-time deployment}
%\item {Group work: analysis of a case study of big data applications to real economy}
%\end{itemize}
				} \tabularnewline \hline %\midrule				
				
				%%%%
				
				\cellcolor{cyan} 9. & \textbf{Final public conference} & 2 & \cnrsentity & 42 & Yes \tabularnewline \hline
				\multicolumn{6}{|p{0.975\textwidth}|}{					
We will hold a five-day conference which will showcase the work of the network to the wider scientific community. 
As opposed to the yearly meetings, the conference will not feature management meetings, and will be dedicated to presentations on \acronym research.
Invited topical presentations on state-of-the art developments within the HEP and industry.  
Each of the conferences will dedicate between half and one day to dissemination activities by \acronym members. 
The conference will have two additional days reserved for ESRs' presentations and public lectures showcasing the work done during their PhD program with \acronym. 
ESRs will also take active part in the organization of this conference, to add to their transferrable skillset. 
				} \tabularnewline \hline %\midrule
						
				\cellcolor{red!70!black} 10. & \textbf{Closing meeting} & - & \heidelbergentity & 48  \tabularnewline\midrule
				\multicolumn{6}{|p{0.975\textwidth}|}{					
In the closing meeting, the network will take stock of the experience of the ETN and plan the next steps. 
The closing meeting is beyond the doctoral period for most of the ESRs, but they will be invited to participate as network alumni, also to give feedback on the ETN experience. 
The PIs and the ESR alumni will give public lectures. 
				} \tabularnewline \hline %\midrule
			
				\multicolumn{6}{|p{0.975\textwidth}|}{
				Table 1.3: Network events and schools.  
				}		
	
			\end{tabular}

%}%end of resizebox
\end{center}
%\vspace{-5mm}
%\end{table}
%\FloatBarrier
%\vspace{-5mm}

%In addition to network-specific events, ESRs will be encouraged to attend the the Yandex School of Machine Learning for High Energy Physics (MLHEP)  and the International School of Trigger And Data AcQuisition, (ISOTDAQ), as part of their training plan. 
%The organizers of both schools have agreed that editions of this school from 2020 will feature a lecture on RTA taught by \acronym researchers if the network is funded. 

All network-organized schools will be also open to the local students of the beneficiary and partners organizing the school and advertised through the \href{HEP Software Foundation}{http://hepsoftwarefoundation.org} Training group, as part of enhancing the overall training program of the institutions involved in \acronym and broaden participation to the network's activities. 
The HEP Software Foundation will support training activities within \acronym, demonstrated in the Letter of Support at the end of this document. 
External students will be accepted if the optimal capacity of the school is not reached by the participating ESRs alone, and a fee will be charged to non-network participants if the node incurs external expenses due to their participation. 

%%Events of the network and yearly meetings
	
\noindent \color{blue}Local training. \color{black}
Training events organized specifically for \acronym are complemented by local training provided by beneficiary nodes. 
Students from the network will be able to participate in this training when located at the node. 
All nodes include a range of graduate level courses in languages, career management, presentational skills, diversity and inclusion, as  well as pedagogical courses in teaching and learning that the ESRs will be encouraged to follow to obtain the necessary amount of credits. 
Partners will also contribute with individual supervision and local training for employees while students are seconded at their premises. 
Links to the training courses for each node are detailed in their description, while below we list examples of graduate schools and training programs within the nodes that the ESRs will be associated to. 

%%Local training
%\vspace{-4mm}
%\begin{table}[h]
%\caption{Local, existing training across the network.\label{tab:LocalTraining}}
\begin{center}\scriptsize
\begin{tabular}{|p{0.200\textwidth}|p{0.575\textwidth}|}
\hline
\textbf{Beneficiary} & \textbf{Available graduate school or program}
\tabularnewline \hline
\lundentity & \href{http://cbbp.thep.lu.se/compute/Courses.php}{COMPUTE graduate school} in advanced computing techniques for research\\ % (Doglioni and Smirnova in steering group)\\
\cnrsentity & Add something here \\
\dortmundentity & Bi-yearly graduate schools in the special research area \textit{Providing Information by Resource-Constrained Data Analysis} (SFB876)\\
\heidelbergentity & \href{https://www.physik.uni-heidelberg.de/highrr/}{HighRR research training group} on detector development.\\
\helsinkientity & \href{https://weboodi.helsinki.fi/hy/vl_kehys.jsp?Kieli=6&MD5avain=&vl_tila=4&Opas=5703&Org=98574586&KohtTyypHierAuk=33}{HEP graduate school} (40 ECTS points, of which 10 of transferrable skills) \\
\cernentity & \href{http://hr-training.web.cern.ch/hr-training/}{Academic training program} including transferrable skills.\\
\nikhefentity & \href{https://www.nikhef.nl/en/education/onderzoekschool/}{Research school} in sub-atomic physics. \\
\unigeentity & HEP doctoral school in theory and phenomenology. \\
\sorbonneentity and \cnrsentity& [Vava]. \\
\ibmentity & Summer schools from the Association Francaise pour l'Intelligence Artificielle (AFIA).\\
\fleetmaticsentity & Dedicated courses in computer science and image processing, in association with \uniboentity \\
\tabularnewline\hline

%\textbf{\acronym :} Types of local training provided by the beneficiaries \tabularnewline\midrule 
%\textbf{\helsinkientity:} HEP graduate school including courses equivalent to one year of full time study (40 ECTS points) of which 10 ECTS are to be earned in transferable skills such as e.g. \href{https://weboodi.helsinki.fi/hy/vl_kehys.jsp?Kieli=6&MD5avain=&vl_tila=4&Opas=5703&Org=98574586&KohtTyypHierAuk=33}{scientific/grant writing and efficient communication}. Regular seminars at the Helsinki Institute of Physics and training courses at the Finnish \href{https://www.csc.fi/web/training}{"IT Center For Science"} complement the classes. \\ %
%\textbf{\unigeentity:} HEP doctoral school that provides courses ranging from theory and phenomenology to experimental aspects such as detectors and stats. These courses are given by UNIGE employees or other experts in the field that are especially invited to give lectures. \href{http://ple.unige.ch/fr/}{Dedicated center} organizes soft skills workshops and seminars.\\ %, awarding ECTS credits
%\textbf{\cernentity}: 
%%\href{https://indico.cern.ch/category/345/}{Summer students program}; 
%\href{http://hr-training.web.cern.ch/hr-training/}{wide catalogue} of transferrable skills courses available, \eg, ``Making presentations'', ``Writing of professional documents'', ``Risk Management''\\%
%\textbf{\dortmundentity:} Advanced lectures (Msc. / graduate student level)
%in the departments of physics, computer science and
%mathematics. Graduate school of the collaborative research center
%(SFB876) on Resource-aware Machine Learning. Weekly seminars in
%physics and computer science. Personal and professional development transferrable skills courses in the center for higher education. \\
%\textbf{\cnrsentity:} Sorbonne university provides a full range of \href{http://ed560.ipgp.fr/index.php/Formations_scientifiques}{academic} and \href{http://ed560.ipgp.fr/index.php/Formations_g\%C3\%A9n\%C3\%A9ralistes}{non-academic} (transferrable skills) training courses.\\ %
%\textbf{\nikhefentity}:  \href{https://www.nikhef.nl/en/education/onderzoekschool/}{Research school} in sub-atomic physics, with \href{https://www.nikhef.nl/en/education/onderzoekschool/topical-lectures/}{topical lectures} on subjects ranging from theoretical and experimental physics, to advanced statistical data analysis
%techniques;
%\href{https://www.nikhef.nl/en/education/academic-education/master/}{lectures}
%from Masters program; personal development courses on time
%management, grant writing, C++ and object oriented programming.
%\textbf{\lundentity:} Advanced lectures (Msc. / graduate student
%level) in the departments of physics, computer science and
%mathematics. Yearly course for PhD students organized by the HEP division on a variety of topics (e.g. 2016: Dark Matter). 
%Weekly seminars in physics, mathematics, astronomy and computer science. \href{http://cbbp.thep.lu.se/compute/Courses.php}{COMPUTE graduate school} with advanced courses on computing in research, awarding ECTS credits. \href{https://www.lunduniversity.lu.se/international-admissions/professional-education/professional-education-paid-by-your-employer}{Transferrable skills courses} such as "entrepreneurship and soft skills".\\
%%Lectures in statistics and reproducibility in data science, statistics and astrophysics\\
%\textbf{\heidelbergentity:} \href{https://www.physik.uni-heidelberg.de/highrr/}{HighRR graduate school}, lectures and tutorials on HEP detector development. School on physics beyond the SM, where lectures are prepared and delivered by advanced students, are also part of the training.%
%\tabularnewline
%\textbf{\ibmentity :} One of the \href{http://www.rudebaguette.com/2014/03/26/ibm-france-lab-hotbed-innovation-made-france/}{largest, top research labs in France}, IBM Research Lab France provides expert supervision and employee in algorithms, mobile app integration from design to market, interaction between large companies and startups, cloud computing and cognitive systems. \\
%\textbf{\dqentity :} Expert supervision in deep learning, real-time control, project management, software development, insurance provision and risk assessment.\\
%\midrule
\end{tabular}
%\vspace{-4mm}
\end{center}
As \acronym also intends to prepare ESRs to teach others, specific credits will also be assigned to teaching skills. 
This includes taking pedagogical courses, as well as supervising Bachelor's and Master's students at the local node. 

\vskip2pt
\noindent \color{blue}\acronym credits system: \color{black}

\begin{wraptable}{l}{0.4\textwidth}
    \vspace{-2mm}
	\caption*{Table 1.4: Example \acronym\ doctoral program\label{tab:docProg}
	}\vspace{4mm}
	%\begin{center}
	\footnotesize
	%\resizebox {\textwidth }{!}{%
	\begin{tabular}{p{45mm}r}
		\midrule
		Type of training & No. of credits \tabularnewline\midrule
		\textbf{Training through research}  & \textbf{135} \tabularnewline
		\hspace{5mm}At host & 75 \tabularnewline
		\hspace{5mm}Through secondment  & 60 \tabularnewline\midrule
		\textbf{Training through lectures, courses and dissemination} &  \textbf{45} \tabularnewline
%		\hspace{5mm}PhD courses  & \tabularnewline
		\hspace{5mm}Technical and Research Training & 30 \tabularnewline
		\hspace{5mm}Transferable Skills Training & 15 \tabularnewline
		\hspace{10mm}of which towards teaching & 5\tabularnewline
        \hspace{10mm}and dissemination & \tabularnewline
		\textbf{Total} &  180 \tabularnewline
		\bottomrule
	\end{tabular}
	%}%
	%\end{center}
    \vspace{-2mm}
\end{wraptable}
%\acronym organizing these topical schools in a three-year rotation period. 

To ensure that such a diverse training program is coherent and recognized across the network, we have designed  a \acronym credit system according to the ECTS standard. 
%The training side of the projects will be developed as a doctoral program.
Each ESR will complete 180 \acronym\ credits, as shown for an example ESR with two secondments in Table~\ref{tab:docProg}.
As PhDs in all participating institutions are awarded based on local regulations, \acronym credits also ensure that ESRs receive the appropriate training. 
Each network-wide event below includes the amount of assigned \acronym\ credits.

We have assigned 1 \acronym\ credit per each $1/2$ day of lectures. 
All students will have to explicitly include 15 credits of transferrable skills training within their PCDP. 
% - they will be able to choose among the programs of their local institute, or the institutes / industries they are seconded in. 
By attending the yearly meetings of the network and the final conference, around $1/2$ of the \acronym\ credits that they will need to complete in both ``Technical and Research'' and ``Transferable Skills'' categories, as presented at the beginning of the section, will be provided in network-wide events. 
ESRs will have freedom how to complete the rest of required credits through the local resources. 
Finally, ESRs will be required to present their work to at least one conference outside the network in their area of expertise. 
A list of conferences of interest for \acronym topics is given in Sec.~\ref{sec:dissemination}.
The amount of credits awarded will depend on the targeted conference.  
The attribution of ECTS credits will require institutes to explicitly include the network events in their course plan, but the conversion from \acronym to ECTS credits will be justified and straightforward. 

%\vspace{-5mm}

%\vspace{-5mm}
%\end{table}

%\textbf{\Tstrut Transferable skills\Bstrut} \\
%{\parbox{\textwidth}{%
%\Tstrut 
%Gender and unconscious bias (\cern); \\ \hline
%\end{tabular}
%}%
%\end{center}
%%\vspace{-5mm}
%\end{table} 
%\vspace{-2mm}
\subsubsection{Role of non-academic sector in the training program}
%MLD I think this sentence is way too hard. Not everyone will agree to this!
%Basic research is at the heart of its PhD projects in the ESRs, but 
\acronym dynamically and directly addresses the challenges of the academic, industrial, and entrepreneurial sectors. 
This enriches all sectors through the transfer of best practices, knowledge, and expertise.
For this reason \acronym includes a comprehensive program of non-academic training, with hands-on experience in solving practical problems during secondments at industrial partners, discussed further in Sec.~\ref{sec:qualityInteraction}.
The non--academic partners of \acronym will have the following roles in the training of the ESRs:
\begin{itemize}
\item \textbf{Training through research: secondments}. One of \acronym's most important objectives is to increase the exposure of the students to the private sector, solving practical problems with RTA and creating commercial value. 
Most ESRs will have secondments at private companies relevant for their tools and research topics.
These will place a particular emphasis on common methods between the commercial applications which the non--academic partners specialize in, and the academic goals of the ESR projects. 
The connections are emphasized in the ESR project descriptions (Sec.~\ref{sec:FellowProj}). 
\item \textbf{Training through mentoring and supervision:} Industrial secondment supervisors are co-supervisors for the overall PhD project, while ESRs that don't have an industrial secondment will be assigned a non--academic mentor throughout their PhD.
\item \textbf{Training through lectures: } All industrial beneficiaries and partners have also agreed to provide lectures and courses on transferrable skills and on their experience in occasion of the yearly meetings, in the non-academic training workshop and in the industry school. In particular, participants from \lightboxentity, \pointeightentity and \fleetmaticsentity have an academic background, participants from \uniboentity transitioned from industry back to academia, and participants from \ximantisentity have a start-up in parallel to their academic position. This places the consortium in the best position to prepare the ESRs for a fluid career in academia and industry. 
%Since an additional goal of \acronym is to develop sustainable software that is commonplace in industry, dedicated lectures will be given in the introductory school.
\end{itemize}

%allowing the seconded students to participate in their local training events. 
%
%We profit from the experience of the companies that take part in the project, and include secondments with them
%for most of the ESRs
%based at academic nodes. The secondments will be an essential contribution of the non--academic sector to the 
%training, also because 
%The secondments will ensure that ESRs are trained in a variety of skills throughout their participation in \acronym. 
%The industrial beneficiaries and partners in \acronym all share the challenge of high energy physics research
%of taking decisions fast and efficiently, and the secondments are naturally embedded in the research programs
%for each ESR as discussed in Sec.~\ref{sec:coherence} providing \textbf{training through research}. 
%  
%
%
%Finally, the non--academic sector is also expected to contribute intensely to the network-wide 
%events, both in the Technical/Research and Transferable Skills sides. 
%There will be several lectures on industry-related 
%Transferable Skills covered by experts from the non-academic sector.


%%%%%%%%

%\noindent \textbf{\color{blue}Importance of the secondments to the partner companies for the training purposes of the network\color{black}}
%\textbf{Text from last year's application: why the secondments to these companies are good} 
% Add drawing and explanation of the intersectoral secondments,but also take the broader look of all  intersectoral exposure and link to the solutions as for example mentorship by industry. 
% Consider the consequences for the Individual Research Projects. 
% Consider the consequences for the Gantt chart


%ESRs will be trained during the network-wide events
%and in addition receive one-to-one mentoring and supervision specific to their projects,
%as described in Tab.~\ref{tab:LocalTraining}. 
%All ESRs will work on an industrial project relevant
%to their research and deliverables during these secondments. 
%The precise topics will be defined in the PCDPs,
%to give the ESRs some freedom to follow their interests. %Some example topics are
%\vspace{-3mm}
%\begin{multicols}{2}[]
%\begin{enumerate}{\leftmargin=1em}
%    \item{Time analysis of a sequence of speech signals %(\dq);}%\vspace{-2mm}
%    \item{Combined speech/image signals analysis %(\dq);}%\vspace{-2mm}
%\end{enumerate} 
%\end{multicols}
%\vspace{-3mm}


%Below: TBC, but it looks like it's in
%\technopolis will play a key role in this area because
%of their unparalleled experience with training scientists to address policy makers and commercialize
%the results of their research.
%We are particularly aware of the very different ways that job applications work inside
%and outside academia, and this will be addressed directly, both in the lectures and as part of the
%final industry school which all ESRs will attend.

%Examples are: preparation for recruitment and interviews, management, and transitioning
%from academia to industry. 


%% HEADER PLEASE READ!!!!!!!!
%% HEADER PLEASE READ!!!!!!!!
%% HEADER PLEASE READ!!!!!!!!
%% HEADER PLEASE READ!!!!!!!!
%% HEADER PLEASE READ!!!!!!!!
%% ALL NODES TO FILL IN THE TWO TABLES AT THE BOTTOM OF THIS TEX FILE DEFINING THE TRAINING
%% HEADER PLEASE READ!!!!!!!!
%% HEADER PLEASE READ!!!!!!!!
%% HEADER PLEASE READ!!!!!!!!
%% HEADER PLEASE READ!!!!!!!!
%% HEADER PLEASE READ!!!!!!!!
%
%%%From instructions
%%4.1 Research and Training Activities
%%Applicants will primarily propose a dedicated and high-level joint research training program that focuses on promoting scientific excellence and exploiting the specific research expertise and infrastructure of the beneficiaries and of the collective expertise of the network as a whole. These training programs will address in particular the development and broadening of the research competences of the ESRs. Such training activities might include:
%% Training through research by means of individual, personalised projects, including meaningful exposure to different sectors;
%% Development of network-wide training activities (e.g. workshops, summer schools) that exploit the inter/multi-disciplinary and intersectoral aspects of the project and expose the researchers to different schools of thought. Such events could also be open to external researchers. For doctoral programs (i.e. EID and EJD), the broad structure of the curriculum should be outlined and preferably quantifiable by ECTS points;
%% Provision of structured training courses (e.g. tutorials, lectures) that are available either locally or at another participant. Training programs between the participants are expected to be coordinated to maximise added value (e.g. joint syllabus development, opening up of local training to other network teams, joint PhD programs, etc.);
%% Exchanging knowledge with the members of the network through undertaking intersectoral visits and secondments. A strong networking component is expected in each proposal;
%% Invitation of visiting researchers originating from the academic or non- academic sector. This would be aimed at improving the skills and know-how of the researchers and should be duly justified in the context of the training program. The network can cover costs of visiting researchers under the Research, Training and Networking cost category.
%%Further training activities with a particular view to widening the career prospects of the researchers would include transferable skills training both within and outside the network. Topics of interest could include:
%%Marie Skłodowska-Curie Actions, Guide for Applicants Innovative Training Networks 2016
%%Page 15 of 48
%% Training related to research and innovation: management of IPR, take up and exploitation of research results, communication, standardisation, ethics, scientific writing, personal development, team skills, multicultural awareness, gender issues, research integrity, etc.
%% Training related to management or grant searching: involvement in the organisation of network activities, entrepreneurship, management, proposal writing, enterprise start-up, task co-ordination, etc.
%%Each researcher recruited for a period of more than 6 months will establish, together with her/his personal supervisor(s) in the host organisation/s, a personal Career Development Plan. This plan shall aid in the provision of the research training program that best suits the researcher's needs. Attention should be paid to the quality of the joint research training program, with provision for supervision and mentoring arrangements and career guidance. Furthermore, the meaningful exposure of each researcher to other disciplines and sectors represented in the network through visits, secondments and other training events shall also be ensured.
%%Although mutual recognition is mandatory only for EJD, it is expected that both beneficiaries and partner organisations will mutually recognise the quality of the research and training and, if possible, of diplomas and other certificates awarded. The size of the joint research training program and of the network will depend on the nature and scope of the training activities to be undertaken by the network, as well as on considerations regarding management and effective interaction among the partners.
%

