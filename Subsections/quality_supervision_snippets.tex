%In all cases, 
\begin{center}\scriptsize
\begin{tabular}{|p{0.07\textwidth}|p{0.88\textwidth}|}
\hline
\textbf{ESR} & \textbf{Supervisors profile and complementary qualifications}
\tabularnewline \hline
%\ESRa & \helsinkientity & \helsinkientity & 8 & 36 & Voutilainen [4/8] & Pierini [6/13], \textit{Taccari} \tabularnewline \hline %Eerola [19], 

\ESRa  & 
\textbf{C. Doglioni (\ESRj, \ESRm)}: Associate senior lecturer. Expert on RTA and dark matter/new physics in ATLAS. ERC Starting Grant on RTA in ATLAS, 2015. Author of first trigger-level search in ATLAS. Convener of HEP Software Foundation Trigger and Reconstruction group. Steering group of COMPUTE research school. Responsible for IPPOG Masterclasses at \lundentity. 
\tabularnewline
& \textbf{O. Smirnova (\ESRm)}: Senior lecturer. Computing grid and distributed data analysis expert. National Grant on RTA and grid computing. Steering group of COMPUTE research school. 
\tabularnewline
& \textbf{P. Christiansen (\ESRk)}: Professor. Expert in real-time detector reconstruction and analysis in ALICE. National Grants on Heavy Ion physics and detector building. Extensive teaching and outreach experience. 
\\
\hline
\cnrsentity& 
\textbf{V. Gligorov (\ESRx)}: Senior researcher. Expert in RTA, ML in trigger systems and flavor physics. ERC Consolidator Grant 2016. LHCb Real-time analysis project coordinator, former LHCb High Level Trigger project leader and deputy Physics Coordinator. Initiator and coordinator of LHCb's Masterclass programme. 
\tabularnewline
& \textbf{F. Crescioli (\ESRf)}: Research engineer. Expert in ASIC design and design and commissioning of highly parallel FPGA based reconstruction systems, such as FTK for ATLAS. Technical coordinator of LPNHE ATLAS group and national projects. 
\tabularnewline
& \textbf{B. Malaescu (\ESRf)}: Senior researcher. Expert in RTA, statistics, SM and BSM physics. ATLAS Standard Model Working Group convener. \\
\hline
\dortmundentity & 
\textbf{J. Albrecht (\ESRd, \ESRe)}: Lecturer. Expert in reconstruction, RTA and LFV/LFU. ERC Starting Grant on RTA and LFV/LFU, 2016. LHCb physics coordinator, various position of responsibilities leading trigger and physics groups in LHCb. 
\\
\hline
\heidelbergentity & 
\textbf{M. Dunford (\ESRl)}
\tabularnewline
& \textbf{P. Starovoitov (\ESRl)}
\tabularnewline
& \textbf{S. Hansmann-Menzemer (\ESRn)}
\\
\hline
\helsinkientity & 
\textbf{M. Voutilainen (\ESRa)}
\\
\hline
\cernentity & 
\textbf{B. Petersen (\ESRc, \ESRi)}
\tabularnewline
& \textbf{M. Pierini (\ESRa)}
\tabularnewline
& \textbf{R. Shahoyan (\ESRk)}
\tabularnewline
& \textbf{A. Pearce  (\ESRe)}
\tabularnewline
& \textbf{R. Matev  (\ESRd)}
\\
\hline
\nikhefentity & 
\textbf{O. Igonkina (\ESRh)}
\tabularnewline
& \textbf{G. Raven (\ESRi)}
\\
\hline
\unigeentity & 
\textbf{A. Sfyrla (\ESRb)}
\\
\hline
\sorbonneentity & 
\textbf{L. Lacassagne (\ESRh)}
\\
\hline
\ibmentity & 
\textbf{P. Julli\'{e} (\ESRj)}
\tabularnewline
& \textbf{P. Feillet (\ESRx)}
\\
\hline
\fleetmaticsentity & 
\textbf{F. Sambo (\ESRf, \ESRm)}
\tabularnewline
& \textbf{L. Taccari (\ESRa)}
\\
\hline
\ximantisentity & 
\textbf{A. Sopasakis (\ESRd, \ESRm, \ESRk, \ESRh)}
\\
\hline
\pointeightentity & 
\textbf{T. Brambach (\ESRi)}
\tabularnewline
& \textbf{K. Dungs (\ESRe, \ESRn)}
\\
\hline
\lightboxentity & 
\textbf{P. Catastini (\ESRb, \ESRc)}
\tabularnewline
& \textbf{F. Borri (\ESRg)}
\\
\hline
\pisaentity & 
\textbf{C. Roda (\ESRf)}
\tabularnewline
& \textbf{A. Annovi (\ESRf)}
\\
\hline
\santiagoentity & 
\textbf{D. Martinez-Santos (\ESRb, \ESRd, \ESRn)}
\\
\hline
\oregonentity & 
\textbf{D. Strom (\ESRh, \ESRl)}
\\
\hline
\ohioentity & 
\textbf{A. Boveia (\ESRj, \ESRl)}
\\
\hline
\liegesentity & 
\textbf{G. Louppe (\ESRj)}
\\
\hline
\uniboentity & 
\textbf{F. Di Stefano (\ESRm)}
\tabularnewline
& \textbf{S. Salti (\ESRm)}
%\ibmentity & \href{http://hr-training.web.cern.ch/hr-training/}{Academic training program} including transferrable skills.\\
\tabularnewline\hline
\end{tabular}
\end{center}

%\begin{itemize}
%\item \textbf{\ESRa} will work on \textbf{ML and RTA for Higgs boson measurements and fleet control}. Voutilainen (\helsinkientity) will provide expertise on physics with RTA, as recipient of national grants for precision measurements at the LHC and as one of the pioneers of RTA in the CMS experiment. The expertise in Deep Learning (DL) techniques and RTA will be provided by Pierini (\cernentity CoG 2017 on ML at the LHC), while Taccari will follow the application of DL methods to in-vehicle edge computing. 
%analysis will work on the 
%\item \textbf{ESR1}
%\end{itemize}
