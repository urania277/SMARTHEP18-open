Most supervisors in \acronym are \textbf{experts in HEP and industrial real-time data analysis}. 
Gligorov (\parisUentity), Raven (\nikhefentity), Strom (\oregonentity) and Sfyrla (\unigeentity) have held positions of responsibility in Trigger and Data Acquisition  in LHCb and ATLAS respectively. 
Pierini (\cernentity),  Voutilainen (\helsinkientity), Gligorov, Albrecht (\dortmundentity), Doglioni (\lundentity), Boveia (\ohioentity), Starovoitov and Dunford in (\heidelbergentity) and Schramm (\unigeentity) are pioneers of the first fully RTA-based searches and measurements in their experiments, leading groups of PhD students to peer-reviewed publication on high impact journals. Gligorov, Albrecht and Doglioni hold ERC grants on this kind of analyses and application to LFV/LFU in LHCb and dark matter in ATLAS, respectively. 
Further trigger expertise is brought to the consortium by Petersen (\cern) and Igonkina (\nikhefentity), who have successfully designed dedicated trigger selections for Supersymmetry and LFV in ATLAS together with their students. 
RTA in industry is a specialty of all industrial partners, from industrial digital services (Catastini and Borri, \lightboxentity), transport optimization (Dungs and Brambach, \pointeightentity), fleet safety (Sambo and Taccari, \fleetmaticsentity) and fraud detection (Juille and Shaw, \ibmentity). 

A strong expertise in \textbf{software and computing tools} is crucial for the success of the research projects in this ITN. This is provided by Raven, one of the main authors of the LHCb software over the last decade, Matev (\cernentity) who is a lead author of the LHCb trigger software, and Smirnova (\lundentity), who is one of the architects of the Worldwide LHC Computing Grid. From the industrial side, having \ibmentity, one of the largest producers of computing hardware and software in the world, is an unique asset for \acronym with the participation of experienced members such as De Sainte Marie.  
Christiansen (\lundentity), Shahoyan (\cernentity) complement the software competences with detector hardware expertise, as leading developers of the ALICE Time Projection Chamber.  

\textbf{Machine learning techniques} and their software implementations are an emerging research tool that all ESRs will be exposed to. Here \acronym ESRs can rely on the expertise of Louppe (\liegesentity), a ML expert with many ongoing collaborations in particle physics; Pierini, holder of an ERC Consolidator Grant for ML at the LHC; Ustyuzhanin from the Yandex School of Computing and \cernentity, and for specific ML computing vision expertise within the CVLab founded by Di Stefano (\uniboentity).  Sopasakis (\ximantis), Sambo and Taccari (\fleetmaticsentity) use advanced ML techniques for transport and fleet control on a day to day basis.  

Expertise in \textbf{hybrid computing architectures} is provided by Lacassagne (\sorbonneentity), leader of the Hardware and Software for Embedded Systems team at LIP6. 
The know-how of Annovi and Roda from \pisaentity, Boveia of \ohioentity, and Crescioli of \cnrsentity in FPGA design and commissioning from their experience on the ATLAS FTK, provide training and research using one of the first hybrid supercomputers in HEP as testing ground. 
GPU expertise is provided by Santos of \santiagoentity, also holding an ERC StG on applications of GPUs for HEP. 

The thesis topics of all ESRs will involve a component of \textbf{data analysis in high energy physics}. 
Those working on Dark Sectors and physics beyond the SM will benefit from the knowledge of the LHC Dark Matter Working Group organizers, 
Boveia and Doglioni, as well as from the expertise of in supersymmetric dark matter and new physics searches of Petersen, Sfyrla, 
Voutilainen, Dunford and . 
The network expertise in LFV/LFU is given by Albrecht, Gligorov, Raven, Igonkina and . 
Malaescu (\cnrsentity) is currently the ATLAS Standard Model convener, while Christiansen holds several national grants for heavy ion measurements in ALICE. 

For \textbf{career mentoring purposes}, Dungs and Catastini transitioned to industry after a successful research experience in HEP, Salti (\uniboentity) spent a period in industry before returning to academia, and Sopasakis is the CEO of a start-up alongside his assistant professorship at \lundentity, so they are all extremely well placed to supervise and mentor students with intersector projects and career perspectives. 


%\begin{itemize}
%\item \textbf{\ESRa} will work on \textbf{ML and RTA for Higgs boson measurements and fleet control}. Voutilainen (\helsinkientity) will provide expertise on physics with RTA, as recipient of national grants for precision measurements at the LHC and as one of the pioneers of RTA in the CMS experiment. The expertise in Deep Learning (DL) techniques and RTA will be provided by Pierini (\cernentity CoG 2017 on ML at the LHC), while Taccari will follow the application of DL methods to in-vehicle edge computing. 
%analysis will work on the 
%\item \textbf{ESR1}
%\end{itemize}
