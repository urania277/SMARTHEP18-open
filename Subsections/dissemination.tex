%In addition to publications and patents, communication of the Marie Sk?lodowska-Curie actions should aim to demonstrate the ways in which research is contributing to a European "Innovation Union."9 It should also account for public spending by providing tangible evidence that the funded research adds value by:
%??? showing how European collaboration in the ITN has achieved more than would have otherwise been possible, notably in achieving scientific excellence, contributing to competitiveness and, where relevant, solving societal challenges;
%??? showing how the outcomes are relevant to our everyday lives, by creating jobs, training skilled researchers, introducing novel technologies, or making our lives more comfortable in other ways;
%??? promoting results, which may possibly influence policy-making or ensure follow-up by industry and the scientific community.

%YR advice:
%In 2015 and previous calls, this section was referring to effectiveness to proposed measures. As of 2016 the quality of the proposed measure seems to get more attention, we advise to be specific about the targeted journals and conferences (in case of dissemination) and about the quality of the local Tech Transfer offices of the participating universities.

%All researchers should ensure, in compliance with their contractual arrangements, that the results of their research are disseminated and exploited, e.g. communicated, transferred into other research settings or, if appropriate, commercialised. Senior researchers, in particular, are expected to take a lead in ensuring that research is fruitful and that results are either exploited commercially or made accessible to the public (or both) whenever the opportunity arises.

%which are the target audiences for dissemination:
In order for \acronym results to make an effective impact within
and beyond the consortium, and to ensure a high visibility for ESRs, 
we identify three distinct audiences for the dissemination strategy. 
%Sentence before, removed. 
%This will ensure a high visibility to ESRs as well as contributing to the dissemination
%of the novel techniques developed to an  audience much larger
%than \acronym's.
The first %target dissemination audience beyond the consortium
is composed of the \textbf{scientists members of the LHC collaborations}
that the ESRs are connected to. In this context,
intermediate work can be shown by the ESRs,
normally after discussion and preview by the supervisor. 
In addition, the ESRs will present their work
regularly within \acronym at the occasion of the yearly meetings,    
and we confirm that all four LHC collaborations allow 
the presentations of intermediate results within the context of regional or
training network meetings after a light review process. 

When the results are mature, they will be
made public to \textbf{the wider HEP and Data Science communities}. 
This step will happen after internal peer-review from the collaborations for physics results,
and after the necessary exploitation measures have been taken 
in case of algorithms with a commercial value. 
\acronym results will be published in high-impact Open Access journals of
the most closely corresponding research field, 
whether physics or Data Science. 
The physics publications produced within the ALICE, ATLAS, CMS and
LHCb collaborations fall under the CERN regulations on intellectual
property. CERN publishes following the Open Access conditions, as
defined by the SCOAP3\footnote{SCOAP3, \href{http://scoap3.org/}{SCOAP3
    Sponsoring Consortium for Open Access Publishing in Particle
    Physics.}} initiative and adopting the Creative Commons
Attribution. Papers with experimental results based on CERN data will
mention "Copyright CERN, for the benefit of the  Collaboration".
The CERN publication policy aims to exercise the copyrights to permit
the widest possible dissemination and use of the obtained results, 
and all \acronym publications will follow the same publication philosophy.
Examples in which \acronym members have
regularly published are \href{http://journals.aps.org/prl/}{PRL}\footnote{We are aware that
  PRL is not Open Access or SCOAP3  compliant, but CERN articles are
  treated as a special case by PRL and \textbf{are} Open Access, and
  all \acronym publications sent to PRL will be in this category.},  
\href{http://www.journals.elsevier.com/physics-letters-b/}{PLB}, \href{http://jhep.sissa.it}{JHEP}, 
\href{http://jmlr.csail.mit.edu}{JMLR}, \href{http://www.springer.com/computer/ai/journal/10994}{ML}, \href{http://www.computer.org/web/tpami}{IEEE PAMI}.
All publications will also be made available on the CDS and arXiv
preprint servers. Each ESR will be responsible for the
publication of 1-2 journal papers and 1-2 conference reports during their PhD. 
\acronym members have extended experience preparing
publications in the most qualified peer-reviewed journals of their
respective fields and will consequently supervise and \textbf{train}
ESRs to disseminate their results.  
\acronym results will be presented in
international conferences, e.g.
\href{https://indico.cern.ch/event/577856/}{EPS},
\href{http://www.ichep2018.org}{ICHEP},
\href{http://moriond.in2p3.fr}{Moriond  EW\&QCD},
\href{http://nips.cc}{NIPS}, \href{http://icml.cc}{ICML}, \href{http://www.aaai.org/Conferences/IAAI/iaai15.php}{AAAI}, \href{http://www.ecmlpkdd2014.org}{ECML},
\href{https://sites.google.com/site/representationlearning2014/}{ICLR}, \href{http://www.aistats.org}{AISTATS}, \href{http://chep2018.org}{CHEP}, \href{http://cdsr.net/papers/}{CDSR}, 
\href{https://indico.cern.ch/event/543031/}{IEEE Real Time Conference}, etc. 

Finally, reports and whitepapers that are deliverables of WP3-6
will summarise the overall WP results with respect to the initial Network objectives,
%and research objectives, 
and the recommendations
made towards adopting RTA and the computing resources needed to do so. 
The target audiences here are the \textbf{EU Commission, the European researchers subscribed to the 
EURAXESS community}, who may find interesting links and connections between
their research and the research undertaken in \acronym and want to collaborate, and policymakers
and research councils who can be pointed to whitepapers and recommendations in the 
same fashion as for the HEP Software Foundation Whitepaper\footnote{http://hepsoftwarefoundation.org/activities/cwp.html}. Prof. M. Sokoloff from \cincinnatientity is one of the leaders of this important effort
that joins HEP and Data Science, and V. Gligorov, A. Boveia,
F. Winklmeier, J. Albrecht and C. Doglioni contributed to the sub-chapter on Software Trigger and Event Reconstruction. A. Boveia and C. Doglioni also have experience being the main organizers and editors of the successful Dark Matter Forum and Dark Matter Working Group whitepapers that gather recommendations on Dark Matter benchmark models for the entire LHC theory and experimental community. Together, \acronym is equipped with  key people to achieve this dissemination goal. 

Brian Petersen is responsible for dedicated Dissemination
and Outreach Work Package. Together with the PC, 
he will assist the network members (ESRs and supervisors) on
publication matters. He will ensure that publications, reports and contributions of
\acronym to conferences will be highlighted on the \acronym web page, social media 
and local institute press releases with the corresponding links to the relevant journal articles, 
preprints and proceeding papers.
