The main academic results of \acronym are software, peer-reviewed papers and whitepapers. 

As mentioned in the previous section, the physics publications produced within the ALICE, ATLAS, CMS and LHCb collaborations fall under the CERN regulations on intellectual
property. 
Similar terms apply to the data and software tools generated within the academic institutions and the CERN laboratory. 
\acronym will conform to the open access policies of the four LHC collaborations when it comes to the data collected by these detectors. 

The developed software will be released as described in Sec.~\ref{sec:ipr}, and used in the ALICE, ATLAS, CMS and LHCb collaborations and disseminated for further use in research.
%We will actively encourage the wider exploitation of these toolkits in our dissemination strategy.
It be licensed under Open Source licenses such as Apache/MIT/BSD/GPL depending on the use case, ensuring free usage and distribution. 
An exception applies to results produced by ESRs working in the private companies. 
In this case the internal rules of the  companies concerning the Intellectual Property of the work need to be respected to guarantee its potential commercial  exploitation.
Preliminary arrangements with the concerned companies have been made, so that the obtained development are published in journal papers, while their actual implementation remains the Intellectual Property of the company. More details of IPR can be found in Sec.~\ref{sec:ipr}.

The main commercial results of \acronym are the improved \ximantisentity traffic prediction application, fleet software for edge computing and mobile platforms, 
software for real-time anomaly detection through automated inference and rule induction in \ibmentity, and Internet-of-Things sensors for industrial production processes in \lightbox. Both beneficiary institutes and secondment hosts will benefit from the exploitations in joint intellectual property agreements. 
%Some industrial beneficiaries and partners have carried out preliminary benefit analyses of these results,
%based on their revenues and margins from current products.
%For the toolkit developed by ESRZ, 
%the overall life insurance market is worth
%around \euro1.7 trillion while the deep learning products are
%estimated to be worth billions of Euro.  
%\dq estimates that
%the successful completion of deliverable 2.3 will lead to better
%prediction of risk and hence reduced insurance costs. \dq estimates 
%this will lead to generated economic value of O(2)~MEuro per year. 
%\dq will retain the right to commercially exploit the results, other
%consortium members will retain fair-use rights, and simplified
%proof-of-principle products will be made available
%under Open Access in order to facilitate further innovation and
%development stemming from those. 
As an example, the adoption of predictive maintenance in industrial production processes thanks to the  Internet-of-Things sensors development during the \lightbox secondment, 
is estimated to on average reduce maintenance costs by more than 25\%, reduce breakdowns by more than 70\%, reduce downtime by more than 35\% and increase productivity
by more than 20\%. 
While we cannot quantify the commercial impact of all industrial products in the same way, we are convinced that the other secondment projects which the ESRs will perform with their industrial partners, and the innovation of methods which \acronym will more generally foster, will lead to significant additional economic value.

We have preliminarily reserved 15000 EUR from the management budget~\footnote{Quote as per discussion with LU Innovation, for filing first a national and then an 
international patent} for filing a possible patent on one of these commercial deliverables, to boost the exploitation of the most promising algorithms
and toolkits in \acronym. 
The use of this fund for patent filing will have to be approved unanimously by the Supervisory Board. 