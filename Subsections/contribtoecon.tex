%\textbf{Instructions: Contribution to structuring
%doctoral/early-stage research training at the European level and to
%strengthening European innovation capacity, including the potential
%for Meaningful contribution of the non-academic sector to the
%doctoral / research training (as appropriate to the implementation
%mode and research field)}  

\subsubsection*{Contribution to structuring doctoral training at the European level}

%%Structuring

The academic and industrial training offered by \acronym
greatly benefits ESRs by widening their career choices
beyond the academic model of student-postdoc-professor.
%normally portrayed in academia. 
%As discussed in Sec.~\ref{sub:trainingOverview}, 
%\acronym structures its training following the Salzburg Principles, 
%\acronym provides a supportive environment for the development of each ESR.
Both \acronym recruitment and \textbf{training} proceeds in a diverse, inclusive, and
supportive environment with the help of specifically tasked officers, complementing
taught courses at the beneficiary institutes with academic research and 
non-academic industrial secondments. 
Such a diverse training program mandates the use of the credit system
described in Sec.~\ref{sub:trainingOverview} to quantify and
evaluate ESR training progress. % of the ESR training as ,
%agreed upon by student and supervisor in the PCDP
%and then approved by the Supervisory Board. 
%Although the \acronym credit system is voluntary, 
%and its adoption is left to the PhD-awarding organizations, 
Its direct link to the ECTS credit system serves as example for the 
structuring of early-stage research training within \acronym. 

\acronym training happens within and outside the consortium, 
blending with existing training programs hosted by beneficiaries and partners, as well 
as with ongoing ITN networks (we have made preliminary contact with INSIGHTS and MCNet as some of the nodes are involved
in those as participants, and schools are already open to external ESRs).
As well as providing complementary training for ESRs, 
these collaborations in training offer an excellent platform to share best practices and seek complementary 
funding sources beyond this ITN, increasing the internationalization
of the participating organizations. Discussions have already started among the network participants
on how to share training and find funding beyond this ITN 
to attract and retain more talented students from abroad. The participation of the network
to ISOTDAQ, the most important school of HEP data acquisition, goes in this direction, by 
disseminating network results and techniques and at the same time allowing ESRs to learn the state of the art. 

%%Meaningful contribution of non-academic sector to the training
\acronym links training in HEP with
commercial applications within the common ground of RTA, 
so that the non-academic sector strongly contributes to the increased societal
and economic relevance of the outcome of the ESR research and training path. 
The majority of ESRs will spend a minimum of three
months seconded to an industrial beneficiary or partner, and
will explicitly use their academic training to further the development
of industrial products; in turn, the industrial training they receive will boost their
academic research. 
Within \acronym, the non-academic sector is placed at the heart of the training program
in a concrete and practical way, so that ESRs employ techniques developed for research
in industry and vice versa. This is not yet the norm in 
physics graduate courses, even though studies show a need
for this kind of interdisciplinary and intersectoral mobility
experience\footnote{Roach M, Sauermann H (2017) The declining interest in an academic career. PLOS ONE 12(9): e0184130, \url{http://journals.plos.org/plosone/article?id=10.1371/journal.pone.0184130}}. 
Moreover, history shows that such
interdisciplinary cross-pollination of methods and problem-solving skills
strengthens innovation (e.g. 
Bell Labs in the USA). 
We hope the success of \acronym, if funded, and of
H2020 ITN programs across Europe will encourage embedding non-academic training in PhD degrees. 
\vspace{-2mm}

\subsubsection*{Contribution to strenghtening the European innovation capacity}

The research and industry goals in \acronym be met using innovative techniques that are the focal point 
of a European-wide ESR education, benefitting both the European higher education system
and Europe's innovation and growth potential. 

The economic importance of physics in European industries overall has been detailed by the
CEBR\footnote{\href{http://www.eps.org/resource/resmgr/policy/EPS_economyReport2013.pdf}{The
importance of physics to the economies of Europe}, Centre for
Economics and Business Research Ltd, London, 2013} in a report
commissioned by the European Physical Society.  
Industries for which the use
of physics in terms of technologies and expertise is critical to their
existence generated a turnover of \euro3.8 trillion in 2010, 15\% of the total turnover Europe's economy. 
Moreover, the productivity per employee in the physics-based 
industries largely outperforms that of other industries. 

%Data Science is one of the fastest
%growing sectors in the world. 
As noted by a recent report by the EU's High Level
Expert Group on the European Open Science Cloud\footnote{"500,000 data scientists needed
in European open research data",
A. Offermann, \url{https://joinup.ec.europa.eu/news/500000-data-scientists-need}}, 
expertise in Data Science is a skill needed to strenghten innovation in Europe, 
and one that \acronym ESRs will gain. \acronym will prepare 
a well-trained addition to the European workforce that will give 
a competitive edge to the academic institutes and industries where the ESRs will
undertake their training, as well as those where they will be employed after their PhD.  
This will be increased by the strength of the network as a whole, 
acting coherently to improve the deployment of RTA 
techniques. In the data-rich environment that 
will dominate both HEP and industry, 
making decisions fast and efficiently becomes a priority to make an impact and
be at the forefront of innovation and developments.  

%The impact of physics is underestimated in this study, as the
%contributions of physics-trained people to the booming field of DS
%are not considered. The skill set of physicists are particularly
%appreciated in this sector, and it represents one of the main
%employers for physicists who leave academia. This can only increase
%the figures from the Cebr, establishing the impact of fundamental
%research in physics as  impactful on the European economy.


 
