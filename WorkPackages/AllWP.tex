%Table 3.1.a

%WP1
%Description of Work and Role of Specific Beneficiaries / Partner Organisations
%(possibly broken down into tasks), indicating lead participant and role of other participants
\begin{center}\small
\resizebox {\textwidth }{!}{%
\begin{tabular}{|p{10mm}|p{45mm}|p{25mm}|p{35mm}|p{35mm}|p{30mm}|}
\hline
\textbf{\Tstrut WP1} & \textbf{Lead Beneficiary}: \lundentity & \textbf{Duration: 1-48} & 
\textbf{\Tstrut Title:} Management &  Activity type: MGT & ESR: 1 elected
\tabularnewline\hline
\multicolumn{6}{|p{202mm}|}{%
\pbox{202mm}{\textbf{\Tstrut Objectives:} Create and  maintain a high level of collaboration between the \acronym consortium members to ensure the optimal functioning of the network, and report to the EU on network activities and progress. }}\tabularnewline\hline
\multicolumn{6}{|p{202mm}|}{
	\pbox{202mm}{\textbf{\Tstrut Description of Work and Role of Partners:}
Doglioni as Project Coordinator (PC) will oversee the smooth running of the planned research and training activities for the network.
\lund has been chosen to coordinate \acronym %and as lead beneficiary of WP0 
as a large institution with dedicated
support and experience with projects of this kind, and due to Doglioni's experience in coordinating HEP communities
and groups within and beyond ATLAS  (e.g. LHC Dark Matter Working Group, more than 300 participants). 
Doglioni is an alumna of the \href{http://www.collegiocavalieri.it/en}{Collegio Universitario Lamaro Pozzani},
%\tablefootnote{The \href{http://www.collegiocavalieri.it/en}{"Lamaro Pozzani"} College has been founded in 1971 by the Federazione Nazionale dei Cavalieri del Lavoro in Rome and }, 
funded by the Italian National Federation of Holders of the Order of Merit for Labour, 
that hosts up to 15 selected Italian students per year, complementing their regular university education with courses and lectures in entrepreneurship, law, IT and languages.
This gives her an unique perspective on cross-talk between academia and industry that started developing as early as her undergraduate education. 
Administrative support to the PC will be hired, at the level of at least 50\% FTE. %(\task{1.1}). 
The PC will keep close contact with other WP coordinators to monitor the execution of tasks, and  compliance to the defined milestones.
The PC's first task will be to convene discussions leading to the signature of the Consortium Agreement (CA). %(\task{1.2}).
Management will coordinate the hiring of all ESRs together with a dedicated recruitment officer.% (\task{1.3}, \task{1.4}).
The PC is responsible for preparing and chairing the consortium administrative meetings in the EB and SB, 
for the preparation of reports for the EU together with the project manager,
%(\task{1.5}), 
and for overseeing the election of ESR representatives to the EB and SB.% after their recruitment. 
%The first of those, the Kick-off meeting is of prime importance to oversee the recruitment of the ESRs.
The PC and the relevant WP coordinators oversee the organization of the planned workshops and conferences % (\task{1.6}). 
The PC is responsible that optimal communication happens. This 
builds lasting collaborations between the nodes, enabling knowledge transfer and
forming a legacy of future research collaborations.
\Bstrut}}\tabularnewline\hline
\multicolumn{6}{|p{202mm}|}{
	\pbox{202mm}{\textbf{\Tstrut Deliverables:}
\deli{1.1} Hiring of a dedicated project manager (month 1); 
\deli{1.2} Signature of the CA by all parties (month 2); 
\deli{1.3} Launch of the website and Twitter accounts as outward-facing communication and organization tools (month 3); 
\deli{1.4} Advertisement of the recruitment and recruitment completion (month 3, 8); 
\deli{1.5} Prepare reports for each Supervisory Board meetings, including mandatory annual progress report to EU (months 3, 13, 25, 37, 48); 
\deli{1.6} Collect documentation and reports following all Network events: workshops, conferences, outreach events (months 3, 13, 19, 25, 37,43, 48).\Bstrut}}\tabularnewline\hline
\end{tabular}
}%
\vspace{-4mm}
\end{center}
%Table 3.1.a


%WP2
%Description of Work and Role of Specific Beneficiaries / Partner Organisations
%(possibly broken down into tasks), indicating lead participant and role of other participants
\begin{center}\small
\resizebox {\textwidth }{!}{%
\begin{tabular}{|p{10mm}|p{45mm}|p{25mm}|p{35mm}|p{35mm}|p{30mm}|}
\hline
\textbf{\Tstrut WP2} & \textbf{Lead Beneficiary}: \unige & \textbf{Duration: 1-48} &
\textbf{\Tstrut Title: Training } &  Activity type: training & ESR: All\tabularnewline\hline
\multicolumn{6}{|p{202mm}|}{%
\pbox{202mm}{\textbf{\Tstrut Objectives:} Organisation of the Network-wide training events, overview of the personal training for each ESR, see Sec.~\ref{sec:training}.}}\tabularnewline\hline
\multicolumn{6}{|p{202mm}|}{
	\pbox{202mm}{\textbf{\Tstrut Description of Work and Role of Partners:}
The WP2 coordinator is Sfyrla from \unige, with experience in student supervision,
public lectures and course organization. 
The WP2 coordinator oversees the preparation of the Personal Career Development 
Plans and follows up with the supervisors for the intermediate and final reports %(\task{2.1}). 
Prof. Sfyrla oversees the organization of the lectures for the development of technical, research and transferable skills
(see Sec.~\ref{sec:trainingcontrib}) taking place at the Network-wide events. %(\task{2.2})..
The WP2 coordinator makes sure the lecturers give training of excellent quality and that adequate documentation is provided to the ESRs, 
and provides advertisement and follow-up material together with the PC.  
Note that short lecture proceedings will be prepared
in advance of the lectures by the speakers,% (\task{2.3}), 
so that they can be disseminated without delay following 
he events (as for example with the CHEP conference series). All nodes and partners will
benefit and contribute to the training of the ESRs, by playing an active role through supervision of
PhD programs, research and industry projects that will train excellent 
scientists with a wide range of skills and experiences. Together with the responsible for WP7, 
Sfyrla will ensure the quality and feedback of the presentation of ESRs to \acronym events.% (\task{2.4})
\Bstrut}}\tabularnewline\hline
\multicolumn{6}{|p{202mm}|}{
	\pbox{202mm}{\textbf{\Tstrut Deliverables:}
\deli{2.1} Personal Career Development Plans for each ESR, intermediate and final monitoring (month 12, 24, 36, 42):
\deli{2.2} Design and organization of network-wide schools together with beneficiaries responsible \acronym events (months 10, 18, 27-29, 38);
\deli{2.3} Proceedings of lectures given at the \acronym events (months 11, 19, 30, 39); %this is also a milestone?
\deli{2.4} Presentations from all ESR to the \acronym events and at the final conference (months 12, 24, 36, 42);\Bstrut}}\tabularnewline\hline
\end{tabular}
}%
\vspace{-4mm}
\end{center}
%Table 3.1.a

%WP3
%Description of Work and Role of Specific Beneficiaries / Partner Organisations
%(possibly broken down into tasks), indicating lead participant and role of other participants
\begin{center}\small
\resizebox {\textwidth }{!}{%
\begin{tabular}{|p{10mm}|p{35mm}|p{22mm}|p{60mm}|p{35mm}|p{18mm}|}
\hline
\textbf{\Tstrut WP3} & \textbf{Lead Beneficiary}: \cnrs & \textbf{Duration: 8-48} &
\textbf{\Tstrut Title: Machine Learning and data analysis } &  Activity type: research & ESR: ALL\tabularnewline\hline
\multicolumn{6}{|p{202mm}|}{%
\pbox{202mm}{\textbf{\Tstrut Objectives:} Deployment of advanced Machine Learning (ML) and data analysis techniques to enable real-time analysis.}}\tabularnewline\hline
\multicolumn{6}{|p{202mm}|}{
	\pbox{202mm}{\textbf{\Tstrut Description of Work and Role of Partners:}
The WP3 coordinator is Gligorov from \cnrs, who led the first large-scale implementation of real-time ML at an LHC experiment in 2011, led LHCb's HLT during 2014 and 2015, and oversaw LHCb's physics programme as deputy physics coordinator during 2016 and 2017. The WP3 coordinator is responsible for the coherence and interaction of the ESRs investigating Machine Learning techniques throughout \acronym, for their development and their application to HEP and commercial use cases according to the network-wide strategy described in Fig.~\ref{fig:implementation}. 
He will also ensure that code is written alongside documentation that makes the software useful and usable, and  
that frameworks and toolkits are made public in a timely manner and respecting IP clauses in case of commercial exploitation. The three main tasks for WP3 
match the research objectives, and their completion enables RTA. WP3 will produce novel algorithms to reconstruct objects and events in real-time (with the know-how of \nikhefentity and \cernentity), and develop ML techniques for both event reconstruction and fast data analysis (exploiting the expertise of \ibmentity, \dqentity and \unigeentity). The algorithms for both reconstruction and ML will also be benchmarked and optimized in the projects hosted at \nikhefentity and \cernentity. 
\Bstrut}}\tabularnewline\hline
\multicolumn{6}{|p{202mm}|}{
	\pbox{202mm}{\textbf{\Tstrut Deliverables:}
The deliverables of WP3 follow the description in Fig. 1 of Sec.~\ref{sec:introRO}. The first
whitepaper is the output of the research of the ESRs on the state of the art on machine learning for real-time analysis~\deli{\deliverableWhitepaperStateOfTheArtWPThree} (month \deliverableWhitepaperStateOfTheArtWPThreeMonth) and their original work is collected in a paper detailing implementation and deployment of ML and reconstruction techniques~\deli{\deliverableWhitepaperDevelopmentWPThree} (month \deliverableWhitepaperDevelopmentWPThreeMonth). These contribute to the software that enables real-time analysis
at LHC experiments that is included in the software releases before the LHC data taking periods~\deli{\deliverableTriggerExperimentalSoftwareWPThree} (months \deliverableTriggerExperimentalSoftwareWPThreeMonth) and it is exploited for the research objectives of WP5 and WP6.
}}\tabularnewline\hline
\end{tabular}
}%
\vspace{-4mm}
\end{center}

\begin{center}\small
\resizebox {\textwidth }{!}{%
\begin{tabular}{|p{10mm}|p{35mm}|p{25mm}|p{45mm}|p{35mm}|p{30mm}|}
\hline
\textbf{\Tstrut WP4} & \textbf{Lead Beneficiary}: \sorbonneentity & \textbf{Duration: 8-48} &
\textbf{\Tstrut Title:  Hybrid Architectures } &  Activity type: research & ESR: 3, 4, 7, 10, 11, 15\tabularnewline\hline
\multicolumn{6}{|p{202mm}|}{%
\pbox{202mm}{\textbf{\Tstrut Objectives:} Study of hybrid computing architectures to enable real-time Analysis.}}\tabularnewline\hline
\multicolumn{6}{|p{202mm}|}{
	\pbox{202mm}{\textbf{\Tstrut Description of Work and Role of Partners:}
The WP4 coordinator is Lacassagne from \sorbonneentity. He is a system architect with extensive experience in
the benchmarking and use of hybrid architectures. 
The WP4 coordinator is responsible for the coherence and interaction of the ESRs developing and testing code for hybrid architectures, following the strategy described in Fig.~\ref{fig:implementation}. 
He is responsible for the  publication and exploitation of successful deliverables related to hybrid architectures, 
as well as many studies that demonstrated that standard architectures could be improved for the purpose of RTA (see e.g. Lemaitre, Lacassagne, \href{https://hal.archives-ouvertes.fr/hal-01361204/document}{Batched Cholesky Factorization for tiny matrices}, DASIP 2016).
The work is divided in three tasks corresponding to the research objectives: the use of FPGAs (e.g. FTK, expertise of \ohioentity and \pisaentity, \heidelberginstrumentsentity), the use of GPUs for speeding up parallel algorithms in industry and HEP (with the contribution of \santiagoentity), and employing parallel and multithreaded algorithms (exploiting the competences of \lightboxentity)
Together with the partners, the WP4 coordinator also oversees the training program on each specific architecture, and ensures the quality of lectures and their proceedings with the coordinator of WP2. 
He also makes sure that code written for non-standard architectures satisfies %sustainability and 
documentation standards.\Bstrut}}\tabularnewline\hline
\multicolumn{6}{|p{202mm}|}{
	\pbox{202mm}{\textbf{\Tstrut Deliverables:}
The first deliverable of WP4 is a whitepaper describing the state of the art on hybrid architectures in real-time analysis~\deli{\deliverableWhitepaperStateOfTheArtWPFour} (month \deliverableWhitepaperStateOfTheArtWPFourMonth). The ESR's subsequent work will be made public in 
a second whitepaper describing the advancements in optimization of hybrid architectures, on FTK and on the LHC trigger systems~\deli{\deliverableWhitepaperDevelopmentWPFour} (month \deliverableWhitepaperDevelopmentWPFourMonth). WP4 will also deliver toolkits to optimize the choice of 
hybrid architecture, and oversee the commercial delivery of a similar optimization toolkit for parallelization~\deli{\deliverableParallelizationOptimizationWPFour} (month \deliverableParallelizationOptimizationWPFourMonth). The related HEP publications are deliverables in collaboration with WP5. A joint deliverable of WP2 and WP4 is the design, organization and documentation of the FPGA and GPU schools (\deli{2.2}). 
}}\tabularnewline\hline
\end{tabular}
}%
\vspace{-4mm}
\end{center}

%WP5
%Description of Work and Role of Specific Beneficiaries / Partner Organisations
%(possibly broken down into tasks), indicating lead participant and role of other participants
\begin{center}\small
\resizebox {\textwidth }{!}{%
\begin{tabular}{|p{10mm}|p{40mm}|p{25mm}|p{45mm}|p{35mm}|p{25mm}|}
\hline
\textbf{\Tstrut WP5} & \textbf{Lead Beneficiary}: \dortmundentity & \textbf{Duration: 8-48} &
\textbf{\Tstrut Title: Real-time decision making} &  Activity type: research & ESR: 1-9,11-15\tabularnewline\hline
\multicolumn{6}{|p{202mm}|}{%
\pbox{202mm}{\textbf{\Tstrut Objectives:} Applying real-time analysis to decision making in physics and society.}}\tabularnewline\hline
\multicolumn{6}{|p{202mm}|}{
	\pbox{202mm}{\textbf{\Tstrut Description of Work and Role of Partners:}
WP5's goal is to enable fast and efficient decision making with RTA, in physics through the use of the trigger systems, and in society to improve safety and efficiency of transport in ways that would not be possible without RTA. 
WP5 is coordinated by Albrecht (\dortmundentity as main beneficiary node) and by Sopasakis (\ximantis). 
They have been chosen to fill this role for their expertise the trigger strategies that are crucial for the physics analyses that are the focus of \acronym, and for their expertise on RTA decision-making and transport. They will ensure that the RTA techniques enabled by WP3 and WP4 are applied to advance both triggers and society by the various ESRs working on the experiment trigger systems, that the exchange between academia and industry is fruitful in both directions through cross-pollination of techniques, and the results are documented in peer-reviewed papers. 
Together with the WP2 and WP3 coordinators, they oversee the trigger lectures in the introductory school at \lund and the physics and ML school at \unigeshort, and coordinate the preparation of the ISOTDAQ lectures and ESR contributions.\Bstrut}}\tabularnewline\hline
\multicolumn{6}{|p{202mm}|}{
	\pbox{202mm}{\textbf{\Tstrut Deliverables:}
The first deliverable is an introductory review of the state of the art of the triggers of all collaborations that the ESRs will compile as preliminary work to their physics analyses~\deli{\deliverableWhitepaperStateOfTheArtWPFive} (month \deliverableWhitepaperStateOfTheArtWPFiveMonth).
The second deliverable is the whitepaper collecting results from all real-time physics analyses in a review~\deli{\deliverableWhitepaperCollectionPapersWPFive} (month \deliverableWhitepaperCollectionPapersWPFiveMonth).
It will include publications on dark sectors and Higgs boson searches, LFV/LFU and precision measurements. 
Deliverables improving transport and logistics are the updates to the Ximantis app~\deli{\deliverableXimantisML, \deliverableXimantisHybrid} (month \deliverableXimantisMLMonth, \deliverableXimantisHybridMonth) 
and the work performed within \pointeightentity (~\deli{\deliverableLogisticsOptimisation}, month \deliverableLogisticsOptimisation). 
A joint deliverable of WP2 and WP5 is the design, organization and documentation of the trigger and physics lectures in the introductory and Physics/ML schools (\deli{2.2}). 
}}\tabularnewline\hline
\end{tabular}
}%
\vspace{-4mm}
\end{center}

%WP6
%Description of Work and Role of Specific Beneficiaries / Partner Organisations
%(possibly broken down into tasks), indicating lead participant and role of other participants
\begin{center}\small
\resizebox {\textwidth }{!}{%
\begin{tabular}{|p{10mm}|p{35mm}|p{25mm}|p{55mm}|p{39mm}|p{16mm}|}
\hline
\textbf{\Tstrut WP6} & \textbf{Lead Beneficiary}: \ibm & \textbf{Duration: 8-48} &
\textbf{\Tstrut Title: Real-time monitoring and discoveries} &  Activity type: Exploitation & ESR: ALL\tabularnewline\hline
\multicolumn{6}{|p{202mm}|}{%
\pbox{202mm}{\textbf{\Tstrut Objectives:} Applying real-time analysis to monitor complex systems and detect anomalies, in physics and society.\Bstrut}}\tabularnewline\hline
\multicolumn{6}{|p{202mm}|}{
	\pbox{202mm}{\textbf{\Tstrut Description of Work and Role of Partners:}
The goal of WP6 is to employ real-time analysis to detect novelty or anomalies while monitoring complex systems and streams of data. 
These data streams range from LHC collision events (\lundentity), to financial transactions (\ibmentity), to data from vehicle dashboard cameras (\fleetmaticsentity), to sensor data from industrial processes (\lightboxentity). 
A sub-goal that is novel to \acronym and essential to introduce such techniques in HEP trigger systems is the accountability and reproducibility of the algorithms employed, developed in \ESRx. 
For this reason, \ibmentity is chosen as the lead beneficiary of WP6 with De Sainte Marie as main coordinator, given his extensive experience in supervision of student projects and his expertise on symbolic knowledge systems. 
WP6's academic co-coordinator is Pierini (\cern), one of the pioneers of anomaly detection in LHC experiments. 
WP6 will work in close collaboration with WP3 to design new algorithms and combine the best of both symbolic knowledge and numerical algorithms towards application to HEP triggers and society. 
The deliverables of WP6 match the research objectives and include applications both in HEP and in the commercial sector, as the algorithms developed can be ported to the different kinds of data. 
Aided by the WP7 coordinator, by LU Innovation from the PC side and by the H2020 IPR helpdesk, WP6 coordinators ensure that these commercial deliverables are disseminated and documented after their exploitation, and correctly handled in terms of IP and \href{http://ec.europa.eu/justice/data-protection/index_en.htm}{EU General Data Protection Regulation}.
Together with the PC, WP6 coordinators are responsible for the organization of the \acronym school focused on industrial and commercials topics. 
\Bstrut}}\tabularnewline\hline
\multicolumn{6}{|p{202mm}|}{
	\pbox{202mm}{\textbf{\Tstrut Deliverables:}
Algorithms for fraud detection and HEP triggers (\deli{\deliverableRule}, month~\deliverableRuleMonth).
Toolkits using ML and AI for real-time in-fleet monitoring within \fleetmaticsentity  (with ~\deli{\deliverableFleetmaticsMLMobile} and \deliverableDashboardCam, months \deliverableFleetmaticsMLMobileMonth and \deliverableDashboardCamMonth).
Software for sensors for Internet-of-things and industrial process optimization in collaboration with \lightboxentity (~\deli{\deliverablePredictiveMaintenance} and ~\deli{\deliverableParallelization}, months \deliverablePredictiveMaintenance and \deliverableParallelizationMonth). 
}}\tabularnewline\hline
\end{tabular}
}%
\vspace{-4mm}
\end{center}

%WP7
%Description of Work and Role of Specific Beneficiaries / Partner Organisations
%(possibly broken down into tasks), indicating lead participant and role of other participants
\begin{center}\small
\resizebox {\textwidth }{!}{%
\begin{tabular}{|p{10mm}|p{45mm}|p{25mm}|p{50mm}|p{30mm}|p{20mm}|}
\hline
\textbf{\Tstrut WP7} & \textbf{Lead Beneficiary}: \cern & \textbf{Duration: 1-48} &
\textbf{\Tstrut Title: Outreach and dissemination} &  Activity type: OUT & ESR: ALL\tabularnewline\hline
\multicolumn{6}{|p{202mm}|}{%
\pbox{202mm}{\textbf{\Tstrut Objectives:} Relay \acronym scientific activities to the general public, monitor delivery of results as journal papers, ensure ESR visibility in conferences.\Bstrut}}\tabularnewline\hline
\multicolumn{6}{|p{202mm}|}{
	\pbox{202mm}{\textbf{\Tstrut Description of Work and Role of Partners:}
WP7 is detailed further in Sec.~\ref{sec:CommPub}.  
\cern is chosen as the lead beneficiary of WP7
given their extensive experience in communicating with the public. 
The WP7 coordinator organises the communication of \acronym to both the general public and the scientific community and together with the PC runs the  
%The first task consists in setting up a 
dedicated outreach portal  %for the communication platform 
\url{www.smarthep.org}, supported by social media activities.
Communication to the scientific communities is achieved by poster and talk contributions to conferences. The WP7 coordinator will monitor the quality of the material presented by the ESRs
ensuring they benefit from an outstanding international visibility. A crucial scientific deliverable will be the articles reporting on the Network activities. 
The coordinator will ensure a consistent top quality of papers produced within the Network. Outreach to the general public will take the form of visits to schools, guided visits to CERN facilities, 
public lectures and "science on tap" at Network events. \acronym dissemination activities will completed by delivery of the \acronym Masterclass exercise.
These tasks will ensure the general public grasps the impact of \acronym on academia and HEP, industry and everyday life.
All academic and industrial partners will take an active part in the WP7 activities, and in particular will simultaneously host 
%provide the material and local connection for 
the International Masterclass exercises, both for the International Day of Women in Science and for the regular date. 
\Bstrut}}\tabularnewline\hline
\multicolumn{6}{|p{202mm}|}{
	\pbox{202mm}{\textbf{\Tstrut Deliverables:}
\deli{7.1} \url{www.smarthep.org} and Facebook and Twitter accounts as a platform for outreach (month 3);
\deli{7.2} Talks and Posters in conferences (month 12-48); 
\deli{7.3} Publication of scientific results in peer-reviewed journals (month 12-48); 
\deli{7.4} Outreach to the general public (month 8-48, but especially in coincidence with network events).
\deli{7.5} Deliver \acronym Masterclass exercise (months 17-18, 30-31, 41-42).
}}\tabularnewline\hline
\end{tabular}
}%
\end{center}