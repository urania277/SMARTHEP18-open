%Table 3.1.a

%WP1
%Description of Work and Role of Specific Beneficiaries / Partner Organisations
%(possibly broken down into tasks), indicating lead participant and role of other participants
\begin{center}\scriptsize
%\resizebox {\textwidth }{!}{%
\begin{tabular}{|p{0.05\textwidth}|p{0.15\textwidth}|p{0.23\textwidth}|p{0.15\textwidth}|p{0.12\textwidth}|}
\hline
\cellcolor{red!70!black} \textbf{\color{white}WP1\color{black}} & \textbf{Management} &  \textbf{Lead Beneficiary}: \lundentity & \textbf{Duration}: 1-48 & 
 ESR: 1 elected
\tabularnewline\hline
\multicolumn{5}{|p{0.975\textwidth}|}{%
\textbf{\Tstrut Objectives:} Create and  maintain a high level of collaboration between the \acronym consortium members to ensure the optimal functioning of the network, and report to the EU on network activities and progress.}\tabularnewline\hline
\multicolumn{5}{|p{0.975\textwidth}|}{\textbf{\Tstrut Description of Work and Role of Partners:}
Doglioni as Project Coordinator (PC) will oversee the smooth running of the planned research and training activities for the network.
\lund has been chosen to coordinate \acronym as a large institution with dedicated support and experience with projects of this kind, and due to Doglioni's experience in coordinating HEP communities and groups within and beyond ATLAS  (e.g. LHC Dark Matter Working Group, more than 300 participants and HEP Software Foundation trigger and reconstruction WG). 
Doglioni is an alumna of the \href{http://www.collegiocavalieri.it/en}{Collegio Universitario Lamaro Pozzani}, funded by the Italian National Federation of Holders of the Order of Merit for Labour and hosting up to 15 selected Italian students per year, complementing their regular university education with courses and lectures in entrepreneurship, law, IT and languages.
This gives her an unique perspective on cross-talk between academia and industry that started developing as early as her undergraduate education. 
Administrative support to the PC will be hired, at the level of at least 50\% FTE. %(\task{1.1}). 
The PC will keep close contact with other WP coordinators to monitor the execution of tasks, and  compliance to the defined milestones.
The PC's first task will be to convene discussions leading to the signature of the Consortium Agreement (CA). %(\task{1.2}).
Management will coordinate the hiring of all ESRs together with a dedicated recruitment officer.% (\task{1.3}, \task{1.4}).
The PC is responsible for preparing and chairing the consortium administrative meetings in the EB and SB, for the preparation of reports for the EU together with the project manager, and for overseeing the election of ESR representatives to the EB and SB. %(\task{1.5}), 
The PC and the relevant WP coordinators oversee the organization of the planned workshops and conferences % (\task{1.6}). 
The PC is responsible for facilitating communication between the nodes, building lasting collaborations between the nodes, enabling knowledge transfer and forming a legacy of future research collaborations.
\Bstrut}\tabularnewline\hline
\multicolumn{5}{|p{0.975\textwidth}|}{
	\pbox{202mm}{\textbf{\Tstrut Deliverables}: \deli{1.1} (month 2) Hiring of a dedicated project manager (month 1); \deli{1.2} Signature of the CA by all parties; }
	}\tabularnewline
\multicolumn{5}{|p{0.975\textwidth}|}{
\deli{1.3} (month 3) Launch of the website and social media accounts as outward-facing communication and organization tools;
}\tabularnewline
\multicolumn{5}{|p{0.975\textwidth}|}{
\deli{1.4} (month 3, 8) Advertisement of the recruitment and recruitment completion, including ESR declarations; 
}\tabularnewline
\multicolumn{5}{|p{0.975\textwidth}|}{
\deli{1.5}  (month 3,9,23,36,48)  Prepare brief reports for each SB meeting, incl. mandatory annual and mid-term progress report to EU ; 
}\tabularnewline
\multicolumn{5}{|p{0.975\textwidth}|}{
\deli{1.6}  (month 3,9,23,36,48) Joint with WP2, collect documentation, reports and feedback to be discussed at the upcoming SB, following all Network events and schools.
}%%	\pbox{202mm}{}
%%}
\tabularnewline\hline
\end{tabular}
%}%
%\vspace{-8mm}
\end{center}
%Table 3.1.a

%WP2
\begin{center}\scriptsize
\begin{tabular}{|p{0.05\textwidth}|p{0.15\textwidth}|p{0.23\textwidth}|p{0.15\textwidth}|p{0.12\textwidth}|}
\hline

\cellcolor{red} \textbf{\color{white}WP2\color{black}} & \textbf{Training} & \textbf{Lead Beneficiary}: \unige & \textbf{Duration: 1-48} & ESR: All\tabularnewline\hline

\multicolumn{5}{|p{0.975\textwidth}|}{%

\textbf{\Tstrut Objectives:} Organisation of the Network-wide training events, overview of the personal training for each ESR, see Sec.~\ref{sec:training}.}

\tabularnewline\hline
\multicolumn{5}{|p{0.975\textwidth}|}{\textbf{\Tstrut Description of Work and Role of Partners:}
The WP2 coordinator is Sfyrla from \unige, with experience in student supervision, public lectures and course organization. 
The WP2 coordinator oversees the preparation of the Personal Career Development Plans and follows up with the supervisors for the intermediate and final reports %(\task{2.1}). 
Prof. Sfyrla oversees the organization of the lectures for the development of technical, research and transferable skills (see Sec.~\ref{sec:trainingcontrib}) taking place at the Network-wide events. %(\task{2.2})..
The WP2 coordinator makes sure the lecturers give training of excellent quality and that adequate documentation is provided to the ESRs, and provides advertisement and follow-up material together with the PC.  
Short lecture proceedings will be prepared in advance of the lectures by the speakers so that they can be disseminated without delay following the events (as for example with the CHEP conference series). % (\task{2.3}), 
All nodes and partners will benefit and contribute to the training of the ESRs, by playing an active role through supervision of PhD programs, research and industry projects that will train excellent scientists with a wide range of skills and experiences. 
Together with the responsibles for WP7, Sfyrla will ensure the quality and feedback of the presentation of ESRs to \acronym events.% (\task{2.4})
\Bstrut}\tabularnewline\hline
\multicolumn{5}{|p{0.975\textwidth}|}{
	\pbox{202mm}{\textbf{\Tstrut Deliverables}: \deli{2.1}  (month 9,23,36,42)  PCDPs for each ESR, intermediate and final monitoring;}
	}\tabularnewline
\multicolumn{5}{|p{0.975\textwidth}|}{
\deli{2.2}  (month 10,15,27-29,38)  Design and organization of network-wide schools together with responsible beneficiaries;
}\tabularnewline
\multicolumn{5}{|p{0.975\textwidth}|}{
\deli{2.3}  (month 9,23,36,42)  Presentations from all ESR to the \acronym events and at the final conference;
}
\tabularnewline\hline
%\multicolumn{5}{|p{0.975\textwidth}|}{
%\deli{2.4}  (month 11,16,30,39)  Proceedings of lectures given by \acronym participants and ESRs; 
%}\tabularnewline
\multicolumn{5}{|p{0.975\textwidth}|}{
\deli{2.4} (month 44) ESRs with three-year PhD duration receive a degree.
}
\tabularnewline\hline
\end{tabular}
%\vspace{-9mm}
\end{center}

%%WP3
\begin{center}\scriptsize
\begin{tabular}{|p{0.04\textwidth}|p{0.24\textwidth}|p{0.21\textwidth}|p{0.12\textwidth}|p{0.07\textwidth}|}
\hline

\cellcolor{orange} \textbf{\color{black}WP3\color{black}} & \textbf{ML \& advanced data analysis} & \textbf{Lead Beneficiary}: \cnrs & \textbf{Duration: 8-48} & ESR: All \tabularnewline\hline

\multicolumn{5}{|p{0.975\textwidth}|}{%

\textbf{\Tstrut Objectives:} Deployment of advanced Machine Learning (ML) and data analysis techniques to enable real-time analysis.}

\tabularnewline\hline
\multicolumn{5}{|p{0.975\textwidth}|}{\textbf{\Tstrut Description of Work and Role of Partners:}
The WP3 coordinator is Gligorov from \cnrs, who led the first large-scale implementation of real-time ML at an LHC experiment in 2011, led LHCb's HLT during 2014 and 2015, oversaw LHCb's physics programme as deputy physics coordinator during 2016 and 2017 and is currently leader of the Real Time Analysis Project in LHCb (consisting of 30 institutes and around 50 FTE).  
The WP3 coordinator is responsible for the coherence and interaction of the ESRs investigating ML techniques in RTA throughout \acronym, for their development and their application to HEP and commercial use cases according to the network-wide strategy described in Fig.~\ref{fig:implementation}. 
He will also ensure that code is written alongside documentation that makes the software useful and usable, and that frameworks and toolkits are made public in a timely manner and respecting IP clauses in case of commercial exploitation. 
WP3 will produce novel algorithms to reconstruct objects and events in real-time (with the know-how of \nikhefentity and \cernentity), and develop ML techniques for event reconstruction, fast data analysis and outlier detection (exploiting the expertise of \liegesentity, \ibmentity, \fleetmaticsentity and \unigeentity). 
The algorithms for both reconstruction and ML will also be benchmarked and optimized in the projects hosted at \nikhefentity and \cernentity. 
\Bstrut}\tabularnewline\hline
\multicolumn{5}{|p{0.975\textwidth}|}{
\textbf{\Tstrut Deliverables}: \textit{For this and other research WP deliverables, we follow the implementation plan in ~\ref{sec:introRO}, where the ESRs first gain an overview of the state of the art, then developing new techniques, and finally disseminating them as legacy of \acronym.}} 
\tabularnewline
\multicolumn{5}{|p{0.975\textwidth}|}{
\deli{2.2} (joint with WP1, WP2) design, organization and documentation of the contributions to the network ML and MLHEP schools
}\tabularnewline
\multicolumn{5}{|p{0.975\textwidth}|}{
\deli{\deliverableWhitepaperStateOfTheArtWPThree}  (month \deliverableWhitepaperStateOfTheArtWPThreeMonth) 
Whitepaper on the state of the art on ML for real-time analysis, detailing implementation and deployment, capitalizing on the attendance of the MLHEP school in month  24;
}\tabularnewline
\multicolumn{5}{|p{0.975\textwidth}|}{
\deli{\deliverableTriggerExperimentalSoftwareWPThree}  (month \deliverableTriggerExperimentalSoftwareWPThreeMonth) 
Collection of ML algorithms and software toolkits to be exploited for the research objectives in WP5 and WP6; 
}\tabularnewline
\multicolumn{5}{|p{0.975\textwidth}|}{
\deli{\deliverableFinalWhitepaperWPThree}  (month \deliverableFinalWhitepaperWPThreeMonth) .
Review paper collecting description and documentation of techniques for ML in RTA. 
}
\tabularnewline\hline
\end{tabular}
%\vspace{-9mm}
\end{center}


%%WP4
\begin{center}\scriptsize
\begin{tabular}{|p{0.04\textwidth}|p{0.18\textwidth}|p{0.17\textwidth}|p{0.12\textwidth}|p{0.20\textwidth}|}
\hline

\cellcolor{yellow} \textbf{\color{black}WP4\color{black}}  & \textbf{Hybrid architectures} & \textbf{Lead Beneficiary}: \sorbonneentity & \textbf{Duration: 8-48}  & ESR: \ESRsForWPFourText \tabularnewline\hline

\multicolumn{5}{|p{0.975\textwidth}|}{%

\textbf{\Tstrut Objectives:} Study and adoption of hybrid computing architectures to enable RTA.}

\tabularnewline\hline
\multicolumn{5}{|p{0.975\textwidth}|}{\textbf{\Tstrut Description of Work and Role of Partners:}
The WP4 coordinator is Lacassagne from \sorbonneentity. 
He is a system architect with extensive experience in the benchmarking and use of hybrid architectures. 
The WP4 coordinator is responsible for the coherence and interaction of the ESRs developing and testing code for hybrid architectures. 
He is responsible for the  publication and exploitation of successful deliverables related to hybrid architectures, as well as many studies that demonstrated that standard architectures could be improved for the purpose of RTA (see e.g. Lemaitre, Lacassagne, \href{https://hal.archives-ouvertes.fr/hal-01361204/document}{Batched Cholesky Factorization for tiny matrices}, DASIP 2016).
The work is divided in three tasks corresponding to the research objectives: the use of FPGAs (e.g. track triggering for ATLAS, expertise of \ohioentity, \oregonentity, \pisaentity), the use of GPUs for speeding up parallel algorithms in industry and HEP (\santiagoentity), and employing parallel and multithreaded algorithms (\lightboxentity).
Together with the partners, the WP4 coordinator also oversees the training program on each specific architecture, and ensures the quality of lectures and their proceedings with the coordinator of WP2. 
He also makes sure that code written for non-standard architectures satisfies high documentation standards.
\Bstrut}\tabularnewline\hline
\multicolumn{5}{|p{0.975\textwidth}|}{
\textbf{\Tstrut Deliverables}: \deli{2.2} (joint with WP1, WP2) design, organization and documentation of the FPGA and GPU schools.} 
\tabularnewline
\multicolumn{5}{|p{0.975\textwidth}|}{
\deli{\deliverableWhitepaperStateOfTheArtWPFour}  (month \deliverableWhitepaperStateOfTheArtWPFourMonth)  
Whitepaper on the state of the art on hybrid architectures in real-time analysis, capitalizing on attendance of network FPGA/GPU schools;
}\tabularnewline
\multicolumn{5}{|p{0.975\textwidth}|}{
\deli{\deliverableParallelizationOptimizationWPFour}  (month \deliverableParallelizationOptimizationWPFourMonth) 
Software toolkits and hardware improvements in HEP (e.g. FTK for ATLAS, GPU for LHCb reconstruction); 
}\tabularnewline
\multicolumn{5}{|p{0.975\textwidth}|}{
\deli{\deliverableParallelization}  (month \deliverableParallelizationMonth) 
\lightbox software to optimize parallelization of financial transactions and associated publications; 
}\tabularnewline
\multicolumn{5}{|p{0.975\textwidth}|}{
\deli{\deliverableWhitepaperDevelopmentWPFour}  (month \deliverableWhitepaperDevelopmentWPFourMonth)  
Review paper collecting advancements in optimization of hybrid architectures for the LHC trigger systems.
}
\tabularnewline\hline
\end{tabular}
%\vspace{-9mm}
\end{center}

%%WP5
\begin{center}\scriptsize
\begin{tabular}{|p{0.04\textwidth}|p{0.23\textwidth}|p{0.20\textwidth}|p{0.12\textwidth}|p{0.12\textwidth}|}
\hline

\cellcolor{green} \textbf{\color{black}WP5\color{black}} & \textbf{Real-time decision making} & \textbf{Lead Beneficiary}: \dortmundentity & \textbf{Duration: 8-48} 
&   ESR: \ESRsForWPFiveText \tabularnewline\hline

\multicolumn{5}{|p{0.975\textwidth}|}{%

\textbf{\Tstrut Objectives:}  Applying real-time analysis to decision making in physics and society.}

\tabularnewline\hline
\multicolumn{5}{|p{0.975\textwidth}|}{\textbf{\Tstrut Description of Work and Role of Partners:}
WP5's goal is to enable fast and efficient decision making with RTA, in physics through the use of the trigger systems, and in society to improve safety and efficiency of transport in ways that would not be possible without RTA. 
WP5 is coordinated by Albrecht (\dortmundentity as main beneficiary node) with Sopasakis as co-coordinator (\ximantis). 
They have been chosen to fill this role for their complementary expertise in decision-making in HEP and industry that are crucial for \acronym. 
%trigger strategies crucial for the physics analyses in \acronym, and for their expertise on RTA decision-making in transport. 
They will ensure that the RTA techniques enabled by WP3 and WP4 are applied to advance both HEP and industry by the various ESRs working on the experiment trigger systems, that the exchange between academia and industry is fruitful in both directions through cross-pollination of techniques, and that the results are documented in peer-reviewed papers. 
%Together with the WP2 and WP3 coordinators, they oversee the trigger lectures in the introductory event at \lund and the physics and ML school at \unigeshort, and coordinate the preparation of the ISOTDAQ and MLHEP lectures and ESR contributions.
\Bstrut}\tabularnewline\hline
\multicolumn{5}{|p{0.975\textwidth}|}{
\textbf{\Tstrut Deliverables}: \deli{2.2} (joint with WP1, WP2) organization of trigger contributions to network events and ISOTDAQ schools.} 
\tabularnewline
\multicolumn{5}{|p{0.975\textwidth}|}{
\deli{\deliverableWhitepaperStateOfTheArtWPFive}  (month \deliverableWhitepaperStateOfTheArtWPFiveMonth) 
Review of the state of the art of the triggers of LHC collaborations, compiled by ESRs will prior to their physics analyses capitalizing on attendance of ISOTDAQ school and network events; 
}\tabularnewline
\multicolumn{5}{|p{0.975\textwidth}|}{
\deli{\deliverableXimantisHybrid}  (month \deliverableXimantisHybridMonth) 
Improved \ximantisentity app using novel ML techniques (e.g. hybrid networks) and associated publications;
}\tabularnewline
\multicolumn{5}{|p{0.975\textwidth}|}{
\deli{\deliverableTriggerExperimentalSoftwareWPFive}  (month \deliverableTriggerExperimentalSoftwareWPFiveMonth) 
Software for LHC trigger upgrades for Run-3 data taking; 
}\tabularnewline
\multicolumn{5}{|p{0.975\textwidth}|}{
\deli{\deliverableLogisticsOptimisation}  (month \deliverableLogisticsOptimisationMonth) 
Client software for optimization of transport and logistics with \pointeightentity and associated publications;
}
\tabularnewline
\multicolumn{5}{|p{0.975\textwidth}|}{
\deli{\deliverableWhitepaperCollectionPapersWPFive}  (month \deliverableWhitepaperCollectionPapersWPFiveMonth)  
Review paper collecting physics results using trigger selection improvements, including summary of publications on dark sectors, LFV/LFU and precision measurements.}
\tabularnewline\hline

\end{tabular}
%\vspace{-9mm}
\end{center}

%%WP6
\begin{center}\scriptsize
\begin{tabular}{|p{0.04\textwidth}|p{0.3\textwidth}|p{0.18\textwidth}|p{0.12\textwidth}|p{0.2\textwidth}|}
\hline

\cellcolor{cyan} \textbf{\color{black}WP6\color{black}} & \textbf{Real-time monitoring and discoveries} & \textbf{Lead Beneficiary}: \ibm & \textbf{Duration: 8-48} &
ESR: \ESRsForWPSixText \tabularnewline\hline

\multicolumn{5}{|p{0.975\textwidth}|}{%

\textbf{\Tstrut Objectives:}   Applying real-time analysis to monitor complex systems and discover anomalies, in physics and society.}

\tabularnewline\hline
\multicolumn{5}{|p{0.975\textwidth}|}{\textbf{\Tstrut Description of Work and Role of Partners:}
The goal of WP6 is to employ real-time analysis to detect novelty or anomalies while monitoring complex systems and streams of data. 
These data streams range from LHC collision events (\lundentity), to financial transactions (\ibmentity), to data from vehicle dashboard cameras (\fleetmaticsentity), to sensor data from industrial processes (\lightboxentity). 
A sub-goal that is novel to \acronym and essential to introduce such techniques in HEP trigger systems is the accountability and reproducibility of the algorithms employed, developed in \ESRx. 
For this reason, \ibmentity is chosen as the lead beneficiary of WP6 with De Sainte Marie as main coordinator, given his extensive experience in supervision of student projects and his expertise on symbolic knowledge systems. 
WP6's academic co-coordinator is Pierini (\cern), one of the pioneers of anomaly detection in LHC experiments. 
WP6 will work in close collaboration with WP3 to design new algorithms and combine the best of both symbolic knowledge and numerical algorithms towards application to HEP triggers and society. 
The deliverables of WP6 match the research objectives and include applications both in HEP and in the commercial sector, as the algorithms developed can be ported to the different kinds of data. 
Aided by the WP7 coordinator, by LU Innovation from the PC side and by the H2020 IPR helpdesk, WP6 coordinators ensure that these commercial deliverables are disseminated and documented after their exploitation, and correctly handled in terms of IP and \href{http://ec.europa.eu/justice/data-protection/index_en.htm}{EU GDPR}.
\Bstrut}\tabularnewline\hline
\multicolumn{5}{|p{0.975\textwidth}|}{
\textbf{\Tstrut Deliverables}: \deli{2.2} (joint with WP1, WP2) organization of the non-academic training and Industry and career development schools. } 
\tabularnewline
\multicolumn{5}{|p{0.975\textwidth}|}{
\deli{\deliverableWhitepaperStateOfTheArtWPSix} 
Review of the state of the art of fully RTA searches in HEP and recommendations for improvements (companion of \deli{\deliverableWhitepaperStateOfTheArtWPFive})
 (month \deliverableWhitepaperStateOfTheArtWPSixMonth) ; 
}\tabularnewline

\multicolumn{5}{|p{0.975\textwidth}|}{
\deli{\deliverableSoftwareWPSix}  (month \deliverableSoftwareWPSixMonth) 
Software enabling advanced fully RTA-based searches at LHC; 
}\tabularnewline

\multicolumn{5}{|p{0.975\textwidth}|}{
\deli{\deliverableRule}  (month ~\deliverableRuleMonth) 
Algorithms for fraud detection and HEP triggers in \ibmentity and associated publications;
}\tabularnewline

\multicolumn{5}{|p{0.975\textwidth}|}{
\deli{\deliverableFleetmaticsMLMobile}  (month \deliverableFleetmaticsMLMobileMonth) 
Toolkits using ML and AI for real-time in-fleet monitoring within \fleetmaticsentity and associated publications;
}

\tabularnewline
\multicolumn{5}{|p{0.975\textwidth}|}{
\deli{\deliverablePredictiveMaintenance}  (month \deliverablePredictiveMaintenanceMonth) 
Software for sensors for Internet-of-things and industrial process optimization in \lightboxentity 
}
\tabularnewline
\multicolumn{5}{|p{0.975\textwidth}|}{
\deli{\deliverableWhitepaperCollectionPapersWPSix}  (month \deliverableWhitepaperCollectionPapersWPSixMonth)  
Review paper collecting physics results using fully RTA-based analyses, including summary of publications on dark matter mediators, Higgs boson, heavy ion physics. 
%%Companion paper on summary of technical work, and implications outside HEP. 
}
\tabularnewline\hline
\end{tabular}
%\vspace{-9mm}
\end{center}

%%WP7
\begin{center}\scriptsize
\begin{tabular}{|p{0.04\textwidth}|p{0.23\textwidth}|p{0.20\textwidth}|p{0.12\textwidth}|p{0.12\textwidth}|}
\hline

\cellcolor{violet} \textbf{\color{black}WP7\color{black}} & \textbf{Outreach and dissemination} & \textbf{Lead Beneficiary}: \cern & \textbf{Duration: 1-48} & ESR: All ESRs.\tabularnewline\hline

\multicolumn{5}{|p{0.975\textwidth}|}{%


\textbf{\Tstrut Objectives:} Relay \acronym scientific activities to the general public, monitor delivery of results as journal papers, ensure ESR visibility in conferences.}
\tabularnewline\hline

\multicolumn{5}{|p{0.975\textwidth}|}{\textbf{\Tstrut Description of Work and Role of Partners:}
WP7 is detailed further in Sec.~\ref{sec:CommPub}.  
\cern is chosen as the lead beneficiary of WP7 given its extensive experience in communicating with the public, with Petersen as responsible. 
Ustyuzhanin from \yandexentity will co-coordinate the effort benefitting from his extensive expertise with data challenges. 
The WP7 coordinator organise the communication of \acronym to both the general public and the scientific community and ensure that all ESRs and supervisors take part in the effort. 
Together with the PC they run the dedicated communication portal \url{www.smarthep.org}, supported by ESR blogging and social media activities. 
Communication to the scientific communities is achieved by poster and talk contributions to conferences. 
The WP7 coordinator will delegate members of the network to support the ESR supervisors and monitor the quality of the material presented by the ESRs and ensure they benefit from an outstanding international visibility. 
The coordinators will ensure a consistent top quality of papers produced within the Network. 
The main innovation of \acronym's outreach program is the ESR data challenge, organized by \cernentity. 
Further outreach to the general public will take the form of visits to schools, guided visits to CERN facilities, public lectures and "science on tap" at Network events. 
\acronym dissemination activities will be completed by delivery of the specific \acronym Masterclass exercise.
These tasks will ensure the general public grasps the impact of \acronym on academia and HEP, industry and everyday life and actively participates in it through the data challenge.
All academic and industrial partners will take an active part in the WP7 activities, and will simultaneously host online the International Masterclass exercises and World Wide Data Day (WWDD) activities, including on the International Day of Women in Science day.
}\tabularnewline\hline
\multicolumn{5}{|p{0.975\textwidth}|}{
\textbf{\Tstrut Deliverables}: 
\deli{1.3}  (joint with WP1)  Launch of \url{www.smarthep.org} and social media as platform for dissemination, communication and outreach;
}\tabularnewline
\multicolumn{5}{|p{0.975\textwidth}|}{
\deli{7.1}  (month 14)  Data challenge website online and open to the public; 
}\tabularnewline
\multicolumn{5}{|p{0.975\textwidth}|}{
\deli{7.2}  (month 23, 36, 48)  Reports on outreach to the general public and on \acronym Masterclass/WWDD exercise.
}\tabularnewline
\multicolumn{5}{|p{0.975\textwidth}|}{
\deli{7.3}  (month 24, 36, 48)  Report on presentation of results at international conferences; 
}\tabularnewline
\multicolumn{5}{|p{0.975\textwidth}|}{
\deli{7.4}  (month 24, 36, 48)  Report on publication of results in peer-reviewed journals; 
}\tabularnewline

\tabularnewline\hline
\multicolumn{5}{p{0.975\textwidth}}{\textbf{Table 3.1a} Work package description for each work package.}
\end{tabular}
%%\vspace{-9mm}
\end{center}
