%\begin{table}[h]

%\begin{table}[h]
\begin{center}\small
\resizebox {\textwidth }{!}{%
\begin{tabular}{|p{16mm}|p{33mm}|p{28mm}|p{18mm}|p{18mm}|p{67mm}|}
\hline
\textbf{\Tstrut Fellow} &
\textbf{Host} &
\textbf{\phd} &
\textbf{Start} &
\textbf{Duration} &
\textbf{Deliverables}\tabularnewline 
ESR12 &  \nikhef & Yes & Month 6& 36 & \deliverableTechPubMLForOptimisation \deliverableTriggerOptToolkit \deliverableUltrasoundSimulation \deliverableHEPPubLFV \tabularnewline
\hline
\multicolumn{4}{|l|}{\textbf{\Tstrut Work Package:}
WP3, WP5, WP6, WP7} &
\multicolumn{2}{l|}{\textbf{Doctoral programme:} VU University Amsterdam}\tabularnewline\hline
\multicolumn{6}{|p{20.2cm}|}{\textbf{\Tstrut Project Title:  Smart optimization of resources for efficient trigger and analysis and use for LHCb measurements of LFV}
}\tabularnewline\hline
\multicolumn{6}{|p{20.2cm}|}{\textbf{\Tstrut Objectives:}
The project follows closely the description of ESR11, as it will develop the same toolkit for the optimization of the real-time trigger system of the main HEP experiments, this time focusing on the LHCb experiment. Compared to the Atlas experiment, the LHCb experiment generates an order of magnitude less data per collision, but the rate at which collisions will be reconstructed in the trigger is more than an order of magnitude larger. The requirement that the toolkit must work optimally for both experiments implies that it must be sufficiently generic and adaptable. As a result, it will also have applications beyond these two experiments.
The workload for the creation of this toolkit 
will be split between ESR11 and ESR12, as both students will be working in synergy
on the same topics, and will benefit from the supervision of both O. Igonkina and G. Raven.  In addition both ESR will adapt and optimizing the toolkit for their respective experiments.  

The work of ESR12 will be applied to the search for Lepton Flavor Violation in the tau to
muon+photon decay in LHCb, which currently does not have a dedicated trigger chain at LHCb. The high data rate of the upgraded LHCb experiment, in combination with the optimization of the trigger algorithms will allow to collect this kind of events
and study a process that was previously thought to require dedicated experiments. 

}\tabularnewline\hline
\multicolumn{6}{|p{20.2cm}|}{\textbf{\Tstrut Expected Results:}
ESR12, together with ESR11 will produce an inter-experiment toolkit for benchmarking
and optimization of trigger systems (Deliverable~\deliverableTriggerOptToolkit), 
and a related peer-reviewed inter-experiment publication (Deliverable~\deliverableTechPubMLForOptimisation).
The toolkit will also be released for use in industrial applications. 
The physics research will lead to a peer-reviewed publication on Lepton Flavor Violation
(Deliverable~\deliverableHEPPubLFV).
ESR11 will receive a PhD in experimental HEP at VU University Amsterdam.
}\tabularnewline\hline

\multicolumn{6}{|p{20.2cm}|}{\textbf{\Tstrut Secondments:}
}\tabularnewline
\hline
\end{tabular}
}%
\end{center}
%\end{table}
%
