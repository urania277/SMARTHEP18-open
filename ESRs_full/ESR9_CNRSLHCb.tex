%\begin{table}[h]

%\begin{table}[h]
\begin{center}\small
\resizebox {\textwidth }{!}{%
\begin{tabular}{|p{16mm}|p{33mm}|p{28mm}|p{18mm}|p{18mm}|p{67mm}|}
\hline
\textbf{\Tstrut Fellow} &
\textbf{Host} &
\textbf{\phd} &
\textbf{Start} &
\textbf{Duration} &
\textbf{Deliverables}\tabularnewline 
ESR9 &  \cnrs & Yes & Month 6& 36 & \deliverableAdversarialFramework, \deliverableTimeOrderedSourcesFramework, \deliverableTechPubTimeOrderedSourcesFramework, \deliverableTechPubAdversarialFramework, \deliverableHEPPubAdversarialLFV \tabularnewline
\hline
\multicolumn{4}{|l|}{\textbf{\Tstrut Work Package:}
WP3, WP5, WP6, WP7} &
\multicolumn{2}{l|}{\textbf{Doctoral programme:} \parisUlong }\tabularnewline\hline
\multicolumn{6}{|p{20.2cm}|}{\textbf{\Tstrut Project Title: Real-time analysis and machine learning in industry and new physics searches with LHCb data}
}\tabularnewline\hline
\multicolumn{6}{|p{20.2cm}|}{\textbf{\Tstrut Objectives:}
The goal of ESR9 is to develop new methods for real-time data analysis. 
This includes the identification and elimination of adversarial examples in classifier training,
the development of recurrent neural networks and identification of the most important patterns
which these networks rely on for their classification,
and the use of heterogenous time-ordered and non-time-ordered datasets in real-time analysis. 
During the training period, ESR9 will benefit from being employed at \dq, who are industry leaders in these topics,
particularly when applied to financial or medical insurance applications.

ESR9 will work on developing \dq's real-time analysis software in two main ways. 
First of all, they will work on a dedicated software framework for identifying, classifying, and eliminating
adversarial examples, which look similar to the human eye but differ at the pixel level.
For instance, a self-driving car could be misled by pixel-level changes to a road-sign which a human would not even notice.
ESR9's developed framework will automate the process of generating adversarial examples by systematically varying training
datasets for classifiers, and then teach the resulting network to ignore such changes in its training. ESR9 will also work on \dq's deep-learning
infrastructure, and in particular in understanding which patterns of information DreamQuark's recurrent neural networks use to
take decisions. This will then allow ESR9 to improve this framework to allow the use of non-time-ordered data
sources together with time-ordered data in order to take better decisions, particularly about financial investments or medical insurance. 

Finally, the adversarial example framework's generality will allow it to be used in a search for New Physics in the decays of strange hadrons with the LHCb
experiment. This is the third objective of this ESR. In particular, the method will be applied to search for lepton-flavour decays
of strange hadrons, which are particularly interesting in light
of hints of lepton non-universality seen by LHCb and other experiments. The major backgrounds in such searches are in fact adversarial examples, generated by
much more frequent Standard Model decays of strange hadrons with identical topologies but different final state particles. The use of the framework developed
by ESR9 will allow these analyses to be performed in real-time, and increase their sensitivity by two orders of magnitude.
}\tabularnewline\hline
\multicolumn{6}{|p{20.2cm}|}{\textbf{\Tstrut Expected Results:}
ESR9 will produce two concrete commercial deliverables: a framework for identifying and eliminating adversarial examples in real-time
algorithms (Deliverable~\deliverableAdversarialFramework), and a framework for combining non-time-ordered data sources together
with time series in real-time analysis (Deliverable~\deliverableTimeOrderedSourcesFramework). Each of these commercial deliverables will be released as a software package
and have a technical publication to document it  (Deliverables~\deliverableTechPubTimeOrderedSourcesFramework, \deliverableTechPubAdversarialFramework).
It will also lead to an academic deliverable : a peer-reviewed paper which will use the adversarial
example framework to search for New Physics in rare decays of strange hadrons with the LHCb detector
(Deliverable~\deliverableHEPPubAdversarialLFV). ESR9 will receive a PhD in experimental HEP at \parisU.
}\tabularnewline\hline
%\multicolumn{6}{|p{20.2cm}|}{\textbf{Doctoral program:} Cambridge}\tabularnewline\hline
\multicolumn{6}{|p{20.2cm}|}{\textbf{\Tstrut Secondments:}
A four-month secondment at CERN will allow the ESR to receive physics training and interaction with the LHCb collaborators involved in the physics analysis, 
as well as training in the development of new software tools under the supervision of Dr. Rosen Matev. A five month secondment at \santiago
will allow ESR9 to adapt and deploy the adversarial example framework to the search for New Physics, under the supervision
of Veronika Chobanova.
}\tabularnewline
\hline
\end{tabular}
}%
\end{center}
%\end{table}
%

%The ATLAS trigger infrastructure has an unique hardware processor to reconstruct the trajectory (tracking) of the charged particles that cross the silicon inner tracker of the experiment. The tracking information is an essential tool for effective real-time event selection and has a central role in the whole ATLAS physics program especially in the HL-LHC phase. The current hardware processor, FTK, is a complex system made by several custom electronics  boards based on FPGAs and Associative Memory chips. The latter are unique computing devices developed for the FTK algorithm. The hardware tracking will be also a central part of the Phase-II Upgrade of ATLAS, with upgraded version of FTK called FTK++.


