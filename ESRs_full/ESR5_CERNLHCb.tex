%\begin{table}[h]

%\begin{table}[h]
\begin{center}\small
\resizebox {\textwidth }{!}{%
\begin{tabular}{|p{16mm}|p{33mm}|p{28mm}|p{18mm}|p{18mm}|p{67mm}|}
\hline
\textbf{\Tstrut Fellow} &
\textbf{Host} &
\textbf{\phd} &
\textbf{Start} &
\textbf{Duration} &
\textbf{Deliverables}\tabularnewline 
ESR5 & \cern & Yes & Month 6& 36 & , YY \tabularnewline
\hline
\multicolumn{4}{|l|}{\textbf{\Tstrut Work Packages:} WP3, WP5, WP6 
} &
\multicolumn{2}{l|}{\textbf{Doctoral programme:} \dortmund }\tabularnewline\hline
\multicolumn{6}{|p{20.2cm}|}{\textbf{\Tstrut Project Title: Search for LFV in tau, strange and charmed mesons decays using real-time analysis}
}\tabularnewline\hline
\multicolumn{6}{|p{20.2cm}|}{\textbf{\Tstrut Objectives:}
LHCb will have a Phase-I upgrade in 2020 where the experiment will be fully read out at 40 MHz, allowing for a very flexible 
full software trigger. This upgrade will be accumulating the same amount of data accumulated by LHCb between 2010 and
2020 in a single year, and multiply by more than a factor six the total accumulated statistics by 2029. However,
the trigger algorithms will need to be revisited to deal with a higher rate and pileup. In particular,
to maintain the LHCb trigger performance at low particle energy one needs to rethink completely the reconstruction algorithms. 
ESR5 will be working on ways to speed up the reconstruction used at the trigger level, 
with emphasis on the unique sensitivity to low energy particles that LHCb has with respect to other LHC experiments. 
This key property of LHCb will allow to search for Lepton Flavour Violation processes in low-mass objects
like taus, kaons or charm mesons, which will benefit from the much larger statistics available after the Phase-I upgrade.
The first objective of ESR5 will therefore be to develop novel, optimized algorithms to identify and reconstruct these low-mass
objects, to be implemented in the LHCb software trigger. 
These algorithms will be then used by ESR5 to search for a violation of lepton flavour in decays
of taus and other light mesons, with will allow synergies with ERS6 (based in Dortmund) searching for
similar phenomena in unflavoured mesons. Both searches will need to have a large part of their event selection
implemented as real time analysis to maintain high signal efficiencies for
these difficult experimental signatures. ESR5 and ESR6 will work in parallel on this topic, 
for a more efficient development. The second objective of ESR5 is to finalize an event selection
that can be implemented online, and help reaching the third objective of searching for LFV in decays 
of taus and other light mesons with the first Run-3 data. 
}\tabularnewline\hline
\multicolumn{6}{|p{20.2cm}|}{\textbf{\Tstrut Expected Results:}
ESR5 will publish three peer-reviewed papers. The first one will describe the reconstruction algorithms for low-energy objects
developed for the LHCb trigger system (Deliverable \deliverableHEPPubLFVReco), the second one will be
written together with ESR6 on the improvements to the trigger selection 
(Deliverable \deliverableHEPPubLFVTrigger) and the third one will describe the results of the search for LFV in decays of taus and other light mesons
(Deliverable \deliverableHEPPubLFVFlavored).
ESR5 will receive a PhD in experimental HEP at \dortmund.
}\tabularnewline\hline
%\multicolumn{6}{|p{20.2cm}|}{\textbf{Doctoral program:} Cambridge}\tabularnewline\hline
\multicolumn{6}{|p{20.2cm}|}{\textbf{\Tstrut Secondments:}
\dortmund, 4 months, J. Albrecht. Optimisation of real-time event selection for LFV searches with low-energy objects in LHCb. 
%\yandex, 3 months, Dr. A. Ustyuzhanin, %\yandex school of data analysis and research project on MVA in industry.
%CERN, 3 months, Dr. A. Hoecker, definition of the optimal variables. 
}\tabularnewline
\hline
\end{tabular}
}%
\end{center}
%\end{table}
%
