%\begin{table}[h]

%\begin{table}[h]
\begin{center}\small
\resizebox {\textwidth }{!}{%
\begin{tabular}{|p{16mm}|p{33mm}|p{28mm}|p{18mm}|p{18mm}|p{67mm}|}
\hline
\textbf{\Tstrut Fellow} &
\textbf{Host} &
\textbf{\phd} &
\textbf{Start} &
\textbf{Duration} &
\textbf{Deliverables}\tabularnewline 
ESR8 &  \cnrs & Yes & Month 6& 36 & , YY \tabularnewline
\hline
\multicolumn{4}{|l|}{\textbf{\Tstrut Work Package:}
WP3, WP4, WP5, WP6, WP7} &
\multicolumn{2}{l|}{\textbf{Doctoral programme:} \parisUlong }\tabularnewline\hline
\multicolumn{6}{|p{20.2cm}|}{\textbf{\Tstrut Project Title: Real-time particle trajectory reconstruction for online event selection and analysis in ATLAS}
}\tabularnewline\hline
\multicolumn{6}{|p{20.2cm}|}{\textbf{\Tstrut Objectives:}
This project aims to train researchers in the usage of advanced computing techniques for real-time analysis in the fields
of Physics and mobile platforms, as well as to contribute to searches for physics beyond the Standard Model ("new physics") 
with the ATLAS detector. The main focus of ESR8 is to study and improve the real time reconstruction of the trajectories of charged particles
in the ATLAS experiment using the hardware tracking processor FTK and FTK++.
The tracking performances such as efficiency and resolution of the parameters are not determined
just by the hardware capabilities, but mostly by a database (pattern bank and geometrical constants)
loaded into FTK/FTK++ and produced by a training procedure. By improving and tuning the training procedure
it is possible to continuously improve the tracking performance and optimize its use for new physics searches.
The first objective of the ESR will be to develop advanced statistical and computing techniques,
such as Principal Component Analysis and Graph Clustering toolkits, in order to produce new databases
for FTK and evaluate their impact on raw performances and physics analyses. As a second objective, the developed toolkits will
also allow to extend the application of FTK-like algorithms to applications outside the field of High Energy Physics,
such as real-time image recognition or genomic data analysis, by providing flexible and powerful tools to train datasets.
These inter-sector toolkits will be developed during the secondment at \fleetmatics, and tested at \pisalong. 
The geographical proximity of the two secondments allows for a seamless integration of the work done in both research and 
industrial context. 
The four months industrial secondment at \fleetmatics will involve online learning, i.e. continuous training of 
machine learning models based on streams of labelled data, and will exploit both the \apachespark\enspace environment for parallel programming and the Amazon Web Services cloud computing infrastructure for massively parallel computations.
The student will develop an online learning tool to continuously process real-time GPS data from Fleetmatics customers.
The developed tool will be exploited to provide customers with smart insights, improving customer experience and thus
increasing engagement with the Fleetmatics products. 
After the secondment, the student will have acquired expertise in the Spark processing framework, specifically involving machine learning on data streams, and in the Amazon Web Services infrastructure. 
The acquired skills will be exploited later in the main research project, applying the computing techniques learned at Fleetmatics to the FTK working dataset production.
The third objective of the ESR will be to investigate in detail the impact of new training on selected physics cases.
In particular FTK tracks will be used to improve the jet reconstruction and calibration, namely for the suppression
of pile-up jets and the track-based components of the global sequential calibration. This will enhance the
sensitivity of the new physics searches using a real-time trigger level analysis of dijet mass distributions.
}\tabularnewline\hline
\multicolumn{6}{|p{20.2cm}|}{\textbf{\Tstrut Expected Results:}
The research of ESR8  will lead to two peer-reviewed papers, one documenting the toolkits developed
and the improvements on the tracking performance (Deliverable~\deliverableHEPPubATLASFTK), and one documenting the real-time trigger level
analysis of dijet mass distributions using FTK tracks (Deliverable~\deliverableHEPPubATLASTLAFTK). The toolkits developed for the training of
FTK datasets will be adapted for usage outside of the ATLAS experiment and released (Deliverable~\deliverableToolkitTrainingFTK). 
ESR8 will receive a PhD in experimental HEP at \parisU.
}\tabularnewline\hline
%\multicolumn{6}{|p{20.2cm}|}{\textbf{Doctoral program:} Cambridge}\tabularnewline\hline
\multicolumn{6}{|p{20.2cm}|}{\textbf{\Tstrut Secondments:}
\fleetmatics, 4 months, Dr. F. Sambo, development of an online learning tool for GPS data processing.
At the end of the secondment, ESR8 will be familiar with the Apache Spark technology and with elements of parallel computing.
\pisa, 3 months, Dr. C. Roda and A. Annovi, use of toolkits for creation of FTK pattern banks. 
}\tabularnewline
\hline
\end{tabular}
}%
\end{center}
%\end{table}
%

%The ATLAS trigger infrastructure has an unique hardware processor to reconstruct the trajectory (tracking) of the charged particles that cross the silicon inner tracker of the experiment. The tracking information is an essential tool for effective real-time event selection and has a central role in the whole ATLAS physics program especially in the HL-LHC phase. The current hardware processor, FTK, is a complex system made by several custom electronics  boards based on FPGAs and Associative Memory chips. The latter are unique computing devices developed for the FTK algorithm. The hardware tracking will be also a central part of the Phase-II Upgrade of ATLAS, with upgraded version of FTK called FTK++.


