%\begin{table}[h]

%\begin{table}[h]
\begin{center}\small
\resizebox {\textwidth }{!}{%
\begin{tabular}{|p{16mm}|p{33mm}|p{28mm}|p{18mm}|p{18mm}|p{67mm}|}
\hline
\textbf{\Tstrut Fellow} &
\textbf{Host} &
\textbf{\phd} &
\textbf{Start} &
\textbf{Duration} &
\textbf{Deliverables}\tabularnewline 
ESR7 &  Dortmund & Yes & Month 6& 36 & \deliverableHEPPubLFVUnflavored, \deliverableHEPPubLFVTrigger, \deliverableComputingOptimisation \tabularnewline
\hline
\multicolumn{4}{|l|}{\textbf{\Tstrut Work Package:}
WP3, WP5, WP6, WP7} &
\multicolumn{2}{l|}{\textbf{Doctoral programme:} \dortmund }\tabularnewline\hline
\multicolumn{6}{|p{20.2cm}|}{\textbf{\Tstrut Project Title: Event Triggering in LHCb}
}\tabularnewline\hline
\multicolumn{6}{|p{20.2cm}|}{\textbf{\Tstrut Objectives:}
%Trigger systems of modern HEP experiments reduce the event data
%processed in real time by several orders of magnitude. %
%The paradigm so
To date, modern HEP triggers have first reconstructed objects (e.g. jets or
tracks), and then performed a selection on these objects. ESR7 will develop triggers based on global event properties
rather than objects, enabling the event to be classed as interesting without the time-intensive reconstruction of
objects. Initial studies in the ERC Starting grant of
J. Albrecht have shown that this approach has a great potential to
identify interesting event characteristics, such as
the primary vertices where protons have collided, however much more detailed studies need
to be performed. If successful, event triggers can allow new classes
of physics selections and therefore open new windows for searches for
physics beyond the SM.  
The first objective of ESR7 will be to analyze the direct
pattern of detector measurements to identify interesting primary
vertices and generally events with an enhanced content of interesting
physical processes. This initial analysis will be used to design the
event trigger selection and benchmark its performance against a more
traditional object-based approach.  
Through this ESR7 will be trained in state-of-the-art trigger selections,
also developing expertise crucial for physics analyses. 
The experience in holistic event selections in RTA obtained through
the first part of this project will be used to work with
\wildtree. There, ESR7 will learn to deploy their global analysis on monitoring streams of computing clusters to predict failures, which
can potentially save a great deal of money to cluster operators.
As part of this ESR7 will be trained in modern AI methods such as deep learning using recurrent neural networks, which have shown great success in speech recognition or translation, and use them to further improve their global analysis tools.
%expensive equipment, for
%example in computing clusters. 
%This equipment produces a large amount of different monitoring streams
%that can be used to predict failures before they occur. 
%Each second the equipment is not running costs the owner large amounts
%of money.
%State of the art models of failure prediction rely on humans to create
%relevant, high level features based on raw data. In this
%secondment we will apply deep learning technology to remove this need
%for human expertise. Motivated by the success of deep learning in
%other fields like human speech recognition and translation we will
%apply deep recurrent neural networks to this problem. 
%Hence the ESR
%will learn to apply modern AI methods in real time in an industrial
%environment. 
In the academic secondment, as their second objective, ESR7 will apply these methods
%developed at the secondment at \wildtree 
to all computing clusters in
the \acronym network, assisted by the the Lund computing
group and performing the first implementation in Lund. 
This will train the ESR to apply
developed methods to real-world problems
The third and final objective of ESR7 will be the analysis of
semileptonically decaying beauty decays, that can be used to test
Lepton Flavour Universality (LFU). These decays are frequent enough to allow the event trigger developed by ESR7 to be benchmarked against more traditional approaches, providing more real-world experience. 
}\tabularnewline\hline
\multicolumn{6}{|p{20.2cm}|}{\textbf{\Tstrut Expected Results:}
ESR7 will publish two peer-reviewed papers. The first one will describe the improvements to the trigger selection 
(Deliverable \deliverableHEPPubLFVTrigger) and the second one will contain the results of the search for LFV in decays of unflavoured mesons
(Deliverable \deliverableHEPPubLFVUnflavored). The 
deep recurrent neural network based framework for the 
optimisation of power consumption and monitoring of computing clusters
will be released as open source software (Deliverable \deliverableComputingOptimisation). ESR7 will receive a PhD in experimental HEP at \dortmund.
}\tabularnewline\hline
%\multicolumn{6}{|p{20.2cm}|}{\textbf{Doctoral program:} Cambridge}\tabularnewline\hline
\multicolumn{6}{|p{20.2cm}|}{\textbf{\Tstrut Secondments:}
3 months at \wildtree, Dr. Tim Head. Topic: deep recurrent neural
networks applied to the optimisation of power consumption and
monitoring of computing clusters. 
2 months at \lundlong, Dr. Caterina Doglioni. Topic: Implementation of
the methods developed at \wildtree in the Lund computing cluster.
}\tabularnewline
\hline
\end{tabular}
}%
\end{center}
%\end{table}
%
