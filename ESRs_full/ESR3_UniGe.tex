%\begin{table}[h]

\begin{center}\small
\resizebox {\textwidth }{!}{%
\begin{tabular}{|p{16mm}|p{33mm}|p{28mm}|p{18mm}|p{18mm}|p{67mm}|}
\hline
\textbf{\Tstrut Fellow} &
\textbf{Host} &
\textbf{\phd} &
\textbf{Start} &
\textbf{Duration} &
\textbf{Deliverables}\tabularnewline 
%Add list of deliverables here
%2.1 & First \acronym conference proceedings & R & PU & 19 & 2 & \saclay & Write and publish proceedings of the first \acronym conference\tabularnewline\midrule 
%2.2 & Toolkit for deep learning & R & PU & 44 & 2 & \saclay & Implementation and release of toolkit for deep learning within HEP\tabularnewline\midrule
%2.3 & Medical insurance provision & R & PU & 44 & 2 & \dq & Application of developed methods to medical insurance provision\tabularnewline\midrule
%2.4 & Deep learning documentation & R & PU & 12,24,36,48 & 2 & \saclay & Publication of research in peer-reviewed journals\tabularnewline\midrule 
%3.1 & Second \acronym conference proceedings & R & PU & 43 & 3 & \dortmund & Proceedings of the second \acronym conference\tabularnewline\midrule
%3.2 & Novel MVA triggers& R & PU & 44 & 3 & \dortmund & Implement novel MVA strategies in LHCb trigger\tabularnewline\midrule
%3.3 & Review of DS methods & R & PU & 44 & 3 & \saclay & Review paper on relationship between DS methods and datasets\tabularnewline\midrule
%3.4 & New figures of merit documentation & R & PU & 12, 24, 36, 48 & 3 & \dortmund & Publication of research in peer-reviewed journals\tabularnewline\midrule
ESR3 &  \unigelong & Yes & Month 6& 36 &\deliverableLLPTrackingToolkit, \deliverableTechPubLLPGPU, \deliverableHEPPubLLP, \deliverablePredictiveMaintenance \tabularnewline
\hline
\multicolumn{4}{|l|}{\textbf{\Tstrut Work Package:}
WP3, WP4, WP5, WP6, WP7} &
\multicolumn{2}{l|}{\textbf{Doctoral programme:} \unige }\tabularnewline\hline
\multicolumn{6}{|p{20.2cm}|}{\textbf{\Tstrut Project Title: 
Machine learning for pattern recognition in searches for exotic physics at the LHC ATLAS experiment}
}\tabularnewline\hline
\multicolumn{6}{|p{20.2cm}|}{\textbf{\Tstrut Objectives:}
%Objective 1: ML-based tracking
%Objective 2: ML-based IoT
%Objective 3: Are GPU useful?
%Objective 4: Apply ML and/or GPU to LLP searches
One of the biggest challenges in hadron collider physics is the presence of multiple proton interactions 
that occur in every bunch collision. This creates extremely 
busy images in the detectors, that need to be deciphered fast and efficiently. Reconstructing 
particle tracks under these conditions in real-time is a major task that %becomes even more challenging in
%terms of real-time needs of the trigger system. 
%This challenge will only 
will only increase in the future, with the planned LHC upgrade. 
The first objective of ESR3 is the development of ML-based tracking reconstruction as a 
replacement to existing tracking algorithms that are too slow to be used in real time. ESR3 will be trained as an expert in both track reconstruction and modern ML techniques and tools for pattern recognition. The ML expertise acquired in this process will 
be crucial for ESR3's second objective, carried out during their secondment, in which they will utilise these techniques to collect and analyse 
data from Internet-Of-Things-ready industrial production chains in order to improve real-time data analysis and forecast 
when the production machinery needs intervention or replacements.  
Track reconstruction is an inherently parallelisable task, but the optimal architecture on which to deploy it must also be evaluated.  The third 
objective of ESR3 is the evaluation of GPUs for track reconstruction, especially at higher 
pile-up conditions of the LHC upgrades. The ESR will also compare the performance of their GPU-optimized ML reconstruction to both CPU-based reconstruction and dedicated hardware solutions (e.g. FPGAs) proposed for use in ATLAS. Through these activities the ESR will be trained in optimizing algorithms for modern computing architectures.
%Currently, hardware tracking is planned as an alternative to software tracking 
%for triggering purposes. The option of using GPUs with fast machine-learning-based tracking has never been 
%evaluated, and the ESR assigned to this project will have the right expertise to answer this question. 
The fourth objective of the project will be to apply the previous developments to a novel real-time selection of displaced vertices in ATLAS. 
This is one of the most promising signatures for new physics and yet one of the most experimentally 
challenging. The ESR will use ATLAS data collected with that real-time selection to perform a unique search for 
exotic long-lived signatures. 
This represent a significant step beyond ATLAS's existing capabilities, will answer crucial questions for the future of the
experiment, and will open new avenues in the searches for new physics. The work will be conducted within the
ATLAS \unige team and will be well integrated in the ATLAS experiment thanks to the geographical proximity to CERN. 
}\tabularnewline\hline
\multicolumn{6}{|p{20.2cm}|}{\textbf{\Tstrut Expected Results:}
ESR3 will produce software for the ATLAS experiment for tracking reconstruction at the trigger level, including the 
case of long-lived particles (Deliverable \deliverableLLPTrackingToolkit).
ESR3's research is expected to yield two original research papers: a technical publication
of the ML-based track reconstruction and its evaluation on GPUs, and a physics 
publication documenting the results of the search for new physics using this technique (Deliverables
\deliverableTechPubLLPGPU and \deliverableHEPPubLLP). 
ESR3 will receive a PhD in experimental HEP at \unigelong. 
%TODO: link this with other GPU-related things
}\tabularnewline\hline
\multicolumn{6}{|p{20.2cm}|}{\textbf{\Tstrut Secondments:}
\lightboxlong, 6 months, Dr. P. Catastini. Machine learning for real-time data acquisition and analysis of industrial sensor data
and development of a predictive maintenance software framework (Deliverable \deliverablePredictiveMaintenance). 
The ESR will acquire knowledge of 1) how industrial production chains are organized; 2) the type of sensors deployed by
IoT ready production plants, how they work and how they are connected; 3)
how to collect and aggregate sensor data in real-time; 3) analysis
techniques used in forecasting parts breakdowns and failures.
}\tabularnewline
\hline
\end{tabular}
}%
\end{center}

%Text for the rest of the application

%Internet of Things (IoT) refers to the use of sensors and other Internet-connected devices to track and control physical objects through the industrial production chain and their subsequent end-user delivery steps. Through IoT, companies may monitor 1) machine status and performance continuously and 2) schedule maintenance only when necessary. The training project we propose is related to 1) and 2).
%In particular the deployment of sensors and systems that are connected and can exchange information allows companies to acquire real-time information of the operational status of each critical component in an industrial production chain. The combination of real-time and historical measurements data is used to infer when the production chain is getting close to a ?critical? status that may require actions and to predict when a specific part of a machine will have to be fixed or replaced. The adoption of these techniques, called predictive maintenance, on average is estimated to reduce maintenance costs by more than 25\%, reduce breakdowns by more than 70\%, reduce downtime by more than 35\% and increase productivity by more than 20\%.

%The data acquired in real-time by the connected sensors are of different nature, unstructured and complex, such as: parts vibration data, lubricant and fuel quality, wear particle data, temperature measurements, ultrasonic noise detection and flow, infrared thermorgraphy, electrical monitoring, etc. These data need to be collected, cleaned, aggregated and analyzed in real-time using both complex event processing infrastructure and statistical analysis techniques. 

%The project will focus on a specific type of production chain with the goal of improving real-time data analysis and forecast by means of machine learning (ML) techniques. Given the heterogeneity of the sensor data, ML techniques are beneficial in forecasting when machinery parts need intervention or should be replaced. 


