%\begin{table}[h]

%\begin{table}[h]
\begin{center}\small
\resizebox {\textwidth }{!}{%
\begin{tabular}{|p{16mm}|p{33mm}|p{28mm}|p{18mm}|p{18mm}|p{67mm}|}
\hline
\textbf{\Tstrut Fellow} &
\textbf{Host} &
\textbf{\phd} &
\textbf{Start} &
\textbf{Duration} &
\textbf{Deliverables}\tabularnewline 
ESR6 &  Dortmund & Yes & Month 6& 36 & \deliverableHEPPubLFVTrigger, \deliverableHEPPubLFVUnflavored \tabularnewline
\hline
\multicolumn{4}{|l|}{\textbf{\Tstrut Work Package:}
WP3, WP5, WP7} &
\multicolumn{2}{l|}{\textbf{Doctoral programme:} \dortmund }\tabularnewline\hline
\multicolumn{6}{|p{20.2cm}|}{\textbf{\Tstrut Project Title: Real-time
    implementation of multivariate analysis for LFV in unflavoured meson decays}
}\tabularnewline\hline
\multicolumn{6}{|p{20.2cm}|}{\textbf{\Tstrut Objectives:}
The objective of this ESR is to advance the level of studies of Lepton
Flavour Violation (LFV) to the next level. Currently, several
experiments are proposed and constructed to investigate LFV.  So far,
no process violating Lepton Flavour is observed. This ESR will search
for LFV in neutral meson decays in the dataset of LHCb. Current tests
by older experiments reach a precision of about one in a million for
decays of $\phi$, $J/\psi$ and $Y(1S)$ mesons, for example. At the
upgraded LHCb detector, which will start data taking in 2020, many
orders of magnitude more of these mesons are produced, allowing in
principle to push the precision by orders of magnitude. Challenges are
the difficult separation between these decays and the overwhelming
background levels. 
The first objective of the ESR will be a development of a real time
selection for $\phi$, $J/\psi$ and $Y(1S)$ mesons decaying into the different
lepton species. Because of the difficult experimental signatures, fast
multivariate selections need to be developed. An extension of this
work is the development of unified RTA selections for these LFV decay
modes. 
%that inclusively select all particle decays to two different lepton
%species. 
This will be done in collaboration with ESR5.  
The third objective of the ESR will be the analysis and publication of
the data prepared in the first stage. The dataset has the potential to
boost the precision of these tests for LFV by several orders of
magnitude. This work will be conducted within, but will represent a
significant extension to the ERC Starting grant that J. Albrecht holds
at TU Dortmund.   
The ESR will get trained in RTA methods used in the LHCb trigger
system. The CERN secondment will allow the ESR to collaborate closely
with ESR5 that works on the related subject of LFV decays of
$\tau^-$-leptons. 
ESR6 will become one of the trigger experts of LHCb's rare decay
group. The ESR will profit from a secondment at Ximantis, which will
cross-fertilise the real-time use of AI in the CERN experiment and in
industry. 
}\tabularnewline\hline
\multicolumn{6}{|p{20.2cm}|}{\textbf{\Tstrut Expected Results:}
ESR6 will publish two peer-reviewed papers. The first one will be
written together with ESR6 on the improvements to the trigger selection 
(Deliverable \deliverableHEPPubLFVTrigger) and the second one will describe the results of the search for LFV in decays of unflavoured mesons
(Deliverable \deliverableHEPPubLFVUnflavored). ESR6 will receive a PhD in experimental HEP at \dortmund.
}\tabularnewline\hline
%\multicolumn{6}{|p{20.2cm}|}{\textbf{Doctoral program:} Cambridge}\tabularnewline\hline
\multicolumn{6}{|p{20.2cm}|}{\textbf{\Tstrut Secondments:}
CERN, 5 months, Dr. Frederic Teubert. Topic: joint development and
implementation of LFV trigger selections. Ximantis, 4 months,
Dr. Alexandros Sopasakis. Enhancement of the existing artificial
intelligent modelling algorithms for traffic prediction, using
Convolutional Neural Networks (CNNs). The ESR will gain hands-on
experience within the training of those neural networks, as well as
theoretical knowledge about the field of ML, AI and best current
practices in general. 
}\tabularnewline
\hline
\end{tabular}
}%
\end{center}
%\end{table}
%
