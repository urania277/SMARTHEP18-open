\begin{center}
\footnotesize
\begin{tabular}{|p{0.1\textwidth}|p{0.85\textwidth}|}
%\resizebox{\textwidth}{!}{%
%%\begin{tabular}{@{}p{25mm}|p{190mm}@{}}
\toprule
\multicolumn{2}{c}{\large\textbf{Partner organization: \oregonlong}}\tabularnewline\hline
\pbox{8cm}{\Tstrut General\\Description\Bstrut} & %
\pbox{0.85\textwidth}{\Tstrut 
 The  \oregonlong was founded in 1876 in Eugene, Oregon, USA.  Today it enrolls more than 
24,000 students from all 50 states in the US and from 95 countries.   The Department of 
Physics has more than 30 research faculty members and is especially well known in the fields 
of optics (Nobel prize winner David Wineland will be joining the faculty later in 2018), soft 
condensed matter and particle physics. The particle physics phenomenology group at Oregon includes
well-known particle-physics phenomenologists, the LIGO gravitational wave detector, the Linear Collider collaboration, and ATLAS experiment
with members in positions of responsibility. 
%Chang, Cohen, Kribs and Soper.  
%The experimental group was founded in 1989 and consists of: Brau,
%Frey, Farr, Jeanty (starting later in 2018), Majewski, Strom and
%Torrence. Frey and Farr mainly work on the LIGO gravitational wave
%detectors.  Brau is the Associate Director of the Linear Collider
%Collaboration. Oregon particle physics group members have been elected to positions on the ATLAS
%Executive Board that include Data Preparation Coordinator (Torrence), Trigger Coordinator (Strom, Winklmeier) and TDAQ Project Leader (Strom).
\Bstrut}\tabularnewline\hline

\pbox{8cm}{\Tstrut Role and\\Commitment\\of Key persons} & %
{\vspace{-5mm}
\begin{enumerate}%[topsep=0pt,itemsep=-2pt,leftmargin=*]
\item Prof. David Strom is Professor of Physics.% at the University of Oregon.
  He was ATLAS Deputy Trigger Coordinator from 2010-2011, ATLAS
  Trigger Coordinator from 2011-2012.  
%In 2013 he became a member of the the ATLAS Trigger Data Acquisition
%Team (TDMT).  As part of the TDMT he served on the ATLAS Executive
%Board in 2015-2016 and continues to serve in 2017-2018.   
During this period Strom was in charge of all aspects of ATLAS TDAQ
including operations, Phase 1 upgrade and Phase 2
upgrade. \textbf{Role:} second supervisor of \ESRl and \ESRh. Committment: 10\%
% \item Stephanie Majewski is Associate Professor of Physics.% at the
%                                 % University of Oregon.  
% She is Level 3 manager for US ATLAS Phase 1 upgrade of the ATLAS
% Liquid Argon Calorimeter and is Level 3 manager of the US ATLAS
% HL-LHC Global Trigger Firmware upgrade, as well as %responsible for
% %the topological clustering firmware for the Global Trigger. She is
% %also 
% editor of the ATLAS TDAQ Technical Design Report.% which is
% %currently under preparation. 
% \textbf{Role:} supervision of seconded student (ESR15). Committment: 10\%
\vspace{-2mm}%\belowdisplayskip}
\end{enumerate}} \tabularnewline\hline

\pbox{8cm}{\Tstrut Key Research\\Facilities,\\Infrastructure\\and Equipment} & %
\pbox{0.85\textwidth}{\Tstrut 
The facilities at the University of Oregon relevant to the particle physics group include the Center for Advanced Materials Characterization in Oregon as well electrical and mechanical workshops. The supervision of seconded students will take place in the Oregon offices at CERN. 
} \tabularnewline\hline
%
\multicolumn{2}{l}{\hspace{-1ex}Independent \Tstrut  research premises\Bstrut: yes}\tabularnewline\hline
%\pbox{8cm}{\Tstrut Past \& current\\involvement\\in Research and\\Training\\Programmes} & 
% \pbox{0.85\textwidth}{\Tstrut 
% \oregon has not previously been involved with EU research projects or grants.
% } \tabularnewline\hline\Tstrut
\pbox{8cm}{\Tstrut Relevant\\Publications} &%
{\vspace{-3mm}
\begin{description}%[topsep=0pt,itemsep=-2pt,leftmargin=*]

\item [ATLAS Collaboration], Search for a scalar partner of the top quark in the jets plus missing transverse momentum final state at $\sqrt{s}$=13 TeV with the ATLAS detector, JHEP {\bf 1712} (2017) 085. 
\item [ATLAS Collaboration],Search for new phenomena in dijet events using 37 fb$^{-1}$ of $pp$ collision data collected at $\sqrt{s}=$13 TeV with the ATLAS detector, Phys. Rev. {\bf D96 } (2017) 052004.
\item [ATLAS Collaboration], Performance of the ATLAS Trigger System in 2015, 
Eur. Phys. J. C {\bf 77} (2017) 317. 
%\item [ATLAS Collaboration], Search for quantum black-hole production in high-invariant-mass lepton-jet final states at $\sqrt{s} = 8$\,TeV with the ATLAS detector,  Phys. Rev. Lett. {\bf 112} (2014) 091804.  
%\item [ATLAS Collaboration], Observation of a new particle in the search for the Standard Model Higgs boson with the ATLAS detector at the LHC,  Phys. Lett. B {\bf 716}, 1 (2012). 

\end{description}}\tabularnewline\bottomrule

\end{tabular}
%}%
\end{center}
