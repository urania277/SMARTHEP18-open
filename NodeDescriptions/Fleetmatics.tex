\begin{center}
\footnotesize
\begin{tabular}{|p{0.1\textwidth}|p{0.85\textwidth}|}
%\resizebox{\textwidth}{!}{%
%\begin{tabular}{@{}p{25mm}|p{190mm}@{}}
\toprule
\multicolumn{2}{c}{\large\textbf{Partner organization: \fleetmatics}}\tabularnewline\hline
\pbox{8cm}{\Tstrut General\\Description\Bstrut} & %
\pbox{0.85\textwidth}{\Tstrut 
%KKT is the Italian subsidiary of Verizon Connect. KKT became part of Verizon Connect in 2016, after having acquired Fleetmatics. 
\fleetmatics is an Italian company, part of the global Verizon Connect group, that focuses on data science and machine learning research and development. \fleetmatics develops software products in the domain of Data Science, Machine Learning and Connected Vehicles.
\fleetmatics specializes in particular in the development of software and data products in the domain of machine learning applied to transportation, connected vehicles, traffic safety. The 30+ employees of \fleetmatics have substantial expertise especially in Machine Learning, Computer Vision, Mathematical Optimization, and Software Development. The company carries out a number of industrial research activities and scientific collaborations, including an industrial PhD program in collaboration with the University of Florence, scientific cooperations with several other universities, participation in European projects.
\Bstrut}\tabularnewline\hline

\pbox{8cm}{\Tstrut Role and\\Commitment\\of Key persons} & %
{\vspace{-5mm}
\begin{enumerate}%[topsep=0pt,itemsep=-2pt,leftmargin=*]
\item Dr. Francesco Sambo holds a Ph.D. in Artificial Intelligence and Bioinformatics from the university of Padova, Italy. He has been PostDoctoral researcher at the University of Padova for 6 years and is author or co-author of more than 30 papers on scientific journals or international scientific conferences. Since October 2015, he is Senior Data Scientist at \fleetmatics. Francesco will be the main industrial advisor of the PhD candidate recruited for the project.
Commitment: 15\%

\item Leonardo Taccari is a lead scientist in the Data Science Team. He has a Ph.D. in Mathematical Optimization from the Politecnico of Milan, Italy. He is author or co-author of more than 15 papers on scientific journals or international scientific conferences. Since October 2015, he is a Senior Data Scientist at \fleetmatics.
Commitment: 15\%

\vspace{-2mm}%\belowdisplayskip}
\end{enumerate}} \tabularnewline\hline

\pbox{8cm}{\Tstrut Key Research\\Facilities,\\Infrastructure\\and Equipment} & %
\pbox{0.85\textwidth}{\Tstrut 
KKT is located in an office in Florence. The site has been declared strategic at a worldwide level by the relevant Business Lines. The staff comprises 10+ PhDs and 20+ Software engineers.
Supermicro SuperServer 2028GR-TRT with NVidia Tesla P100 GPU as computing and prototyping platform,
Microsoft Analytics Platform System data warehouse for fast data retrieval. 
Premises: open office space + 3 video conference rooms, training room, study space and kitchen. 
} \tabularnewline\hline
%
\multicolumn{2}{l}{\hspace{-1ex}Independent \Tstrut  research premises\Bstrut: yes}\tabularnewline\hline
\pbox{8cm}{\Tstrut Past \& current\\involvement\\in Research and\\Training\\Programmes} & 
\pbox{0.85\textwidth}{\Tstrut 
Since it was founded by a professor of the University of Florence and 3 PhDs, \fleetmatics has hired several PhD students and started an industrial graduate program with the University of Florence (1 current PhD student). 
KKT currently hosts 1 PhD student and 2 Master interns, and will recruit 1 more PhD student per year for the next 3 years.
} \tabularnewline\hline\Tstrut
\pbox{8cm}{\Tstrut Relevant\\Publications} &%
{\vspace{-3mm}
\begin{itemize}%[topsep=0pt,itemsep=-2pt,leftmargin=*]

\item L. Taccari, F. Sambo, L. Bravi, S. Salti, L. Sarti, M. Simoncini, A. Lori. Classification of Crash and Near-Crash Events from Dashcam Videos and Telematics. IEEE 21st International Conference on Intelligent Transportation Systems, ITSC 2018;

\item M. Simoncini, L. Taccari, F. Sambo, L. Bravi, S. Salti, A. Lori. Vehicle classification from low-frequency GPS data with recurrent neural networks. Transportation Research Part C: Emerging Technologies 91, 176-191, 2018;

\item F. Sambo, S. Salti, L. Bravi, M. Simoncini, L. Taccari, A. Lori, Integration of GPS and satellite images for detection and classification of fleet hotspots. IEEE International Conference on Intelligent Transportation Systems (ITSC), 2017

\item Stop Purpose Classification from GPS Data of Commercial Vehicle Fleets
L Sarti, L Bravi, F Sambo, L Taccari, M Simoncini? - IEEE Conference of Data Mining Workshops (ICDMW), 2017

\item Sarti  L, Sambo F, Salti S, et al.  Stop Purpose Classification from GPS Data of Commercial Vehicle Fleets.  IEEE Conference of Data Mining Workshops (ICDMW), 2017

%\item Sambo F, Salti S, et al.  Integration of GPS and Satellite Images for Detection and Classification of Fleet Hotspots. ITSC conf.proc., 2017
%\item  Simoncini M, Sambo F, Salti S, et al. Vehicle Classification from Low Frequency GPS data. ICDMW 2016
%\item Palossi D, Salti S, et al.GPU-SHOT: parallel optimization for
%  real-time 3D local description. IEEE CVPRW 2013.
%
%\item Sarti  L, Bravi L, Sambo F, Taccari L, Simoncini M, Salti S, Lori A. Stop Purpose Classification from GPS Data of Commercial Vehicle Fleets. ICDMW conference proceedings, 2017
%\item Sambo F, Salti S, Bravi L, Simoncini M, Taccari L, Lori A.  Integration of GPS and Satellite Images for Detection and Classification of Fleet Hotspots. ITSC conference proceedings, 2017
%\item  Simoncini M, Sambo F, Taccari L, Bravi L, Salti S, Lori A. Vehicle Classification from Low Frequency GPS data. ICDMW 2016
%\item Palossi D, Tombari F, Salti S, Ruggiero M, Di Stefano L, Benini L. GPU-SHOT: parallel optimization for real-time 3D local description. IEEE CVPRW 2013.
\end{itemize}}\tabularnewline\bottomrule

\end{tabular}
%}%
\end{center}
