\begin{center}
\footnotesize
\begin{tabular}{|p{0.1\textwidth}|p{0.85\textwidth}|}
%\resizebox{\textwidth}{!}{%
%\begin{tabular}{@{}p{25mm}|p{190mm}@{}}
\toprule
\multicolumn{2}{c}{\large\textbf{Partner organization: \ximantis}}\tabularnewline\hline
\pbox{8cm}{\Tstrut General\\Description\Bstrut} & %
\pbox{0.85\textwidth}{\Tstrut 
\ximantis is a Swedish traffic forecasting company founded in 2014 and is part of the Lund University Innovation System. \ximantis produces forecasts of upcoming traffic congestion in real time thus allowing users to avoid them. The forecasting capabilities of \ximantis have been repeatedly tested and validated on different occasions in real traffic with data provided by the Federal Traffic Safety Administration of the USA. It has been shown to be capable to produce detailed traffic evolution and congestion information at specific, chosen, road locations and specific future times. This capability allows for significant savings in time and energy resources. Given the huge impact of traffic in $CO_2$ emissions and other harmful particulates as well as the waste of precious energy recourses due to congestion the environmental benefits proposed by the \ximantis forecasting provide clear incentives for this technology.
\Bstrut}\tabularnewline\hline

\pbox{8cm}{\Tstrut Role and\\Commitment\\of Key persons} & %
{\vspace{-5mm}
\begin{enumerate}%[topsep=0pt,itemsep=-2pt,leftmargin=*]
\item Dr. Alexandros Sopasakis,
% received his Bachelors (1992), Masters (1996) and PhD (2000) in Mathematics from Texas A\&M University. He subsequently held a postdoc position at Chalmers (2000-2001) funded under a mobility program from EU and academic positions in mathematical research at Georgia Tech, Berkeley, New York University and University of Massachusetts. From 2007-2011 he held a tenure track position at the University North Carolina and   
since 2011 a tenured position at Lund University where he also received his Lecturer qualification. A. Sopasakis received two National Science Foundation (NSF) grants as main PI and several grants in the USA and Sweden as co-researcher. 
Because of his diverse profile in research, Dr. Sopasakis has been able to work in many areas including descriptions of micromagnetic behavior, planetary weather prediction, multiscale particle interactions and traffic flow evolution. 
He is the CEO and Founder of \ximantis. \textbf{Role:} industrial
supervisor of \ESRd, \ESRh, \ESRk and \ESRl. Commitment: 10\%  
%\vspace{-\belowdisplayskip}
\vspace{-2mm}
\end{enumerate}
} \tabularnewline\hline
\pbox{8cm}{\Tstrut Key Research\\Facilities,\\Infrastructure\\and Equipment} & %
\pbox{0.85\textwidth}{\Tstrut 
Ideon innovation offices, offices at Lund University. Key computing infrastructure: supercomputer machine with 54 Intel CPUs a Tesla K40 GPU and 2 RTX 2080 GPUs. 
} \tabularnewline\hline
%
\multicolumn{2}{l}{\hspace{-1ex}Independent \Tstrut  research premises\Bstrut: yes}\tabularnewline\hline
\pbox{8cm}{\Tstrut Past \& current\\involvement\\in Research and\\Training\\Programmes} & 
\pbox{0.85\textwidth}{ \Tstrut 
\ximantis received H2020 phase 1 SME grant in 2018  and Vinnova grant for SMEs in 2016
} \tabularnewline\hline\Tstrut
\pbox{8cm}{\Tstrut Relevant\\Publications} &%
{
\vspace{-3mm}
\begin{itemize}%[topsep=0pt,itemsep=-2pt,leftmargin=*]
\item A. Sopasakis, MA Katsoulakis, Information metrics for improved traffic model fidelity through sensitivity analysis and data assimilation. Transportation Research Part B: Methodological 86, 1-18, 2016
\item A. Sopasakis, Novel Updating Mechanisms for Stochastic Lattice-Free Traffic Dynamics, Progress in Industrial Mathematics at ECMI 2012, 285-289, 2014
%\item A. Sopasakis, Traffic updating mechanisms for stochastic lattice-free dynamics, Procedia-Social and Behavioral Sciences 80, 837-845, 2013
\item A. Sopasakis, Lattice Free Stochastic Dynamics, Communications in Computational Physics 12(3). p.691-702, 2012
%\item Ximantis Mobile app for traffic forecasting, currently tested in the AWS webserver, for the iTunes Apple store.
\vspace{-2mm}
\end{itemize}
}\tabularnewline\bottomrule
\end{tabular}
%}%
\end{center}