\begin{center}
\footnotesize
\begin{tabular}{|p{0.1\textwidth}|p{0.85\textwidth}|}
%\resizebox{\textwidth}{!}{%
%\begin{tabular}{@{}p{25mm}|p{190mm}@{}}
\toprule
\multicolumn{2}{c}{\large\textbf{Beneficiary: \nikhef}}\tabularnewline\hline
\pbox{8cm}{\Tstrut General\\Description\Bstrut} & %
\pbox{0.85\textwidth}{\Tstrut \nikhef is the Dutch National Institute for Subatomic Physics, coordinating and 
leading the Dutch experimental activities in the fields of accelerator-based particle physics and 
astroparticle physics, with the mission of studying the interactions and structure of all elementary 
particles and fields at the smallest distance scale and the highest attainable energy. \nikhef is a 
partnership between the Foundation for Fundamental Research on Matter (FOM, part of NWO, the 
Netherlands Organisation for Scientific Research) and four universities: Radboud University 
Nijmegen, University of Amsterdam, Utrecht University and VU University Amsterdam. The research at 
NIKHEF relies on the development of innovative technologies. The knowledge and technology transfer 
to third parties, \ie, industry, civil society and general public, is an integral part of 
\nikhef mission. 
The \nikhef collaboration consists of about 200 physicists (60 tenured staff, 40 postdocs, 100 \phd 
students), 75 technical and engineering staff and 25 support staff and has at present seven 
experimental research lines (ATLAS, ALICE, LHCb, Gravitational Waves, Dark Matter, Neutrino 
Telescopes and Cosmic Rays), and in addition research lines on Theoretical Physics, Detector R\&D 
and Physics Data Processing, supporting the experimental effort. Several members of the tenured staff
also are professor at the partner universities, and through these channels \phd students can be
awarded their doctoral degree at the partner universities.
The FOM-institute \nikhef is located 
in the Amsterdam Science Park. \Bstrut}\tabularnewline\hline
\pbox{8cm}{\Tstrut Role and\\Commitment\\of Key persons} & %
{\vspace{-5mm}
\begin{enumerate}%[topsep=0pt,itemsep=-2pt,leftmargin=*]
\item Prof.~Olga Igonkina, professor of experimental particle
physics in Radboud University of Nijmegen, senior researcher at Nikhef,
ATLAS trigger menu coordinator in Run 1, convener of ATLAS Exotics
lepton+X physics group (2013-2014), receiver of several dutch grants
(VIDI, VICI,  projectruimte). Expertise in $\tau$ reconstruction,
B-physics, lepton flavor violation (LFV), data analysis, triggering.
Role: supervisor of \ESRh, co-supervisor of \ESRi, tutoring of seconded
students, diversity and inclusion officer, commitment: 20\%.
\item Prof.~Gerhard Raven, professor of experimental particle physics, 
former LHCb Trigger project leader, LHCb time-dependent CP-violation physics group convener, 
receiver of several dutch grants. Expertise in data analysis, detector alignment, and triggering.  
Role: supervisor of \ESRi, co-supervisor of \ESRh, tutoring of seconded students, commitment: 20\%;
%\item Dr.~Matt Kenzie, physics, researcher, expertise in data analysis, statistical methods, triggering.  
%Role: co-supervisor of ESR8 and ESR6, tutoring of seconded students, commitment: 20\%;
%\item Dr.~Noam Tal Hod, physics, researcher, ATLAS exotics lepton+X convener (2014-2015), ATLAS exotics trigger liaison. Expertise in data analysis,
%reconstruction and statistics tools.
%Role: tutoring of ESR7 and seconded students, commitment: 20\%.
\vspace{-\belowdisplayskip}
\end{enumerate}} \tabularnewline\hline
\pbox{8cm}{\Tstrut Key Research\\Facilities,\\Infrastructure\\and Equipment} & %
\pbox{0.85\textwidth}{\Tstrut \nikhef has three technical divisions together with 75 staff members: Mechanic Technology (MT), Electronics Technology (ET) 
and Computing Technology CT. \nikhef is equipped with state-of-the-art tools and equipment for engineering design optimisation (3D CAD, material studies, etc.), 
analogue, digital and mixed-signal electronics and micro-electronics design, production and testing (Mentor Graphics, signal generators and analysers, etc.) 
and a powerful computing infrastructure for data processing, consisting of European EGEE Grid clusters and Giga data storage. \nikhef is the Netherlands LHC Tier 1 
and hosts the AMS-IX Internet exchange. \nikhef has  a long tradition in statistical data analysis, and is home to the RooFit Toolkit for data modeling. 
\nikhef also hosts the Particle and Astro-particle track of the joint Physics Master of the two universities of Amsterdam.} \tabularnewline\hline
\multicolumn{2}{l}{\hspace{-1ex}Independent \Tstrut  research premises\Bstrut: yes}\tabularnewline\hline
\pbox{8cm}{\Tstrut Past \& current\\involvement\\in Research and\\Training\\Programmes} & 
\pbox{0.85\textwidth}{\Tstrut \nikhef has been and is currently involved in several EU funded projects, in particular in theory, detector R\&D and e-infrastructure (computing and data processing). \nikhef also hosts several ERC (advanced) grants. NIKHEF is involved in the following Initial Training Networks: MC-PAD (completed), LHCPhenoNet, TALENT, INFIERI and HiggsTools.%
} \tabularnewline\hline\Tstrut
\pbox{8cm}{\Tstrut Relevant\\Publications} &%
{\vspace{-3mm}
\begin{itemize}%[topsep=0pt,itemsep=-2pt,leftmargin=*]
%Removed one paper because we can only quote 5
\item   O.Igonkina for ATLAS collaboration, ``ATLAS trigger menu and performance in Run 1 and
 prospects for Run 2 ``, presented at IEEE 2013, doi:10.1109/NSSMIC.2013.6829554
\item  ATLAS collaboration, O.Igonkina et al., ``Technical Design Report for the Phase-I Upgrade of the ATLAS TDAQ System'', CERN-LHCC-2013-018
\item ATLAS collaboration, O.Igonkina et al., ``Search for new
  phenomena in events with three or more charged leptons in pp
  collisions at $\sqrt{s}=8$ TeV with the ATLAS detector'',
   	JHEP08 (2015) 138
%\item O.Igonkina et al., ``Performance of the ATLAS Trigger System in 2010'', Eur.Phys.J. C72 (2012) 1849
\item J.Albrecht, C.Fitzpatrick, V.Gligorov and G.Raven,  ``The upgrade of the LHCb trigger system'', JINST 9 (2014) C10026
%\item J.Albrecht, V.Gligorov, G.Raven and S.Tolk, ``Performance of the LHCb High Level Trigger in 2012'', J.Phys.Conf.Ser. 513 (2014) 012001.
\item The LHCb collaboration, G.Raven et al., ``Precision measurement of CP violation in $\Bs\to J\psi\Kp\Km$ decays'', Phys. Rev. Lett. 114(2015)041801
% \item    O.Igonkina et al., ``The ATLAS tau trigger'', ATL-DAQ-PROC-2008-008

\end{itemize}}\tabularnewline\bottomrule
\end{tabular}
%}%
\end{center}

%\checkme{PS: Some references seem to be a left-over}
