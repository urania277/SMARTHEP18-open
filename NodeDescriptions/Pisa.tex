\begin{center}
\footnotesize
\begin{tabular}{|p{0.1\textwidth}|p{0.85\textwidth}|}
%\resizebox{\textwidth}{!}{%
%\begin{tabular}{@{}p{25mm}|p{190mm}@{}}
\toprule
\multicolumn{2}{c}{\large\textbf{Partner organization: \pisalong}}\tabularnewline\hline
\pbox{8cm}{\Tstrut General\\Description\Bstrut} & %
\pbox{0.85\textwidth}{\Tstrut 
 INFN is a research, non-profit, government organisation with about 1500 researchers in 20 research structures spread over the Italy and 5 national laboratories. INFN conducts theoretical and experimental research in sub-nuclear, nuclear and astro-particle physics within a framework of international competition and in close collaboration with Italian universities.
\Bstrut}\tabularnewline\hline

\pbox{8cm}{\Tstrut Role and\\Commitment\\of Key persons} & %
{\vspace{-5mm}
\begin{enumerate}%[topsep=0pt,itemsep=-2pt,leftmargin=*]
\item Dr. Alberto Annovi is a staff researcher at INFN Pisa, and a member of the ATLAS collaboration, where his main research interests are in tracking and trigger systems. He is coordinating the Hardware Tracking system for the ATLAS upgrade, and has been the project leader of the ATLAS Fast-Tracker upgrade. He is also a member of the CDF collaboration (trigger, electroweak physics and B-physics). 
%He has been the second supervisor of 2 undergraduate students at University of Roma3. 
He has been member of the organizing committee for the international workshops: Vertex 2016, MOCAST 2015 and 2016, Workshop on Intelligent Trackers (WIT) 2012, 2014, 2016. \textbf{Role:} additional supervisor for \ESRf. Committment: 15\%. 

\item  Prof. Chiara Roda is the Pisa ATLAS group team leader (20 members). She has or has had leading and coordination roles in ATLAS both in detector performance and in physics analysis. She has been editor for publications on jet cross-section, diboson production, exotic searches in diboson. She has supervised 4 PhD theses, 10 master theses and many diploma thesis at the University of Pisa. She co-teaches the HEP laboratory course for master students. \textbf{Role:} second supervisor for \ESRf. Committment: 10\%. 
\vspace{-\belowdisplayskip}
\end{enumerate}} \tabularnewline\hline

\pbox{8cm}{\Tstrut Key Research\\Facilities,\\Infrastructure\\and Equipment} & %
\pbox{0.85\textwidth}{\Tstrut 
EGO (European Gravitational Observatory) in Cascina (Pisa) is a Consortium run by INFN and the French CNRS. INFN manages a major distributed computing facility, CNAF, providing resources and support for INFN research. 
} \tabularnewline\hline
%
\multicolumn{2}{l}{\hspace{-1ex}Independent \Tstrut  research premises\Bstrut: yes}\tabularnewline\hline
\pbox{8cm}{\Tstrut Past \& current\\involvement\\in Research and\\Training\\Programmes} & 
\pbox{0.85\textwidth}{\Tstrut 
87 INFN projects were selected for funding in FP7 (20 in the PEOPLE program; 8 of them still running are 3 ITN and 4 IRSES). In H2020, 28 INFN projects are founded (8 are MSCA: 2 ITN, 1 IF, 4 RISE and 1 NIGHT).
INFN runs the Gran Sasso Science Institute at L'Aquila, organises several post-graduate schools and funds partially PhD programmes in Italy. 
} \tabularnewline\hline\Tstrut
\pbox{8cm}{\Tstrut Relevant\\Publications} &%
{\vspace{-2mm}
\begin{itemize}%[topsep=0pt,itemsep=-2pt,leftmargin=*]
%\item ATLAS Collaboration. The FastTracker Real Time Processor and Its Impact on Muon Isolation, Tau and b-Jet Online Selections at ATLAS. IEEE Trans. Nucl. Sci. 59, 348. (2012) %http://dx.doi.org/10.1109/TNS.2011.2179670
\item Annovi A et al. A Multi-Core FPGA-based 2D-Clustering Implementation for Real-Time Image Processing, ATL-COM-DAQ-2014-015 %& https://cds.cern.ch/record/1695793/ doi:10.1109/TNS.2014.2364183
\item Annovi A, Crescioli F et al. AM06: the Associative Memory chip for the Fast TracKer in the upgraded ATLAS detector, JINST12(2017)C04013
\item Annovi A et al. Highly Parallelized Pattern Matching Hardware for Fast Tracking at Hadron Colliders, IEEE Trans.Nucl.Sc.63(2016) 2 
%\item ATLAS Collaboration. Fast TracKer (FTK) Technical Design Report, CERN-LHCC-2013-007 ; ATLAS-TDR-021
\end{itemize}}\tabularnewline\bottomrule

\end{tabular}
%}
\end{center}