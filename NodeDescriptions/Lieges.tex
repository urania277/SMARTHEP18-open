\begin{center}
\resizebox{\textwidth}{!}{%
\begin{tabular}{@{}p{25mm}|p{190mm}@{}}
\toprule
\multicolumn{2}{c}{\large\textbf{Partner organization: \lieges}}\tabularnewline\hline
\pbox{8cm}{\Tstrut General\\Description\Bstrut} & %
\pbox{19cm}{\Tstrut 
The University of Li\`{e}ge \liegesentity is a major public university in Belgium, counting more than 24000 students and 3000 academics across all fields of research.
The Department of Electrical Engineering and Computer Science, known as the Montefiore Institute, is part of the School of Applied Sciences. The department gathers around 150 full-time equivalent whose research activities span electrical engineering, electronics, systems and modeling, optimization, machine learning and artificial intelligence. Keeping computer science within the department has always proven to be very fruitful research wise, encouraging the development of interdisciplinary topics, such as data analysis and machine learning applied to engineering problems. 
\Bstrut}\tabularnewline\hline

\pbox{8cm}{\Tstrut Role and\\Commitment\\of Key persons} & %
{\vspace{-8mm}
Prof. Gilles Louppe is an Associate Professor in artificial intelligence and deep learning at \liegesentity, Belgium. Before that, he was a postdoctoral associate at New York University with the Physics Department and the Center for Data Science. Gilles Louppe is also an Analysis Consultant and Expert (ACE) with the ATLAS experiment at CERN.
%As a researcher, Gilles Louppe's interests lie at the intersection of machine learning, artificial intelligence and physical sciences, with close ties with the ATLAS experiment at CERN. His far ambition is to unlock discoveries of a new kind by making artificial intelligence a cornerstone of the modern scientific method. Using fundamental sciences, such as particle physics, as a test bed, his present research interests circle around how to use or design new machine learning algorithms to approach data-driven scientific problems in new and transformative ways. Current topics of research include: simulator-based likelihood-free inference, deep generative models, probabilistic programming, and developments towards the automation of science.
Commitment: 10\%  

} \tabularnewline\hline
\pbox{8cm}{\Tstrut Key Research\\Facilities,\\Infrastructure\\and Equipment} & %
\pbox{19cm}{\Tstrut 
Of immediate interest for the project, researchers at \liegesentity have access to a national wide consortium of computing clusters, totaling more than 10000 nodes and mixing both CPUs and GPUs with large amounts of memory and disk space. The University offers a diverse range of facilities to its students and staff, including not only library, office, laboratory and learning
spaces, but also restaurant, sports facilities, clubs or museums.
} \tabularnewline\hline

\multicolumn{2}{l}{\hspace{-1ex}Independent \Tstrut  research premises\Bstrut: yes}\tabularnewline\hline
\pbox{8cm}{\Tstrut Past \& current\\involvement\\in Research and\\Training\\Programmes} & 
\pbox{19cm}{ \Tstrut 
Within FP7, \liegesentity was the beneficiary of 32 Marie-Curie projects, in which 9 ITN, 9
Individual fellowships and 1 COFUND project for Post-Doc training (600405-BelPD, still ongoing). %To be updated
Within H2020, \liegesentity is the beneficiary of 14 Marie-Curie projects, of which 8 are ITN and 4 are
Individual fellowships. %numbers didn't match
} \tabularnewline\hline\Tstrut
\pbox{8cm}{\Tstrut Relevant\\Publications} &%
{
\vspace{-3mm}
\begin{itemize}%[topsep=0pt,itemsep=-2pt,leftmargin=*]
\item Albertsson, Kim, et al. "Machine Learning in High Energy Physics Community White Paper." Journal of Physics: Conference Series. Vol. 1085. No. 2. IOP Publishing, 2018.
\item Baydin, Atilim Gunes, et al. "Efficient Probabilistic Inference in the Quest for Physics Beyond the Standard Model." arXiv preprint arXiv:1807.07706 (2018).
%\item Brehmer, Johann, et al. "Constraining Effective Field Theories with Machine Learning." arXiv preprint arXiv:1805.00013(2018).
\item Brehmer, Johann, et al. "A Guide to Constraining Effective Field Theories with Machine Learning." arXiv preprint arXiv:1805.00020 (2018).
\item Louppe, Gilles, and Kyle Cranmer. "Adversarial Variational Optimization of Non-Differentiable Simulators." arXiv preprint arXiv:1707.07113 (2017).
%\item Louppe, Gilles, et al. "QCD-aware recursive neural networks for jet physics." arXiv preprint arXiv:1702.00748 (2017).
\item Louppe, Gilles, Michael Kagan, and Kyle Cranmer. "Learning to pivot with adversarial networks." Advances in Neural Information Processing Systems. 2017.
%\item Cranmer, Kyle, Juan Pavez, and Gilles Louppe. "Approximating likelihood ratios with calibrated discriminative classifiers." arXiv preprint arXiv:1506.02169 (2015).
\vspace{-5mm}
\end{itemize}
}\tabularnewline\bottomrule
\end{tabular}
}%
\end{center}