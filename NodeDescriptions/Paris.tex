\begin{center}
\footnotesize
\begin{tabular}{|p{0.1\textwidth}|p{0.85\textwidth}|}
%\resizebox{\textwidth}{!}{%
%\begin{tabular}{@{}p{25mm}|p{190mm}@{}}
\toprule
\multicolumn{2}{c}{\large\textbf{Beneficiary: CNRS}}\tabularnewline\hline 
\pbox{8cm}{\Tstrut General\\Description\Bstrut} &%
\pbox{0.85\textwidth}{\Tstrut 
LPNHE is a particle, astroparticle, and nuclear physics laboratory within the IN2P3 institute of CNRS, with around 110 researchers and support personnel. The laboratory participates in several world-wide experiments (ATLAS, LHCb, Auger, LSST, etc.), and local teams are formed around each experiment. The laboratory is attached to the Sorbonne Universite, a recently founded mega-university which is home to around 58000 students and 200 laboratories hosting 7700 professor-researchers and over 5000 doctoral students, as well as over 50 ERC grants and 45 industry sponsored research chairs. This ensures that the \acronym students recruited or seconded to the CNRS node will receive the best and most modern training available. Note that although CNRS does not itself give out PhDs, the embedding of the lab in Sorbonne Universite means that PhD students are hosted in LPNHE, paid by CNRS, and receive their degrees from Sorbonne Universite. 
}
\tabularnewline\hline
\pbox{8cm}{\Tstrut Role and\\Commitment\\ of Key persons} &%
{\vspace{-5mm}
\begin{enumerate}%[topsep=0pt,itemsep=-2pt,leftmargin=*]
\item  Dr.~Vladimir V. Gligorov, senior researcher at LPNHE, is the former LHCb High Level Trigger project leader and deputy Physics Coordinator, and now leads LHCb's Real Time Analysis project (consisting of 30 institutes and around 50 FTE). Together with M. Williams, Gligorov was responsible for the first large scale use of Machine Learning in the trigger system of an LHC experiment, with over 1/3 of LHCb's data between 2010 and 2012 taken using the novel boosted decision tree which they had designed to be safe for real-time use. Gligorov launched and subsequently coordinated LHCb's masterclass programme, and is the PI of ERC Consolidator Grant GA724777 "RECEPT". He has co-supervised one PhD, one Masters, and several CERN summer students and is currently supervising two PhD students. Gligorov's PhD student won the LHCb thesis prize in 2016. Expertise in triggering, real-time reconstruction, machine learning, data analysis, outreach. Role: co-supervisor of ESR8, LHCb contact person, tutoring of seconded students, ML, AI, and data analysis WP coordinator. Commitment: 20\%
\item Francesco Crescioli, research engineer at LPNHE and member of the ATLAS collaboration. He is an expert in ASIC design and design and commissioning of highly parallel FPGA based reconstruction systems. He has co-lead the development of the Associative Memory (AM) chip for the FTK processor in ATLAS and he is currently the AM representative in the FTK Coordination Board in ATLAS. He is the technical coordinator of the French Agence Nationale de la Recherche project "FastTrack" (ANR-13-BS05-0011). He has been the technical coordinator of the SATT IDF Innov valorization project "SPAD" (no. 268).  He has been the coordinator of the WP6 of the EU IAPP project "FTK" (grant agreement no. 324318). He is the technical coordinator of ATLAS Group at LPNHE. He has co-supervised two Master students, at University of Pisa and LPNHE, and he is currently supervising a small team of engineers. Role: co-supervisor of ESR6, lecturer on hybrid architectures, supervision of seconded students, diversity officer. Commitment: 20\%
\item Bogdan Malaescu, senior researcher at LPNHE and member of the ATLAS collaboration, expert in statistics and physics data analysis with jets. He has been nominated convener of the "Standard Model" group in ATLAS and he has been convener of the "Statistics  Forum", "Jets \& photons" subgroup of the "Standard Model" and "Jet energy scale and jet energy resolution" subgroup of the "JetEtmiss" group. He has been co-supervisor of a PhD student and supervisor of 3 Master students. He is currently co-supervising a PhD student. Role: co-supervisor of ESR6. Commitment: 20\%
\vspace{-2mm}
\end{enumerate}
} \tabularnewline\hline   
\pbox{8cm}{\Tstrut Key Research\\Facilities,\\Infrastructure\\and Equipment\Bstrut} & %
\pbox{0.85\textwidth}{\Tstrut
 The LPNHE lab hosts a large computing cluster, with both x86 and non-x86 (GPU/FPGA/hybrid) architectures, which the researchers can use in their work. LPNHE also has an extensive staff of full-time mechanical and electronics engineers who can provide support to researchers in their work. Further computing resources including personal cloud storage are available through the CNRS cloud computing platforms. Appropriate office space, secretarial, administrative, and outreach support, as well as access to all relevant scientific literature is provided.
} 
\tabularnewline\hline
\multicolumn{2}{l}{\hspace{-1ex}Independent \Tstrut  research premises\Bstrut: yes}\tabularnewline\hline
\pbox{8cm}{\Tstrut Past \& current\\involvement\\in Research and\\Training\\Programmes\Bstrut} &  
\pbox{0.85\textwidth}{
\Tstrut CNRS has hosted 460 ERC grants since 2007 and participated or coordinated over 450 H2020 programmes since 2014. Full lists can be found  \href{http://erc.cnrs.fr/en/tous-les-laureats/}{here} and \href{http://www.fabiodisconzi.com/open-h2020/per-country/fr/centre+national+de+la+recherche+scientifique+cnrs/index.html}{here}. The most relevant grants currently hosted in the participating labs are GA654168 "AIDA-2020" and GA724777 "RECEPT", both part of the H2020 programme.
} 
\tabularnewline\hline\Tstrut
\pbox{8cm}{\Tstrut Relevant\\Publications} &%
{
\begin{itemize}%[topsep=0pt,itemsep=-2pt,leftmargin=*]
\item R. Aaij et al., Tesla : an application for real-time data analysis in High Energy Physics, Comput.Phys.Commun. 208 (2016) 35-42.
%\item V. V. Gligorov, Real-time data analysis at the LHC: present and future, J.Mach.Learn.Res. 42 (2015) 1-18
\item V.V. Gligorov and M. Williams, Efficient, reliable and fast high-level triggering using a bonsai boosted decision tree, JINST 8 (2013) P02013
\item C.L. Sotiropoulou et al., The Associative Memory System Infrastructures for the ATLAS Fast Tracker, IEEE Trans.Nucl.Sci. 64 (2017) no.6, 1248-1254
\item B. Malaescu and P. Starovoitov, Evaluation of the Strong Coupling Constant alphas Using the ATLAS Inclusive Jet Cross-Section Data, Eur. Phys. J. C 72 (2012) 2041
\item M. Aaboud et al. (ATLAS Collaboration), Search for new phenomena in dijet events using 37 fb-1 of pp collision data collected at 13 TeV with the ATLAS detector, Phys. Rev. D 96 (2017) no.5,  052004
%Had to remove one because we're up to 5
%\item M.Ali Mirzaei et al., Heterogeneous computing system platform for high-performance pattern recognition applications, proceedings of MOCAST 2017
\end{itemize}
}\tabularnewline\hline
\end{tabular}
%}%
\end{center}

%\checkme{List of Publications}
