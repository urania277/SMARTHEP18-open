\begin{center}
\resizebox{\textwidth}{!}{%
\begin{tabular}{@{}p{25mm}|p{190mm}@{}}
\toprule
\multicolumn{2}{c}{\large\textbf{Beneficiary: \parisUlong}}\tabularnewline\hline
\pbox{8cm}{\Tstrut General\\Description\Bstrut} & %
\pbox{19cm}{\Tstrut 
Born from the merger of Universite Pierre et Marie Curie and \parisUlong, whose campuses are in the heart of Paris, \parisUlong covers all major disciplinary fields and offers new 
transversal academic and research programs. \parisUlong becomes a fully multidisciplinary research-intensive university with three faculties: Humanities and Social Sciences, Medicine  and Sciences \& Engineering. With more than 53 400 students (among 10 200 international  students), 4400 doctoral students and 6300 researchers, \parisUlong is one of the leading  French universities. The university is involved in numerous European and International partnership agreements and has France's largest scientific library and infrastructures bringing together the best talent in a wide array of these disciplines. With 8,500 publications per year (approx. 10\% of all publications in France), \parisUlong is a major player in international knowledge and innovation economy, offering transversal academic and research programs. The EU office will manage all the financial, administrative and legal aspects for the participation of \parisUlong in this project. 
\Bstrut}\tabularnewline\hline

\pbox{8cm}{\Tstrut Role and\\Commitment\\of Key persons} & %
{\vspace{-8mm}
\begin{enumerate}%[topsep=0pt,itemsep=-2pt,leftmargin=*]
\item Lionel Lacassagne, full professor at \parisUlong, leader of the Hardware and Software for Embedded Systems team at LIP6. 
Expertise in designing embedded systems and highly parallel heterogeneous computing architectures. Commitment: 20\%
\item Quentin Meunier, associate professor at \parisUlong, expert in architectures and software for efficient parallelisation, including cache protocols, coherence management, fixed-point and floating-point arithmetic. Commitment: 20\%

\item Andrea Pinna, associate professor at \parisUlong, expert in algorithm (Symbolic, NNs, CNN, Fuzzy logic three) implementation for embedded system architecture, digital VLSI system design, CMOS vision system on chip. Commitment: 20\%
\vspace{-\belowdisplayskip}
\end{enumerate}} \tabularnewline\hline

\pbox{8cm}{\Tstrut Key Research\\Facilities,\\Infrastructure\\and Equipment} & %
\pbox{19cm}{\Tstrut 
The LIP6 lab hosts a large computing cluster, with both x86 and non-x86 (GPU/FPGA/hybrid) architectures, which the researchers can use in their work. The lab also has extensive facilities for designing new computing architectures, with dedicated support from a team of full-time experienced engineers for the work of researchers. Further computing resources including personal cloud storage are available, and access to all relevant scientific literature is provided.
} \tabularnewline\hline
%
\multicolumn{2}{l}{\hspace{-1ex}Independent \Tstrut  research premises\Bstrut: yes}\tabularnewline\hline
\pbox{8cm}{\Tstrut Past \& current\\involvement\\in Research and\\Training\\Programmes} & 
\pbox{19cm}{\Tstrut 
The European Affairs office, which is in charge of the EU projects at the university, has managed so far 150 FP7 and 85 H2020 projects (35 ERC grants and 45 industry-sponsored research chairs).  \parisUlong is currently involved in 23 Marie Curie actions, including 12 MSCA-IF, 9 MSCA-ITN and 2 MSCA-RISE.
} \tabularnewline\hline\Tstrut
\pbox{8cm}{\Tstrut Relevant\\Publications} &%
{\vspace{-3mm}
\begin{itemize}%[topsep=0pt,itemsep=-2pt,leftmargin=*]
\item F. Lemaitre, L. Lacassagne, Batched Cholesky Factorization for tiny matrices, Design and Architectures for Signal and Image Processing (DASIP), Rennes, France, pp. 1-8
\item L. Cabaret, L. Lacassagne, D. Etiemble, Parallel Light Speed Labeling: an efficient connected component algorithm for labeling and analysis on multi-core processors, Real-Time Image Proc (2016)
\item H. Liu, Q. L. Meunier, A. Greiner, Decoupling Translation Lookaside Buffer Coherence from Cache Coherence, ISVLSI'17, 2017, Bochum, Germany
\item  O. L. C. Camacho, A. Pinna, X. Dray, and B. Granado, “Polyps. Recognition Using Fuzzy Trees,” in Biomedical and Health Informatics 
(IEEE BHI’17), Orlando, United States, 2017
\end{itemize}}\tabularnewline\bottomrule

\end{tabular}
}%
\end{center}
