\begin{center}
\resizebox{\textwidth}{!}{%
\begin{tabular}{@{}p{25mm}|p{190mm}@{}}
\toprule
\multicolumn{2}{c}{\large\textbf{Partner organization: \radboudentity}}\tabularnewline\hline
\pbox{8cm}{\Tstrut General\\Description\Bstrut} & %
\pbox{19cm}{\Tstrut 
Radboud University of Nijmegen is one of the leading Dutch
Universities with 21\,000 students and more than 100 bachelor and
master programmess. The EHEF group of Radboud University makes
predictions and performs experiments at the high energy particle
physics frontier and studies the properties of spacetime and
fields. The ATLAS group focuses on the Higgs, Dark matter and Beyond
the Standard Model physics with deep involvement in the trigger, the
readout of the muon detector, the reconstruction of muons, and the
identification and measurement of b-quark jets and tau leptons.  
\Bstrut}\tabularnewline\hline

\pbox{8cm}{\Tstrut Role and\\Commitment\\of Key persons} & %
{\vspace{-8mm}
\begin{enumerate}%[topsep=0pt,itemsep=-2pt,leftmargin=*]
\item Prof.~Olga Igonkina, physics, professor of experimental particle
physics in Radboud University of Nijmegen, more details in hte \nikhef
node description.
Role: supervisor of ESR8, co-supervisor of ESR9, tutoring of seconded
students, diversity and inclusion officer, commitment: 20\%.
\item Prof.~M. Van der Toorn, lawyer, administrative support. 
\vspace{-\belowdisplayskip}
\end{enumerate}} \tabularnewline\hline

\pbox{8cm}{\Tstrut Key Research\\Facilities,\\Infrastructure\\and Equipment} & %
\pbox{19cm}{\Tstrut 
The ESR will be based at \nikhef, \radboudentity is the institution to
give the PhD. 
} \tabularnewline\hline
%
\multicolumn{2}{l}{\hspace{-1ex}Independent \Tstrut  research premises\Bstrut: yes}\tabularnewline\hline
\pbox{8cm}{\Tstrut Past \& current\\involvement\\in Research and\\Training\\Programmes} & 
\pbox{19cm}{\Tstrut 
Radboud University is involved in a number of H2020 grants, including 1 Synergy grant, 16 Advanced grants, 12 Consolidator grants and 37 Starting grants.
The full list of grants and programmes can be found \href{http://www.ru.nl/english/research/prizes-achievements/more-prizes-achievements/}{on the Radboud University website}. 
} \tabularnewline\hline\Tstrut
\pbox{8cm}{\Tstrut Relevant\\Publications} &%
{\vspace{-3mm}
\begin{itemize}%[topsep=0pt,itemsep=-2pt,leftmargin=*]

\item See \nikhef node description.

\end{itemize}}\tabularnewline\bottomrule

\end{tabular}
}%
\end{center}
