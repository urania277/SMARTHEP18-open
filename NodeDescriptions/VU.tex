\begin{center}
\resizebox{\textwidth}{!}{%
\begin{tabular}{@{}p{25mm}|p{190mm}@{}}
\toprule
\multicolumn{2}{c}{\large\textbf{Partner organization: \amsterdam}}\tabularnewline\hline
\pbox{8cm}{\Tstrut General\\Description\Bstrut} & %
\pbox{19cm}{\Tstrut 
Academic research and education at VU Amsterdam is characterised by a
high level of ambition, and encourages free and open communication and
ideas. In 2016, VU hosted approximately 22,000 students and over 2,500
scientific staff. The total research output in 2016 translated to over
3,900 scientific publications, and 271 doctoral theses.  
\Bstrut}\tabularnewline\hline

\pbox{8cm}{\Tstrut Role and\\Commitment\\of Key persons} & %
{\vspace{-8mm}
\begin{enumerate}%[topsep=0pt,itemsep=-2pt,leftmargin=*]
\item Prof.~Gerhard Raven, physics, professor of experimental particle
  physics, more details in the \nikhef node description. 
Role: supervisor of ESR8, co-supervisor of ESR6, tutoring of seconded students, commitment: 20\%;
\vspace{-\belowdisplayskip}
\end{enumerate}} \tabularnewline\hline

\pbox{8cm}{\Tstrut Key Research\\Facilities,\\Infrastructure\\and Equipment} & %
\pbox{19cm}{\Tstrut 
The ESR will be based at \nikhef, \amsterdam is the institution to
give the PhD. 
} \tabularnewline\hline
%
\multicolumn{2}{l}{\hspace{-1ex}Independent \Tstrut  research premises\Bstrut: yes}\tabularnewline\hline
\pbox{8cm}{\Tstrut Past \& current\\involvement\\in Research and\\Training\\Programmes} & 
\pbox{19cm}{\Tstrut 
In FP7, VU Amsterdam has acquired close to 220 grants across all
pillars and priorities, among which 70 as coordinator. A total of 54
Marie Curie grants were obtained out of FP7, of which 27 training
networks. \\
In Horizon 2020, VU Amsterdam has acquired approximately 100 grants
across all pillars and priorities, among which around 40 as
coordinator. A total of  32 Marie Curie grants were obtained in
2014-2017, of which 13 ITNs.  
} \tabularnewline\hline\Tstrut
\pbox{8cm}{\Tstrut Relevant\\Publications} &%
{\vspace{-3mm}
\begin{itemize}%[topsep=0pt,itemsep=-2pt,leftmargin=*]

\item See \nikhef node description.

\end{itemize}}\tabularnewline\bottomrule

\end{tabular}
}%
\end{center}
