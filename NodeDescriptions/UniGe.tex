\begin{center}
\footnotesize
\begin{tabular}{|p{0.1\textwidth}|p{0.85\textwidth}|}
%\resizebox{\textwidth}{!}{%
%\begin{tabular}{@{}p{25mm}|p{190mm}@{}}
\toprule
\multicolumn{2}{c}{\large\textbf{Beneficiary: \unigelong}}\tabularnewline\hline
\pbox{8cm}{\Tstrut General\\Description\Bstrut} & %
\pbox{0.85\textwidth}{\Tstrut  \unigelong (\unigeshort) is Switzerland's second largest university with more than 17\,000 students of 150 different nationalities and more than 3950 researchers of 113 nationalities, who study and work in 9 different faculties. 
The university provides an international environment for education and research. It has long history of strong ties with international research organisations, such as \cern. 
As a result of its history and its strategic choices, the \unigeshort made possible a diversity of research areas to emerge in which the institution excels. 
High energy physics, profiting strongly from the geographical vicinity to CERN, is one such area of excellence.   
\Bstrut}\tabularnewline\hline
\pbox{8cm}{\Tstrut Role and\\Commitment\\of Key persons} & %
{\vspace{-5mm}
\begin{enumerate}%[topsep=0pt,itemsep=-2pt,leftmargin=*]
\item Prof.~Anna Sfyrla, physics, assistant professor, experimental particle physics and a member of the ATLAS collaboration. 
She had various responsibility positions at ATLAS in areas spanning the trigger, searches for new physics  and the HL-LHC upgrade. 
She is currently the thesis director of 2 PhD students, and supervises a post-doctoral researcher. 
She has supervised many PhD, master and summer students in trigger and analysis projects as a CERN staff and fellow. 
She has organised, hosted and convened numerous workshops related to her areas of research. 
More details: \url{http://dpnc.unige.ch/~sfyrla/}. 
Role: supervisor of \ESRb, responsible for WP2. 
Commitment: 20\%. 
\item Dr.~Steven Schramm, physics, senior postdoc, member of the ATLAS Collaboration. 
 He has previously or currently holds positions in areas relating to the trigger, machine learning, and the reconstruction of physics objects used for dark matter searches: ATLAS jet trigger coordinator (2015-2017), convener of ATLAS jet substructure group (2017-present), founder and coordinator of LHC Inter-experimental Machine Learning group (2015-2018). 
 Scientific secretary to the ATLAS trigger and data acquisition steering group (2017-present). 
 Recipient of Canadian Banting fellowship. Expertise in jet reconstruction, searches for new physics, machine learning, data analysis, triggering. 
 He has previously and continues to co-supervise many PhD, MSc, and summer students in the areas of triggers, jet performance, and analysis.   
 Role: additional supervisor of ESR2, tutoring of students seconded at CERN. Commitment: 20\%.
\item Prof. Giuseppe Iacobucci, full professor (Head of the particle physics department, Faculty of Science, UniGe). Expert in detector \& sensor design and RnD and a broad spectrum of physics (from QCD measurements to  DM searches). Role: senior advisor. 
\vspace{-2mm}
% \vspace{-\belowdisplayskip}
\end{enumerate}}
\tabularnewline\hline
\pbox{8cm}{\Tstrut Key Research\\Facilities,\\Infrastructure\\ and Equipment} & %
\pbox{0.85\textwidth}{\Tstrut  
The Department of Nuclear and Particle Physics of the \unigelong studies the fundamental structures and laws of nature following three complementary directions: collider physics at \cern s LHC; neutrino physics in collider experiments; astroparticle physics experiments on the group and in space. 
The ATLAS group of the department has made significant contributions to the construction and operation of the experiment. 
It is presently contributing to event reconstruction and searches for new physics, as well as the HL-LHC upgrade.
\unigeshort has computing clusters available for the ATLAS affiliated students and researchers, comprised of more than 3000 computing cores. 
The ATLAS group owns several hundreds of TB of disk storage space, as well as high-performance GPUs obtained recently to further the research and development of machine learning tools and real-time applications. 
Many internal training programmes and facilities are available to the ESR affiliated to \unigeshort. 
The student will have access to the universities Doctoral School, where experimental and theoretical aspects of High Energy Physics at PhD level are taught. 
The student will also have access to career forums organised by the University of Geneva. 
} \tabularnewline\hline
\multicolumn{2}{l}{\hspace{-1ex}Independent \Tstrut research premises\Bstrut: yes}\tabularnewline\hline
\pbox{8cm}{\Tstrut Past \& current\\involvement\\in Research and\\Training\\Programmes\Bstrut} & 
\pbox{0.85\textwidth}{\Tstrut 
\unige researchers are granted numerous prizes and distinctions each year. 
Almost 50 researchers from \unige were granted a prestigious European Research Council Grant (FP7 and H2020).
Open to the world, the \unige leads research projects in collaboration with almost 100 countries. 
At a European level, the UNIGE actively participates in many EU research programmes, particularly to the Framework Programmes for Research with more than 250 participations in FP7 including 20 coordinations and 36 prestigious European Research Council Grants and over 50 participations in Horizon 2020 projects. 
\unige is also involved in 50 COST networks and research projects and many other European and international research and innovation programmes (IMI, ESA, INTERREG, NIH, etc.). 
}\tabularnewline\hline
\pbox{8cm}{\Tstrut Relevant\\Publications} &%
{\vspace{-3mm}
\begin{itemize}%[topsep=0pt,itemsep=-2pt,leftmargin=*]
\item   ATLAS Collaboration: ``Search for new phenomena with large jet multiplicities and missing transverse momentum using large-radius jets and flavour-tagging at ATLAS in 13 TeV proton-proton collisions''  JHEP 1712(2017)034
\item  ATLAS Collaboration: ``Jet reconstruction and performance using particle flow with the ATLAS detector''  EPJC 77(2017)466.
\item  ATLAS Collaboration: `Performance of the ATLAS Trigger System in 2015`''  EPJC77(2017)317
\item  ATLAS Collaboration: ``Search for new phenomena in final states with large jet multiplicities and missing transverse momentum at $\sqrt{s}$=8 TeV proton-proton collisions using the ATLAS experiment'' JHEP 10(2013)130
\end{itemize}
}\tabularnewline\hline
\end{tabular}
%}%
\end{center}
%\checkme{PS : Maybe one more reference?}