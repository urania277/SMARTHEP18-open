\begin{center}
\resizebox{\textwidth}{!}{%
\begin{tabular}{@{}p{25mm}|p{190mm}@{}}
\toprule
\multicolumn{2}{c}{\large\textbf{Partner organization: \santiagolong}}\tabularnewline\hline
\pbox{8cm}{\Tstrut General\\Description\Bstrut} & %
\pbox{19cm}{\Tstrut 
The Instituto Galego de Fisica de Altas Enerxias (IGFAE) is a mixed institute between the regional government of Galicia and the Universidade de Santiago de Compostela (USC). 
Universidade de Santiago de Compostela (USC) is one of the oldest universities in Europe and one of Spains top universities. The IGFAE is one the most scientifically productive centers of the USC. Indeed, it is the only institute in the region recognized as Unit of Excellence (via de Maria de Maeztu Program of the Spanish government) and one of the four on Particle Physics in Spain. The IGFAE has 28 faculty members, 13 postdocs, 33 PhD students and 2 emeriti. \Bstrut}\tabularnewline\hline

\pbox{8cm}{\Tstrut Role and\\Commitment\\of Key persons} & %
{\vspace{-8mm}
\begin{enumerate}%[topsep=0pt,itemsep=-2pt,leftmargin=*]
\item Diego Martinez is Distinguished Researcher at Axencia Galega de Innovacion (GAIN), from the industry department of the regional government. He has extensive experience on LHCb data analysis and is PI of an ERC-StG-639068. 
\item Veronika Chobanova is convener of B decays to charmonia, being the only member of a Spanish institution at the Physics Planning Group of LHCb. She is also co-PI of a national project of 133k euro, and outreach convener of IGFAE. 
\vspace{-\belowdisplayskip}
\end{enumerate}} \tabularnewline\hline

\pbox{8cm}{\Tstrut Key Research\\Facilities,\\Infrastructure\\and Equipment} & %
\pbox{19cm}{\Tstrut 
The group has a CPU Tier-3 with a computing power of 5.4kHEPSpec. In addition, a GPU cluster is being constructed, and currently triples the power of the Tier-3 for parallelizable problems.
} \tabularnewline\hline
%
\multicolumn{2}{l}{\hspace{-1ex}Independent \Tstrut  research premises\Bstrut: yes}\tabularnewline\hline
\pbox{8cm}{\Tstrut Past \& current\\involvement\\in Research and\\Training\\Programmes} & 
\pbox{19cm}{\Tstrut 
It hosted 10 FP7 projects (1 StG/CoG, 6 INFRAS, 2 MSCA-IRSES, 1 MSCA-ERG) and 3 H2020 projects (1 ERC-StG, 1 INFRAS, 1 MSCA-IF).
The European Projects Office of USC takes care of the financial managing of EU funded projects.
} \tabularnewline\hline\Tstrut
\pbox{8cm}{\Tstrut Relevant\\Publications} &%
{\vspace{-3mm}
\begin{itemize}%[topsep=0pt,itemsep=-2pt,leftmargin=*]
\item      D. Martinez Santos et al.” Ipanema-$\beta$: tools and examples for HEP analysis on GPU”  arXiv:1706.01420
\item  Veronika Chobanova et al. Sensitivity of LHCb and its upgrade in the measurement of $B(K_S^0\rightarrow \pi^0\mu^+\mu^-)$. CERN-LHCb-PUB-2016-017
\item  CMS \& LHCb, 'Observation of the rare Bs to mumu  decay from the combined analysis of CMS and LHCb data'. Nature522(2015)68 �72
%\item R. Aaij et al. [LHCb Collaboration], “Test of lepton universality with $B_0 \rightarrow K^∗\ell^+\ell^−$ decays”, Journal of HEP 08 (2017) 055, arXiv:1705.05802 [hep-ex]
%\item     LHCb Collaboration “Precision measurement of CP violation in $B_s \rightarrow J/\psi KK$ decays”, Phys. Rev. Lett. 114, 041801 (2015)
\end{itemize}}\tabularnewline\bottomrule

\end{tabular}
}%
\end{center}
