\begin{center}
\resizebox{\textwidth}{!}{%
\begin{tabular}{@{}p{25mm}|p{190mm}@{}}
\toprule
\multicolumn{2}{c}{\large\textbf{Partner organization: \cincinnatilong}}\tabularnewline\hline
\pbox{8cm}{\Tstrut General\\Description\Bstrut} & %
\pbox{19cm}{\Tstrut 
The University of Cincinnati is a public research university with more than 40,000 students.  The Physics Department currently enrolls 55 students in its Ph.D. program and typically graduates 10 per year. Five faculty members actively pursue research in experimental and six in theoretical particle physics, all but one studying flavor physics and/or dark matter physics.
\Bstrut}\tabularnewline\hline

\pbox{8cm}{\Tstrut Role and\\Commitment\\of Key persons} & %
{\vspace{-8mm}
\begin{enumerate}%[topsep=0pt,itemsep=-2pt,leftmargin=*]

\item Michael D. Sokoloff is a Professor of Physics who has worked in flavor physics since 1981 and has supervised 11 Ph.D. students.  He has multiple U.S. NSF awards that support his work on physics, triggering, and data processing for LHCb.% [NSF PHY-1505719], developing GPU-friendly algorithms [NSF PHY-1414736],  using machine learning in the real-time data ingestion and reduction system for the LHCb Run 3 trigger [NSF OAC-1740102], developing data intensive analysis tools for high energy physics [NSF OAC-1450319], and conceptualization of a scientific software institute for high energy physics [NSF OAC-1558219].  Commitment: 5\%

\item Jure Zupan is a Professor of Physics and an expert in flavor and dark matter physics phenomenology.  He has published more than 80 peer-reviewed journal articles and presented more than 100 invited talks and seminars in 6 years. Commitment: 10\%

\item Stefania Gori is an Assistant Professor of Physics and an expert in flavor and dark matter physics phenomenology, recently awarded a CAREER award by the U.S. NSF, their most prestigious award supporting junior faculty. Commitment: 10\%

\vspace{-4mm}
\end{enumerate}} \tabularnewline\hline

\pbox{8cm}{\Tstrut Key Research\\Facilities,\\Infrastructure\\and Equipment} & %
\pbox{19cm}{\Tstrut 
Access to computing resources in the Physics Department, including an NVIDIA DGX-1 Volta AI platform optimized for deep learning. We also have access to resources at the Ohio Supercomputer Center that can be used for research and education.
} \tabularnewline\hline
%
\multicolumn{2}{l}{\hspace{-1ex}Independent \Tstrut  research premises\Bstrut: yes}\tabularnewline\hline
% \pbox{8cm}{\Tstrut Past \& current\\involvement\\in Research and\\Training\\Programmes} & 
% \pbox{19cm}{\Tstrut 
% N/A
% } \tabularnewline\hline\Tstrut
\pbox{8cm}{\Tstrut Relevant\\Publications} &%
{\vspace{-3mm}
\begin{itemize}%[topsep=0pt,itemsep=-2pt,leftmargin=*]

%\item  A.~Giri, Y.~Grossman, A.~Soffer and J.~Zupan, Determining gamma using B->DK with multibody D decays, Phys. Rev. D 68 (2003) 054018

\item   S.~Fajfer, J.~F.~Kamenik, I.~Nisandzic and J.~Zupan, Implications of Lepton Flavor Universality Violations in B Decays, Phys. Rev. Lett.  109 (2012) 161801

\item   D.~Curtin, R.~Essig, S.~Gori and J.~Shelton, Illuminating Dark Photons with High-Energy Colliders, JHEP 1502 (2015) 157

%\item   R.~Andreassen, B.~T.~Meadows, M.~de Silva, M.~D.~Sokoloff and K.~Tomko, GooFit: A library for massively parallelising maximum-likelihood fits, J. Phys. Conf. Ser.  513 (2014) 052003

\item   A.~A.~Alves, Jr, J. Albrecht, C. Doglioni, V. Gligorov,
  M. Sokoloff  et al., A Roadmap for HEP Software and Computing R\&D for the 2020s,
  arXiv:1712.06982
\vspace{-4mm}
\end{itemize}}\tabularnewline\bottomrule
\end{tabular}
}%
\end{center}
