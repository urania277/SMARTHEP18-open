\begin{center}
\resizebox{\textwidth}{!}{%
\begin{tabular}{@{}p{25mm}|p{190mm}@{}}
\toprule
\multicolumn{2}{c}{\large\textbf{Beneficiary: \lundlong}}\tabularnewline\hline
\pbox{8cm}{\Tstrut General\\Description\Bstrut} & %
\pbox{19cm}{\Tstrut 
\lundlong was founded in 1666,
% and for a number of years has been ranked among the world’s top 100 universities~\footnote{See \href{http://www.lunduniversity.lu.se/about/about-lund-university/university-world-rankings}{http://www.lunduniversity.lu.se/about/about-lund-university/university-world-rankings}}. 
the University has 42\,000 students and more than 7\,500 staff based
in Lund, Helsingborg and Malmoe. 
The Faculty of Science conducts research and education within Biology,
Astronomy, Physics, Geosciences, Chemistry, Mathematics and
Environmental Sciences. 
%The Faculty is organized into ten departments, gathered in the
%northern campus area. 
The Faculty has approximately 1\,900 students, 330 PhD students and 700 employees. 
The Department of Physics is with a staff of about 350 scientists and
educators one of the largest departments within \lundlong. There
are seven research divisions and a number of research centers within
the department. The research activities at the department cover a
broad spectrum of modern physics.
% For more information, see \url{www.fysik.lu.se/english}.  
\Bstrut}\tabularnewline\hline

\pbox{8cm}{\Tstrut Role and\\Commitment\\of Key persons} & %
{\vspace{-8mm}
\begin{enumerate}%[topsep=0pt,itemsep=-2pt,leftmargin=*]
\item Dr.~Caterina Doglioni, physics, senior associate lecturer, 
%convener of ATLAS Exotics jet+X physics group (2012-2014), 
convener of ATLAS Astroparticle forum (2014-2016), LHC Physics Center Dark Matter working group organizer (2015-present), 
receiver of ERC Starting Grant 2015 (DARKJETS, Grant Agreement no. 679305). 
Expertise in jet reconstruction, dark matter, triggering, real-time analysis in ATLAS.
\textbf{Role:} Project Coordinator, supervisor of ESR13, tutoring of seconded students. Commitment: 30\%.
\item Dr.~Peter Christiansen, physics, associate professor of experimental particle physics, 
expert in real-time detector reconstruction and analysis for the ALICE experiment. 
\textbf{Role:} supervisor of ESR14, lectures and organization of training, commitment: 20\%. 
\item Dr.~Oxana Smirnova, physics, associate professor of experimental particle physics, 
computing grid and distributed data analysis expert, Grid Architect in 2002-2003 and
Sweden's representative in the International Computing Board.
\textbf{Role:} co-supervisor of ESR13, tutoring of seconded students, commitment: 20\%.
\vspace{-\belowdisplayskip}
\end{enumerate}} \tabularnewline\hline

\pbox{8cm}{\Tstrut Key Research\\Facilities,\\Infrastructure\\and Equipment} & %
\pbox{19cm}{\Tstrut Lund University is active 
with a strong, long-term contribution
to the ATLAS Experiment and the ALICE Experiment.
%, Lund University group is well known in the field of particle physics. 
This ITN fits well the research interests and existing efforts in the group: 
an ongoing ERC StG, held by Doglioni, includes real-time data analyses
in searches for new physics in the two-jet final state.   
The opportunities for collaboration within the group are strong as the ESRs in this project
extend the research program, and there is sufficient room for a clearly defined, original line of research
for the students in this project.   
The expertise that can be found in the theoretical physics 
department of Lund is also invaluable for all physics searches and measurements
in this ITN: the \textsc{Pythia} event 
generator developed in Lund is the most used for the simulation of both signal and background 
events for jet searches in the ATLAS Collaboration. Collaboration with the theory division towards the implementation and 
testing of some of the benchmark physics models in this project within~\textsc{Pythia}, 
and on the validation and tuning of QCD backgrounds with ATLAS data will
be precious for searches that rely on the good modelling of the background
to New Phenomena. In \lundlong, the ESRs will also 
benefit from the use of a computing cluster connected to the NorduGrid network, 
which can be used for fast and efficient data analysis. 
The combination of a strong understanding of the theoretical issues for the comparison of 
collider Dark Matter searches and a hands-on experience with LHC data will 
be conducive to developments and collaboration between the beneficiaries of this project, 
other Swedish Universities contributing to Dark Matter searches within ATLAS
and the Swedish NORDITA Dark Matter program
that can be undertaken during and beyond the time frame of this project. 
Furthermore, there are two major research facilities in Lund: MAX IV, a world-leading synchrotron radiation laboratory
which has opened in 2016, and the European Spallation Source, a European facility that will host the most
powerful neutron source. Even though this research is not directly connected to the topics within this ITN, cross-pollination of ideas from the visiting
researchers and industrial partners joining the will be beneficial for the training of students either resident or seconded in Lund.
} \tabularnewline\hline
%
\multicolumn{2}{l}{\hspace{-1ex}Independent \Tstrut  research premises\Bstrut: yes}\tabularnewline\hline
\pbox{8cm}{\Tstrut Past \& current\\involvement\\in Research and\\Training\\Programmes} & 
\pbox{19cm}{\Tstrut \lundlong has been and is currently involved in several EU funded projects, in a number of disciplines, hosting several ERC (starting and advanced) grants. One of them is the DARKJETS ERC, concerning discovery strategies for Dark Matter and other new phenomena at the LHC, which has a connection to this project. An ETN that Lund particle physics is directly involved in is INSIGHTS (GA No.765710)  notable , concerning statistics for physics and society, with which \acronym plans to collaborate. Even though Doglioni was the original \lundentity responsible for INSIGHT, the role of local node coordinator has been assigned to Else Lytken, allowing Doglioni to have time to be PC of \acronym if funded. We also plan to collaborate with the~\href{http://www.montecarlonet.org/}{MCNet ETN.}, that has a long tradition in Lund.} \tabularnewline\hline\Tstrut
\pbox{8cm}{\Tstrut Relevant\\Publications} &%
{\vspace{-3mm}
\begin{description}%[topsep=0pt,itemsep=-2pt,leftmargin=*]

\item [Oliver Buchmueller, Caterina Doglioni and Lian-Tao Wang] Search for dark matter at colliders. Nature Physics 13, 217223 (2017). 

\item [ATLAS Collaboration] Performance of the ATLAS Trigger System in 2015. Eur.Phys.J. C77 (2017) no.5, 317

\item [ALICE Collaboration] The ALICE TPC, a large 3-dimensional tracking device with fast readout for ultra-high multiplicity events. Nucl. Instrum. Meth. A 622, 316 (2010) 1

\item [ALICE Collaboration] Centrality dependence of the nuclear modification factor of charged pions, kaons, and protons in Pb-Pb collisions at $\sqrt{s_NN}$=2.76 TeV. Phys. Rev. C 93, 034913 (2016)

%Search for new phenomena in dijet mass and angular distributions from $pp$ collisions at  $\sqrt{s}=13$~TeV with the ATLAS detector
%[ATLAS Collaboration]. Physics Letters B 754 (2016) 302-322

%\item (2015) Search for new phenomena in the dijet mass distribution using $pp$ collision data at $\sqrt{s}=8$~TeV with the ATLAS detector
%[ATLAS Collaboration]. Phys.Rev. D91 (2015) 052007

\item [O. Smirnova] Current Grid operation and future role of the Grid?, in Proceedings of CHEP 2012, J. Phys.: Conf. Ser. 396 042055 (2013)

%Published in \href{http://10.1140/epjc/s10052-014-3190-y}{}. 

\end{description}}\tabularnewline\bottomrule

\end{tabular}
}%
\end{center}
