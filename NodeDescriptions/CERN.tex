\begin{center}
\resizebox{\textwidth}{!}{%
\begin{tabular}{@{}p{25mm}|p{190mm}@{}}
\toprule
\multicolumn{2}{c}{\large\textbf{Beneficiary: \cern}}\tabularnewline\hline
\pbox{8cm}{\Tstrut General\\Description\Bstrut} & %
\pbox{19cm}{\Tstrut 
\cern is an International European Organization and is the worlds largest particle physics centre, providing technologically-advanced facilities for particle physics. 
CERN has 22 member states. 
Close to 13,000 scientists from 650 institutes worldwide are involved in the research and technology programme. 
CERN's mission is focused on 4 topics: research, technology, collaboration and education, including a long and strong training tradition via the Fellows, Associates and Students programmes. 
It has its own Learning and Development service providing almost 14,000 person days of technical management, communication, academic, safety and language training per year. 
The Large Hadron Collider, the largest and most powerful human-made particle collider, is located at CERN, together with its four main experiments: ALICE, ATLAS, CMS and LHCb. 
Researchers from all four collaborations, who spearheaded real-time analysis in their own experiments, are involved in \acronym.  
CERN hosts the largest particle accelerator in the world, the Large Hadron Collider (LHC), which generates data rates of over 100~Exabytes per year.  
CERN's aims include education and it offers a number of educational and training  programmes to students and teachers, as well as playing a leading role in the International Masterclass programme. 
\Bstrut}\tabularnewline\hline
\pbox{8cm}{\Tstrut Role and\\Commitment\\of Key persons} & %
{\vspace{-8mm}
\begin{enumerate}%[topsep=0pt,itemsep=-2pt,leftmargin=*]
\item Dr.~Monica Pepe-Altarelli is a senior physicist at CERN and a member of the LHCb collaboration in which she held several positions of responsibility.
She is Vice-President Elected at Large of the Executive Council of the International Union of Pure and Applied Science  (\href{http://iupap.org/executive-council-and-commission-chairs/executive-council-officers-2017-2020/}{IUPAP}) as well as Associate Member Delegate to the EPS Council for the period April 2017-2021. She has expertise in data analysis in the NA32, ALEPH and LHCb collaborations.
% Dr.~Monica Pepe-Altarelli, CERN staff. She has been LHCb deputy spokesperson and member of LHCb's Gender, Equality, and Diversity task force. 
%She has also been Vice-President Elected at Large of the Executive Council of the International Union of Pure and Applied Science 
%(\href{http://iupap.org/about-us/executive-council/executive-council-officers-2014-2017/}{IUPAP}).
% Expertise in data analysis in NA32, ALEPH and LHCb.
Role: supervision and mentoring of visiting students, lectures. 
Committment: 10\%.
\item Dr. Rosen Matev, CERN LD staff, expertise in software and trigger design and maintenance, RTA, luminosity measurements. 
Has co-supervised three PhD and a MSc student, currently co-supervising a CERN doctoral student. 
Received LHCb Early Career Scientist award, shared with four other colleagues, on the Run II software trigger. 
Role: co-supervisor of \ESRd.
Commitment: 20\%.
\item  Dr. Benjamin Couturier, CERN staff. After graduating from the Ecole Superieure d'Electricite (Supelec) and the University of Paris-Sud in 1996, Benjamin Couturier has been working as Software Engineer and Software Architect for various companies before joining CERN. He is now part of the team developing the software framework used to process and analyze data from the LHCb experiment at CERN.
Role: co-supervisor of \ESRg and \ESRi, lectures, non-academic training.
Commitment: 20\%.
\item Dr. Andrey Ustyuzhanin is the head of Yandex-CERN joint projects as well as the head of Laboratory of Methods for Big Data Analysis at National Research University Higher School of Economics. 
%His team participates in LHCb and SHiP (Search for Hidden Particles). 
He is an expert in machine learning and gives regular lectures on Machine Learning applied to High Energy Physics. % at the Yandex School of Data Analysis. 
Role: lectures, data challenge responsible, non-academic training.
Commitment: 10\%. 
\item Dr. Brian Petersen is a CERN staff scientist, and a member of the ATLAS collaboration. 
He has been the coordinator of the ATLAS trigger group of the ATLAS upgrade physics group.
He has supervised multiple CERN fellows and was chair of the 2017 CERN-Fermilab Hadron Collider Physics Summer School.
Role: supervisor of \ESRc and seconded students, lectures, commitment: 20\%
\item Dr.~Maurizio Pierini is the coordinator of the Physics Performances and Dataset (PPD) area in CMS, and the initiator of the Data Scouting analyses at the trigger level in CMS. 
He is a CERN staff scientist, and a LPC Distinguished Researcher of the Fermilab Physics Center.  
Role: co-supervisor of \ESRa, lectures.
Commitment: 15\%
\item Dr. Ruben Shahoyan is an expert on software and coordinates the reconstruction and calibration activities within the  O2 software project for ALICE Run3/4 upgrade. 
He coordinates the ALICE upgrade activity on calibration and reconstruction, and has played a critical role in calibrating the space point distortions of the existing Time Projection Chamber.
Role: supervisor of \ESRk, lectures.
Commitment: 15\%
\vspace{-3mm}
\end{enumerate}} \tabularnewline\hline
\pbox{8cm}{\Tstrut Key Research\\Facilities,\\Infrastructure\\ and Equipment} & %
\pbox{19cm}{\Tstrut World-class accelerator facilities: PS / SPS / LHC complexes. 
In-house engineering/technology/detector physics groups, prototyping, material science services, mechanical and electronics workshop, etc.
Due to its position as a focal point for research into elementary particle physics and associated technologies, CERN has state-of-the-art technological infrastructure and equipment. 
This spans a very large range of facilities such as accelerators and particle detectors, a forefront informatics backbone including Grid developments, state-of-the-art laboratories for mechanical, electronic, microelectronic and optoelectronic engineering and large cryogenics installations.
%In addition to standard office space with fibre-optic network connectivity and a library with access to over 2000 journals relevant to HEP and DS, CERN provides extensive support and infrastructure for both data analysis and computing.
%This includes a 100 node computing cluster and 2~Tb of disk space for each researcher's own work.
%In addition ESRs will have access to a dedicated cluster of around 10 nodes composed of the latest available GPU processors, and a second dedicated cluster of around 10 nodes composed of the latest available CPU 
ESRs will have access to dedicated clusters with the latest available GPU and CPU processors.
} \tabularnewline\hline
\multicolumn{2}{l}{\hspace{-1ex}Independent \Tstrut research premises\Bstrut: Yes
}\tabularnewline\hline
\pbox{8cm}{\Tstrut Past \& current\\involvement\\in Research and\\Training\\Programmes\Bstrut} & 
\pbox{19cm}{\Tstrut 
CERN has a learning and development programme offering about 15 Academic Training courses per year on subjects ranging from theoretical and experimental particle physics, to advances in technologies, computing and engineering. 
%It offers summer programmes for students and high school teachers, including dedicated physics and computing summer schools, as well as technical and doctoral students programmes. 
CERN also has the ``Openlab'' programme collaborating with industry on the development of IT technologies.
CERN has participated in and coordinated numerous European training projects, some recent examples being the ACEOLE, LA3NET, CATHI, EDUSAFE, PACMAN, and ICE-DIP training networks.\\
EU projects (FP7): Coordinator of 15 ITNs (11 completed / 4 ongoing), beneficiary or partner in 14 ITNs ( 8 / 6), coordinator of 5 COFUND grants (4 / 1), coordinator of 2 RISE and benefiociary in 2 RISE. 
%Completed projects : (FP6) Coordinator of 7 EST, 1 RTN, 8 individual fellowships + partner in 2 RTNs; 
% Coordinator of 4 COFUND grants; Coordinator of 15 individual fellowships; partner in 2 IAPPs. 
%Ongoing projects : (FP7) Coordinator of 1 COFUND grant; (H2020) 
%Coordinator of 1 ongoing COFUND grant, coordinator of another grant for a call of more than 50 applicants; Coordinator of 9 ongoing individual fellowships; Coordinator of 2 RISE + beneficiary in 2 others;
}\tabularnewline\hline
\pbox{8cm}{\Tstrut Relevant\\Publications} &%
{\vspace{-3mm}
\begin{itemize}%[topsep=0pt,itemsep=-2pt,leftmargin=*]
\item R. Aaij et al., The LHCb Trigger and its Performance in 2011, JINST {\bf 8} (2013) P04022.
%\item J. Albrecht et al., Event building and reconstruction at 30 MHz using a CPU farm, JINST 9 (2014) C10029
\item R. Aaij et al., Performance of the LHCb trigger and full real-time reconstruction in Run 2 of the LHC, arXiv:1812.10790
\item G.~Aad {\it et al.} [ATLAS Collaboration], Performance of the ATLAS Trigger System in 2010, Eur.\ Phys.\ J.\ C {\bf 72} (2012) 1849
\item M.~Aaboud {\it et al.} [ATLAS Collaboration], Performance of the ATLAS Trigger System in 2015, Eur.\ Phys.\ J.\ C {\bf 77}, no. 5, 317 (2017)
\vspace{-4mm}
\end{itemize}
}\tabularnewline\hline
\end{tabular}
}%
\end{center}