\begin{center}
\footnotesize
\begin{tabular}{|p{0.1\textwidth}|p{0.85\textwidth}|}
%\resizebox{\textwidth}{!}{%
%\begin{tabular}{@{}p{25mm}|p{190mm}@{}}
\toprule
\multicolumn{2}{c}{\large\textbf{Beneficiary: \ibmlong}}\tabularnewline\hline
\pbox{8cm}{\Tstrut General\\Description\Bstrut} & %
\pbox{0.85\textwidth}{\Tstrut 
IBM France is the French subsidiary of IBM, a hardware, software and services company based in Armonk, New York, United States. 
IBM France has founded IBM France Lab in 2011, after having acquired ILOG. IBM France Lab gathers all the R\&D entities based in France as a result of several mergers and acquisitions (in particular ILOG, Rational, SPSS). IBM France Lab develops software products in the domain of Cognitive Computing, namely Decision Engineering. 
IBM France Lab is the software and hardware development organization inside IBM France, that specializes in particular in the development of software products in the domains of decision systems and DevOps. IBM France Lab covers a large spectrum of in-depth expertise especially in the realms of Operational Research, Constraint Programming, Mathematical Optimization and Rule-based Reasoning. IBM Is amongst the world leaders in commercial-grade technology in these domains.
In addition to its own computing facilities, IBM France Lab can draw on IBM cloud pods and HPC facilities located in France and in Europe for its research purposes.
IBM France has a startup accelerator program, named Scalezone and located in its headquarters in Bois-Colombes, near Paris, France.
The French Center for Advanced Studies (CAS) is the research department inside IBM France Lab, which coordinates the industrial Research activity of the 600+ employees of France Lab, with a focus on collaborative R\&D activities (scientific cooperations, PhD theses, European and national R\&D projects).
To extend its research presence in France and especially with regards to Artificial Intelligence, IBM France is opening a new co-innovation center in Orsay, on the academic and industrial research Campus Paris-Saclay. The new co-innovation center will focus on the next wave of AI, combining symbolic AI (e.g. knowledge-based systems) and numeric AI (e.g. data-based machine learning technology such as deep learning). As part of this program, IBM France CAS is currently training and hiring PhD students on related subjects, such as to combining probabilistic data (such as the output of predictive models) and rule-based systems, explanations in recommender systems, detecting data unfit for a deep learning model (such as malicious attacks), or using machine learning to improve the performance of constraint-based optimisation. PhD theses defended by students trained in IBM France recently include research on explaining rule-based decisions or adjusting decision rules automatically in an evolving environment.
The participation of IBM France in the SMARTHEP project is an integral part of this research programme.
\Bstrut}\tabularnewline\hline

\pbox{8cm}{\Tstrut Role and\\Commitment\\of Key persons} & %
{\vspace{-5mm}
\begin{enumerate}%[topsep=0pt,itemsep=-2pt,leftmargin=*]
\item Dr Christian de Sainte Marie is the leader of IBM France Center for Advanced Studies. In this role, he contributed to mentor 9 PhD students AI in the last 10 years, and, in this and his previous role at ILOG, he led ILOG, then IBM France contributions in of 30 national, European and international collaborative R\&D projects. He also leads the design of the new AI co-innovation center's research program. Role: WP6 responsible, co-supervision of IBM students. Commitment: 10\%. 
\item Pierre Feillet is a software engineer in the Decision Lab. He works as a software architect contributing to a decision automation platform. covering the rule execution and its integration with machine learning and big data. Role: main supervisor of the PhD candidate recruited for the project on rule induction (\ESRx). He is a board member of Association Francaise pour l'Intelligence Artificielle (industrial/academic). Commitment: 15\%.
\item Dr.~Hugues~Juill\'{e} is a software engineer in the
  Prescriptive Analytics \& Decision Optimization Team. He works on the
  design and development of tools for modelling optimization
  problems. He also conducts research on algorithms taking advantage
  of OR algorithms for improving ML performance. Role: main
  supervisor of \ESRj, commitment: 15\%.
\item Paul Shaw is a research scientist and leader in
  constraint programming working in the Decision Optimization team.
  His background is in constraint programming, optimization modelling,
  and local search.  He works on the CP Optimizer solver and the use
  of optimization tools in machine learning. Role: additional supervision of \ESRj, commitment: 10\%.
\vspace{-2mm}%\belowdisplayskip}
\end{enumerate}
} \tabularnewline\hline

\pbox{8cm}{\Tstrut Key Research\\Facilities,\\Infrastructure\\and Equipment} & %
\pbox{0.85\textwidth}{\Tstrut 
IBM France Lab is located in three principal premises in France:
Gentilly, Nice and Pornichet. These three sites have been declared
strategic at a worldwide level by the relevant Business Lines. IBM
France Lab conducts its technological activity thanks to the
contribution of highly-skilled scientists and developers. In
particular, the staff comprises 100+ PhD, several Distinguished
Engineers and Senior Technical Staff Members. 
} \tabularnewline\hline

\multicolumn{2}{l}{\hspace{-1ex}Independent \Tstrut  research premises\Bstrut: yes}\tabularnewline\hline
\pbox{8cm}{\Tstrut Past \& current\\involvement\\in Research and\\Training\\Programmes} & 
\pbox{0.85\textwidth}{\Tstrut 
Since the creation of the Center for Advanced Studies in 2010, IBM France has hired 11 PhD students from major Engineering Schools in France (4 have defended in 2017),
and participated in several academic projects, including with Ecole Polytechnique and Universite Claude Bernard. Moreover, IBM France has contributed in several 
collaborative EU research projects, e.g. Ontorule (2009-2012), Matrics (2014-2016), Ideas (2012-2015) and, in association with IBM Italy MINO (2012-2016).
%for EU projects. 
Rider (2010-2014), and OptimodLyon (2012-2014) are examples of French-subsidized research projects performed by IBM France. 
IBM France Lab currently hosts 2 PhD students, and is in the process of recruiting 4 more in the coming months, including one in the framework of ITN MINOA, in which IBM France is a beneficiary. 
Members of IBM France Lab teach in universities and major Engineering schools, and the Lab plans to host a number of Master internship starting in Q2 2018. In addition, IBM France Lab contributes in the AFIA academic association, and organizes 4 scientific conferences each year, called Les Vendredis du CAS, that gather researchers from inside and outside IBM. 
} \tabularnewline\hline\Tstrut
\pbox{8cm}{\Tstrut Relevant\\Publications} &%
{\vspace{-3mm}
\begin{description}%[topsep=0pt,itemsep=-2pt,leftmargin=*]
\item[IBM CPLEX Optimizer] https://www.ibm.com/analytics/data-science/prescriptive-analytics/cplex-cp-optimizer, IBM. 
\item[P. Laborie, J. Rogerie, P. Shaw, P. Vilim.] IBM ILOG CP Optimizer or Scheduling.  20+ Years of Scheduling with Constraints at IBM/ILOG, Constraints, 2018. 
\item[P. Vilim, P. Laborie, P. Shaw.] Failure-directed Search for Constraint-based Scheduling." Proceedings of CPAIOR 2015.
\item[P. Shaw.]  Constraint Programming and Local Search Hybrids. In "Hybrid Optimization: The Ten Years of CPAIOR.  M. Milano, P. van Hentenryck, eds." Springer. pp 271-303, 2011.
\item [Hugues Juille and Jordan B. Pollack.] Coevolutionary learning and the design of complex systems. 
%\item[Hugues Juille and Jordan B. Pollack.] ��Coevolutionary Learning and the Design of Complex Systems. 
Advances in Complex Systems, Volume 02, Issue 04, December 1999.
\end{description}}
\tabularnewline\bottomrule

\end{tabular}
%}%
\end{center}
