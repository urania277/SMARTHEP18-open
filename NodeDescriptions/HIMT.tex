\begin{center}
\resizebox{\textwidth}{!}{%
\begin{tabular}{@{}p{25mm}|p{190mm}@{}}
\toprule
\multicolumn{2}{c}{\large\textbf{Partner organization: \heidelberginstrumentslongline}}\tabularnewline\hline
\pbox{8cm}{\Tstrut General\\Description\Bstrut} & %
\pbox{19cm}{\Tstrut 
\heidelberginstrumentslongline(\heidelberginstrumentsentity) is one of the world leaders in the production of laser lithography systems, with more than thirty years of experience in maskless lithography and with an installation base of more than 800 systems worldwide. The company offers a variety of maskless pattern generator systems: these range from small and easy to use tabletop systems to highly complex photomask production equipment with exposure areas of several square meters. \heidelberginstrumentsentity systems are installed in academic and industrial sites in more than 50 countries and are used in research, development and production. Applications include MEMS, BioMEMS, Nanotechnology, ASICS, TFT, Micro Optics and others. In 2016, \heidelberginstrumentsentity reached a turnover of 30 million Euros. The Management Board of \heidelberginstrumentsentity consists of Martin Wynaendts, President and CEO, Steffen Diez (Chief Sales Officer,
CSO) and Konrad R\"ossler (Chief Technical Officer, CTO). More than 170 employees worldwide work for \heidelberginstrumentsentity. \heidelberginstrumentsentity operates Customer Service Offices in Germany, USA, Japan, South Korea, Taiwan and China.
\Bstrut}\tabularnewline\hline

\pbox{8cm}{\Tstrut Role and\\Commitment\\of Key persons} & %
{\vspace{-8mm}
\begin{enumerate}%[topsep=0pt,itemsep=-2pt,leftmargin=*]
\item Mr. Roland Kaplan 
%received his Diploma in Physics from Heidelberg University in 1984. During his studies, he developed software to analyze emissions of nuclear power plants through a project financed by the German Government. Later, at the European Molecular Biology Laboratory (EMBL) in Heidelberg, he took part in the software-supported analysis of fluorescent spectrums. For his Diploma thesis - also at the EMBL - he worked on the upgrade of a Confocal Laser Microscope to a 3D raster scanning microscope. After his studies 
joined \heidelberginstruments as a software developer for real time image processing applications in 1985 and by 1990 he was in charge of the Software Department. 
%With the installation of the first direct write laser lithography systems for wafer fabrication in 1992, he took over various functions in the application group. Later, he was a key player in development of the first UV based large area exposure system, introduced in 1997.
Since 2002 he has been leading the Research and Development group at  \heidelberginstruments, responsible for various innovations and technology implementations. Mr. Kaplan has co-authored a number of patents in the field of maskless lithography.
Supervision commitment: 10\%
\item Mr. Konrad R\"ossler has been in charge of the Global Customer Support and Customer Application Center at \heidelberginstruments{} since 2005. 
%He graduated from Heidelberg University with a Diploma in Physics in 1993. After a short period at Leica Lasertechnik, he joined \heidelberginstruments{} in 1994. There, he worked partly for R\&D and customer applications, experienced the development and success of the first MW~800 lithography systems, especially in Asia. He was a key contributor in start up and management of Customer Support offices in Taiwan (1997-1999), China (1999) and Japan (2002-2004). During this period he learned the requirements and needs of \heidelberginstruments{} equipment in production companies.
\vspace{-\belowdisplayskip}
\end{enumerate}} \tabularnewline\hline

\pbox{8cm}{\Tstrut Key Research\\Facilities,\\Infrastructure\\and Equipment} & %
\pbox{19cm}{\Tstrut 
As a company designing complex optical-mechanical systems \heidelberginstruments{} has all departments and equipment needed. In the electronic design department 10 engineers are working on high speed data processing modules using CAD-tools from ALTIUM and FPGA-tools from XILINX.
\heidelberginstrumentsentity{} has several clean rooms, where the largest is used as Process and Application Laboratory
with 4 of our own Laser Lithography Systems, photoresist coating and process equipment, inspection tools like optical microscopes, confocal microscopes and scanning-electron microscope
} \tabularnewline\hline
%
\multicolumn{2}{l}{\hspace{-1ex}Independent \Tstrut  research premises\Bstrut: yes}\tabularnewline\hline
\pbox{8cm}{\Tstrut Past \& current\\involvement\\in Research and\\Training\\Programmes} & 
\pbox{19cm}{\Tstrut 
MICROCOMP (2004 - 2006): Partner in Collaboration (Leader: Ehrfeld Mikrotechnik BTS GmbH): Grayscale lithography on SU-8 photoresist.
OLAE+ (2013 - 2016): “Digital Lith” Collaboration (Leader: Centre for Process Innovation, UK): Development of a demonstrator for direct writing OLEDs on film.
} \tabularnewline\hline\Tstrut
\pbox{8cm}{\Tstrut Relevant\\Publications} &%
{\vspace{-3mm}
\begin{itemize}%[topsep=0pt,itemsep=-2pt,leftmargin=*]
\item \heidelberginstrumentsentity{} publication summarised on the company website : https://himt.de/index.php/publications.html
\item Patents in the field maskless lithography
\end{itemize}}\tabularnewline\bottomrule

\end{tabular}
}%
\end{center}
