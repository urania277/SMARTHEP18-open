The Dortmund node description is ready but has a problem with UTF encoding to be fixed.


%\begin{center}
%\resizebox{\textwidth}{!}{%
%\begin{tabular}{@{}p{25mm}|p{190mm}@{}}
%\toprule
%\multicolumn{2}{c}{\large\textbf{Beneficiary: \dortmundLong}}\tabularnewline\hline 
%\pbox{8cm}{\Tstrut General\\Description\Bstrut} &%
%\pbox{19cm}{\Tstrut 


%TU Dortmund University was founded in 1968 and has 16 faculties
%ranging from natural science and 
%engineering to social sciences and humanities. The university
%currently counts over 34,000 students and�� about 4,000 of those being
%international. TU Dortmund has been ranked among the world's top 50 in
%Nature Index'�s 2018 list of "Rising Stars" mainly due to the number of
%articles from physical sciences. The   
%department of Physics counts about 50 lecturers and post-doctoral
%researchers and about 150 PhD students. 
%Research focuses are particle physics, solid state physics, and accelerator physics. 
%%The attractive study program of the department has led to an enormous increase in undergraduate students with currently more than 1,200.
%The group experimental physics 5 covers a large area of research in particle physics, focused on data analysis and detector development. 
%The group is a member of the LHCb collaboration since 2004 and is central to the development of the High Level Trigger of the experiment and was leading the efforts towards the upgrade trigger Technical Design Report. 
%In addition, the group is significantly contributing to the upgraded
%tracking detector (SciFi-tracker) and core physics of LHCb. 
%%The group is also strongly involved in the core physics analyses of the LHCb experiment. 
%The group intensely collaborates with the local computer science
%department in the frame of the Collaborative Research Center CRC876
%"Providing Information by Resource-Constrained Data Analysis" and the
%recently founded Dortmund Data Science Center.
%%The Initial Training Network proposal fits extremely well with the group's existing research profile on advanced data analysis techniques.
}  
%\tabularnewline\hline
%\pbox{8cm}{\Tstrut Role and\\Commitment\\ of Key persons} &%
%{\vspace{-8mm}
%\begin{enumerate}%[topsep=0pt,itemsep=-2pt,leftmargin=*]
%\item Dr. Johannes Albrecht, lecturer at the faculty of physics,
%  receiver of a Emmy-Noether Grant (2013 - 2018) and the ERC Starting
%  Grant PRECISION in 2016. 
%Expertise in track reconstruction, triggering, real time data analysis, multivariate analyses. 
%LHCb trigger deputy project leader (2011-2014), LHCb upgrade trigger
%project leader (2012-2014), physics analysis group convener
%(2011-2013), deputy physics coordinator (since 2018). Has supervised 8
%PhD and 18 MSc students (finished and ongoing). 
%Role: supervisor of \ESRd and \ESRe. 
%Commitment: 20\%
%
%\item Prof. Bernhard Spaan, head of experimental physics 5, team leader for the LHCb experiment. 
%Collaboration board chair of the LHCb experiment (2013-2017). 
%Project leader in the computer science collaborative research center 
%%special research area "Providing Information by Resource-Constrained
%%Data Analysis" 
%CRC876, former Dean of the faculty of physics, former elected head of the
%committee of German particle physicists (KET). 
%Expertise in data analysis in multiple experiments, including LHCb,
%BABAR, CLEO and ARGUS. 
%Has supervised about 100 MSc and PhD students. 
%Role: co-supervisor of \ESRd and \ESRe, commitment: 10\%
%
%\end{enumerate}
%}
%\tabularnewline\hline   
%\pbox{8cm}{\Tstrut Key Research\\Facilities,\\Infrastructure\\and Equipment\Bstrut} & %
%\pbox{19cm}{\Tstrut 
%The department of physics is involved in data analysis at the CERN based experiments LHCb and ATLAS, in neutrino experiments (Magic, Ice Cube, Cobra) and also has a strong particle physics theory department. 
%ESRs benefit from the close link to the theory part of the department and from the intense collaboration between the department of physics and the department of computer science, which is also formalised in the
%participation of two research groups in the Collaborative Research Center (CRC876). 
%The group has access to excellent computing resources, including a local computing cluster and are eligible to perform distributed analyses on the Grid. 
%} \tabularnewline\hline 
%\multicolumn{2}{l}{\hspace{-1ex}Independent \Tstrut  research premises\Bstrut: yes}\tabularnewline\hline
%\pbox{8cm}{\Tstrut Past \& current\\involvement\\in Research and\\Training\\Programmes\Bstrut} &  
%\pbox{19cm}{\Tstrut  
%The TU Dortmund is involved in or coordinating several national and
%international graduate schools and training programs including five
%MSCA-ITNs – one of them in coordinating function. The group
%experimental physics 5 participates in the  Collaborative Research
%Center  (CRC876), which ideally complements the proposed ITN. Its
%integrated bi-yearly graduate schools are open to the members of the
%ITN. The department has an extensive programme on graduate and
%post-graduate courses. Additionally, the local ESRs will benefit from
%the services of the Graduate Center of TU Dortmund and of the Research
%Academy Ruhr, one of the largest and most powerful platforms in
%Germany, supported by the three University Alliance Ruhr universities
%of Bochum, Dortmund, and Duisburg-Essen, with the goal to support
%young researchers and prepare them for careers inside and outside
%academia. 
%% The TU Dortmund currently co-ordinates or participates in 36 EU-projects, amongst them 3 Marie-Curie ITNs. 
%% The group experimental physics 5 participates in the special research area \textit{Providing Information by Resource-Constrained Data Analysis} (SFB876), which ideally complements the proposed ITN. 
%% Its bi-yearly graduate schools are also open to the members of the ITN. 
%% The department has an extensive programme on graduate and post-graduate courses. 
%% Additionally, the local ESRs will become members of the \textbf{Research Academy Ruhr}, one of the largest and most powerful platforms in Germany to support young researchers and prepare them for careers inside and outside academia at the University alliance of three UA Ruhr Universities: Bochum, Duisburg-Essen and TU Dortmund.   
%} \tabularnewline\hline\Tstrut
%\pbox{8cm}{\Tstrut Relevant\\Publications} &%
%{\vspace{-3mm}
%\begin{itemize}%[topsep=0pt,itemsep=-2pt,leftmargin=*]
%\item The LHCb and CMS Collaborations, Observation of the rare $B^0_s\to\mu^+\mu^-$ decay from the combined analysis of CMS and LHCb data, submitted to Nature.
%\item The LHCb and CMS Collaborations, "Observation of the rare $B^0_s \rightarrow \mu^+ \mu^-$ decay", Nature 522 (2015) 68–72
%\item R. Aaij, J. Albrecht, et al., "The LHCb Trigger and its Performance in 2011", JINST {\bf 8} (2013) P04022. 
%\item The LHCb Collaboration, "Search for the lepton flavour violating decay $\tau\to\mu\mu\mu$, JHEP 02 (2015) 121
%\item The LHCb Collaboration, ``LHCb Trigger and Online Upgrade Technical Design Report'', CERN-LHCC-2014-016
%\item The ARGUS Collaboration, "Observation of $B^0-\bar{B^0}$ mixing", Phys.Lett. B192 (1987) 245
%\end{itemize}
%}\tabularnewline\hline

%\end{tabular}
%}%
%\end{center}