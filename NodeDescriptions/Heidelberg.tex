\resizebox{\textwidth}{!}{%
\begin{tabular}{@{}p{25mm}|p{190mm}@{}}
\toprule
\multicolumn{2}{c}{\large\textbf{Beneficiary: \heidelberglong}}\tabularnewline\hline
\pbox{8cm}{\Tstrut General\\Description\Bstrut} & %
\pbox{19cm}{\Tstrut  
\heidelberglong  was established in 1386 and is the oldest university in Germany.  
It has twelve faculties with a total of more than 30,000 students. 
The Department of Physics and Astronomy is the largest physics department in Germany with
350 new incoming students and more than 100 PhD degrees per year.
It follows the idea of teaching rooted in research and sees its research programme at the borders of knowledge as a pre-requisite for teaching and training its students at high quality. 
The both experimental institutes involved in this application, the Kirchhoff Institute for Physics (KIP) and the Physikalisches Institut (PI), perform research in the areas of classic complex systems, quantum systems, heavy ion physics, atomic physics  as well as in fundamental particle physics.  
\Bstrut}\tabularnewline\hline

\pbox{8cm}{\Tstrut
  Role and\\Commitment\\of Key persons} & %
{\vspace{-8mm}
\begin{enumerate}%[topsep=0pt,itemsep=-2pt,leftmargin=*]

\item Prof.~Dr.~Stephanie Hansmann-Menzemer is a full professor at the PI of the \hd.
  She has a strong background in flavour physics, with focus on mixing and CP violation. She is an expert in
  tracking algorithms and core software development for online and offline application and she is involved in the development of the
  readout-system for the upgrade of the LHCb forward tracking system. 
  Prof.~Dr.~Hansmann-Menzemer won several prestigious research prizes, among others the EPS young researcher award (2007), an Emmy-Noether young investigator grant (2006) and an ERC starting grant (2010).
  She is vice dean of the faculty and co-spokesperson of the research training {\itshape Particle Physics Beyond the Standard Model} of the German research foundation (DFG). She has supervised 15 PhD students as primary advisor.
  Role: supervisor of \ESRn and academic contact. Commitment: 10\%

\item Dr.~Martino Borsato is a postdoctoral researcher at \hd. He has a strong background in rare $B$ decays, Lepton Flavour Universality tests, searches for dark sector particles and phenomenology of SUSY. For these analysis he has developed advanced ML techniques as well as modern parallel architectures. He is currently convener of the LHCb Run 1-2 performance working-group and has convened the LHCb Exotica physics-analysis subgroup in 2017 and 2018. He has a European-wide profile as he studied in Italy, got his PhD in France and has been a PostDoc in Spain and Germany.
Role: co-supervisor of \ESRn. Commitment: 20\%. 

\item apl.~Prof.~Dr.~Monica Dunford is researcher group leader at the University of Heidelberg. She has been a member of the ATLAS collaboration since 2006 and has been active in the hadronic calorimeter, the trigger system, precision measurements of the Standard Model as well as searches for Dark Matter. She was also a member of the SNO collaboration, where she did hardware development and energy reconstruction algorithms for Cherenkov detectors. She has supervised or co-supervised eight PhD students in addition to many masters and bachelors students.
  Role: co-supervisor of \ESRl and academic contact. Commitment: 10\%.


\item Dr.~Pavel Starovoitov is a postdoctoral researcher  at \hd, and a member of the ATLAS Collaboration since 2000. He is leading the Tile Calorimeter Upgrade Simulation group of the ATLAS experiment. He was the convener of the PDF Fit Forum and of the Jets and Photon ATLAS physics working group. He has supervised and co-supervised in total six PhD students at the University of Hamburg, the Belarus State University and at Heidelberg University and worked with several Master and Bachelor students. Dr.~Starovoitov was the holder of the  Marie-Curie Incoming International Fellowship, CERN-INTAS Young Scientist Fellowship.
Role: co-supervisor of \ESRl and recruitment officer. Commitment: 20\%. 
\end{enumerate}
} 
\tabularnewline\hline
\pbox{8cm}{\Tstrut Key Research\\Facilities,\\Infrastructure\\and Equipment} & %
\pbox{19cm}{\Tstrut %}\tabularnewline\hline
%
The ATLAS and LHCb physics group at KIP and PI have their own computing farms with \~800 worker nodes and more than a Petabyte of  disk space for fast local data analysis. 
The farms are connected to the Worldwide HEP grid network as well as to the German national analysis facility (NAF) with several thousands computing cores and Petabytes of disk space. The both experimental institutes are equipped with mechanical and electronic workshops with all together 65 technicians and engineers, which made leading contributions  to the design, construction and commissioning of e.g. the ALICE Transition Radiation Detector, the LHCb Outer tracker and the Scintillating Fibre Tracker for the upgrade and the ATLAS Calorimeter. %
%
} \tabularnewline\hline
%
\multicolumn{2}{l}{\hspace{-1ex}Independent \Tstrut  research premises\Bstrut: yes}\tabularnewline\hline
\pbox{8cm}{\Tstrut Past \& current\\involvement\\in Research and\\Training\\Programmes} & 
\pbox{19cm}{\Tstrut
  Due to the large students body and the large number of professors -- about 45 full professors and 30 professors from associated research institutes such as for example the Max-Planck-Institute for Nuclear Physics, the Max-Planck-Institute for Astronomy or the Heidelberg Institute for
  Theoretical Studies -- the physics curriculum is very broad and offers many attractive specialisations. 
  Furthermore all PhD students from \hdshort ATLAS and LHCb groups are enrolled in the Heidelberg Graduate School for Physics (HGSFP). In addition to providing excellent  education in astronomy and cosmics physics, particle physics, quantum dynamics,	cosmology, mathematical physics, HGSFP  aims to train young scientists to be able to  cross the boundaries between different fields of fundamental physics. Furthermore, the department is running three research training networks funded by DFG and the Max-Planck-Society:  
  GRK1940  {\itshape Particle Physics Beyond the Standard Model}, GRK2058 {\itshape High Rate and High Resolution Detectors for Nuclear and Particle Physics} and {\itshape Precision Tests of Fundamental Symmetries}.
  The standard curriculum as well as the dedicated courses of the HGSFP and the three research training groups in particle physics are open for students of the SMARTHEP ETN.
The \heidelberglong participates in  112  projects within FP7 and 49 projects within H2020 programs, where it leads 49 and 26 projects respectively. It includes both collaborative and individual grants.  %For example, The \hdshort group has running EU FP7 training network grant PicoSEC-MCNet 
} \tabularnewline\hline\Tstrut
\pbox{8cm}{\Tstrut Relevant\\Publications} &%
{\vspace{-3mm}
\begin{itemize}%[topsep=0pt,itemsep=-2pt,leftmargin=*]
\item   ATLAS Collaboration: ``Measurement of detector-corrected observables sensitive to the anomalous production of events with jets and large missing transverse momentum in $pp$ collisions at $\sqrt{s}=13$  TeV using the ATLAS detector'' EPJC 77(2017)11
\item   ATLAS Collaboration: ``Jet energy scale measurements and their systematic uncertainties in proton-proton collisions at $\sqrt{s} = 13$ TeV with the ATLAS detector'' Phys.Rev.D 7(2017)072002
\item LHCb Collaboration: ``Search for the lepton flavour violating decay $\tau^-\to\mu^-\mu^+\mu^-$'' JHEP 02(2015)121
\item LHCb Collaboration: ``Search for dark photons produced in 13 TeV pp collisions'' Phys.Rev.Lett. 120(2018)061801
\item LHCb Collaboration: ``Measurement of angular and CP asymmetries in $D^0\to \pi^+\pi^-\mu^+\mu^-$ and $D^0\to K^+K^-\mu^+\mu^-$ decays'', Phys.Rev.Lett. 121(2018)091801

\end{itemize}}\tabularnewline\bottomrule
\end{tabular}
}%

