\begin{center}
\resizebox{\textwidth}{!}{%
\begin{tabular}{@{}p{25mm}|p{190mm}@{}}
\toprule
\multicolumn{2}{c}{\large\textbf{Beneficiary: \heidelberglong}}\tabularnewline\hline
\pbox{8cm}{\Tstrut General\\Description\Bstrut} & %
\pbox{19cm}{\Tstrut  
\heidelberglong  was established in 1386 and is  oldest university in Germany.  
It is also one of the strongest research universities in Europe. 
\heidelberglong has twelve faculties with a total of more than 30,000 students and a research and teaching staff of more than 5,000 scientists - among them 450 professors. 
The Department of Physics and Astronomy follows the idea of teaching rooted in research and sees its research programme at the borders of knowledge as a pre-requisite for teaching and training its students at high quality. 
The Kirchhoff Institute for Physics performs the research in the areas of classic complex systems, quantum systems as well as in fundamental particles and interactions domain.  
It continues the tradition of diverse scientific research and education.  
\Bstrut}\tabularnewline\hline

\pbox{8cm}{\Tstrut Role and\\Commitment\\of Key persons} & %
{\vspace{-8mm}
\begin{enumerate}%[topsep=0pt,itemsep=-2pt,leftmargin=*]

\item Dr.~Pavel Starovoitov is a postdoctoral researcher  at \hd, and a member of the ATLAS Collaboration since 2000. He is leading the Tile Calorimeter Upgrade Simulation group of the ATLAS experiment.  He was the convener of the PDF Fit Forum and of the Jets and Photon Physics working group  in  ATLAS Collaboration. He has been the second supervisor of 1 PhD student at the University of Hamburg,  1 PhD student at the Heidelberg University, and currently supervises 1 PhD student at the Belarus State University and co-supervises 4 PhD students at the Heidelberg University. He has co-supervised several Master and Bachelor students. Dr.~Starovoitov was the holder of the  Marie-Curie Incoming International Fellowship, CERN-INTAS Young Scientist Fellowship.
Role: main supervisor of \ESRl and recruitment officer. Commitment: 20\%. 

\item Priv.~Doz.~Dr.~Monica Dunford is a young researcher group leader at \hd. She has been a member of the ATLAS collaboration since 2006 and has been active in the hadronic calorimeter, the trigger system, precision measurements of the Standard Model as well as searches for Dark Matter. She was also a member of the SNO collaboration, where she did hardware development and energy reconstruction algorithms for Cherenkov detectors. She has supervised or co-supervised eight PhD students in addition to many masters and bachelors students. She is on the programming boards for several international conferences in LHC physics.
Role: co-supervisor of \ESRl and academic contact. Commitment: 10\%. 

\item Prof. Dr. Stefanie Hansmann-Menzemer ...

\item Dr. Martino Borsato is a Postdoctoral Researcher at the PI of \hd. He has a strong background in rare B decays, Lepton Flavour Universality tests, direct searches for light dark sector particles as well as SUSY and Dark Matter phenomenology. For these analysis he has developed advanced machine learning techniques as well as modern parallel architectures. He is currently convener of the LHCb Run 1-2 performance working-group and has convened the LHCb ?Exotica? physics-analysis subgroup in 2017 and 2018. He has a European-wide profile as he studied in Italy, got his PhD in France and has been a PostDoc in Spain and Germany.
Role: co-supervisor of \ESRn. Commitment: 20\%. 

%\vspace{-\belowdisplayskip}
\end{enumerate}
} 
\tabularnewline\hline
\pbox{8cm}{\Tstrut Key Research\\Facilities,\\Infrastructure\\and Equipment} & %
\pbox{19cm}{\Tstrut %}\tabularnewline\hline
%
\hdshort ATLAS group has its own computing farm with about 500 worker nodes and 600TB disk space for a fast local data analysis. 
The farm is connected to the Worldwide HEP grid network as well as to the German national analysis facility (NAF) with several thousands computing cores and Petabytes of disk space. Students will benefit from enrollment in the Heidelberg graduate schools for fundamental Physics that combines doctoral projects from the forefront of international research with a broad and deep 
teaching program in these areas of fundamental physics and emphasizes their interrelations, as well as from strong connections to the Institute of Theoretical Physics, performing 
research in QCD phenomenology, Dark Matter particles and Higgs Physics domains. 
%
\hdshort has fully equipped electronics laboratory with clean room, component placer, high-frequency oscilloscopes, etc. There are ten electronics engineers, two of them work 100\% with the ATLAS group.
} \tabularnewline\hline
%
\multicolumn{2}{l}{\hspace{-1ex}Independent \Tstrut  research premises\Bstrut: yes}\tabularnewline\hline
\pbox{8cm}{\Tstrut Past \& current\\involvement\\in Research and\\Training\\Programmes} & 
\pbox{19cm}{\Tstrut  
All PhD students from the \hdshort ATLAS group are enrolled in the Heidelberg Graduate School of Fundamental Physics (HGSFP). In addition to providing excellent  education in astronomy and cosmics physics, particle physics, quantum dynamics,	cosmology, mathematical physics, HGSFP  aims to train young scientists to be able to  cross the boundaries between different fields of fundamental physics. 
The \heidelberg participates in  112  projects within FP7 and 49 projects within H2020 programs, where it leads 49 and 26 projects respectively. It includes both collaborative and individual grants.  %For example, The \hdshort group has running EU FP7 training network grant PicoSEC-MCNet 
} \tabularnewline\hline\Tstrut
\pbox{8cm}{\Tstrut Relevant\\Publications} &%
{\vspace{-3mm}
\begin{itemize}%[topsep=0pt,itemsep=-2pt,leftmargin=*]
\item   ATLAS Collaboration: ``Measurement of inclusive jet and dijet cross-sections in proton-proton collisions at $\sqrt{s}=13$ TeV with the ATLAS detector'' arXiv:1711.02692
\item  ATLAS Collaboration: ``Measurement of differential cross sections and $W^+/W^-$  cross-section ratios for $W$ boson production in association with jets at $\sqrt{s}=8$ TeV with the ATLAS detector'' arXiv:1711.03296
\item   ATLAS Collaboration: ``Measurement of three-jet production cross-sections in $pp$ collisions at 7 TeV centre-of-mass energy using the ATLAS detector'' EPJC 75(2015)228
\item   ATLAS Collaboration: ``Measurement of detector-corrected observables sensitive to the anomalous production of events with jets and large missing transverse momentum in $pp$ collisions at $\mathbf{\sqrt{s}=13}$  TeV using the ATLAS detector'' EPJC 77(2017)11
\item   ATLAS Collaboration: ``Jet energy scale measurements and their systematic uncertainties in proton-proton collisions at $\sqrt{s} = 13$ TeV with the ATLAS detector'' Phys.Rev.D 7(2017)072002
\end{itemize}}\tabularnewline\bottomrule
\end{tabular}
}%
\end{center}

