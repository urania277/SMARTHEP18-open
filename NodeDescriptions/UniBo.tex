\begin{center}
\resizebox{\textwidth}{!}{%
\begin{tabular}{@{}p{25mm}|p{190mm}@{}}
\toprule
\multicolumn{2}{c}{\large\textbf{Partner organization: \unibolong}}\tabularnewline\hline
\pbox{8cm}{\Tstrut General\\Description\Bstrut} & %
\pbox{19cm}{\Tstrut 
UNIBO  is the second largest university in Italy and one of the most active in research and technology transfer. It stands among the most important institutions of higher education in EU with 87,000 enrolled students
%, 2,857 Academic staff, 1,198 post-docs 
and 1,606 PhDs.
%, 3,014 administrative and technicians staff units. The activities are implemented in 5 Campuses based in the Emilia-Romagna Region (Bologna, Forli', Cesena, Ravenna, Rimini) and a permanent headquarter in Buenos Aires, Argentina. UNIBO offers 51 PhD/Doctoral degree programs and 64 Master's degree programs. 
%The Department of Computer Science and Engineering at UNIBO is the single point of contact for all the research and teaching activities concerning computer science of the University of Bologna. 
The research activities at the Department of Computer Science and Engineering covers the full spectrum of the most active and relevant topics in computer science. At UNIBO, the ESR will be hosted within the the Computer Vision lab (CVLab), which is a research laboratory active in the field of computer vision since more than 20 years. Research carried out at CVLab has lead to many publications in the most prestigious venues of the field (CVPR, ICCV, ECCV,PAMI, IJCV,TIP) and several patents.
\Bstrut}\tabularnewline\hline

\pbox{8cm}{\Tstrut Role and\\Commitment\\of Key persons} & %
{\vspace{-8mm}
\begin{enumerate}%\begin{description}%[topsep=0pt,itemsep=-2pt,leftmargin=*]
\item Prof. Luigi Di Stefano, Full Professor at DISI-UNIBO and founder of CVLab. He teaches Computer Architecture, Computer Vision and Image Processing and has supervised 11 PhD students and many BSc and MSc students. %many undergraduate and master students as well as 11 PhD students.
% and  5 post-docs and 8 research scholarship winners. He is a member of the Board of the PhD program in Computer Science and Engineering at DISI. He has given invited lectures in PhD schools and has been member of the Thesis Defense Committee or Thesis Reviewer for several PhD candidates in Italy and abroad. 
He has coordinated many academic research projects (grants from public national, European or private company sources)
%grants as well as by private companies 
and is author of more than 150 papers and several patents. 
%Recent several scientific events he has contributed to include 3DV2018, %(Demo\&Exhibit Chair), 
%2018 EMVA Forum (Local Host), 2018 SIAM Conference on Imaging Science (MoC), ICCV2017 (Area Chair).
Role: academic supervision of \ESRm. 
Commitment: 10\%
\item Dr. Samuele Salti, assistant professor at DISI, %Dr. Samuele Salti holds a Ph.D. in Computer Engineering from the University of Bologna, Italy. He 
is co-author of 33 papers on international scientific journals or conference proceedings and of 1 US patent. 
%Since 2018 he is Assistant Professor at DISI. 
He has supervised more than 20 students at the bachelor and master level. 
Role: co-supervision of \ESRm. 
Commitment: 10\%
%\vspace{-\belowdisplayskip}
%\end{description}} 
\end{enumerate}} 
\tabularnewline\hline

\pbox{8cm}{\Tstrut Key Research\\Facilities,\\Infrastructure\\and Equipment} & %
\pbox{19cm}{\Tstrut 
Key Research Facilities, Infrastructure, and Equipment: 
The laboratory features about 10 PCs equipped with high-performance NVIDIA GPU boards
%(GEFORCE GTX 1080 Ti, Tesla K40,TESLA V100)
and deep learning software frameworks. 
As UNIBO is member of CINECA, CVlab can apply to get free access to D.A.V.I.D.E., the most powerful european deep learning supercomputer (180 Tesla P100 GPUs). 
Other equipment available at CVLab include embedded and mobile computer vision platforms, RGB, RGB-D and stereo cameras, Augmented/Virtual Reality Headsets.} \tabularnewline\hline
%%
\multicolumn{2}{l}{\hspace{-1ex}Independent \Tstrut  research premises\Bstrut: yes}\tabularnewline\hline
\pbox{8cm}{\Tstrut Past \& current\\involvement\\in Research and\\Training\\Programmes} & 
\pbox{19cm}{\Tstrut 
%At EU level UNIBO ranks 2nd University in Italy and 37th in Europe in the ranking of top 50 Higher or secondary education organizations in FP7 signed grant agreements in terms of counts of participations (EC 7th Monitoring Report). 
UNIBO is involved in 270 projects funded in FP7
%(87.7 MEUR granted)
, 41 of which are PEOPLE projects, with 11 ITN (3 coordinated by UNIBO). 
In H2020, UNIBO is involved in 182 funded projects,
% (71,64 MEUR granted)
38 of which are MSCA projects (20 ITN - 4 coordinated by UNIBO).
%-; 8 RISE -3 of which coordinated by UNIBO, 8 IF ? of which 3GF and 5 EF; 2 NIGHT - 1 coordinated by UNIBO). 
According to the DG EAC (April 2014) UNIBO ranks 1st University in Italy participating in MCA.
%At UNIBO, research activities are promoted, coordinated and supported by the 33 Scientific Departments and by the Research and Knowledge Transfer Division, with the support of the European Research \& Innovation Office which has more than 10 years experience with European projects.
} \tabularnewline\hline\Tstrut
\pbox{8cm}{\Tstrut Relevant\\Publications} &%
{\vspace{-3mm}
\begin{itemize}%[topsep=0pt,itemsep=-2pt,leftmargin=*]

\item P. Z. Ramirez, M. Poggi, F. Tosi, S. Mattoccia, L. Di Stefano. Geometry meets semantics for semi-supervised monocular depth estimation. Proceedings of ACCV 2018 and \href{arXiv:1810.0493}{https://arxiv.org/abs/1810.04093}
\item A. Tonioni, M. Poggi, S. Mattoccia, L. Di Stefano. Unsupervised adaptation for deep stereo. 
Proc. of ICCV 17%}{http://openaccess.thecvf.com/content_ICCV_2017/papers/Tonioni_Unsupervised_Adaptation_for_ICCV_2017_paper.pdf}
\item D. Palossi, F. Tombari, S. Salti, M. Ruggiero, L. Di Stefano, L. Benini. GPU-SHOT: Parallel Optimization for Real-Time 3D Local Description. Proc. of CVPR Workshops 2013%}{https://www.cv-foundation.org/openaccess/content_cvpr_workshops_2013/W10/html/Palossi_GPU-SHOT_Parallel_Optimization_2013_CVPR_paper.html}
\end{itemize}}\tabularnewline\bottomrule

\end{tabular}
}%
\end{center}
 