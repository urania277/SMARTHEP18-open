\begin{center}
\resizebox{\textwidth}{!}{%
\begin{tabular}{@{}p{25mm}|p{190mm}@{}}
\toprule
\multicolumn{2}{c}{\large\textbf{Beneficiary: \helsinkilong}}\tabularnewline\hline
\pbox{8cm}{\Tstrut General\\Description\Bstrut} & %
\pbox{19cm}{\Tstrut 
The \helsinkilong  is nearly 400 years old and is the leading university in Finland. 
It was placed number 90 in the most recent Times Higher Education World University Rankings. 
The University has 40,000 students of which 4,700 are doctoral students (1/4 international) distributed on 32 doctoral programs. 
The university is a top research university and is e.g. a member of the League of European Research Universities. 
The Department of Physics is one of the largest departments of the university and has 30 professors and about 330 annual person-years.
\Bstrut}\tabularnewline\hline
\pbox{8cm}{\Tstrut Role and\\Commitment\\of Key persons} & %
{\vspace{-8mm}
\begin{enumerate}%[topsep=0pt,itemsep=-2pt,leftmargin=*]
\item  Prof~Mikko Voutilainen, Assistant Professor of Experimental
  Elementary Particle Physics, \helsinkilong. 
He is a world expert on jet energy corrections and jet measurements,
and recipient of two prestigious prizes in this field (URA Thesis
Award for PhD, and Wu-Ki Tung Award for post-doc). 
%He is project leader of the CMS Experiment project at HIP, co-convener of CMS-TOP-mass subgroup and PI of the Academy of Finland project “Top quarks and gluons”. Role: main supervisor of ESR1. Commitment: 15%
He is project leader of the CMS Experiment project at HIP, co-convener
of CMS-SMP-Jets group and PI of the UH project ``Precision
measurements of QCD'' and the Academy of Finland project ``Top quarks
and gluons''. 
Role: main supervisor of \ESRa. 
Commitment: 15\% 
\item Dr.~Henning Kirschenmann, Postdoctoral Researcher at the Helsinki Institute of Physics. 
He has a strong background in jet physics (top mass, jet calibration), searches for new physics (SUSY), and jets at the trigger level. 
He is currently CMS-SMP-hadronic subgroup co-convener in CMS and
focuses on innovative use of DNNs for improving jet performance.  
Role: additional supervisor of \ESRa. 
Commitment: 20\%
\item Prof~Paula Eerola, Professor of Experimental Elementary Particle
  Physics, \helsinkilong. 
Vice-Rector at the University of Helsinki. She is a leading expert in
all aspects of B physics at LHC-CMS, physics beyond the Standard Model
at LHC-CMS. Novel data processing techniques. Application of particle
physics detectors to radiation detection. 
%Director of Helsinki Institute of Physics. She is a leading expert in all aspects of B physics at LHC-CMS, physics beyond the Standard Model at LHC-CMS. 
%Novel data processing techniques. 
%Application of particle physics detectors to radiation detection. 
Role: Senior advisor, co-supervisor of \ESRa. 
Commitment: 5\% 
%\vspace{-\belowdisplayskip}
\end{enumerate}
} \tabularnewline\hline
\pbox{8cm}{\Tstrut Key Research\\Facilities,\\Infrastructure\\ and Equipment} & %
\pbox{19cm}{ \helsinkientity is a Tier-2 site in the LHC Computing Grid and extensive local computing resources are available for physics analyses. 
The \helsinkientity Detector laboratory is playing a critical role in upgrades to the CMS tracker and there is a possibility to cooperate with the local theory community as well.  
}
\tabularnewline\hline
\multicolumn{2}{l}{\hspace{-1ex}Independent \Tstrut research premises\Bstrut: yes}\tabularnewline\hline
\pbox{8cm}{\Tstrut Past \& current\\involvement\\in Research and\\Training\\Programmes\Bstrut} & 
\pbox{19cm}{\Tstrut The department of physics has participated in 3 FP7 MSC-ITN (CLOUD-ITN, CLOUD-TRAIN, HEXACOMM) projects and coordinated 2 FP7 IRSES (LAIC, GHG-LAKE) and 1 FP7 IAPP (MeChanICs) projects.
UH is currently participating in 11 H2020 MSC ITN and 8 RISE projects and hosting 19 MSC Individual Fellowships.
The department of physics hosts 4 MSCA-IF (nanoCAVa, OXFLUX, FRoST, LAWINE) projects and is participating in 1 MSCA-RISE (NonMinimalHiggs) project and in 1 MSCA-ITN project (CLOUDMOTION).
}
\tabularnewline\hline
\pbox{8cm}{\Tstrut Relevant\\Publications} &%
{\vspace{-3mm}
\begin{itemize}%[topsep=0pt,itemsep=-2pt,leftmargin=*]
\item (2017) Jet energy scale and resolution in the CMS experiment in pp collisions at 8 TeV. [CMS Collaboration]. JINST 12 (2017) no.02, P02014
\item (2011) Determination of jet energy calibration and transverse momentum resolution in CMS. [CMS Collaboration]. JINST 6 (2011) P11002
\item (2017) Search for dijet resonances in proton–proton collisions at $\sqrt{s}=13$ TeV and constraints on dark matter and other models. [CMS Collaboration]. Phys.Lett. B769 (2017) 520-542
\item (2017) Identification of heavy-flavour jets with the CMS detector in pp collisions at 13 TeV. [CMS Collaboration].  arXiv:1712.07158 [physics.ins-det]
\vspace{-4mm}
\end{itemize}
}\tabularnewline\hline
\end{tabular}
}%
\end{center}
%\checkme{PS: Publications are missing}
