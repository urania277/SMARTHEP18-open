\begin{center}
\resizebox{\textwidth}{!}{%
\begin{tabular}{@{}p{25mm}|p{190mm}@{}}
\toprule
\multicolumn{2}{c}{\large\textbf{Partner organization: \ohiolong}}\tabularnewline\hline
\pbox{8cm}{\Tstrut General\\Description\Bstrut} & %
\pbox{19cm}{\Tstrut 
Ohio State University was founded in 1870, and is one of the largest and most comprehensive campuses in the US, and one of the nation's top-20 public university. Ohio State has an academic staff of more than 6500, and teaches more than 60000 students . Of particular interest to this network's research program is CCAPP, the Center for Cosmology and AstroParticle Physics (CCAPP), to build upon the unique environment between the OSU Departments of Astronomy and Physics, to pursue collaborative research at the interface of cosmology, astrophysics, and high energy physics in interdisciplinary research.
\Bstrut}\tabularnewline\hline

\pbox{8cm}{\Tstrut Role and\\Commitment\\of Key persons} & %
{\vspace{-8mm}
\begin{enumerate}%[topsep=0pt,itemsep=-2pt,leftmargin=*]
\item Prof. Antonio Boveia is an assistant professor in experimental particle physics and a CCAPP member. He has covered positions of responsibility in ATLAS, such as Jets and Dark Matter group convenor and Astroparticle Forum convenor, and he is currently one of the LHC Dark Matter Forum organizers. 
%He has extensive experience in tracking from the CDF experiment and from his postdoctoral building and commissioning the ATLAS FTK. 
He is the contact person of the first trigger-level analysis in ATLAS. \textbf{Role:} FPGA lectures at the introductory school, supervision of seconded students (ESR13). Commitment: 10\%
\item Prof. Linda Carpenter is an associate professor in theoretical particle physics and a CCAPP member. She is an expert in Dark Matter and Supersymmetry theories at colliders, as well as in comparisons between collider and non-collider experiments. \textbf{Role:} responsible for the DM theory and collider/non-collider lectures at the introductory school, theory advice. Commitment: 10\%  
\vspace{-4mm}
\end{enumerate}} \tabularnewline\hline

\pbox{8cm}{\Tstrut Key Research\\Facilities,\\Infrastructure\\and Equipment} & %
\pbox{19cm}{\Tstrut 
CCAP will provide expertise in Dark Matter theory and searches throughout the network. The ATLAS group of which Boveia is a member hosts postdoctoral researchers and engineers that are expert in FPGA programming and in FTK monitoring, and will provide training and secondment expertise for \acronym based in their offices at \cernentity.  
} \tabularnewline\hline
%
\multicolumn{2}{l}{\hspace{-1ex}Independent \Tstrut  research premises\Bstrut: yes}\tabularnewline\hline
% \pbox{8cm}{\Tstrut Past \& current\\involvement\\in Research and\\Training\\Programmes} & 
% \pbox{19cm}{\Tstrut 
% The Ohio State University has not previously been involved with EU research projects or grants.} \tabularnewline\hline\Tstrut
\pbox{8cm}{\Tstrut Relevant\\Publications} &%
{\vspace{-3mm}
\begin{description}%[topsep=0pt,itemsep=-2pt,leftmargin=*]

\item [A. Boveia, C. Doglioni (editors) et al.], Dark Matter Benchmark Models for Early LHC Run-2 Searches: Report of the ATLAS/CMS Dark Matter Forum, arXiv:1507.00966
%\item [The ATLAS Collaboration], Trigger-object Level Analysis with the ATLAS detector at the Large Hadron Collider: summary and perspectives (written as complement to HEP Software Foundation whitepaper), ATL-DAQ-PUB-2017-003
%\item [The ATLAS Collaboration], Performance of the ATLAS Trigger System in 2015. Eur.Phys.J. C77 (2017) no.5, 317
\item [L. Carpenter et al.], Mono-Higgs: a new collider probe of dark matter, Phys. Rev. D 89, 075017 (2014)
\item [L. Carpenter et al.], Collider searches for dark matter in events with a Z boson and missing energy, Phys.Rev. D87 (2013) no.7, 074005 
\vspace{-4mm}
\end{description}}\tabularnewline\bottomrule
\end{tabular}
}%
\end{center}
