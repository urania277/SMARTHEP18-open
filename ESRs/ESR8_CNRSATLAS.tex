%\begin{table}[h]

%\begin{table}[h]
\begin{center}\small
\resizebox {\textwidth }{!}{%
\begin{tabular}{|p{19mm}|p{20mm}|p{30mm}|p{30mm}|p{23mm}|p{58mm}|}
\hline
\textbf{\Tstrut Fellow} \,\,ESR8&
\textbf{Host} \,\, \cnrs&
\textbf{\phd} \,\, Yes&
\textbf{Start} \,\, Month 8&
\textbf{Duration} \,\, 36&
\textbf{Deliverables} \,\, \deliverableWhitepaperStateOfTheArtWPThree,\deliverableWhitepaperDevelopmentWPThree,\deliverableTriggerExperimentalSoftwareWPThree,\deliverableWhitepaperStateOfTheArtWPFour,\deliverableWhitepaperDevelopmentWPFour, \deliverableWhitepaperStateOfTheArtWPFive,\deliverableWhitepaperCollectionPapersWPFive,\deliverableFleetmaticsMLMobile \tabularnewline 
%ESR8 &  \cnrs & Yes & Month 6& 36 & , YY \tabularnewline
\hline
\multicolumn{2}{|l|}{\textbf{\Tstrut Work Package:}
WP3,4,5,6} &
\multicolumn{2}{l|}{\textbf{Doctoral programme:} \parisUlong }& %\tabularnewline\hline
\multicolumn{2}{l|}{\textbf{\Tstrut Title: Real-time trajectory reconstruction in ATLAS} %for online event selection and analysis
}\tabularnewline\hline
\multicolumn{6}{|p{20.2cm}|}{\textbf{\Tstrut Objectives:}
ESR8 will be trained in the real-time reconstruction of trajectories and images in both HEP and industry, and apply them to the reconstruction and analysis of particle trajectories in ATLAS as well as GPS data processing. 
The hardware tracking processors FTK/FTK++ allow ATLAS to find particle trajectories in real time. 
Their performances such as efficiency and resolution are not determined
just by the hardware, but also by a learned database of pattern bank and geometrical constants.% loaded into FTK/FTK++. 
ESR8's first objective will be to deploy statistical and computing techniques,
e.g. Principal Component Analysis and Graph Clustering, in order to produce new databases
for FTK and evaluate their impact on raw performances and physics analyses.
ESR8's second objective will be to extend the application of FTK-like algorithms outside of HEP, by providing flexible and powerful tools to train datasets.
%such as real-time image recognition or genomic data analysis, 
These inter-sector toolkits will be developed during the secondment at \fleetmaticsentity, and tested at \pisaentity, 
whose geographical proximity allows for a seamless integration of the work done in both.% the research and industrial context. 
At \fleetmaticsentity ESR8 will be trained in the continuous online learning of 
ML models based on streams of labelled data, and in both the \apachespark\enspace 
%environment for parallel programming 
and the Amazon Web Services infrastructures for massively parallel computations.
ESR8 will develop an online learning tool to continuously process real-time GPS data from Fleetmatics customers,
%The developed tool will be exploited to 
providing customers with smart insights, improving customer experience, and thus
increasing engagement with Fleetmatics products. 
%After the secondment, the student will have acquired expertise in the Spark processing framework, specifically involving machine learning on data streams, and in the Amazon Web Services infrastructure.
%At the end of the secondment, ESR8 will be familiar with the Apache Spark technology and with elements of parallel computing.
%This project aims to train researchers in the usage of advanced computing techniques for real-time analysis in the fields
%of Physics and mobile platforms, as well as to contribute to searches for physics beyond the Standard Model ("new physics") 
%with the ATLAS detector. The main focus of ESR8 is to study and improve the real time reconstruction of the trajectories of charged particles
%in the ATLAS experiment using the hardware tracking processor FTK and FTK++.
%By improving and tuning the training procedure
%it is possible to continuously improve the tracking performance and optimize its use for new physics searches.
%The first objective of the ESR will be to  
%As a second objective, the developed toolkits will
%also allow to extend  
The acquired skills will feed back into the main research project and be applied to the FTK working dataset production. %, applying the computing techniques learned at Fleetmatics to .
ESR8's third objective will be to investigate in detail the impact of new training on selected physics cases.
In particular FTK tracks will be used to improve jet reconstruction and calibration, namely for the suppression
of pile-up jets and the track-based components of the global sequential calibration. This will enhance the
sensitivity of RTA NP searches with dijet mass distributions.
}\tabularnewline\hline
\multicolumn{6}{|p{20.2cm}|}{\textbf{\Tstrut Expected Results:}
Two peer-reviewed papers, one documenting the toolkits
and the improvements on the tracking performance, and one documenting the RTA of dijet mass distributions using FTK tracks. The toolkits developed for the training of
FTK datasets will be adapted for usage outside of ATLAS and released. 
ESR8 will receive a PhD in experimental HEP at \sorbonneentity. 
}\tabularnewline\hline
%\multicolumn{6}{|p{20.2cm}|}{\textbf{Doctoral program:} Cambridge}\tabularnewline\hline
\multicolumn{6}{|p{20.2cm}|}{\textbf{\Tstrut Secondments:}
\fleetmaticsentity, 4 months, Sambo, development of an online learning tool for GPS data processing.
\pisaentity, 3 months, Roda and Annovi, use of toolkits for creation of FTK pattern banks. 
}\tabularnewline
\hline
\end{tabular}
}%
\end{center}
%\end{table}
%

%The ATLAS trigger infrastructure has an unique hardware processor to reconstruct the trajectory (tracking) of the charged particles that cross the silicon inner tracker of the experiment. The tracking information is an essential tool for effective real-time event selection and has a central role in the whole ATLAS physics program especially in the HL-LHC phase. The current hardware processor, FTK, is a complex system made by several custom electronics  boards based on FPGAs and Associative Memory chips. The latter are unique computing devices developed for the FTK algorithm. The hardware tracking will be also a central part of the Phase-II Upgrade of ATLAS, with upgraded version of FTK called FTK++.


