%\begin{table}[h]

\begin{center}\small
\resizebox {\textwidth }{!}{%
\begin{tabular}{|p{25mm}|p{23mm}|p{18mm}|p{28mm}|p{34mm}|p{60mm}|}
\hline
\textbf{\Tstrut Fellow} \,\,\ESRb&
\textbf{Host} \,\,\unigeentity&
\textbf{\phd} \,\,Yes&
\textbf{Start (mo.)} \,\,8&
\textbf{Duration (mo.)} \,\,36&
\textbf{Deliverables} \,\,\deliverableWhitepaperStateOfTheArtWPThree, \deliverableWhitepaperDevelopmentWPThree, \deliverableTriggerExperimentalSoftwareWPThree, \deliverableWhitepaperStateOfTheArtWPFour, \deliverableWhitepaperDevelopmentWPFour,  \deliverableWhitepaperStateOfTheArtWPFive, \deliverableWhitepaperCollectionPapersWPFive, \deliverablePredictiveMaintenance  \tabularnewline 
\hline
\multicolumn{2}{|l|}{\textbf{\Tstrut Work Package:}
\WPESRb} &
\multicolumn{2}{l|}{\textbf{Doctoral programme:} \unigeentity }&%\tabularnewline\hline
\multicolumn{2}{l|}{\textbf{\Tstrut Title: 
ML pattern recognition for exotic physics and industry}
}\tabularnewline\hline
\multicolumn{6}{|p{21.2cm}|}{\textbf{\Tstrut Objectives:}
Real time particle track reconstruction in LHC experiments is particularly difficult because of extremely busy detector images created by the multiple proton interactions (pile-up) in each bunch collision. 
This challenge will only increase in the future LHC upgrade.
\ESRb's first objective is to develop ML-based track reconstruction in ATLAS, as a replacement to algorithms that are too slow to be used in the trigger.
\ESRb will be trained in track reconstruction, modern ML techniques, and general pattern recognition tools. 
This expertise will be crucial for \ESRb's second objective and secondment with \lightboxentity, in which \ESRb will utilise ML techniques to collect and analyze data from IoT sensors in industrial production chains in order to forecast in real-time when the machinery needs intervention.  
\ESRb's third objective will be to evaluate GPUs for track reconstruction at the higher pile-up conditions of the LHC upgrades.
\ESRb will compare GPU-optimized ML reconstruction to CPU-based reconstruction and dedicated hardware (e.g. FPGA) solutions proposed for ATLAS. 
Through this the ESR will be trained in optimizing algorithms for modern computing architectures.
\ESRb's will then apply their knowledge to a novel real-time displaced vertex selection in ATLAS, one of the most promising and experimentally challenging NP signatures. 
\ESRb's search for exotic long-lived (LLP) signatures with this selection will be a significant step beyond ATLAS's current capabilities, answer crucial questions for the future of ATLAS and open new avenues in the searches for new physics. 
}\tabularnewline\hline
\multicolumn{6}{|p{21.2cm}|}{\textbf{\Tstrut Expected Results:}
1. ML-based trigger-level tracking reconstruction software in ATLAS (peer-reviewed paper). 
2. Intervention forecasting algorithms for IoT-ready plants in \lightboxentity. 
3. Comparative assessment of GPU-based tracking. 
4. Application to long-lived particle search (peer-reviewed paper). 
\ESRb will receive a PhD in experimental HEP at \unigelong. 
%TODO: link this with other GPU-related things
}\tabularnewline\hline
\multicolumn{6}{|p{21.2cm}|}{\textbf{\Tstrut Secondments:}
\lightboxlong, 6 months, Catastini, ML for real-time industrial sensor data acquisition and analysis. 
}\tabularnewline
\hline
\end{tabular}
}%
\end{center}



