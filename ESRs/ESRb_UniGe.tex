%\begin{table}[h]

\begin{center}\small
\resizebox {\textwidth }{!}{%
\begin{tabular}{|p{19mm}|p{37mm}|p{16mm}|p{21mm}|p{23mm}|p{64mm}|}
\hline
\textbf{\Tstrut Fellow} \,\,\ESRb&
\textbf{Host} \,\,\unigelong&
\textbf{\phd} \,\,Yes&
\textbf{Start} \,\,Month 8&
\textbf{Duration} \,\,36&
\textbf{Deliverables} \,\,\deliverableWhitepaperStateOfTheArtWPThree, \deliverableWhitepaperDevelopmentWPThree, \deliverableTriggerExperimentalSoftwareWPThree, \deliverableWhitepaperStateOfTheArtWPFour, \deliverableWhitepaperDevelopmentWPFour, \deliverableWhitepaperStateOfTheArtWPFive, \deliverableWhitepaperCollectionPapersWPFive, \deliverablePredictiveMaintenance \tabularnewline 
%Add list of deliverables here
%2.1 & First \acronym conference proceedings & R & PU & 19 & 2 & \saclay & Write and publish proceedings of the first \acronym conference\tabularnewline\midrule 
%2.2 & Toolkit for deep learning & R & PU & 44 & 2 & \saclay & Implementation and release of toolkit for deep learning within HEP\tabularnewline\midrule
%2.3 & Medical insurance provision & R & PU & 44 & 2 & \dq & Application of developed methods to medical insurance provision\tabularnewline\midrule
%2.4 & Deep learning documentation & R & PU & 12,24,36,48 & 2 & \saclay & Publication of research in peer-reviewed journals\tabularnewline\midrule 
%3.1 & Second \acronym conference proceedings & R & PU & 43 & 3 & \dortmund & Proceedings of the second \acronym conference\tabularnewline\midrule
%3.2 & Novel MVA triggers& R & PU & 44 & 3 & \dortmund & Implement novel MVA strategies in LHCb trigger\tabularnewline\midrule
%3.3 & Review of DS methods & R & PU & 44 & 3 & \saclay & Review paper on relationship between DS methods and datasets\tabularnewline\midrule
%3.4 & New figures of merit documentation & R & PU & 12, 24, 36, 48 & 3 & \dortmund & Publication of research in peer-reviewed journals\tabularnewline\midrule
%\ESRb &  \unigelong & Yes & Month 6& 36 &\deliverableLLPTrackingToolkit, \deliverableTechPubLLPGPU, \deliverableHEPPubLLP, \deliverablePredictiveMaintenance \tabularnewline
\hline
\multicolumn{2}{|l|}{\textbf{\Tstrut Work Package:}
WP3,4,5,6} &
\multicolumn{2}{l|}{\textbf{Doctoral programme:} \unige }&%\tabularnewline\hline
\multicolumn{2}{l|}{\textbf{\Tstrut Title: 
ML pattern recognition for exotic physics and industry}
}\tabularnewline\hline
\multicolumn{6}{|p{20.2cm}|}{\textbf{\Tstrut Objectives:}
\ESRb will study RTA in both HEP and industry, be trained in state-of-the art ML and AI methods, and use these to improve ATLAS track reconstruction and monitoring of industrial machinery.
RTA track reconstruction in ATLAS is particularly difficult because of extremely busy detector images created by the multiple proton interactions (pile-up) in each bunch collision. 
This challenge will only increase in the future LHC upgrade.
\ESRb's first objective is to develop ML-based track reconstruction as a replacement to algorithms too slow to be used in real time.
\ESRb will be trained in track reconstruction, modern ML techniques, and general pattern recognition tools. 
The acquired ML expertise will be crucial for \ESRb's second objective and secondment, in which \ESRb will be trained in industrial production chains, sensors used in IoT-ready plants, collecting and aggregating sensor data in real-time, and forecasting analysis techniques. 
\ESRb will then utilise ML techniques to collect and analyze data from Internet-Of-Things-ready industrial production chains in order to forecast in real-time when the machinery needs intervention.  
\ESRb's third objective will be to evaluate GPUs for track reconstruction, especially at higher pile-up conditions of the LHC upgrades.
\ESRb will compare GPU-optimized ML reconstruction to both CPU-based reconstruction and dedicated hardware (e.g. FPGA) solutions proposed for ATLAS. 
Through this the ESR will be trained in optimizing algorithms for modern computing architectures.
\ESRb's will then apply their reconstruction to a novel real-time  displaced vertex selection in ATLAS, one of the most promising and experimentally challenging NP signatures. 
\ESRb's  search for exotic long-lived signatures with this selection will be a significant step beyond ATLAS's current capabilities, answer crucial questions for the future of ATLAS and open new avenues in the searches for NP. 
}\tabularnewline\hline
\multicolumn{6}{|p{20.2cm}|}{\textbf{\Tstrut Expected Results:}
Trigger-level tracking reconstruction software, including long-lived particles, for ATLAS.
Two original research papers: a technical publication of the ML-based track reconstruction and its evaluation on GPUs, and a physics publication documenting the results of the search for NP using this technique. 
\ESRb will receive a PhD in experimental HEP at \unigelong. 
%TODO: link this with other GPU-related things
}\tabularnewline\hline
\multicolumn{6}{|p{20.2cm}|}{\textbf{\Tstrut Secondments:}
\lightboxlong, 6 months, Catastini, ML for real-time industrial sensor data acquisition and analysis. 
}\tabularnewline
\hline
\end{tabular}
}%
\end{center}



