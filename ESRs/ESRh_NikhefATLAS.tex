\begin{center}\small
\resizebox {\textwidth }{!}{%
\begin{tabular}{|p{25mm}|p{23mm}|p{18mm}|p{28mm}|p{34mm}|p{50mm}|}
\hline
\textbf{\Tstrut Fellow} \,\,\ESRh&
\textbf{Host} \,\,\nikhefentity&
\textbf{\phd} \,\,Yes&
\textbf{Start (mo.)} \,\,8&
\textbf{Duration (mo.)} \,\,36&
\textbf{Deliverables}\,\, \deliverableTriggerExperimentalSoftwareWPThree, \deliverableWhitepaperDevelopmentWPThree, \deliverableWhitepaperStateOfTheArtWPFive, \deliverableTriggerExperimentalSoftwareWPFive, \deliverableWhitepaperCollectionPapersWPFive, \deliverableFleetmaticsMLMobile \tabularnewline 
\hline
\multicolumn{2}{|l|}{\textbf{\Tstrut Work Package:}
\WPESRa} &
\multicolumn{2}{l|}{\textbf{Doctoral programme:} \radboudentity }&%\tabularnewline\hline
\multicolumn{2}{l|}{\textbf{\Tstrut Title: Optimization of RTA resources and ATLAS LFV search}
}\tabularnewline\hline
\multicolumn{6}{|p{21.2cm}|}{\textbf{\Tstrut Objectives:} Instead of focusing on specific physics processes to choose what is "interesting", \ESRh will investigate a more generic approach - what are the bottlenecks in our trigger algorithms, what prevents us to record events we want and how to do more interesting physics with the same or even fewer resources. 
\ESRh's first objective will be to simplify, streamline and optimize the process of testing and benchmarking ATLAS triggers, by creating software tools that will analyze the performance of each trigger line, the resources used by each line, and their commonalities. 
\ESRh will also study how to reduce the consumption of resources by trigger lines using ML algorithms. 
A secondment at CERN under the supervision of  physicists from the \oregonentity will prove that these tools are useful within the ATLAS trigger, testing the performance of the algorithms developed by \ESRh on chains that make use of FTK.
As the tools developed by \ESRh will be experiment-independent, they will be portable and useful for applications for other LHC and future experiments. 
Specifically, we will design the tools together with LHCb trigger colleagues (\ESRi), and for different computing architectures (\ESRg).
The secondment at \ximantisentity will allow \ESRh to use test a specific kind of algorithms, Attention networks, to focus both traffic prediction and resource consumption algorithms to a limited number of features. 
\ESRh will monitor the performance of these algorithms compared to the existing ones in the app, in an effort to make the developed tools fully environment independent, and provide a set of improvements for the app as its second objective. 
\ESRh will then test whether this kind of networks can help increasing the efficiency of trigger lines. 
\ESRh's third objective will be to search for LFV in the $\tau\to 3\mu$ process with ATLAS Run 3 data, where the optimization of the analysis chain resource consumption is critical and will use previously developed tools.
}\tabularnewline\hline
\multicolumn{6}{|p{21.2cm}|}{\textbf{\Tstrut Expected Results:}1. With \ESRi, determine resources and bottlenecks in trigger systems, and identify how to optimize them (toolkit and peer-reviewed paper). 
2. Establish optimal algorithms among new ML techniques for \ximantis traffic app.
3. Implement optimization for FTK triggers chains (with \oregonentity at CERN) and share code with LHCb (with \cernentity).  
4. Search for LFV with ATLAS Run 3 data with optimized triggers (peer-reviewed paper). 
\ESRh will receive a PhD in experimental HEP at \radboudlong.
}\tabularnewline\hline
\multicolumn{6}{|p{21.2cm}|}{\textbf{\Tstrut Secondments:}
\oregonentity, \cernentity (at CERN), 5 months, Strom, Couturier. Benchmarking of algorithms for FTK in the ATLAS trigger and for LHCb. 
\ximantisentity, 4 months, Sopasakis. Optimization of the \ximantis app and testing of Attention networks. 
}\tabularnewline
\hline
\end{tabular}
}%
\end{center}
%\begin{center}\small
%\resizebox {\textwidth }{!}{%
%\begin{tabular}{|p{25mm}|p{23mm}|p{18mm}|p{28mm}|p{34mm}|p{60mm}|}
%\hline
%\textbf{\Tstrut Fellow} \,\,\ESRh&
%\textbf{Host} \,\,\nikhef&
%\textbf{\phd} \,\,Yes&
%\textbf{Start (mo.)} \,\,8&
%\textbf{Duration (mo.)} \,\,36&
%\textbf{Deliverables}\,\
%\hline
%\multicolumn{2}{|l|}{\textbf{\Tstrut Work Package:}
%\WPESRh} &
%\multicolumn{2}{l|}{\textbf{Doctoral programme:} \radboud }&%\tabularnewline\hline
%\multicolumn{2}{l|}{\textbf{\Tstrut Title: Optimization of RTA resources and LFV search with ATLAS}
%}\tabularnewline\hline
%\multicolumn{6}{|p{20.2cm}|}{\textbf{\Tstrut Objectives:}
%\ESRh will study and be trained in advanced ML methods for RTA in both HEP and industry, and apply these methods to the optimization of resources for RTA, searches for LFV, and the optimization of ultrasound image simulation.
%Instead of focusing on specific physics processes and separation of signal versus background based on their characteristics, \ESRh investigate a more generic approach - what are the bottlenecks in our trigger algorithms, what prevents us to record events we want and how to do more exciting physics with the same or even less resources we have. 
%\ESRh's first objective will be to simplify, streamline and optimize the process of testing and benchmarking ATLAS triggers, by creating software tools that will analyze the performance of each trigger line, 
%the resources used by each line, and their commonalities. \ESRh will also study how to reduce the consumption of resources by trigger lines using ML algorithms. 
%A secondment at CERN under the supervision of  physicists from the \oregonentity will prove that these tools are useful within the ATLAS trigger, testing the performance of the algorithms developed by \ESRh on the upcoming chains using the FTK.
%As the tools developed by \ESRh will not use any physics characteristics they will be portable and useful for applications independently of the experiment. 
%Specifically, we will design the tools together with LHCb trigger colleagues (\ESRi), and for different computing architectures (ESR10).
%\ESRh's second objective will be to search for LFV in the $\tau\to 3\mu$ process with ATLAS data, where the optimization of the analysis chain resource consumption is critical and will use the tools developed by \ESRh.
%\ESRh will be seconded with Ximantis and work on hybrid networks (e.g. Convolutional Neural Network followed by Long Short-Term Memory recurrent networks) which can capture features in different dimensionalities (e.g. space and time for traffic monitoring and forecasting). 
%\ESRh will monitor the performance of these algorithms compared to the existing ones in the app, in an effort to make the developed tools fully environment independent. 
%\ESRh will then test whether this kind of networks can help monitoring trigger lines and making them more efficient. 
%}\tabularnewline\hline
%\multicolumn{6}{|p{20.2cm}|}{\textbf{\Tstrut Expected Results:}
%Inter-experiment toolkit for trigger benchmarking and optimization (with \ESRi), a related peer-reviewed publication, a technical publication on used ML methods. 
%The toolkit will be released for use in industry, and contribute to the optimization of the \ximantisentity app using hybrid networks. 
%The physics research will lead to a peer-reviewed publication on LFV.
%\ESRh will receive a PhD in experimental HEP at \radboudentity.
%}\tabularnewline\hline
%
%\multicolumn{6}{|p{20.2cm}|}{\textbf{\Tstrut Secondments:}
%\ximantisentity, 4 months, Sopasakis. Optimization of the app and testing of hybrid networks. 
%\oregonentity (at CERN), 5 months, Strom. Benchmarking of algorithms for FTK in the ATLAS trigger. 
%}\tabularnewline
%\hline
%\end{tabular}
%}%
%\end{center}
%%\end{table}
%%

%
%\multicolumn{2}{l|}{\textbf{Doctoral programme:} \radboud }&%\tabularnewline\hline
%\multicolumn{2}{l|}{\textbf{\Tstrut Title: Optimization of RTA resources and LFV search with ATLAS}
%}\tabularnewline\hline
%\multicolumn{6}{|p{21.2cm}|}{\textbf{\Tstrut Objectives:}
%
%
%}\tabularnewline\hline
%\multicolumn{6}{|p{21.2cm}|}{\textbf{\Tstrut Expected Results:}
%}\tabularnewline\hline
%\multicolumn{6}{|p{21.2cm}|}{\textbf{\Tstrut Secondments:}
%\ximantisentity, 4 months, Sopasakis. Optimization of the app and testing of hybrid networks. 
%\oregonentity (at CERN), 5 months, Strom. Benchmarking of algorithms for FTK in the ATLAS trigger. 
%}\tabularnewline
%\hline
%\end{tabular}
%}%
%\end{center}