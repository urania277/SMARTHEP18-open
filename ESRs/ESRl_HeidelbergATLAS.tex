%\begin{table}[h]
\begin{center}\small
\resizebox {\textwidth }{!}{%
\begin{tabular}{|p{21mm}|p{19mm}|p{15mm}|p{8mm}p{12mm}|p{19mm}|p{39mm}|p{38mm}|}
\hline
\textbf{\Tstrut Fellow} \ESRl&
\textbf{Host} \hdshort&
\textbf{\phd} Yes&
\textbf{Start} & Month 8&
\textbf{Duration} 36&
\textbf{Deliverables} \deliverableWhitepaperStateOfTheArtWPFour,\deliverableWhitepaperDevelopmentWPFour,\deliverableTriggerExperimentalSoftwareWPThree,\deliverableWhitepaperStateOfTheArtWPFive,\deliverableWhitepaperCollectionPapersWPFive & 
\textbf{Work Package:} WP4,5,6\tabularnewline 
%\ESRl &  \hdshort & Yes & Month 6& 36 & \deliverableHEPPubATLASTLAMultijet, \deliverableHEPPubPileupNoiseCaloFTK \tabularnewline
\hline
%\multicolumn{2}{|l|}{\textbf{\Tstrut Work Package:}
%WP4,5,6} &
\multicolumn{4}{|l|}{\textbf{Doctoral programme:} \heidelberg }&%\tabularnewline\hline
\multicolumn{4}{l|}{\textbf{\Tstrut Project Title: Real-time noise reduction new physics searches and industry}
}\tabularnewline\hline
\multicolumn{8}{|p{20.2cm}|}{\textbf{\Tstrut Objectives:} 
\ESRl will be trained in real-time methods for noise reduction which they will apply to the analysis of ATLAS data.
This project will provide \ESRl with an expert-level understanding of the installation, calibration  and operation of the large-scale high-energy physics experiments trigger systems, knowledge in the statistical analysis of experimental data in searches for new physics phenomena. 
\ESRl will learn how to code the programmable hardware for the high-speed real-time data processing. 
The first objective of \ESRl is the development and validation of the pileup noise reduction algorithms in the ATLAS calorimeter trigger system. This information will be used in real-time and at HL-LHC calibrated using the tracking information from the FTK. This will be aided by two secondments at CERN, supervised by calorimeter and FTK experts. 
The second objective of the project is the suppression of the known SM backgrounds in a search for Dark Matter particles using an angular analysis that depends critically on the noise reduction and calibration \ESRl developed.
The use of a RTA will allow the subtraction of noise and identification of an interesting events at the earliest possible stage.
TO BE UPDATED: SECONDMENT
}
\tabularnewline\hline
\multicolumn{8}{|p{20.2cm}|}{\textbf{\Tstrut Expected Results:}
Two peer-reviewed papers: on the new system of the preprocessing of the calorimeter signals and on the search for new physics in the events with multiple hadron jets in the final-state.  
\ESRl will receive a PhD degree in HEP at \heidelbergentity.
}
\tabularnewline\hline
%\multicolumn{6}{|p{20.2cm}|}{\textbf{Doctoral program:} Cambridge}\tabularnewline\hline
\multicolumn{8}{|p{20.2cm}|}{\textbf{\Tstrut Secondments:}
\oregonentity, \pisaentity (at CERN), 3 months, Strom and Annovi. Operations of FTK for trigger-level pileup suppression algorithm. 
}\tabularnewline
\hline
\end{tabular}
}%
\end{center}
%\end{table}
%
