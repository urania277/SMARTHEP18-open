%\begin{table}[h]
\begin{center}\small
\resizebox {\textwidth }{!}{%
\begin{tabular}{|p{25mm}|p{23mm}|p{18mm}|p{28mm}|p{34mm}|p{60mm}|}
\hline
\textbf{\Tstrut Fellow} \,\,\ESRl&
\textbf{Host} \,\,\heidelbergentity&
\textbf{\phd} \,\,Yes&
\textbf{Start (mo.)} \,\,8&
\textbf{Duration (mo.)} \,\,36&
\textbf{Deliverables}\,\,\deliverableWhitepaperStateOfTheArtWPThree, \deliverableWhitepaperDevelopmentWPThree, \deliverableTriggerExperimentalSoftwareWPThree, \deliverableWhitepaperStateOfTheArtWPFive, \deliverableWhitepaperCollectionPapersWPFive, \deliverableFleetmaticsMLMobile \tabularnewline 
\hline
\multicolumn{2}{|l|}{\textbf{\Tstrut Work Package:}
\WPESRl} &
\multicolumn{2}{l|}{\textbf{Doctoral programme:} \heidelbergentity }&%\tabularnewline\hline
\multicolumn{2}{l|}{\textbf{\Tstrut Title: Real-time noise reduction new physics searches}
}\tabularnewline\hline
\multicolumn{6}{|p{21.2cm}|}{\textbf{\Tstrut Objectives:}
\ESRl will be trained in real-time methods for noise reduction which they will apply to the analysis of ATLAS data.
This project will provide \ESRl with an expert-level understanding of the installation, calibration  and operation of the large-scale high-energy physics experiments trigger systems, knowledge in the statistical analysis of experimental data in searches for new physics phenomena. 
\ESRl will learn how to code the programmable hardware for the high-speed real-time data processing of the detectors measuring particle energy in ATLAS (calorimeters). 
The first objective of \ESRl is the development and validation of the pileup noise reduction algorithms in the ATLAS calorimeter trigger system. This information will be used in real-time and cross-calibrated using the tracking information from the FTK, as a second objective. This will be aided by a secondment at CERN, supervised by calorimeter and FTK experts. 
The third objective of the project is the suppression of the known SM backgrounds in a search for Dark Matter particles using an angular analysis that depends critically on the noise reduction and calibration \ESRl developed.
The use of a RTA will allow the subtraction of noise and identification of interesting events at the earliest possible stage.
This will benefit from a secondment at \lundentity under Doglioni's supervision, given her and her institute's expertise on this kind of searches, and from a secondment at \ximantisentity for the investigation of Attention networks to focus the traffic prediction algorithms and the search selection on a limited number of features. 
}\tabularnewline\hline
\multicolumn{6}{|p{21.2cm}|}{\textbf{\Tstrut Expected Results:}
1. Develop algorithms for better noise suppression in calorimeter hardware (peer-reviewed paper). 
2. Deploy in ATLAS trigger with the aid of tracker information from FTK (software, conference paper).
3. Apply technique to angular searches (peer-reviewed paper).
\ESRl will receive a PhD degree in HEP at \heidelberglong. 
}\tabularnewline\hline
\multicolumn{6}{|p{21.2cm}|}{\textbf{\Tstrut Secondments:}
\oregonentity, \pisaentity (at CERN), 3 months, Strom, Annovi. Operations of FTK for trigger-level pileup suppression algorithm. 
\lundentity, 2 months, Doglioni, application of noise reduction techniques to angular searches.
\ximantisentity, 3 months, Sopasakis. Enhancement of AI modelling algorithms for traffic prediction, using Attention networks.} 
\tabularnewline
\hline
\end{tabular}
}%
\end{center}

%%%
%%%
%%%%\end{table}
%
