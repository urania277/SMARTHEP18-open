%\begin{table}[h]
\begin{center}\small
\resizebox {\textwidth }{!}{%
\begin{tabular}{|p{25mm}|p{23mm}|p{18mm}|p{28mm}|p{34mm}|p{60mm}|}
\hline
\textbf{\Tstrut Fellow} \,\,\ESRk&
\textbf{Host} \,\,\lundentity&
\textbf{\phd} \,\,Yes&
\textbf{Start (mo.)} \,\,8&
\textbf{Duration (mo.)} \,\,36&
\textbf{Deliverables}\,\,\deliverableWhitepaperStateOfTheArtWPSix,\deliverableSoftwareWPSix,\deliverableWhitepaperCollectionPapersWPSix \tabularnewline 
\hline
\multicolumn{2}{|l|}{\textbf{\Tstrut Work Package:}
\WPESRk} &
\multicolumn{2}{l|}{\textbf{Doctoral programme:} \lundentity }&%\tabularnewline\hline
\multicolumn{2}{l|}{\textbf{\Tstrut Title: Real-time calibration and analysis of the ALICE TPC}
}\tabularnewline\hline
\multicolumn{6}{|p{21.2cm}|}{\textbf{\Tstrut Objectives:}
\ESRk will study and commission the ALICE Time Projection Chamber (TPC) detector.
In 2019-2020 the LHC will be shut down to upgrade and prepare the experiments for Run 3. 
The goal of the ALICE upgrade is to be able to analyze the full rate of 50000 events per second, increasing the sensitivity for most measurements by between one and two orders of magnitude.
The main goal on the detector side is the upgrade of the TPC with a Gas Electron Multiplier (GEM) readout that will allow continuous operation. This continuous readout requires a whole new software framework denoted O2 (online-offline), whose goal is to do full calibration and reconstruction in real time.  
\ESRk's first objective will be to contribute to the development of the  ALICEO2 framework and use this expertise in the analysis of the first data from Run 3, which will start in 2021. 
Particularly, due to the space charge build up fluctuations, the upgraded TPC will have to be calibrated every 5 milliseconds over a space volume of 90 cubic meters. 
This demands fast, effective and robust algorithms that \ESRk will have to develop, tune and benchmark.
A 6-months secondment at \cern will allow the implementation of the real-time calibration algorithms in the ALICE software, as well as training and interaction with the core of the ALICE O2 development team. 
The second objective of \ESRk is the analysis of heavy-ion data, which will first be available at the end of 2021.  
This analysis will be a measurement of the nuclear modification factor, since this measurement will be very sensitive to the corrections and a perfect testing ground for the algorithms developed. 
Even though \ESRk will not have an industrial secondment, they will receive local mentoring by Sopasakis from \ximantisentity (also a lecturer in mathematics at \lundentity) on matters of industry/academia interaction. 
}\tabularnewline\hline
\multicolumn{6}{|p{21.2cm}|}{\textbf{\Tstrut Expected Results:}
1. Contribute to development of the O2 algorithms for ALICE TPC reconstruction (peer-reviewed paper). 
2. Apply algorithms to the measurement of the nuclear modification factor (peer-reviewed paper).
\ESRk will receive a PhD in experimental HEP at \lundlong.
}\tabularnewline\hline
\multicolumn{6}{|p{21.2cm}|}{\textbf{\Tstrut Secondments:}
\cern, 5 months, Shahoyan, implementation of algorithms in ALICE trigger software. 
}\tabularnewline
\hline
\end{tabular}
}%
\end{center}

%%%%
%%%%
%%%%

%In 2019-2020 LHC will be shut down to upgrade and prepare the experiments for Run 3. The goal of the ALICE upgrade is to be able to analyse the full rate of events, 50kHz, online thereby increasing the sensitivity for most measurements by between an order and two orders of magnitude.
%PC is involved in both detector upgrades and data analysis. The main goal on the detector side is the upgrade of the TPC with a Gas Electron Multiplier (GEM) readout that will allow continuous operation. The continuous readout requires a whole new software framework denoted O2 (online-offline). The goal of the framework is to do full calibration and reconstruction in real time (RTA).  
%The Lund heavy-ion group is since long time active in the ALICE TPC project. Sweden has contributed with around 21MSEK (~2M¤) to the TPC detector. PC is one of two coordinators of ALICE TPC GEM upgrade simulation activities responsible for the implementation of the GEM TPC simulations in the O2 framework.

%Ruben Shahoyan is an expert on software and one of the main developers
%for the ALICE O2 project. He coordinates the ALICE upgrade activity on
%calibration and reconstruction, and has played a critical role in
%calibrating the space point distortions of the existing TPC.


%In 2019-2020, LHC will be shut down to upgrade and prepare the
%experiments for Run 3. The goal of the ALICE upgrade is to be able to
%analyze the full rate of events, 50kHz, thereby increasing the
%sensitivity for most measurements by between an order and two orders of
%magnitude. To achieve this leap in sensitivity, the experiment is
%implementing continuous readout. On the software side a new Online-
%Offline (O2) framework will be implemented. The goal of this framework
%is to achieve the massive required reduction of data volume while still
%preserving the excellent precision of Run 1 and Run 2. Real Time
%Analysis and Machine Learning offers important tools for both
%reduction, calibration and quality control that will be explored in
%this project.
%
%
%The Lund heavy-ion group is since long time active in the ALICE Time
%Projection Chamber (TPC) project, which, with its 90 cubic meter gas
%volume and 500,000 readout channels, is the largest TPC in the world.
%Sweden has contributed with around 21MSEK (~2M¤) to the TPC detector
%(~1/3 of the total cost). The Lund group has been heavily involved both
%in the research and development as well as the construction,
%installation and commissioning of the TPC. This has led to a detector
%working well, from the start of data taking in 2009. 
%To enable 50 kHz running, the ALICE TPC will be upgraded with a
%completely new GEM readout for LHC Run 3. The Lund group will develop
%and fabricate the front-end-readout electronics with the main
%collaborator Oak Ridge National Laboratory (ORNL). Lund has taken on
%the charge to lead the characterization of the front-end chip, SAMPA,
%which is a new development for readout of GEM based TPCs. Lund will
%also contribute to the installation and commissioning on the TPC in
%2019-20.
%At the same time, Peter Christiansen as one of two coordinators of
%ALICE TPC GEM upgrade simulation activities will be responsible for the
%implementation of the GEM TPC simulations in the new O2 software
%framework. This development is coupled to the major challenge on the
%TPC calibration and reconstruction side of the upgrade, where the large
%rates in Run 3 and on will lead to space charge build up in the gas
%volume. The resulting space charge distortions will be of order
%centimeters and have to be calibrated to better than 200 microns to not
%affect the precision of the reconstruction. 