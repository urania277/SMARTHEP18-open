
%\begin{table}[h]

%\begin{table}[h]
\begin{center}\small
\resizebox {\textwidth }{!}{%
\begin{tabular}{|p{19mm}|p{26mm}|p{25mm}|p{21mm}|p{23mm}|p{66mm}|}
\hline
\textbf{\Tstrut Fellow} \,\,\ESRc&
\textbf{Host} \,\,\cern&
\textbf{\phd} \,\,Yes&
\textbf{Start} \,\,Month 8&
\textbf{Duration} \,\,36&
\textbf{Deliverables} \,\, \deliverableWhitepaperDevelopmentWPThree, \deliverableTriggerExperimentalSoftwareWPThree, \deliverableWhitepaperStateOfTheArtWPFour, \deliverableWhitepaperDevelopmentWPFour, \deliverableWhitepaperStateOfTheArtWPFive, \deliverableWhitepaperCollectionPapersWPFive, \deliverableParallelization \tabularnewline 
%\ESRc &  \cern & Yes & Month 6& 36 & \deliverableTechPubMultithreaded, \deliverableHEPPubLLP \tabularnewline
\hline
\multicolumn{2}{|l|}{\textbf{\Tstrut Work Package:}
\WPESRc} &
\multicolumn{2}{l|}{\textbf{Doctoral programme:} \unige } &%\tabularnewline\hline
\multicolumn{2}{l|}{\textbf{\Tstrut Title: Efficient RTA in ATLAS using multi-threaded processing }
}\tabularnewline\hline
\multicolumn{6}{|p{20.2cm}|}{\textbf{\Tstrut Objectives:}
\ESRc will be trained in algorithm optimization for highly parallel computing architectures in both HEP and industry, deploy this to improve the performance of both ATLAS RTA and commercial investment code, and search for LLPs with the first Run 3 ATLAS data. 
Multithreaded (MT) programming is crucial to make best use of today's parallel computing architectures, but until recently most HEP code was unable to run MT.
Because of associated overheads, MT is particularly challenging for RTA. \ESRc's first objective will be to implement new monitoring within the ATLAS real-time code, measure algorithm scheduling and performance as well as the overhead of MT, and identify improvements that maximize MT performance.
This will be done in synergy with ESR11 and ESR12, and integrated in the ATLAS real-time software. 
Working on this objective will result in \ESRc becoming trained in advanced techniques of developing and evaluating code for highly parallel architectures. 
This will be crucial for their secondment to \lightbox to study the optimal parallelization of algorithms for investment strategies, trading infrastructures and integrated business processes.
\ESRc's second objective will be to be trained in these commercial tasks and then produce a commercial framework with figures of merit for their real-time optimization.
\ESRc's will be seconded to \cnrs and \parisU, whose physical proximity enables \ESRc to benefit from the expertise in MT and parallelization more generally of both. 
\ESRc will receive further MT and parallelization training and apply this and their earlier results on MT optimization to the creation of pattern banks for FTK, working closely with \ESRf.
\ESRc's fourth objective is to use the gained insights and knowledge to implement new RTA capabilities in the ATLAS trigger for LLP signatures, including dedicated pattern recognition algorithms for example for long-lived charged particles decaying in the middle of the detector. 
This would increase the trigger acceptance for such particles in ATLAS Run~3 data and \ESRc will lead the search for LLPs with the first Run~3 data, in collaboration with \ESRb and benefiting from the supervision of Sfyrla, as a fifth objective. 
}\tabularnewline\hline
\multicolumn{6}{|p{20.2cm}|}{\textbf{\Tstrut Expected Results:}
One peer-reviewed paper will document the multi-threaded implementation of the ATLAS HLT, while the second one will report on the search for LLPs
in the first ATLAS Run-3 data. The industrial secondment will produce a commercial toolkit for the real-time optimization of parallel/sequential complex tasks.
\ESRc will receive a PhD in experimental HEP at \unige.
}\tabularnewline\hline
\multicolumn{6}{|p{20.2cm}|}{\textbf{\Tstrut Secondments:}
\lightbox 4 months, Catastini, improved of efficiency for complex tasks by real-time decision of sequential/parallel processing. 
\cnrs and \parisU, 5 months, Crescioli and Lacassagne, optimization of parallel code for FTK pattern bank creation. 
}\tabularnewline
\hline
\end{tabular}
}%
\end{center}
%\end{table}
%
