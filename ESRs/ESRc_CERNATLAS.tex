
%\begin{table}[h]

%\begin{table}[h]
\begin{center}\small
\resizebox {\textwidth }{!}{%
\begin{tabular}{|p{25mm}|p{23mm}|p{18mm}|p{28mm}|p{34mm}|p{60mm}|}
\hline
\textbf{\Tstrut Fellow} \,\,\ESRc&
\textbf{Host} \,\,\cernentity&
\textbf{\phd} \,\,Yes&
\textbf{Start (mo.)} \,\,8&
\textbf{Duration (mo.)} \,\,36&
\textbf{Deliverables} \,\, \deliverableWhitepaperDevelopmentWPThree,\deliverableTriggerExperimentalSoftwareWPThree,\deliverableWhitepaperStateOfTheArtWPFour,\deliverableWhitepaperDevelopmentWPFour,\deliverableWhitepaperStateOfTheArtWPFive, \deliverableWhitepaperCollectionPapersWPFive,\deliverableParallelization \tabularnewline 
%\ESRc &  \cern & Yes & Month 6& 36 & \deliverableTechPubMultithreaded, \deliverableHEPPubLLP \tabularnewline
\hline
\multicolumn{2}{|l|}{\textbf{\Tstrut Work Package:}
\WPESRc} &
\multicolumn{2}{l|}{\textbf{Doctoral programme:} \unigeentity } &%\tabularnewline\hline
\multicolumn{2}{l|}{\textbf{\Tstrut Title: Efficient RTA in ATLAS using multi-threaded processing }
}\tabularnewline\hline
\multicolumn{6}{|p{21.2cm}|}{\textbf{\Tstrut Objectives:}
Multithreaded (MT) programming is crucial to make best use of today's parallel computing architectures, but until recently most HEP code was unable to run MT.
Because of associated overheads, MT is particularly challenging for RTA. 
\ESRc's first objective will be to implement new monitoring within the ATLAS real-time code, measure algorithm scheduling and performance as well as the overhead of MT, and identify improvements that maximize MT performance.
This will be done in synergy with \ESRh and \ESRi's benchmarking work, and integrated in the ATLAS real-time software. 
Working on this objective will result in \ESRc becoming trained in advanced techniques of developing and evaluating code for highly parallel architectures. 
This will be crucial for their secondment to \lightbox, where they will devise an optimal parallelization of algorithms for investment strategies, trading infrastructures and integrated business processes.
\ESRc's second objective will be to be trained in these commercial tasks and then produce a commercial framework with figures of merit for their real-time optimization.
\ESRc's will also be seconded to \cnrs and \parisU, whose physical proximity enables \ESRc to benefit from the expertise in MT and parallelization more generally of both. 
\ESRc will receive further MT and parallelization training and apply their knowledge on MT optimization to the creation of pattern banks for track triggers, working closely with \ESRf.
\ESRc's third objective is to use the gained insights and knowledge to implement new RTA capabilities in the ATLAS trigger for LLP signatures, including dedicated pattern recognition algorithms for charged particles decaying in the middle of the detector. 
This will increase the trigger acceptance for such particles in ATLAS Run~3 data, so that \ESRc can lead a search with the first data, in collaboration with \ESRb and benefiting from the supervision of Sfyrla, as a fourth objective. 
}\tabularnewline\hline
\multicolumn{6}{|p{21.2cm}|}{\textbf{\Tstrut Expected Results:}
1. Implementation and monitoring of MT algorithms in ATLAS trigger (peer-reviewed papers). 
2. Commercial toolkit for the real-time optimization of parallel/sequential complex tasks in \lightboxentity. 
3. Use of MT in FTK pattern bank creation (\cnrs, \parisU). 
4. New trigger algorithms for LLP and search (peer-reviewed papers). 
\ESRc will receive a PhD in experimental HEP at \unige.
}\tabularnewline\hline
\multicolumn{6}{|p{21.2cm}|}{\textbf{\Tstrut Secondments:}
\lightbox 4 months, Catastini, improved efficiency for complex tasks by real-time decision of sequential/parallel processing. 
\cnrs and \parisU, 5 months, Crescioli and Lacassagne, optimization of parallel code for FTK pattern bank creation. 
}\tabularnewline
\hline
\end{tabular}
}%
\end{center}
%\end{table}
%
