%\begin{table}[h]

%\begin{table}[h]
\begin{center}\small
\resizebox {\textwidth }{!}{%
\begin{tabular}{|p{19mm}|p{20mm}|p{25mm}|p{21mm}|p{23mm}|p{72mm}|}
\hline
\textbf{\Tstrut Fellow} \,\,ESR5 &
\textbf{Host} \,\,CERN &
\textbf{\phd} \,\,Yes &
\textbf{Start} \,\,Month 8 &
\textbf{Duration} \,\,36 &
\textbf{Deliverables} \,\,\deliverableWhitepaperStateOfTheArtWPThree, \deliverableWhitepaperDevelopmentWPThree, \deliverableTriggerExperimentalSoftwareWPThree, \deliverableWhitepaperStateOfTheArtWPFive, \deliverableWhitepaperCollectionPapersWPFive \tabularnewline 
\hline
\multicolumn{2}{|l|}{\textbf{\Tstrut Work Packages:} WP3,5} &
\multicolumn{2}{l|}{\textbf{Doctoral programme:} \dortmund }&%\tabularnewline\hline
\multicolumn{2}{l|}{\textbf{\Tstrut Title: Using RTA to search for LFV in LHCb}
}\tabularnewline\hline
\multicolumn{6}{|p{20.2cm}|}{\textbf{\Tstrut Objectives:}
%LHCb will have a Phase-I upgrade in 2020 where the experiment will be fully %read out at 40 MHz, allowing for a very flexible 
%full software trigger. This upgrade will be accumulating the same amount of %data accumulated by LHCb between 2010 and
%2020 in a single year, and multiply by more than a factor six the total %accumulated statistics by 2029. 
ESR5 will speed up the LHCb trigger reconstruction and be trained in writing real-time reconstruction algorithms on modern parallel computing architectures. ESR5 will use this work to search for Lepton Flavour Violation (LFV) in LHCb data.
%a key prediction of many physical theories which improve on the Standard Model, yet
%never observed so far. 
LFV, a never observed yet key prediction of many NP theories, is now particularly interesting in light
of hints of the related lepton flavour non-universality (LFUV) seen by LHCb and other experiments.
The LHCb upgrade will start in 2020 and accumulate data five times faster, processing 3 TB/s in software. 
%However,
%the trigger algorithms will need to be revisited to deal with a higher rate and pileup. In particular,
To maintain LHCb's trigger performance at low particle energies in this environment one needs to rethink completely the reconstruction algorithms, 
with emphasis on the unique sensitivity to low energy particles that LHCb has with respect to other LHC experiments. As LHCb's trigger is entirely software based this work will critically depend on how such algorithms perform on modern parallel computing architectures, and will crucially benefit from being hosted in the CERN group of LHCb which leads the collaboration's efforts on parallel computing and algorithm optimization.
The first objective of ESR5 will be to develop novel, optimized algorithms to identify and reconstruct low-energy
objects for the LHCb trigger. This will allow ESR5 to eventually search for LFV processes in low-mass objects like taus, kaons or charm mesons, and benefit from the much larger statistics of the LHCb upgrade.
%These algorithms will be then used by ESR5 to search for a violation of lepton flavour in decays
%of taus and other light mesons, 
This work will be done in synergy with ERS6 (based in Dortmund) who will be searching for
similar phenomena in unflavoured mesons. Both searches will need to have a large part of their event selection
implemented as RTA to maintain high signal efficiencies for
these difficult experimental signatures. ESR5 and ESR6 will work in parallel on this topic, 
for a more efficient development. The second objective of ESR5 is to finalize an event selection
that can be implemented online, which will allow ESR5 to reach their third objective of searching for LFV in decays 
of taus and other light mesons with the first Run-3 LHCb data. 
}\tabularnewline\hline
\multicolumn{6}{|p{20.2cm}|}{\textbf{\Tstrut Expected Results:}
Peer-reviewed papers on 1/ reconstruction algorithms for low-energy objects
in the LHCb trigger, 2/ the improved trigger selection with ESR6, 3/ the search for LFV in decays of taus and light mesons.
ESR5 will receive a PhD in experimental HEP at \dortmund.
}\tabularnewline\hline
\multicolumn{6}{|p{20.2cm}|}{\textbf{\Tstrut Secondments:}
\dortmund, 3+3 months, Albrecht. Optimisation of real-time event selection for LFV searches with low-energy objects in LHCb. 
%\yandex, 3 months, Dr. A. Ustyuzhanin, %\yandex school of data analysis and research project on MVA in industry.
%CERN, 3 months, Dr. A. Hoecker, definition of the optimal variables. 
}\tabularnewline
\hline
\end{tabular}
}%
\end{center}
%\end{table}
%
