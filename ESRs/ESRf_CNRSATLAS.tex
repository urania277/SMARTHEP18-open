\begin{center}\small
\resizebox {\textwidth }{!}{%
\begin{tabular}{|p{25mm}|p{26mm}|p{18mm}|p{28mm}|p{34mm}|p{50mm}|}
\hline
\textbf{\Tstrut Fellow} \,\,\ESRf&
\textbf{Host} \,\, \cnrs&
\textbf{\phd} \,\, Yes&
\textbf{Start (mo.)} \,\,8&
\textbf{Duration (mo.)} \,\,36&
\textbf{Deliverables} \,\, 
\deliverableWhitepaperStateOfTheArtWPThree, \deliverableWhitepaperStateOfTheArtWPFour, \deliverableSoftwareDevelopmentWPFour, \deliverableWhitepaperDevelopmentWPFour, \deliverableSoftwareWPSix, \deliverableWhitepaperCollectionPapersWPSix, \deliverableFleetmaticsMLWebLearning 
\tabularnewline 
\hline
\multicolumn{2}{|l|}{\textbf{\Tstrut Work Package:}
\WPESRf} &
\multicolumn{2}{l|}{\textbf{Doctoral programme:} \parisU }& %\tabularnewline\hline
\multicolumn{2}{l|}{\textbf{\Tstrut Title: Real-time trajectory reconstruction in ATLAS} 
}\tabularnewline\hline
\multicolumn{6}{|p{21.2cm}|}{\textbf{\Tstrut Objectives:} The hardware tracking processors FTK (Run-3) and HTT (Run-4) allow ATLAS to find particle trajectories in real time. 
Their performances such as efficiency and resolution are not determined just by the hardware, but also by a learned database of patterns (pattern banks) and geometrical constants.
\ESRf's first objective will be to deploy statistical and computing techniques, e.g. Principal Component Analysis and Graph Clustering, in order to produce new pattern banks for FTK and evaluate their impact on raw performances and physics analyses.
\ESRf's second objective will be to extend the application of FTK-like algorithms outside of HEP, by providing flexible and powerful tools to train datasets.
These inter-sector toolkits will be developed during the secondment at \fleetmaticsentity, and tested at \pisaentity, whose geographical proximity allows for a seamless integration of the work.
At \fleetmaticsentity, \ESRf will be trained in the continuous online learning of ML models based on streams of labelled data, and in both the \apachespark\enspace and the Amazon Web Services infrastructures for massively parallel computations.
\ESRf will develop an online learning tool to continuously process real-time GPS data from Fleetmatics customers, providing customers with smart insights, improving customer experience, and thus increasing engagement with \fleetmaticsentity products. 
The acquired ML and online processing skills will feed back into the main research project and will be applied to the FTK working dataset production. 
\ESRf's third objective will be to investigate in detail the impact of new training on selected physics cases.
In particular FTK tracks will be used to improve jet reconstruction and calibration, namely for the suppression of pile-up jets and the track-based components of the global sequential calibration, in collaboration with \ESRl. 
This will enhance the sensitivity of fully RTA-based searches for dark matter mediators in the dijet mass distribution in Run 3 data.
}\tabularnewline\hline
\multicolumn{6}{|p{21.2cm}|}{\textbf{\Tstrut Expected Results:} 
1. Deploy new statistical and computing techniques for FTK/HTT pattern banks (peer-reviewed paper). 
2. Use of algorithms and implementation in \apachespark\enspace / Amazon Web Services for online learning tool of GPS data, cross-talk with HEP (\fleetmaticsentity / \pisaentity). 
3. Use of FTK tracks in improved RTA-based search for dark matter mediators. 
\ESRf will receive a PhD in experimental HEP at \sorbonneentity. 
}\tabularnewline\hline
%\multicolumn{6}{|p{20.2cm}|}{\textbf{Doctoral program:} Cambridge}\tabularnewline\hline
\multicolumn{6}{|p{21.2cm}|}{\textbf{\Tstrut Secondments:}
\fleetmaticsentity, 4 months, Sambo, development of an online learning tool for GPS data processing.
\pisaentity, 3 months, Roda and Annovi, use of toolkits for creation of FTK pattern banks. 
}\tabularnewline
\hline
\end{tabular}
}%
\end{center}

