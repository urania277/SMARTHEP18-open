%\begin{table}[h]

%\begin{table}[h]
\begin{center}\small
\resizebox {\textwidth }{!}{%
\begin{tabular}{|p{19mm}|p{26mm}|p{25mm}|p{21mm}|p{23mm}|p{66mm}|}
\hline
\textbf{\Tstrut Fellow} \,\,ESR4&
\textbf{Host} \,\,\cern&
\textbf{\phd} \,\,Yes&
\textbf{Start} \,\,Month 8&
\textbf{Duration} \,\,36&
\textbf{Deliverables} \,\, \deliverableWhitepaperDevelopmentWPThree, \deliverableTriggerExperimentalSoftwareWPThree, \deliverableWhitepaperStateOfTheArtWPFour, \deliverableWhitepaperDevelopmentWPFour, \deliverableWhitepaperStateOfTheArtWPFive, \deliverableWhitepaperCollectionPapersWPFive, \deliverableParallelization \tabularnewline 
%ESR4 &  \cern & Yes & Month 6& 36 & \deliverableTechPubMultithreaded, \deliverableHEPPubLLP \tabularnewline
\hline
\multicolumn{2}{|l|}{\textbf{\Tstrut Work Package:}
WP3,4,5,6} &
\multicolumn{2}{l|}{\textbf{Doctoral programme:} \unige } &%\tabularnewline\hline
\multicolumn{2}{l|}{\textbf{\Tstrut Title: Efficient RTA in ATLAS using multi-threaded processing }
}\tabularnewline\hline
\multicolumn{6}{|p{20.2cm}|}{\textbf{\Tstrut Objectives:}
ESR4 will be trained in algorithm optimization for highly parallel computing architectures in both HEP and industry, deploy this to improve the performance of both ATLAS RTA and commercial investment code, and search for LLPs with the first Run 3 ATLAS data. 
Multithreaded (MT) programming is crucial to make best use of today's parallel computing architectures, but until recently most HEP code was unable to run MT.
%The computational power boost seen over the last decade in CPUs has come from increasing the number of cores in a single processor rather than the core speed. This makes multi-threaded programming, which divides calculations over multiple cores running in parallel, mandatory for fully exploiting the power of these CPUs. Until recently, however, most HEP code was not able to execute in a multithreaded way.
%Until recently in HEP the multiple cores could be
%utilized fairly efficiently by simply processing a different event on each core (multi-processing), but as the core count has increased this is no longer efficient and instead a more advanced, multi-threaded approach has to be employed. 
Because of associated overheads, MT is particularly challenging for RTA. ESR4's first objective will be to implement new monitoring within the ATLAS real-time code, measure algorithm scheduling and performance as well as the overhead of MT, and identify improvements that maximize MT performance.% of MT processing. 
%The first objective of ESR4 is to ensure an highly efficient, parallel, multi-threaded implementation of the ATLAS high-level trigger system by implementing new monitoring capabilities to measure the algorithm scheduling and algorithm performance as well as the overhead of the multi-threaded system. 
This will be done in synergy with ESR11 and ESR12, and %conducted in the CERN ATLAS team and 
integrated in the ATLAS real-time software. %, but it will focus from the start on the ATLAS trigger algorithms with multithreading. 
%It will be used to analyze the resource usage and identify improvements that maximize the utilization of concurrent processing.
%The work will be conducted in the CERN ATLAS team and integrated in the ATLAS trigger software group developing the ATLAS multithreaded execution framework. 
Working on this objective will result in ESR4 becoming trained in advanced techniques of developing and evaluating code for highly parallel architectures. This will be crucial for their secondment to \lightbox to study
the optimal parallelization of 
%Parallelization is also a task required by complex event systems, such as 
algorithms for investment strategies, trading infrastructures and integrated business processes.
%However, the evolution of such systems is a sequential process. 
%In their secondment at \lightbox, 
%ESR4 will study two orthogonal approaches
%to improve the efficiency of investment code
%running on a multi-core CPU: either splitting the main task into independent %sub-tasks requiring limited
%synchronization and each running on a single core, or revising its %implementation so that fewer cores
%are used. The optimization is intended to be applied in real-time during the process. 
ESR4's second objective will be to be trained in these commercial tasks and then produce a commercial framework with figures of merit for their real-time optimization.
%Full text by Reflexive
%Simulation of complex event systems, like investment strategies, algorithmic trading infrastructures, integrated business processes, require considerable computational power that is hard to obtain from a fully sequential process given the underlying timing constraints. In most scenarios the final user is interested in the outcome of a simulation under potential perturbations or suggested improvements to the controlling algorithms, and rapid feedback is a crucial feature to enhance the effectiveness of decision making processes. On the other hand, the evolution of a complex event system is inherently a sequential process given that, although many activities occur concurrently, the number of necessary synchronizations may hinder the speedup produced by parallelization.
%Our objective is to study two orthogonal approaches to improve the efficiency of a simulation task running on a multi-core CPU. On one hand we aim to separate a simulation task into several independent sub-tasks that require limited synchronization. On the other hand, especially when the number of independent tasks does not cover the total number of available cores, we aim to revise the implementations of our simulators so that the number of synchronization points is reduced at the cost of replicating part of the computation. We plan to identify appropriate measure effectiveness, like for instance the ratio between the time required by a sequential process normalized by the number of cores, and the time required by the multi-threaded process. We plan as well to employ techniques based on competitive ratios to assess how task parallelization can be decided online during the simulation process.
%We expect to start by ad-hoc adaptations of our existing code to a modern multi-core architecture, and use the resulting know-how to identify generic implementation patterns that could be amenable to automatic on-line or off-line optimization.
%Parallelization is also a focus of the ATLAS \cnrs group and ESR8.% working on the %parallelization of the pattern bank creation for FTK. 
ESR4's will be seconded to \cnrs and \parisU, whose physical proximity enables ESR4 to benefit from the expertise in MT and parallelization more generally of both. ESR4 will receive further MT and parallelization training and apply this and their earlier results on MT optimization 
%on parallelization and apply this and the gained knowledge on multi-threading 
%gained from the optimization of the ATLAS trigger software 
to the creation of pattern banks for FTK, working closely with ESR8.%, and see whether the algorithms
%are optimal or can benefit from further improvement. 
ESR4's fourth objective is to use the gained insights and knowledge to implement new RTA
capabilities in the ATLAS trigger for LLP signatures, including dedicated pattern recognition algorithms for example for long-lived charged particles decaying in the middle of the detector. 
%For example, long-lived
%charged particles can decay in middle of the tracking volume, leaving just a %short, straight trajectory of highly
%ionizing hits in the innermost layers. These could be identified already in %the trigger system with a fast,
%dedicated pattern recognition algorithm.
This would increase the trigger acceptance for such particles in ATLAS Run~3 data and ESR4 will lead the search for LLPs with the first Run~3 data, in collaboration
with ESR3 and benefiting from the supervision of Sfyrla, as a fifth objective. 
%ESR4 will then apply this algorithm in a search for LLPs with the first Run-3 LHC experiment data
%collected with the ATLAS experiment, as a fifth and final objective. 
}\tabularnewline\hline
\multicolumn{6}{|p{20.2cm}|}{\textbf{\Tstrut Expected Results:}
One peer-reviewed paper will document the multi-threaded implementation of the ATLAS HLT, while the second one will report on the search for LLPs
in the first ATLAS Run-3 data. The industrial secondment will produce a commercial toolkit for the real-time optimization of parallel/sequential complex tasks.
ESR4 will receive a PhD in experimental HEP at \unige.
}\tabularnewline\hline
\multicolumn{6}{|p{20.2cm}|}{\textbf{\Tstrut Secondments:}
\lightbox 4 months, Catastini, improved of efficiency for complex tasks by real-time decision of sequential/parallel processing. \cnrs and \parisU, 5 months, Crescioli and Lacassagne, optimization of parallel code for FTK pattern bank creation. 
}\tabularnewline
\hline
\end{tabular}
}%
\end{center}
%\end{table}
%
