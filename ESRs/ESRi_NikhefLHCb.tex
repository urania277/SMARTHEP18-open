%\begin{table}[h]

%\begin{table}[h]
\begin{center}\small
\resizebox {\textwidth }{!}{%
\begin{tabular}{|p{21mm}|p{19mm}|p{15mm}|p{8mm}p{12mm}|p{19mm}|p{39mm}|p{37mm}|}
\hline
\textbf{\Tstrut Fellow} \,\,ESR12&
\textbf{Host} \,\,\nikhef&
\textbf{\phd} \,\,Yes&
\textbf{Start} &Month 8&
\textbf{Duration} \,\,36&
\textbf{Deliverables} \deliverableWhitepaperStateOfTheArtWPThree,\deliverableWhitepaperDevelopmentWPThree,\deliverableTriggerExperimentalSoftwareWPThree,\deliverableWhitepaperStateOfTheArtWPFive,\deliverableWhitepaperCollectionPapersWPFive &
\textbf{Work Package:} WP3,5,6\tabularnewline 
%ESR12 &  \nikhef & Yes & Month 6& 36 & \deliverableTechPubMLForOptimisation \deliverableTriggerOptToolkit \deliverableUltrasoundSimulation \deliverableHEPPubLFV \tabularnewline
\hline
%\multicolumn{4}{|l|}{\textbf{\Tstrut Work Package:}
%WP3, WP5, WP6, WP7} &
\multicolumn{4}{|l|}{\textbf{Doctoral programme:} VU University Amsterdam}&%\tabularnewline\hline
\multicolumn{4}{l|}{\textbf{\Tstrut Title:  Optimization of RTA resources and LFV search with LHCb}
}\tabularnewline\hline
\multicolumn{8}{|p{20.2cm}|}{\textbf{\Tstrut Objectives:}
ESR12 will collaborate closely with ESR11 on developing the toolkit for the optimization of the real-time trigger system of the main HEP experiments, and receive training in the same basic methods and techniques. ESR12 will focus on LHCb. Compared to ATLAS, LHCb generates an order of magnitude less data per collision at an order of magnitude larger rate. The requirement that the toolkit must work optimally for both experiments implies that it must be sufficiently generic and adaptable. As a result, it will also have applications beyond these two experiments.
The workload for the creation of this toolkit 
will be split between ESR11 and ESR12, as both students will be working in synergy
on the same topics, and will benefit from the supervision of both O. Igonkina and G. Raven.  In addition both ESRs will adapt and optimizing the toolkit for their respective experiments.
The work of ESR12 will be applied to the search for LFV in the $\tau\to\mu\gamma$ decay in LHCb, which currently does not have a dedicated trigger chain at LHCb. The high data rate of the upgraded LHCb experiment, in combination with the optimization of the trigger algorithms will allow to collect this kind of events
and study a process that was previously thought to require dedicated experiments. A six-month secondment to Dortmund will allow ESR12 to receive expert training in the design of LFV analyses from Albrecht, and benefit from synergy between that of their work and Albrecht's ESRs and StG team.
}\tabularnewline\hline
\multicolumn{8}{|p{20.2cm}|}{\textbf{\Tstrut Expected Results:}
Inter-experiment toolkit for benchmarking
and optimization of trigger systems (with ESR11), 
a related peer-reviewed publication, a related publication on used ML methods.
The physics research will lead to a peer-reviewed publication on Lepton Flavor Violation
ESR11 will receive a PhD in experimental HEP at VU University Amsterdam.
}\tabularnewline\hline

\multicolumn{8}{|p{20.2cm}|}{\textbf{\Tstrut Secondments:} 6 months, Dortmund, Albrecht, training in analysis techniques for LFV searches.
}\tabularnewline
\hline
\end{tabular}
}%
\end{center}
%\end{table}
%
