\begin{center}\small
\resizebox {\textwidth }{!}{%
\begin{tabular}{|p{25mm}|p{23mm}|p{18mm}|p{28mm}|p{34mm}|p{50mm}|}
\hline
\textbf{\Tstrut Fellow} \,\,\ESRi&
\textbf{Host} \,\,\nikhefentity&
\textbf{\phd} \,\,Yes&
\textbf{Start (mo.)} \,\,8&
\textbf{Duration (mo.)} \,\,36&
\textbf{Deliverables} \deliverableWhitepaperStateOfTheArtWPThree,\deliverableWhitepaperDevelopmentWPThree,\deliverableTriggerExperimentalSoftwareWPThree,\deliverableWhitepaperStateOfTheArtWPFive,\deliverableWhitepaperCollectionPapersWPFive \tabularnewline
\hline
\multicolumn{2}{|l|}{\textbf{\Tstrut Work Package:}
\WPESRi} &
\multicolumn{2}{l|}{\textbf{Doctoral programme:} \amsterdamentity }&%\tabularnewline\hline
\multicolumn{2}{l|}{\textbf{\Tstrut Title: Optimization of RTA resources and LHCb LFV search}
}\tabularnewline\hline
\multicolumn{6}{|p{21.2cm}|}{\textbf{\Tstrut Objectives:}
\ESRi will collaborate closely with \ESRh on developing the toolkit for the optimization of the real-time trigger system of the main HEP experiments, and receive training in the same basic methods and techniques. 
\ESRi will focus on LHCb. Compared to ATLAS, LHCb generates an order of magnitude less data per collision at an order of magnitude larger rate. 
The requirement that the toolkit must work optimally for both experiments implies that it must be sufficiently generic and adaptable. 
As a result, it will also have applications beyond these two experiments.
The workload for the creation of this toolkit will be split between \ESRh and \ESRi, as both students will be working in synergy on the same topics, and will benefit from the supervision of both Igonkina and Raven. 
In addition both ESRs will adapt and optimize the toolkits for their respective experiments.
A two-month secondment to Dortmund will allow \ESRi to receive expert training in the design of LFV analyses from Albrecht, and benefit from synergy between the work of Albrecht's group, the work of Albrecht's ESRs and the work of \ESRn who will be also seconded there. 
This will happen in tandem with a secondment at \pointeightentity where optimization and benchmarking techniques will be used for monitoring and decision-making in German industrial transport. 
Like \ESRh, \ESRi will also be seconded to CERN towards the end of their PhD, to implement the benchmarking techniques on ATLAS and LHCb online systems. 
The work of \ESRi will be applied to the search for LFV in the $\tau\to\mu\gamma$ decay in LHCb, which currently does not have a dedicated trigger chain at LHCb. 
The high data rate of the upgraded LHCb experiment, in combination with the optimization of the trigger algorithms will allow to collect this kind of events and study a process that was previously thought to require dedicated experiments. 
}\tabularnewline\hline
\multicolumn{6}{|p{21.2cm}|}{\textbf{\Tstrut Expected Results:}
1. With \ESRi, determine resources and bottlenecks in trigger systems, and identify how to optimize them (toolkit and peer-reviewed paper). 
2. Apply techniques in analysis of public transport and logistics at \pointeightentity.
3. Apply techniques to design an optimized trigger chain for LFV in the $\tau\to\mu\gamma$ decay in LHCb (peer-reviewed paper).
\ESRi will receive a PhD in experimental HEP at \amsterdamlong.
}\tabularnewline\hline
\multicolumn{6}{|p{21.2cm}|}{\textbf{\Tstrut Secondments:}
\dortmundentity and \pointeightentity, 5 months, Albrecht, Design of LFV analysis and RTA optimization of algorithms for public transport optimization. 
%\pointeightentity, 2 months, Brambach, 
\cernentity, 2 months, Petersen and Couturier, application of benchmarking algorithms to ATLAS and LHCb.
}\tabularnewline
\hline
\end{tabular}
}%
\end{center}

%%%%
%%%%
%%%%