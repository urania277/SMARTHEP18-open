%\begin{table}[h]
\begin{center}\small
\resizebox {\textwidth }{!}{%
\begin{tabular}{|p{21mm}|p{19mm}|p{15mm}|p{8mm}p{12mm}|p{19mm}|p{39mm}|p{38mm}|}
\hline
\textbf{\Tstrut Fellow} ESR15&
\textbf{Host} \hdshort&
\textbf{\phd} Yes&
\textbf{Start} & Month 8&
\textbf{Duration} 36&
\textbf{Deliverables} \deliverableWhitepaperStateOfTheArtWPFour,\deliverableWhitepaperDevelopmentWPFour,\deliverableTriggerExperimentalSoftwareWPThree,\deliverableWhitepaperStateOfTheArtWPFive,\deliverableWhitepaperCollectionPapersWPFive, \deliverableHITrigger & 
\textbf{Work Package:} WP4,5,6\tabularnewline 
%ESR15 &  \hdshort & Yes & Month 6& 36 & \deliverableHEPPubATLASTLAMultijet, \deliverableHEPPubPileupNoiseCaloFTK \tabularnewline
\hline
%\multicolumn{2}{|l|}{\textbf{\Tstrut Work Package:}
%WP4,5,6} &
\multicolumn{4}{|l|}{\textbf{Doctoral programme:} \heidelberg }&%\tabularnewline\hline
\multicolumn{4}{l|}{\textbf{\Tstrut Project Title: Real-time noise reduction new physics searches and industry}
}\tabularnewline\hline
\multicolumn{8}{|p{20.2cm}|}{\textbf{\Tstrut Objectives:} 
%The high-luminosity upgrade of the LHC (HL-LHC) will provide copious amount of data to be searches for beyond the Standard Model physics. The high instantaneous luminosity of HL-LHC will result in about 200 simultaneous proton-proton collisions in each event (pile-up). This will deteriorate the performance of the detector unless new algorithms for its mitigation of this noise are implemented, both in offline reconstruction and in real-time.  
ESR15 will be trained in real-time methods for noise reduction which they will apply to the analysis of ATLAS data and the intensity correction of industrial lasers.
This project will provide ESR15 with an expert-level understanding of the installation, calibration  and operation of the large-scale high-energy physics experiments trigger systems, knowledge in the statistical analysis of experimental data in searches for new physics phenomena. ESR15 will learn how to program embedded devices and code the programmable hardware for the high-speed real-time data processing. 
The first objective of ESR15 is the development and validation of the pileup noise reduction algorithms in the ATLAS calorimeter trigger system. This information will be used in real-time and at HL-LHC calibrated using the tracking information from the FTK. This will be aided by two secondments at CERN, supervised by calorimeter and FTK experts. 
The second objective of the project is the suppression of the known SM backgrounds 
%in events with at least four hadron jets in the final-state 
in a search for Dark Matter particles using an angular analysis that depends critically on the noise reduction and calibration ESR15 developed.
%employs large-radius jets and knowledge of their substructure. 
% together with mass-drop techniques.
%Removed detail
%Signals of new physics phenomena in the events with at least four jets can be distinguished from known Standard Model background  processes using the information on the angular correlations between the final-state objects as well as the properties of the dijet masses in the four-jet system. 
The use of a RTA will allow the identification of an interesting events at the earliest possible stage.%  and the use of further discriminants that will improve on the sensitivity to new phenomena. 
%Statistical methods of data analysis will be used to quantify the result of this search and establish the level of agreement between the measurements and predictions. 
The third objective of the project is the transfer of the experience in building and operating the 40 MHz ATLAS trigger to the commercial sector, \heidelberginstrumentsentity, 
%saving space, not clear what this is anyway
%where the  hardware trigger system for the ``Automatic Intensity Correction'' operating at up to 300 kHz rate will be built.  
where a raw input data rate of 300 kHz 
%from a high resolution analog to digital converter 
will be processed in inside a FPGA-based hardware module for a real-time intensity correction to a laser system. The processing steps will include filtering and statistic calculations. ESR15 will adapt and optimize the FPGA code for different systems with different requirements in speed and accuracy, and test their own project on real commercial systems.
}
\tabularnewline\hline
\multicolumn{8}{|p{20.2cm}|}{\textbf{\Tstrut Expected Results:}
Two peer-reviewed papers: on the new system of the preprocessing  of the calorimeter signals and on the search for new physics in the events with multiple hadron jets in the final-state.  
%The first will also contain a description of the real-time algorithms of the pileup corrections in the trigger system . The second will have results of the application of the information on the angular correlations and dijet mass properties in real-time (trigger-level) analysis (Deliverable~). 
Development of the hardware+software system for a RTA correction of the equipment produced by \heidelberginstrumentsentity.
ESR15 will receive a PhD degree in HEP at \heidelbergentity.
}
\tabularnewline\hline
%\multicolumn{6}{|p{20.2cm}|}{\textbf{Doctoral program:} Cambridge}\tabularnewline\hline
\multicolumn{8}{|p{20.2cm}|}{\textbf{\Tstrut Secondments:}
\oregonentity, \pisaentity (at CERN), 3 months, Strom and Annovi. Operations of Level-1 calorimeter trigger including the trigger-level pileup suppression algorithm. 
\heidelberginstrumentsentity, 6 months, Kaplan. Gain experience in programming embedded devices as well as developing code for programmable hardware used for the real-time data processing.
}\tabularnewline
\hline
\end{tabular}
}%
\end{center}
%\end{table}
%
