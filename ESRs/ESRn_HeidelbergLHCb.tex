%\begin{table}[h]
\begin{center}\small
\resizebox {\textwidth }{!}{%
\begin{tabular}{|p{25mm}|p{23mm}|p{18mm}|p{28mm}|p{34mm}|p{50mm}|}
\hline
\textbf{\Tstrut Fellow} \,\,\ESRn&
\textbf{Host} \,\,\heidelbergentity&
\textbf{\phd} \,\,Yes&
\textbf{Start (mo.)} \,\,8&
\textbf{Duration (mo.)} \,\,36&
\textbf{Deliverables}\,\,\deliverableWhitepaperStateOfTheArtWPThree,\deliverableWhitepaperDevelopmentWPThree, \deliverableTriggerExperimentalSoftwareWPThree,\deliverableWhitepaperStateOfTheArtWPFive,\deliverableWhitepaperCollectionPapersWPFive \tabularnewline 
\hline
\multicolumn{2}{|l|}{\textbf{\Tstrut Work Package:}
\WPESRn} &
\multicolumn{2}{l|}{\textbf{Doctoral programme:} \heidelbergentity }&%\tabularnewline\hline
\multicolumn{2}{l|}{\textbf{\Tstrut Title: RTA to search for Dark Photons in LHCb}
}\tabularnewline\hline
\multicolumn{6}{|p{21.2cm}|}{\textbf{\Tstrut Objectives:}
\ESRn will develop an advanced fully RTA-based analysis to efficiently select decays involving displaced di-electron vertices with low invariant mass. 
\ESRn will use these advances to search for light dark photons with the LHCb detector focusing in a range of extremely low coupling so far unexplored by other experiments. 
The upgraded LHCb detector, which is planned to start data-taking in 2021, will allow for the first time to search for this kind of signature. 
It is the novel real-time detector readout in Run-3 that gives the possibility to use RTA to separate dark photon candidates from the overwhelming background online. 
The reconstruction of tracks and displaced vertices has to be carried out online even for low-momentum tracks. 
This challenging task will depend critically on the performance that can be achieved on modern computing architectures. 
\ESRn will be trained in the usage of highly parallel architectures for the LHCb online reconstruction and will profit from a secondment in \santiagoentity on use of GPUs. 
\ESRn will develop advanced ML algorithms to optimise the electron classification and make it fast enough to be run in real time. 
This work will benefit from a secondment in \dortmundentity under Albrecht's supervision, who will host both \ESRn and \ESRi and train them on how to design an online analysis of decays involving electrons with the help of his StG team.
These advances will also allow \ESRn to develop a tool to efficiently retrieve online low momentum photons through their conversion in di-electron pairs, a challenging task that requires an RTA-based approach. 
Even though \ESRn will not have an industrial secondment, they will receive mentoring by Dungs from \pointeightentity (who was previously working in LHCb HEP and worked at Google) while in Dortmund and remotely, on matters of careers in industry starting from academia. 
}\tabularnewline\hline
\multicolumn{6}{|p{21.2cm}|}{\textbf{\Tstrut Expected Results:}
1. Development and testing of algorithms for RTA of electrons (conference paper). 
2. Application to dark photon search with Run 3 data (peer-reviewed paper).
%3. RTA-based search for LFV decays of tau lepton (peer-reviewed paper)
\ESRn will receive a PhD in Experimental HEP from \heidelberglong.
}\tabularnewline\hline
\multicolumn{6}{|p{21.2cm}|}{\textbf{\Tstrut Secondments:}
\dortmundentity, 3 months, Albrecht. Design of LFV physics analysis. 
\santiagoentity, 3 months, Martinez Santos. Use of GPU in LHCb online reconstruction. 
}\tabularnewline
\hline
\end{tabular}
}%
\end{center}