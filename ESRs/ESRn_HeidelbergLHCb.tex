%\begin{table}[h]
\begin{center}\small
\resizebox {\textwidth }{!}{%
\begin{tabular}{|p{21mm}|p{19mm}|p{15mm}|p{8mm}p{12mm}|p{19mm}|p{39mm}|p{38mm}|}
\hline
\textbf{\Tstrut Fellow} \ESRn&
\textbf{Host} \hdshort&
\textbf{\phd} Yes&
\textbf{Start} & Month 8&
\textbf{Duration} 36&
\textbf{Deliverables} \deliverableWhitepaperStateOfTheArtWPThree,\deliverableWhitepaperDevelopmentWPThree, \deliverableTriggerExperimentalSoftwareWPThree,\deliverableWhitepaperStateOfTheArtWPFive,\deliverableWhitepaperCollectionPapersWPFive
\textbf{Work Package:} \WPESRn \tabularnewline 
%\ESRm &  \hdshort & Yes & Month 6& 36 & \deliverableHEPPubATLASTLAMultijet, \deliverableHEPPubPileupNoiseCaloFTK \tabularnewline
\hline
%\multicolumn{2}{|l|}{\textbf{\Tstrut Work Package:}
%WP4,5,6} &
\multicolumn{4}{|l|}{\textbf{Doctoral programme:} \hdshort }&%\tabularnewline\hline
\multicolumn{4}{l|}{\textbf{\Tstrut Project Title: RTA to search for Dark Photons in LHCb}
}\tabularnewline\hline
\multicolumn{8}{|p{20.2cm}|}{\textbf{\Tstrut Objectives:} 
\ESRn will develop an advanced RTA technique to efficiently select decays involving displaced di-electron vertices with low invariant mass. 
\ESRn will use these advances to search for light dark photons with the LHCb detector focusing in a range of extremely low coupling so far unexplored by other experiments. 
If dark matter was produced thermally in the big bang and is light, a light mediator such as a dark photon is necessary to explain its observed relic density. %CD: not sure this is the case
The upgraded LHCb detector, which is planned to start data-taking in 2021, will allow for the first time to search for this kind of signature. 
Indeed, the novel real-time detector readout gives the possibility to use RTA to separate dark photon candidates from the overwhelming background online. 
The reconstruction of tracks and displaced vertices has to be carried out online even for low-momentum tracks. 
This challenging task will depend critically on the performance that can be achieved on modern computing architectures. 
\ESRn will be trained in the usage of highly parallel architectures for the LHCb online reconstruction and will profit from a secondment in \santiago on use of GPUs. 
%The online information from the RICH detectors will be used to efficiently identify the two tracks as electrons and to suppress the fake-track background. %CD: too much detail?
\ESRn will develop advanced ML algorithms to optimise the electron classification and make it fast enough to be run in real time. 
These developments will also allow \ESRn to efficiently retrieve online low momentum photons through their conversion in di-electron pairs. 
Conversion photons have a much better momentum resolution than calorimeter ones and can be used to form vertices with other tracks. 
Finally, \ESRn will design a RTA-based search for LFV decays of the tau lepton by identifying a signature consisting of a muon and a conversion photon forming a displaced vertex, and use it in physics analysis. This work will benefit from a secondment in \dortmund under Albrecht's supervision, who will host both \ESRn and \ESRi and train them how to design a LFV physics analysis with the help of his StG team. 
This will happen in tandem with a secondment at \pointeight where RTA techniques will be used for monitoring and decision-making in German industrial transport. This will complement and strenghten the work of \ESRi, also seconded at \pointeight. 
}
\tabularnewline\hline
\multicolumn{8}{|p{20.2cm}|}{\textbf{\Tstrut Expected Results:}
Peer-reviewed papers on the reconstruction algorithms for low-mass dielectrons and photon conversions in the LHCb trigger, on the search for low mass dark photon decays to di-electrons and finally on the search for LFV in decays of a tau lepton to a photon and a muon. 
\ESRn will receive a PhD in Experimental HEP from \hdshort.
}
\tabularnewline\hline
%\multicolumn{6}{|p{20.2cm}|}{\textbf{Doctoral program:} Cambridge}\tabularnewline\hline
\multicolumn{8}{|p{20.2cm}|}{\textbf{\Tstrut Secondments:}
3 months at \santiago, Martinez Santos. Use of GPU in LHCb online reconstruction. 
2 months at \pointeight, Dungs. Application of RTA to consulting project for German transport industry.  
4 months at \dortmund, Albrecht. Design of LFV physics analysis. 
}\tabularnewline
\hline
\end{tabular}
}%
\end{center}
%\end{table}
%
