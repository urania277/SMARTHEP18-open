%\begin{table}[h]

\begin{center}\small
\resizebox {\textwidth }{!}{%
\begin{tabular}{|p{19mm}|p{37mm}|p{16mm}|p{21mm}|p{23mm}|p{64mm}|}
\hline
\textbf{\Tstrut Fellow} \,\,ESR3&
\textbf{Host} \,\,\unigelong&
\textbf{\phd} \,\,Yes&
\textbf{Start} \,\,Month 8&
\textbf{Duration} \,\,36&
\textbf{Deliverables} \,\,\deliverableWhitepaperStateOfTheArtWPThree, \deliverableWhitepaperDevelopmentWPThree, \deliverableTriggerExperimentalSoftwareWPThree, \deliverableWhitepaperStateOfTheArtWPFour, \deliverableWhitepaperDevelopmentWPFour, \deliverableWhitepaperStateOfTheArtWPFive, \deliverableWhitepaperCollectionPapersWPFive, \deliverablePredictiveMaintenance \tabularnewline 
%Add list of deliverables here
%2.1 & First \acronym conference proceedings & R & PU & 19 & 2 & \saclay & Write and publish proceedings of the first \acronym conference\tabularnewline\midrule 
%2.2 & Toolkit for deep learning & R & PU & 44 & 2 & \saclay & Implementation and release of toolkit for deep learning within HEP\tabularnewline\midrule
%2.3 & Medical insurance provision & R & PU & 44 & 2 & \dq & Application of developed methods to medical insurance provision\tabularnewline\midrule
%2.4 & Deep learning documentation & R & PU & 12,24,36,48 & 2 & \saclay & Publication of research in peer-reviewed journals\tabularnewline\midrule 
%3.1 & Second \acronym conference proceedings & R & PU & 43 & 3 & \dortmund & Proceedings of the second \acronym conference\tabularnewline\midrule
%3.2 & Novel MVA triggers& R & PU & 44 & 3 & \dortmund & Implement novel MVA strategies in LHCb trigger\tabularnewline\midrule
%3.3 & Review of DS methods & R & PU & 44 & 3 & \saclay & Review paper on relationship between DS methods and datasets\tabularnewline\midrule
%3.4 & New figures of merit documentation & R & PU & 12, 24, 36, 48 & 3 & \dortmund & Publication of research in peer-reviewed journals\tabularnewline\midrule
%ESR3 &  \unigelong & Yes & Month 6& 36 &\deliverableLLPTrackingToolkit, \deliverableTechPubLLPGPU, \deliverableHEPPubLLP, \deliverablePredictiveMaintenance \tabularnewline
\hline
\multicolumn{2}{|l|}{\textbf{\Tstrut Work Package:}
WP3,4,5,6} &
\multicolumn{2}{l|}{\textbf{Doctoral programme:} \unige }&%\tabularnewline\hline
\multicolumn{2}{l|}{\textbf{\Tstrut Title: 
ML pattern recognition for exotic physics and industry}
}\tabularnewline\hline
\multicolumn{6}{|p{20.2cm}|}{\textbf{\Tstrut Objectives:}
%Objective 1: ML-based tracking
%Objective 2: ML-based IoT
%Objective 3: Are GPU useful?
%Objective 4: Apply ML and/or GPU to LLP searches
ESR3 will study RTA in both HEP and industry, be trained in state-of-the art ML and AI methods, and use these to improve ATLAS track reconstruction and monitoring of industrial machinery.
RTA track reconstruction in ATLAS is particularly difficult because of extremely busy detector images created by the multiple proton interactions (pile-up) in each bunch collision. This challenge will only increase in the future LHC upgrade.
%One of the biggest challenges in hadron collider physics is the presence of multiple proton interactions 
%that occur in every bunch collision. This creates extremely 
%busy images in the detectors, that need to be deciphered fast and efficiently. Reconstructing 
%particle tracks under these conditions in real-time is a major task that %becomes even more challenging in
%terms of real-time needs of the trigger system. 
%This challenge will only 
%will only increase in the future, with the planned LHC upgrade. 
ESR3's first objective is to develop ML-based track reconstruction as a 
replacement to algorithms too slow to be used in real time. ESR3 will be trained in track reconstruction, modern ML techniques, and general pattern recognition tools. 
The acquired ML expertise will 
be crucial for ESR3's second objective and secondment, in which ESR3 will be trained in industrial production chains, sensors used in IoT-ready plants, collecting and aggregating sensor data in real-time, and forecasting analysis techniques. ESR3 will then utilise ML techniques to collect and analyze 
data from Internet-Of-Things-ready industrial production chains in order to forecast in real-time when the machinery needs intervention.  
%The ESR will acquire knowledge of 1) how industrial production chains are organized; 2) %the type of sensors deployed by
%IoT ready production plants, how they work and how they are connected; 3)
%how to collect and aggregate sensor data in real-time; 3) analysis
%techniques used in forecasting parts breakdowns and failures.
%Track reconstruction is an inherently parallelisable task, but the optimal architecture on which to deploy it must also be evaluated.  
ESR3's third 
objective will be to evaluate GPUs for track reconstruction, especially at higher 
pile-up conditions of the LHC upgrades. ESR3 will compare GPU-optimized ML reconstruction to both CPU-based reconstruction and dedicated hardware (e.g. FPGA) solutions proposed for ATLAS. Through this the ESR will be trained in optimizing algorithms for modern computing architectures.
%Currently, hardware tracking is planned as an alternative to software tracking 
%for triggering purposes. The option of using GPUs with fast machine-learning-based tracking has never been 
%evaluated, and the ESR assigned to this project will have the right expertise to answer this question. 
ESR3's will then apply their reconstruction to a novel real-time  displaced vertex selection in ATLAS, one of the most promising and experimentally challenging NP signatures. ESR3's  search for exotic long-lived signatures with this selection will be a significant step beyond ATLAS's current capabilities, answer crucial questions for the future of ATLAS and open new avenues in the searches for NP. 
%The work will be conducted within the
%ATLAS \unige team and will be well integrated in the ATLAS experiment thanks to the geographical proximity to CERN. 
}\tabularnewline\hline
\multicolumn{6}{|p{20.2cm}|}{\textbf{\Tstrut Expected Results:}
Trigger-level tracking reconstruction software, including long-lived particles, for ATLAS.
Two original research papers: a technical publication of the ML-based track reconstruction and its evaluation on GPUs, and a physics 
publication documenting the results of the search for NP using this technique. 
ESR3 will receive a PhD in experimental HEP at \unigelong. 
%TODO: link this with other GPU-related things
}\tabularnewline\hline
\multicolumn{6}{|p{20.2cm}|}{\textbf{\Tstrut Secondments:}
\lightboxlong, 6 months, Catastini, ML for real-time industrial sensor data acquisition and analysis. %, development of a predictive maintenance software framework. 
}\tabularnewline
\hline
\end{tabular}
}%
\end{center}

%Text for the rest of the application

%Internet of Things (IoT) refers to the use of sensors and other Internet-connected devices to track and control physical objects through the industrial production chain and their subsequent end-user delivery steps. Through IoT, companies may monitor 1) machine status and performance continuously and 2) schedule maintenance only when necessary. The training project we propose is related to 1) and 2).
%In particular the deployment of sensors and systems that are connected and can exchange information allows companies to acquire real-time information of the operational status of each critical component in an industrial production chain. The combination of real-time and historical measurements data is used to infer when the production chain is getting close to a ?critical? status that may require actions and to predict when a specific part of a machine will have to be fixed or replaced. The adoption of these techniques, called predictive maintenance, on average is estimated to reduce maintenance costs by more than 25\%, reduce breakdowns by more than 70\%, reduce downtime by more than 35\% and increase productivity by more than 20\%.

%The data acquired in real-time by the connected sensors are of different nature, unstructured and complex, such as: parts vibration data, lubricant and fuel quality, wear particle data, temperature measurements, ultrasonic noise detection and flow, infrared thermorgraphy, electrical monitoring, etc. These data need to be collected, cleaned, aggregated and analyzed in real-time using both complex event processing infrastructure and statistical analysis techniques. 

%The project will focus on a specific type of production chain with the goal of improving real-time data analysis and forecast by means of machine learning (ML) techniques. Given the heterogeneity of the sensor data, ML techniques are beneficial in forecasting when machinery parts need intervention or should be replaced. 


