%\begin{table}[h]
\begin{center}\small
\resizebox {\textwidth }{!}{%
\begin{tabular}{|p{19mm}|p{23mm}|p{25mm}|p{21mm}|p{23mm}|p{69mm}|}
\hline
\textbf{\Tstrut Fellow} \,\,\ESRa&
\textbf{Host} \,\,\helsinki&
\textbf{\phd} \,\,Yes&
\textbf{Start} \,\,Month 8&
\textbf{Duration} \,\,36&
\textbf{Deliverables}\,\,\deliverableWhitepaperStateOfTheArtWPThree, \deliverableWhitepaperDevelopmentWPThree, \deliverableTriggerExperimentalSoftwareWPThree, \deliverableWhitepaperStateOfTheArtWPFive, \deliverableWhitepaperCollectionPapersWPFive, \deliverableFleetmaticsMLMobile \tabularnewline 
%2.1 & First \acronym conference proceedings & R & PU & 19 & 2 & \saclay & Write and publish proceedings of the first \acronym conference\tabularnewline\midrule 
%2.2 & Toolkit for deep learning & R & PU & 44 & 2 & \saclay & Implementation and release of toolkit for deep learning within HEP\tabularnewline\midrule
%2.3 & Medical insurance provision & R & PU & 44 & 2 & \dq & Application of developed methods to medical insurance provision\tabularnewline\midrule
%2.4 & Deep learning documentation & R & PU & 12,24,36,48 & 2 & \saclay & Publication of research in peer-reviewed journals\tabularnewline\midrule 
%3.1 & Second \acronym conference proceedings & R & PU & 43 & 3 & \dortmund & Proceedings of the second \acronym conference\tabularnewline\midrule
%3.2 & Novel MVA triggers& R & PU & 44 & 3 & \dortmund & Implement novel MVA strategies in LHCb trigger\tabularnewline\midrule
%3.3 & Review of DS methods & R & PU & 44 & 3 & \saclay & Review paper on relationship between DS methods and datasets\tabularnewline\midrule
%3.4 & New figures of merit documentation & R & PU & 12, 24, 36, 48 & 3 & \dortmund & Publication of research in peer-reviewed journals\tabularnewline\midrule
%\ESRa &  \helsinki & Yes & Month 6& 36 & \deliverableCMSHLTGeneralPurposeScouting, \deliverableHEPPubCMSBoosted \tabularnewline
\hline
\multicolumn{2}{|l|}{\textbf{\Tstrut Work Package:}
WP3,5,6} &
\multicolumn{2}{l|}{\textbf{Doctoral programme:} \helsinki }&%\tabularnewline\hline
\multicolumn{2}{l|}{\textbf{\Tstrut Title: ML and RTA for object identification in HEP and industry
%Improvement of jet flavor and heavy object identification techniques and measurement of production of boosted Z and H to bottom jet pairs at the trigger level. 
}
}\tabularnewline\hline
\multicolumn{6}{|p{20.2cm}|}{\textbf{\Tstrut Objectives:}
\ESRa will study real-time object identification in HEP and industry, be trained in state-of-the art ML and AI methods, and use these to improve mobile image processing and measure Higgs and $Z$ boson decays to b-quark pairs.
Deep learning (DL) based algorithms which identify heavy objects (e.g. $b$ quarks) in 
jets, e.g. DNNs, have been tested offline. 
%First steps towards this have been taken in offline reconstruction during Run 2. 
Adopting these to the trigger will significantly improve
trigger performance, allow to record more
interesting data for offline analysis, and 
%enhance the performance of analyses with
%trigger jets, as they will be able 
allow RTA analyses to exploit discriminating features previously only accessible offline. \ESRa's first objective is to understand the
resource cost of DL algorithms, and, if feasible, implement them in the trigger. 
The industrial secondment to Fleetmatics will have the second objective of
adapting DL frameworks, e.g. Tensor Flow and Keras,
in a smartphone environment, for real-time processing of images captured by mobile devices. 
%The ESR will gain hands-on experience within the training of those neural networks, as well as 
%theoretical knowledge about the field of ML, AI and best current practices in general.
%will allow Fleetmatics to explore potential business developments
%in the mobile domain and would be an 
This is an instrumental step towards mobile-phone in-vehicle edge computing
applications for Fleetmatics.
At Fleetmatics \ESRa will be trained in both theory and best practice of DL, and will return with expertise in RTA DL frameworks, which will be key to address the physics challenges.
%After the secondment, the student will have stronger 
% in real-time environment, 
%in improving flavor identification at the trigger level.
\ESRa's third objective is to use their trigger DL algorithms 
to implement a full RTA in which no "safety net" of offline processed events
is needed. By requiring around 1\% of full event storage, this generic "scouting" dataset will allow a much higher event rate and enable a wide range of novel analyses of events with hadronic activity.
%remove the "safety net" of the events that only 
%are processed at a later date in case of discovery. If less 
%storage is needed for these events (e.g. if only 1\% of them is recorded), 
%then the trigger level events can have an increased rate. 
%This can only be done only once the performance of the objects reconstructed in
%real-time is equivalent to offline, and includes the variables in the first objective.
%This generic trigger level dataset will enable a wide range of analyses
%to be performed that rely on datasets with hadronic activity. During 
During the 5-month secondment to CERN \ESRa will be trained in both ML and RTA in the context of HEP, and work on the implementation 
%to collaborate closely with the world experts in ML and real-time analysis
%on the implementation of the flavor identification techniques at the trigger level
and deployment of this general purpose scouting stream.  
%such
%as the scalar sum of jet \pT $H_\textrm{T}$ or the leading jet \pT itself, with much lower thresholds than before. 
\ESRa will validate this stream by measuring the frequent and well understood 
$Z\rightarrow\bar{b}b$ process, before applying it to study 
%A testcase for the general purpose trigger level dataset is the analysis of
modestly boosted $H\rightarrow\bar{b}b$,
complementing the inclusive CMS search for highly boosted $H\rightarrow\bar{b}b$.
%performed on 2016 data by CMS. The measurement of $Z\rightarrow\bar{b}b$ events
%validates the application of these techniques at the trigger level
%on a more frequent and better understood signature. This will lead to
%a peer-reviewed publication. 
%Measurements of direct $H\rightarrow\bar{b}b$ decays
%may resolve the loop induced and tree-level contributions to the gluon fusion production mode
%and provide an alternative approach to study the top quark Yukawa coupling in addition to
%the $ttH$ process, while measuring $Z\rightarrow\bar{b}b$ validates the techniques for trigger analysis.
}\tabularnewline\hline
\multicolumn{6}{|p{20.2cm}|}{\textbf{\Tstrut Expected Results:}
Improve flavor/heavy object tagging at HLT benefiting multiple CMS analyses/triggers, establish high-rate general purpose scouting for Run3.
Validate by measuring $Z\rightarrow\bar{b}b$ cross section, publish technique and measurement. Measure and publish boosted $H\rightarrow\bar{b}b$ production.
%and publish a peer-reviewed paper.
% (Deliverable~\deliverableCMSHLTGeneralPurposeScouting).
Implement mobile ML frameworks for in-vehicle image processing at \fleetmatics.
%(Deliverable~\deliverableFleetmaticsMLMobile) 
%to resolve the long- and short-distance contributions to the gluon fusion process.
%The physics analysis will lead to a peer-reviewed publication on $Z\rightarrow\bar{b}b$ cross section and establishing boosted $H\rightarrow\bar{b}b$ production. 
%(Deliverable~\deliverableHEPPubCMSBoosted). 
%The ESR's progress
%and publications will be part of the Whitepapers of WP3 and WP5. 
\ESRa will receive a PhD in experimental HEP at \helsinkilong.
}\tabularnewline\hline
%\multicolumn{6}{|p{20.2cm}|}{\textbf{Doctoral program:} Cambridge}\tabularnewline\hline
\multicolumn{6}{|p{20.2cm}|}{\textbf{\Tstrut Secondments:}
\cern, 5 months, Pierini, implementing generic data scouting analysis for Run 3; 
\fleetmatics, 4 months, Sambo, In-vehicle image processsing in a smartphone environment. 
%The ESR will gain hands-on experience within the training of those neural networks, as well as 
%theoretical knowledge about the field of ML, AI and best current practices in general.
}\tabularnewline
\hline
\end{tabular}
}%
\end{center}
%\end{table}
%
