%\begin{table}[h]

%\begin{table}[h]
\begin{center}\small
\resizebox {\textwidth }{!}{%
\begin{tabular}{|p{21mm}|p{36mm}|p{24mm}|p{23mm}|p{18mm}|p{58mm}|}
\hline
\textbf{\Tstrut Fellow} \,\,ESR10&
\textbf{Host} \,\,\parisU&
\textbf{\phd} \,\,Yes&
\textbf{Start} \,\,Month 8&
\textbf{Duration} 36&
\textbf{Deliverables} \,\, \deliverableWhitepaperStateOfTheArtWPThree,\deliverableWhitepaperDevelopmentWPThree,\deliverableWhitepaperStateOfTheArtWPFour,\deliverableWhitepaperDevelopmentWPFour,\deliverableParallelizationOptimizationWPFour \tabularnewline 
%ESR10 &  \sorbonnelong & Yes & Month 6& 36 &  \tabularnewline
\hline
\multicolumn{2}{|l|}{\textbf{\Tstrut Work Package:}
WP3,4} &
\multicolumn{2}{l|}{\textbf{Doctoral programme:} \parisU } & %\tabularnewline\hline
\multicolumn{2}{l|}{\textbf{\Tstrut Title: RTA on heterogeneous computing architectures}
}\tabularnewline\hline
\multicolumn{6}{|p{20.2cm}|}{\textbf{\Tstrut Objectives:}
The goal of this project is to, in partnership with a number of computing hardware companies, 
train ESR10 in programming for heterogeneous computing architectures, 
and in particular simultaneously optimizing data formats and processing techniques to enable CPUs, GPUs, FPGAs, 
and hybrids to work together to solve problems which none of these technologies could solve on their own. 
The first objective of ESR10 will be to 
design a novel ML method for optimizing heterogeneous computing architectures,
and deploy it in the context of the specific requirements of real-time data processing in the LHC collaborations, 
with ATLAS and LHCb as main examples to exploit synergies with ESR8 and ESR9 based in the Paris area.
This method takes as input the cost of the various computing architectures under consideration, builds and
emulates test processing systems, tests them using the task being optimized for and finds the most cost-effective one.
The second objective will be to apply this optimization code
to the specific problems of the LHCb and ATLAS real-time data-processing architectures. 
This work will be done in collaboration with ESR11 and ESR12, as they will prepare a software toolkit that
can be used to benchmark these processes. 
}\tabularnewline\hline
\multicolumn{6}{|p{20.2cm}|}{\textbf{\Tstrut Expected Results:}
This ESR project will lead to the release of a software package to optimize heterogeneous computing architectures and 
two peer-reviewed papers, one documenting the novel machine learning based optimization, 
and one documenting the possible applications of this method to the optimization of the LHCb and ATLAS real-time data
processing. ESR10 will receive a PhD in experimental HEP at \sorbonneentity. 
}\tabularnewline\hline
%\multicolumn{6}{|p{20.2cm}|}{\textbf{Doctoral program:} Cambridge}\tabularnewline\hline
\multicolumn{6}{|p{20.2cm}|}{\textbf{\Tstrut Secondments:}
A five-month secondment at \cernentity will allow for training in machine learning methods and interaction with the LHCb and ATLAS
collaborators involved in the optimization of their respective real-time analysis systems, under the supervision of Petersen and Couturier. 
%A four month secondment at NVIDIA will provide ESRXXX with valuable training on the optimization of heterogeneous architectures, under the supervisionof (insert commercial link here).
}\tabularnewline
\hline
\end{tabular}
}%
\end{center}
%\end{table}
%

%The ATLAS trigger infrastructure has an unique hardware processor to reconstruct the trajectory (tracking) of the charged particles that cross the silicon inner tracker of the experiment. The tracking information is an essential tool for effective real-time event selection and has a central role in the whole ATLAS physics program especially in the HL-LHC phase. The current hardware processor, FTK, is a complex system made by several custom electronics  boards based on FPGAs and Associative Memory chips. The latter are unique computing devices developed for the FTK algorithm. The hardware tracking will be also a central part of the Phase-II Upgrade of ATLAS, with upgraded version of FTK called FTK++.


