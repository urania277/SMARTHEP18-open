%\begin{table}[h]

%\begin{table}[h]
\begin{center}\small
\resizebox {\textwidth }{!}{%
\begin{tabular}{|p{25mm}|p{23mm}|p{18mm}|p{28mm}|p{34mm}|p{50mm}|}
\hline
\textbf{\Tstrut Fellow} \,\,\ESRg&
\textbf{Host} \,\,\parisUentity&
\textbf{\phd} \,\,Yes&
\textbf{Start (mo.)} \,\,8&
\textbf{Duration (mo.)} \,\,36&
\textbf{Deliverables} \,\, \deliverableWhitepaperStateOfTheArtWPThree, \deliverableWhitepaperDevelopmentWPThree, \deliverableWhitepaperStateOfTheArtWPFour, \deliverableWhitepaperDevelopmentWPFour, \deliverableParallelizationOptimizationWPFour \tabularnewline 
%\ESRg &  \sorbonnelong & Yes & Month 6& 36 &  \tabularnewline
\hline
\multicolumn{2}{|l|}{\textbf{\Tstrut Work Package:}
\WPESRg} &
\multicolumn{2}{l|}{\textbf{Doctoral programme:} \parisUentity } & %\tabularnewline\hline
\multicolumn{2}{l|}{\textbf{\Tstrut Title: RTA on heterogeneous computing architectures}
}\tabularnewline\hline
\multicolumn{6}{|p{21cm}|}{\textbf{\Tstrut Objectives:}
The goal of this project is to, in partnership with a number of computing hardware companies, train \ESRg in programming for heterogeneous computing architectures, and in simultaneously optimizing data formats and processing techniques to enable CPUs, GPUs, FPGAs, and hybrids to work together to solve problems which none of these technologies could solve on their own. 
The first objective of \ESRg will be to design a novel ML method for optimizing heterogeneous computing architectures, and deploy it in the context of the specific requirements of real-time data processing in the LHC collaborations.
This will use ATLAS and LHCb as main examples to exploit synergies with \ESRf and \ESRx who are also based in the Paris area.
This method takes as input the cost of the various computing architectures under consideration, builds and emulates test processing systems, tests them using the task being optimized for and finds the most cost-effective one.
The second objective will be to apply this optimization code to the specific problems of the LHCb and ATLAS real-time data-processing architectures. 
This work will be done in collaboration with \ESRh and \ESRi, as they will prepare a software toolkit that can be used to benchmark these processes. 
A five-month secondment at \cernentity will allow for training in ML methods and interaction with the LHCb and ATLAS collaborators involved in the optimization of their respective real-time analysis systems, under the supervision of Petersen and Couturier. 
Even though \ESRg does not have an industrial secondment, while seconded at \cernentity, \ESRg will be able to benefit from the mentoring of Catastini from \lightboxentity, to understand the market potential of such advancements of hybrid architectures.  
}\tabularnewline\hline
\multicolumn{6}{|p{21cm}|}{\textbf{\Tstrut Expected Results:}
1. ML-based optimization of heterogeneous computing architectures (peer-reviewed paper, software package). 
2. Application of optimization code to LHCb and ATLAS triggers (peer-reviewed paper).
\ESRg will receive a PhD in experimental HEP at \parisUlong. 
}\tabularnewline\hline
%\multicolumn{6}{|p{20.2cm}|}{\textbf{Doctoral program:} Cambridge}\tabularnewline\hline
\multicolumn{6}{|p{21cm}|}{\textbf{\Tstrut Secondments:}
\cernentity, 5 months, Petersen and Couturier, optimization of trigger systems of LHCb and ATLAS.  
}\tabularnewline
\hline
\end{tabular}
}%
\end{center}


