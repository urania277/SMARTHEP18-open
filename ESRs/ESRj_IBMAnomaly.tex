%\begin{table}[h]
\begin{center}\small
\resizebox {\textwidth }{!}{%
\begin{tabular}{|p{25mm}|p{23mm}|p{18mm}|p{28mm}|p{34mm}|p{50mm}|}
\hline
\textbf{\Tstrut Fellow} \,\,\ESRj&
\textbf{Host} \,\,\ibmentity&
\textbf{\phd} \,\,Yes&
\textbf{Start (mo.)} \,\,8&
\textbf{Duration (mo.)} \,\,36&
\textbf{Deliverables}\,\,\deliverableWhitepaperStateOfTheArtWPThree,\deliverableWhitepaperDevelopmentWPThree,\deliverableTriggerExperimentalSoftwareWPThree,\deliverableWhitepaperStateOfTheArtWPSix, \deliverableRule, \deliverableWhitepaperCollectionPapersWPSix \tabularnewline 
\hline
\multicolumn{2}{|l|}{\textbf{\Tstrut Work Package:}
\WPESRj} &
\multicolumn{2}{l|}{\textbf{Doctoral programme:} \lundentity }&%\tabularnewline\hline
\multicolumn{2}{l|}{\textbf{\Tstrut Title: Novelty detection for industry and ATLAS searches}
}\tabularnewline\hline
\multicolumn{6}{|p{21.2cm}|}{\textbf{\Tstrut Objectives:}
Unlike SM particles, dark matter can have a longer lifetime than known particles and for example decay in the ATLAS calorimeter without leaving a trace in previous detector volumes.
In this case, calorimeter "noise bursts" have a very similar signature and while relatively infrequent occur in 0.015\% of the events, far more frequent than dark matter. 
ATLAS cannot currently distinguish the two in real-time. 
Anomaly detection is an ML technique concerned with the detection of ultra-rare "anomalous" events which do not follow part of the "normal" pattern of input samples and where little is known about the distribution of these anomalies. 
A wealth of different techniques exist, and this project will use existing ML methods to detect novelties, while extending and/or specializing the methods where appropriate to enable their use in RTA.
Of particular interest to IBM is the possibility to combine ML with powerful general mathematical programming solvers, such as the IBM CPLEX commercial product.
The first objective of \ESRj is to develop an algorithm that runs on existing ATLAS data and datasets provided by IBM, and discriminates between an anomalous signal and background, using CPLEX. 
Subsequently, this algorithm will be compared to the open source version, and the latter implemented as an anomaly detection system running in the ATLAS trigger, to be implemented in Run-3.
The physics side of the project will be co-supervised by Doglioni (\lundentity).
While benefiting from her expertise in searches for new physics, this work will represent a significant step beyond the ATLAS physics program, since it targets final states never explored before. 
The 6-months \cern secondment under the supervision of Boveia (\ohioentity) will ensure that this work is well integrated in the trigger system of the ATLAS experiment, and extend it to different detector signatures.
}\tabularnewline\hline
\multicolumn{6}{|p{21.2cm}|}{\textbf{\Tstrut Expected Results:}
1. Develop combination of ML and programming solvers for novelty detection (peer-reviewed paper). 
2. Extend improved novelty detection algorithms to RTA in industry and HEP (conference paper). 
3. Apply to dark matter searches in ATLAS (peer-reviewed paper).
\ESRj will receive a PhD in experimental HEP at \lundlong.
}\tabularnewline\hline
\multicolumn{6}{|p{21.2cm}|}{\textbf{\Tstrut Secondments:}
\cern, 6 months, Boveia. Application of anomaly detection techniques to ATLAS data. 
}\tabularnewline
\hline
\end{tabular}
}%
\end{center}