%\begin{table}[h]

%\begin{table}[h]
\begin{center}\small
\resizebox {\textwidth }{!}{%
\begin{tabular}{|p{19mm}|p{26mm}|p{25mm}|p{21mm}|p{23mm}|p{66mm}|}
\hline
\textbf{\Tstrut Fellow} \,\,ESR7 &
\textbf{Host} \,\,Dortmund&
\textbf{\phd} \,\,Yes&
\textbf{Start} \,\,Month 8&
\textbf{Duration} \,\,36 &
\textbf{Deliverables} \,\,\deliverableWhitepaperStateOfTheArtWPThree, \deliverableWhitepaperDevelopmentWPThree, \deliverableTriggerExperimentalSoftwareWPThree, \deliverableWhitepaperStateOfTheArtWPFive, \deliverableWhitepaperCollectionPapersWPFive, \deliverableComputingOptimisation \tabularnewline 
\hline
\multicolumn{2}{|l|}{\textbf{\Tstrut Work Package:}
WP3,5,6} &
\multicolumn{2}{l|}{\textbf{Doctoral programme:} \dortmund } &
\multicolumn{2}{l|}{\textbf{\Tstrut Title: Event Triggering in LHCb}
}\tabularnewline\hline
\multicolumn{6}{|p{20.2cm}|}{\textbf{\Tstrut Objectives:}
%Trigger systems of modern HEP experiments reduce the event data
%processed in real time by several orders of magnitude. %
%The paradigm so
ESR7 will study deep learning (DL) RTAs of complex systems, in both HEP and industry. ESR7 will receive training in DL, as well as other modern ML and AI methods, and apply these methods to global analyses of HEP collisions and the monitoring of computing clusters.
HEP triggers first reconstruct objects (e.g. jets or
tracks), and then perform a selection on these objects. 
Event triggers select based on global event properties, avoiding time-intensive reconstruction, and may allow to search for BSM physics in areas where traditional object-based selections are be too slow.
% have shown great potential, e.g. identifying
%primary vertices where protons collided from raw detector hits, but a lot of further development must happen before this can be put into production. 
ESR7's first objective will be to analyze the direct
pattern of detector hits to identify events enhanced in interesting
physical processes, building on promising initial studies in the ERC StG of
Albrecht.
%ESR7 will develop triggers based on global event properties
%rather than objects, enabling the event to be classed as interesting without the time-intensive reconstruction of
%objects. If successful,   
ESR7 will design a global trigger selection and benchmark its performance against a
traditional object-based approach.  
%Through this ESR7 will be trained in state-of-the-art trigger selections,
%also developing expertise crucial for physics analyses. 
The experience in event trigger selections will be used to work with
\wildtreeentity. There, ESR7 will be trained in deploying global event analysis for monitoring and predicting computing cluster failures, 
leading to monetary and resource savings.
As part of this ESR7 will be trained in modern AI methods such as DL using recurrent NNs, which have shown great success in speech recognition or translation,
and use them to further improve their global analysis tools.
%expensive equipment, for
%example in computing clusters. 
%This equipment produces a large amount of different monitoring streams
%that can be used to predict failures before they occur. 
%Each second the equipment is not running costs the owner large amounts
%of money.
%State of the art models of failure prediction rely on humans to create
%relevant, high level features based on raw data. In this
%secondment we will apply deep learning technology to remove this need
%for human expertise. Motivated by the success of deep learning in
%other fields like human speech recognition and translation we will
%apply deep recurrent neural networks to this problem. 
%Hence the ESR
%will learn to apply modern AI methods in real time in an industrial
%environment. 
ESR7's second objective will be to apply these methods
%developed at the secondment at \wildtree 
to optimize the computing clusters in
the \acronym network, assisted by the the Lund computing
group and performing the first implementation in Lund. 
This will train ESR7 to apply
developed methods to real-world problems.
ESR7's third objective will be to apply the developed event triggers to the analysis of 
LFUV in semileptonically decaying beauty decays. These are abundant enough to allow the event trigger to be benchmarked against more traditional approaches, providing more real-world experience.
%
%The third and final objective of ESR7 will be the analysis of
%, that can be used to test
%. These decays are frequent enough to allow the event trigger developed by ESR7  
}\tabularnewline\hline
\multicolumn{6}{|p{20.2cm}|}{\textbf{\Tstrut Expected Results:}
Peer-reviewed paper on the search for LFU in semileptonic beauty decays. Open source 
deep recurrent NN based framework for the 
optimisation of power consumption and monitoring of computing clusters. ESR7 will receive a PhD in experimental HEP at \dortmund.
}\tabularnewline\hline
%\multicolumn{6}{|p{20.2cm}|}{\textbf{Doctoral program:} Cambridge}\tabularnewline\hline
\multicolumn{6}{|p{20.2cm}|}{\textbf{\Tstrut Secondments:}
3 months at \wildtreeentity, Head. Deep recurrent NNs for optimisation of power consumption and
monitoring of computing clusters. 
2 months at \lundentity, Smirnova. Implementation of
methods developed at \wildtreeentity in the Lund computing cluster.
}\tabularnewline
\hline
\end{tabular}
}%
\end{center}
%\end{table}
%
