%\begin{table}[h]

%\begin{table}[h]
\begin{center}\small
\resizebox {\textwidth }{!}{%
\begin{tabular}{|p{19mm}|p{26mm}|p{25mm}|p{21mm}|p{23mm}|p{66mm}|}
\hline
\textbf{\Tstrut Fellow} \,\,\ESRe &
\textbf{Host} \,\,Dortmund&
\textbf{\phd} \,\,Yes&
\textbf{Start} \,\,Month 8&
\textbf{Duration} \,\,36 &
\textbf{Deliverables} \,\,\deliverableWhitepaperStateOfTheArtWPThree, \deliverableWhitepaperDevelopmentWPThree, \deliverableTriggerExperimentalSoftwareWPThree, \deliverableWhitepaperStateOfTheArtWPFive, \deliverableWhitepaperCollectionPapersWPFive, \deliverableLogisticsOptimisation \tabularnewline 
\hline
\multicolumn{2}{|l|}{\textbf{\Tstrut Work Package:}
\WPESRe} &
\multicolumn{2}{l|}{\textbf{Doctoral programme:} \dortmund } &
\multicolumn{2}{l|}{\textbf{\Tstrut Title: Event Triggering in LHCb}
}\tabularnewline\hline
\multicolumn{6}{|p{20.2cm}|}{\textbf{\Tstrut Objectives:}
\ESRe will study deep learning (DL) RTAs of complex systems, in both HEP and industry. 
\ESRe will receive training in DL, as well as other modern ML and AI methods, and apply these methods to global analyses of HEP collisions.
As part of this \ESRe will be trained in modern AI methods such as DL using recurrent NNs, which have shown great success in speech recognition or translation, and use them to further improve their global analysis tools.
HEP triggers first reconstruct objects (e.g. jets or tracks), and then perform a selection on these objects. 
Event triggers select based on global event properties, avoiding time-intensive reconstruction, and may allow to search for BSM physics in areas where traditional object-based selections are be too slow.
\ESRe's first objective will be to analyze the direct pattern of detector hits to identify events enhanced in interesting physical processes, building on promising initial studies in the ERC StG of Albrecht.
\ESRe will design a global trigger selection and benchmark its performance against a traditional object-based approach.  
The experience in global event trigger selections will be used to collaborate with \ibmentity's \ESRx, comparing a symbolic approach with a purely stochastic ML one. 
With the secondment at Point 8, \ESRe will work on a real-time data-analysis project in German industry. 
The ESR's task will be analysing client-provided data, applying similar techniques as in HEP to real-time traffic predictions for logistics and transport companies. 
This secondment will train \ESRe to apply the developed methods to real-world problems.
Once back in Dortmund, \ESRe's third objective will be to apply the developed event triggers to the analysis of LFUV in semileptonically decaying beauty decays. 
These are abundant enough to allow the event trigger to be benchmarked against more traditional approaches, providing more real-world experience.
}\tabularnewline\hline
\multicolumn{6}{|p{20.2cm}|}{\textbf{\Tstrut Expected Results:}
Conference paper on RTA for complex systems and its implementation in logistics. 
Peer-reviewed paper on the search for LFU in semileptonic beauty decays. 
\ESRe will receive a PhD in experimental HEP at \dortmund.
}\tabularnewline\hline
\multicolumn{6}{|p{20.2cm}|}{\textbf{\Tstrut Secondments:}
4 months at \nikhefentity, Raven. Optimization of global event triggers. 
4 months at \pointeightentity, Dungs. Analysis of logistics data with RTA methods. 
}\tabularnewline
\hline
\end{tabular}
}%
\end{center}
%\end{table}
%
