\begin{center}\small
\resizebox {\textwidth }{!}{%
\begin{tabular}{|p{25mm}|p{26mm}|p{18mm}|p{28mm}|p{34mm}|p{60mm}|}
\hline
\textbf{\Tstrut Fellow} \,\,\ESRe &
\textbf{Host} \,\,\dortmundentity&
\textbf{\phd} \,\,Yes&
\textbf{Start (mo.)} \,\,8&
\textbf{Duration (mo.)} \,\,36&
\textbf{Deliverables} \,\,\deliverableWhitepaperStateOfTheArtWPThree, \deliverableWhitepaperDevelopmentWPThree,  \deliverableTriggerExperimentalSoftwareWPThree, \deliverableWhitepaperStateOfTheArtWPFive, \deliverableWhitepaperCollectionPapersWPFive, \deliverableLogisticsOptimisation \tabularnewline 
\hline
\multicolumn{2}{|l|}{\textbf{\Tstrut Work Package:}
\WPESRe} &
\multicolumn{2}{l|}{\textbf{Doctoral programme:} \dortmundentity } &
\multicolumn{2}{l|}{\textbf{\Tstrut Title: Global event triggering in LHCb}
}\tabularnewline\hline
\multicolumn{6}{|p{21.2cm}|}{\textbf{\Tstrut Objectives:} HEP triggers first reconstruct objects (e.g. jets or tracks), and then perform a selection on these objects. 
Event triggers select based on overall (global) event properties, avoid time-intensive reconstruction, and permit searching for new physics in areas where traditional object-based selections are too slow.
\ESRe will receive training in DL of complex systems, as well as other modern ML and AI methods, and apply these methods to global analyses of HEP collisions in RTA.
\ESRe will focus on DL using recurrent neural networks, which have shown great success in speech recognition or translation, and use them to further improve global analysis tools.
\ESRe's first objective will be to analyze the overall pattern of detector hits to identify events enhanced in interesting physical processes, building on promising initial studies in Albrecht's ERC StG.
As a first objective, \ESRe will design a global trigger selection and benchmark its performance against a traditional object-based approach. 
This will be done in collaboration with \ESRi, within a secondment at \nikhefentity. 
The experience in global event trigger selections will be used to collaborate with \ibmentity's \ESRx, comparing a symbolic approach with a purely stochastic ML one. 
With the secondment at \pointeightentity, \ESRe will work on a real-time data-analysis project in German industry that requires a global analysis. 
As a second objective, the ESR will analyze client-provided data on real-time traffic prediction for transport companies, applying similar techniques as in HEP. 
Once back in Dortmund, \ESRe's final objective will be to apply the developed event triggers to the analysis of LFU violation in semileptonically decaying beauty mesons, which are abundant enough to allow the event trigger to be benchmarked against more traditional approaches.
}\tabularnewline\hline
\multicolumn{6}{|p{21.2cm}|}{\textbf{\Tstrut Expected Results:} 
1. Global event-based RTA using recurrent NN and other ML algorithms (conference paper).
2. Implementation of global RTA to transport in \pointeightentity.  
3. Application of global event-based RTA and benchmarking against traditional approach (peer-reviewed paper)
\ESRe will receive a PhD in experimental HEP at \dortmund.
}\tabularnewline\hline
\multicolumn{6}{|p{21.2cm}|}{\textbf{\Tstrut Secondments:}
\nikhefentity, 4 months, Raven. Optimization of global event triggers. 
\pointeightentity, 4 months, Dungs. Analysis of logistics data with RTA methods. 
}\tabularnewline
\hline
\end{tabular}
}%
\end{center}