%\begin{table}[h]

%\begin{table}[h]
\begin{center}\small
\resizebox {\textwidth }{!}{%
\begin{tabular}{|p{25mm}|p{26mm}|p{18mm}|p{28mm}|p{34mm}|p{60mm}|}
\hline
\textbf{\Tstrut Fellow} \,\,\ESRd&
\textbf{Host} \,\,\dortmundentity &
\textbf{\phd} \,\,Yes &
\textbf{Start (mo.)} \,\,8&
\textbf{Duration (mo.)} \,\,36&
\textbf{Deliverables} \,\,\deliverableWhitepaperStateOfTheArtWPThree, \deliverableWhitepaperDevelopmentWPThree, \deliverableTriggerExperimentalSoftwareWPThree, \deliverableWhitepaperStateOfTheArtWPFive, \deliverableTriggerExperimentalSoftwareWPFive, \deliverableWhitepaperCollectionPapersWPFive, \deliverableXimantisML \tabularnewline 
\hline
\multicolumn{2}{|l|}{\textbf{\Tstrut Work Package:}
\WPESRd} &
\multicolumn{2}{l|}{\textbf{Doctoral programme:} \dortmundentity } & %\tabularnewline\hline
\multicolumn{2}{l|}{\textbf{\Tstrut Title: Real-time ML for LFV in unflavoured meson decays}
}\tabularnewline\hline
\multicolumn{6}{|p{21.2cm}|}{\textbf{\Tstrut Objectives:}
\ESRd will work closely together with \ESRe, and search for LFV in neutral meson decays. 
The precision of tests by older experiments in LFV in neutral meson decays, in particular for decays of $\phi$, $J/\psi$ and $Y(1S)$ mesons, could be improved by orders of magnitude in the LHCb upgrade. 
This will need a fully RTA-based analysis, because of the difficult separation between these decays and the overwhelming background.
\ESRd's first objective will develop ML-based real time selections for the unflavoured $\phi$, $J/\psi$ and $Y(1S)$ mesons decaying into the different lepton species. 
\ESRd will be trained in optimizing ML selection algorithms for RTA applications, benefiting from the existing expertise of the ERC StG group of J. Albrecht. 
\ESRd will be seconded to \ximantisentity, where they will receive training in current best practice in ML and AI and gain hands-on experience optimizing ML algorithms for traffic prediction applications.  
This additional industrial experience and training will allow \ESRd to gain perspective on what kinds of methods are best suited to different types of real-time problems.
%\ESRd's second objective will be to use this training to enhance existing AI-based traffic modelling at \ximantisentity.
A secondment at \santiagoentity followed by a secondment at \cernentity will allow \ESRd to deploy and commission the RTA techniques for LFV decays of light and unflavoured particles in the LHCb trigger as a third objective. 
\ESRd's fourth objective will be to search for LFV in the decays of unflavoured mesons using Run~3 LHCb data. 
}\tabularnewline\hline
\multicolumn{6}{|p{21.2cm}|}{\textbf{\Tstrut Expected Results:}
1. Study of ML-based RTA for neutral meson decays.  
2. Enhance existing AI-based traffic modelling at \ximantisentity.
3. Deploy and commission trigger in LHCb (peer-reviewed paper). 
4. Search for LFV in decays of unflavoured mesons. 
\ESRd will receive a PhD in experimental HEP at \dortmund.
}\tabularnewline\hline
%\multicolumn{6}{|p{20.2cm}|}{\textbf{Doctoral program:} Cambridge}\tabularnewline\hline
\multicolumn{6}{|p{21.2cm}|}{\textbf{\Tstrut Secondments:}
\santiagoentity, 2 months, Martinez Santos, physics of LFV. 
\cernentity, 2 months, Matev, development and implementation of LFV trigger selections. 
\ximantisentity, 4 months, Sopasakis, enhancement of AI modelling algorithms for traffic prediction, using Convolutional Neural Networks (CNNs).  
}\tabularnewline
\hline
\end{tabular}
}%
\end{center}
%\end{table}
%
