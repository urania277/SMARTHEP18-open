%\begin{table}[h]

%\begin{table}[h]
\begin{center}\small
\resizebox {\textwidth }{!}{%
\begin{tabular}{|p{19mm}|p{26mm}|p{25mm}|p{21mm}|p{23mm}|p{66mm}|}
\hline
\textbf{\Tstrut Fellow} \,\,ESR6&
\textbf{Host} \,\,Dortmund &
\textbf{\phd} \,\,Yes &
\textbf{Start} \,\,Month 8&
\textbf{Duration} \,\,36&
\textbf{Deliverables} \,\,\deliverableWhitepaperStateOfTheArtWPThree, \deliverableWhitepaperDevelopmentWPThree, \deliverableTriggerExperimentalSoftwareWPThree, \deliverableWhitepaperStateOfTheArtWPFive, \deliverableWhitepaperCollectionPapersWPFive, \deliverableXimantisML \tabularnewline 
\hline
\multicolumn{2}{|l|}{\textbf{\Tstrut Work Package:}
\WPESRd} &
\multicolumn{2}{l|}{\textbf{Doctoral programme:} \dortmund } & %\tabularnewline\hline
\multicolumn{2}{l|}{\textbf{\Tstrut Title: Real-time
     MVA for LFV in unflavoured meson decays}
}\tabularnewline\hline
\multicolumn{6}{|p{20.2cm}|}{\textbf{\Tstrut Objectives:}
%The objective of this ESR is to advance the level of studies of Lepton
%Flavour Violation (LFV) to the next level. 
\ESRd will develop unified real-time selections and models in HEP and industry, and be trained in state of the art AI and ML methods and algorithms for real-time applications.
\ESRd will work closely together with ESR5, and search for LFV in neutral meson decays. 
Tests by older experiments reach a precision of about one in a million for decays of $\phi$, $J/\psi$ and $Y(1S)$ mesons, while the LHCb upgrade could improve this precision by orders of magnitude. 
The difficult separation between these decays and the overwhelming background means the analysis must be done in real-time. \ESRd's first objective will develop fast ML-based real time selections for the unflavoured $\phi$, $J/\psi$ and $Y(1S)$ mesons decaying into the different lepton species. 
\ESRd will be trained in optimizing ML selection algorithms for RTA applications, and implementing them within the LHCb trigger, benefiting from the existing expertise of the ERC StG group of J. Albrecht which the ESR will be attached to. 
\ESRd will be seconded to Ximantis, where they will gain hands-on experience optimizing neural networks for traffic prediction applications, and receive training in current best practice in ML and AI. This additional industrial experience and training will allow \ESRd to gain vital perspective on what kinds of ML and AI methods are best suited to different types of real-time problems.
\ESRd's second objective will be to use this training to enhance existing AI-based traffic modelling at \ximantisentity.
The CERN secondment will allow \ESRd to be trained in trigger and RTA development by leading members of the LHCb CERN group.
\ESRd's third objective will be the development of unified RTA selections for LFV decays of light and unflavoured particles in LHCb's trigger, aided by Martinez Santos's expertise during the secondment at \santiagoentity.
\ESRd's fourth objective will be to search for LFV in the decays of unflavoured mesons using the first Run~3 LHCb data. 
}\tabularnewline\hline
\multicolumn{6}{|p{20.2cm}|}{\textbf{\Tstrut Expected Results:}
Two peer-reviewed papers. The first one on the improved trigger selection and the second on the results of the search for LFV in decays of unflavoured mesons. 
\ESRd will receive a PhD in experimental HEP at \dortmund.
}\tabularnewline\hline
%\multicolumn{6}{|p{20.2cm}|}{\textbf{Doctoral program:} Cambridge}\tabularnewline\hline
\multicolumn{6}{|p{20.2cm}|}{\textbf{\Tstrut Secondments:}
\santiagoentity, 2 months, Martinez Santos, physics of LFV. 
\cernentity, 2 months, Matev, development and implementation of LFV trigger selections. 
\ximantisentity, 4 months, Sopasakis, enhancement of AI modelling algorithms for traffic prediction, using Convolutional Neural Networks (CNNs).  
}\tabularnewline
\hline
\end{tabular}
}%
\end{center}
%\end{table}
%
