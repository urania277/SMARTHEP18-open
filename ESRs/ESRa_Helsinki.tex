%\begin{table}[h]
\begin{center}\small
\resizebox {\textwidth }{!}{%
\begin{tabular}{|p{25mm}|p{23mm}|p{18mm}|p{28mm}|p{34mm}|p{60mm}|}
\hline
\textbf{\Tstrut Fellow} \,\,\ESRa&
\textbf{Host} \,\,\helsinkientity&
\textbf{\phd} \,\,Yes&
\textbf{Start (mo.)} \,\,8&
\textbf{Duration (mo.)} \,\,36&
\textbf{Deliverables}\,\,\deliverableWhitepaperStateOfTheArtWPThree, \deliverableTriggerExperimentalSoftwareWPThree, \deliverableFinalWhitepaperWPThree, \deliverableWhitepaperStateOfTheArtWPSix, \deliverableSoftwareWPSix, \deliverableWhitepaperCollectionPapersWPSix, \deliverableFleetmaticsMLMobile \tabularnewline 
\hline
\multicolumn{2}{|l|}{\textbf{\Tstrut Work Package:}
\WPESRa} &
\multicolumn{2}{l|}{\textbf{Doctoral programme:} \helsinkientity }&%\tabularnewline\hline
\multicolumn{2}{l|}{\textbf{\Tstrut Title: ML and RTA for Higgs boson measurements and industry}
%was: ML and RTA for object identification in HEP and industry}
}\tabularnewline\hline
\multicolumn{6}{|p{21.2cm}|}{\textbf{\Tstrut Objectives:} Deep learning (DL) based algorithms which identify heavy objects (e.g. $b$ quarks) in jets, e.g. DNNs, have been tested offline but not yet used in the trigger. 
As a first objective, \ESRa will understand the resource cost of these DNNs, improve them if necessary, and adapt them to be used in the trigger. This will significantly improve trigger performance, record more interesting data, and allow pure RTA-based data taking to exploit discriminating features previously inaccessible. 
The industrial secondment to Fleetmatics will have the second objective of adapting DL frameworks, e.g. Tensor Flow and Keras, in resource-constrained environments (e.g. embedded platforms), for real-time processing of images captured by mobile devices. 
This is an instrumental step towards mobile-phone in-vehicle edge computing applications for \fleetmaticsentity.
At \fleetmaticsentity, \ESRa will be trained in theory and best practice of DL, and will return with expertise in RTA DL frameworks in constrained environments to improve the initial trigger selections towards their application in physics analysis. 
\ESRa's third objective is to exploit DL in trigger algorithms to implement a generic RTA that caters to a variety of different physics analyses with hadronic activity and allows a much higher event rate. 
%By requiring around 1\% of full event storage, this generic "scouting" dataset will allow a much higher event rate and enable a wide range of novel analyses of events with hadronic activity.
During the 5-month secondment to CERN \ESRa will be trained in both ML and RTA in the context of HEP, and work on the implementation and deployment of this general purpose scouting stream.  
\ESRa will validate this stream by measuring the frequent and well understood $Z\rightarrow bb$
%$Z\rightarrow\bar{b}b$ 
process, and then apply it to a new measurement of the 
%$H\rightarrow\bar{b}b$ 
$H\rightarrow bb$ process.
}\tabularnewline\hline
\multicolumn{6}{|p{21.2cm}|}{\textbf{\Tstrut Expected Results:} 1. Improve flavor/heavy object tagging at HLT in CMS (peer-reviewed paper). 
2. Implement mobile ML frameworks for in-vehicle image processing at \fleetmatics.
3. Establish high-rate general purpose RTA stream for Run3.  
4. Validate stream by measuring $Z\rightarrow bb$ cross section, then measure boosted $H\rightarrow bb$ production (peer-reviewed papers).
\ESRa will receive a PhD in experimental HEP at \helsinkilong.
}\tabularnewline\hline
\multicolumn{6}{|p{21.2cm}|}{\textbf{\Tstrut Secondments:}
\fleetmatics, 4 months, Taccari, In-vehicle image processing in resource-constrained environment; 
\cern, 5 months, Pierini, commissioning generic data scouting analysis for Run 3 using DL. 
}\tabularnewline
\hline
\end{tabular}
}%
\end{center}
