%\begin{table}[h]

%\begin{table}[h]
\begin{center}\small
\resizebox {\textwidth }{!}{%
\begin{tabular}{|p{19mm}|p{20mm}|p{30mm}|p{30mm}|p{23mm}|p{58mm}|}
\hline
\textbf{\Tstrut Fellow} \,\,ESR9&
\textbf{Host} \,\,\dq&
\textbf{\phd} \,\, Yes &
\textbf{Start} \,\, Month 8&
\textbf{Duration} \,\,  36&
\textbf{Deliverables}\,\, \deliverableWhitepaperStateOfTheArtWPThree,\deliverableWhitepaperDevelopmentWPThree,\deliverableTriggerExperimentalSoftwareWPThree, \deliverableWhitepaperStateOfTheArtWPFive, \deliverableWhitepaperCollectionPapersWPFive, \deliverableNN\tabularnewline 
%ESR9 &  \cnrs & Yes & Month 6& 36 & \deliverableAdversarialFramework, \deliverableTimeOrderedSourcesFramework, \deliverableTechPubTimeOrderedSourcesFramework, \deliverableTechPubAdversarialFramework, \deliverableHEPPubAdversarialLFV \tabularnewline
\hline
\multicolumn{2}{|l|}{\textbf{\Tstrut Work Package:}
WP3,5,6} &
\multicolumn{2}{l|}{\textbf{Doctoral programme:} \parisUlong } &%\tabularnewline\hline
\multicolumn{2}{l|}{\textbf{\Tstrut Title: Adversarial RTA for industry and NP searches}
}\tabularnewline\hline
\multicolumn{6}{|p{20.2cm}|}{\textbf{\Tstrut Objectives:}
ESR9 will be study and be trained in ML and AI methods for RTA, both in industry and in HEP, apply these methods to the optimization of RTA of financial investments and medical insurance, and search for NP with LHCb data.
%The goal of ESR9 is to develop new methods for real-time data analysis. 
These methods include the identification and elimination of adversarial examples,
%in classifier training,
development and understanding of recurrent neural networks (RNNs) %and understanding of the most important patterns used by them,
%identification of the most important patterns
%which these networks rely on for their classification,
and use of heterogenous time-ordered and non-time-ordered datasets in RTA. 
ESR9 will benefit from being employed at \dqentity, industry leaders in these topics
particularly applied to financial or medical insurance applications, and academic supervision from Gligorov, whose ERC CoG team which ESR9 will integrate in works on closely related academic topics. ESR9 will work on developing \dqentity's RTA software in two main ways. 
First, a dedicated software framework for identifying, classifying, and eliminating
adversarial examples, which look similar to the human eye but differ at the pixel level.
%For instance, a self-driving car could be misled by pixel-level changes to a road-sign which a human would not even notice.
ESR9's framework will automate the generation of adversarial examples by systematically varying training
datasets, then teaching the NN to ignore such changes in its training. ESR9 will also work on \dqentity's DL
infrastructure, in particular understanding which patterns of information \dqentity's RNNs use to
take decisions. ESR9 will improve this framework to allow the use of non-time-ordered data
sources together with time-ordered data in order to take better decisions, particularly about financial investments or medical insurance. 
Finally, the adversarial example framework's generality will allow it to be used in a search for New Physics in the decays of strange hadrons with the LHCb
experiment. In particular, the method will be applied to search for LFV decays
of strange hadrons. 
Major backgrounds in such searches are in fact adversarial examples, generated by
much more frequent SM decays of strange hadrons with identical topologies but different final state particles. The framework developed
by ESR9 will allow these analyses to be performed in real-time, increasing their sensitivity by two orders of magnitude.
}\tabularnewline\hline
\multicolumn{6}{|p{20.2cm}|}{\textbf{\Tstrut Expected Results:}
Real-time frameworks for identifying/eliminating adversarial examples, and combining non-time-ordered data sources with time series. Each will be released as software
and documented in a technical publication. Peer-reviewed paper using adversarial example framework to search for NP in rare decays of strange hadrons. ESR9 will receive a PhD in experimental HEP at \sorbonneentity.
}\tabularnewline\hline
%\multicolumn{6}{|p{20.2cm}|}{\textbf{Doctoral program:} Cambridge}\tabularnewline\hline
\multicolumn{6}{|p{20.2cm}|}{\textbf{\Tstrut Secondments:}
4 months, CERN, Matev, training in development of software tools and physics analysis. 5 months, \santiagoentity, Borsato, deployment of adversarial example framework to the NP search. \dqentity and \cnrsentity are both based around Paris, ensuring coherence of supervision.
}\tabularnewline
\hline
\end{tabular}
}%
\end{center}
%\end{table}
%

%The ATLAS trigger infrastructure has an unique hardware processor to reconstruct the trajectory (tracking) of the charged particles that cross the silicon inner tracker of the experiment. The tracking information is an essential tool for effective real-time event selection and has a central role in the whole ATLAS physics program especially in the HL-LHC phase. The current hardware processor, FTK, is a complex system made by several custom electronics  boards based on FPGAs and Associative Memory chips. The latter are unique computing devices developed for the FTK algorithm. The hardware tracking will be also a central part of the Phase-II Upgrade of ATLAS, with upgraded version of FTK called FTK++.


