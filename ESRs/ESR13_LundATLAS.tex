%\begin{table}[h]
\begin{center}\small
\resizebox {\textwidth }{!}{%
\begin{tabular}{|p{21mm}|p{32mm}|p{15mm}|p{7mm}p{12mm}|p{19mm}|p{30mm}|p{34mm}|}
\hline
\textbf{\Tstrut Fellow} ESR13&
\textbf{Host} \ibmentity &
\textbf{\phd} \,\,Yes&
\textbf{Start} &Month 8&
\textbf{Duration} 36&
\textbf{Deliverables} \deliverableWhitepaperStateOfTheArtWPThree,\deliverableWhitepaperDevelopmentWPThree,\deliverableTriggerExperimentalSoftwareWPThree,\deliverableWhitepaperStateOfTheArtWPFive,\deliverableWhitepaperCollectionPapersWPFive &
\textbf{Work Package:} WP3,5,6\tabularnewline 
\hline
%\multicolumn{2}{|l|}{\textbf{\Tstrut Work Package:}
%WP3,5,7} &
\multicolumn{4}{|l|}{\textbf{Doctoral programme:} \lundentity } &%\tabularnewline\hline
\multicolumn{4}{l|}{\textbf{\Tstrut Title: Novelty detection for industry and ATLAS NP searches}
}\tabularnewline\hline
\multicolumn{8}{|p{20.2cm}|}{\textbf{\Tstrut Objectives:}
%New phenomena at the LHC can manifest themselves in unexpected ways: new particles may be discovered at the LHC given the enormous dataset available, even if a full theory predicting them is not yet available. However, it is crucial to distinguish unexpected, rare new phenomena by equally infrequent detector problems with similarly unexpected experimental signatures in the detector.
ESR13 will be trained in the most advanced industrial ML and AI methods, including development for supercomputer systems, and use these in RTA to search for dark matter particles with ATLAS.
%New "dark sector" 
Unlike SM particles, dark matter can decay within the ATLAS calorimeter without leaving a trace in previous detector volumes.
%A very similar signature would be observed from noisy elements in the calorimeter.
Calorimeter "noise bursts" have a very similar signature and while relatively infrequent
occur in 0.015\% of the events, far more frequent than dark matter. % (expecting fewer than 200000 of them in the full three-year LHC dataset spanning from 2015 to 2018). 
%Neither can be identified or distinguished in the real-time system of the ATLAS detector.  
ATLAS cannot currently distinguish the two in real-time.
Anomaly detection is an ML technique concerned with the detection of ultra-rare "anomalous" events which do not follow part of the "normal" pattern of input samples 
and where little is known about the distribution of these anomalies. 
%Very few novelties occur with respect to normal samples, making these techniques different from standard ML. 
A wealth of different techniques exist, and this project will use existing ML methods to detect novelties, while extending and/or
specializing the methods where appropriate to enable their use in RTA.
Of particular interest to IBM is the possibility to combine ML with powerful general mathematical programming solvers, such as the IBM CPLEX commercial product.
The first objective of ESR13 is to develop an algorithm that runs on existing ATLAS data and datasets provided by IBM, 
%identifies outliers in the calorimeter, 
and discriminates between signal and background, CPLEX. 
Subsequently, this algorithm will be compared to the open source version, and the latter implemented as an anomaly detection system running in the ATLAS
trigger, to be implemented in Run-3.% of the ATLAS detector.
The physics side of the project will be co-supervised by Doglioni (\lundentity).
%, who will supervise the student up to the completion of their PhD thesis. 
While benefiting from her expertise in searches for new physics, this work will represent a significant step beyond the ATLAS physics program,
since it targets final states never explored before. The 3-months \cern secondment will ensure that this work is well integrated in the trigger system of the ATLAS experiment.}\tabularnewline\hline
\multicolumn{8}{|p{20.2cm}|}{\textbf{\Tstrut Expected Results:}
Toolkit for novelty detection to be implemented in the ATLAS trigger, and relative peer-reviewed publications on a proof-of-principle physics analysis.  
ESR13 will receive a PhD in experimental HEP at \lundentity.
}\tabularnewline\hline
\multicolumn{8}{|p{20.2cm}|}{\textbf{\Tstrut Secondments:}
\ohioentity (at CERN), 6 months, Boveia. Benchmarking of algorithms for long-lived particle discrimination in dijet RTA. 
}\tabularnewline
\hline
\end{tabular}
}%
\end{center}
%\end{table}
%