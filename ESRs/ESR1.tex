%\begin{table}[h]

%\begin{table}[h]
\begin{center}\small
\resizebox {\textwidth }{!}{%
\begin{tabular}{|p{16mm}|p{33mm}|p{28mm}|p{18mm}|p{18mm}|p{67mm}|}
\hline
\textbf{\Tstrut Fellow} &
\textbf{Host} &
\textbf{\phd} &
\textbf{Start} &
\textbf{Duration} &
\textbf{Deliverables}\tabularnewline 
ESR1 &  \helsinki & Yes & Month 6& 36 & X.X\tabularnewline
\hline
\multicolumn{4}{|l|}{\textbf{\Tstrut Work Package:}
WP3, WP5, WP6, WP7} &
\multicolumn{2}{l|}{\textbf{Doctoral programme:} \dortmund }\tabularnewline\hline
\multicolumn{6}{|p{20.2cm}|}{\textbf{\Tstrut Project Title: Inclusive low dijet mass search at the trigger level  
}\tabularnewline\hline
\multicolumn{6}{|p{20.2cm}|}{\textbf{\Tstrut Objectives:}
}\tabularnewline\hline
(TO DO Henning)
Do online/offline jet matching to transfer offline final corrections back to trigger level; explore usage of DNN to ???transform??? HLT jets to offline jets (improve resolution).
Devise a joint strategy with ATLAS on how to correlate systematic uncertainties and align procedures to calibrate jets at HLT.
Perform dijet mass analysis in scouting and ???standard??? regime. Extend by considering dijet + hard radiation topology to reduce backgrounds.

The ESR will work with \ximantis's existing artificial intelligent modeling algorithms in order to improve their capabilities. The Stochastic model used to forecast traffic conditions and evolution within the \ximantis app has a number of unknown parameters, which require continuous calibration as traffic conditions are dynamic. Using a Convolutional Neural Network (CNN) to automatically analyze and dynamically calibrate these parameters is a novel and essential ingredient for the success of the predictions. 
Since the ability of CNNs to resolve serious mathematical modeling problems is rather recent, the theory is not yet fully developed for this purpose. The ESR will explore a number of ideas within the field in order to produce models with better predicting capabilities, in particular redesigning the CNN and changing the number of the parameters it is allowed to change on its own.
\multicolumn{6}{|p{20.2cm}|}{\textbf{\Tstrut Expected Results:}
Improved performance of HLT-jets
Contribute to trigger performance publication, presenting new techniques
??Standard?? dijet mass bump hunt, contribute to dark matter interpretation. 
ESR1 will receive a PhD in experimental HEP at \helsinki.
}\tabularnewline\hline
%\multicolumn{6}{|p{20.2cm}|}{\textbf{Doctoral program:} Cambridge}\tabularnewline\hline
\multicolumn{6}{|p{20.2cm}|}{\textbf{\Tstrut Secondments:}
\lund, 5 months, Dr. Caterina Doglioni. Topic: improving the performance of physics objects analysed in real-time using inter-experiment tools. 
\ximantis, 4 months, Dr. Alexandros Sopasakis. Enhancement of the existing artificial intelligent modelling algorithms
for traffic prediction, using Convolutional Neural Networks (CNNs). 
The ESR will gain hands-on experience within the training of those neural networks, as well as 
theoretical knowledge about the field of ML, AI and best current practices in general.
}\tabularnewline
\hline
\end{tabular}
}%
\end{center}
%\end{table}
%
